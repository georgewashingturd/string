\documentclass[aps,preprint,preprintnumbers,nofootinbib,showpacs,prd]{revtex4-1}
\usepackage{graphicx,color}
\usepackage{caption}
\usepackage{subcaption}
\usepackage{amsmath,amssymb}
\usepackage{multirow}
\usepackage{amsthm}%        But you can't use \usewithpatch for several packages as in this line. The search 

\usepackage{cancel}

%%% for SLE
\usepackage{dcolumn}   % needed for some tables
\usepackage{bm}        % for math
\usepackage{amssymb}   % for math
\usepackage{multirow}
%%% for SLE -End

\usepackage{ulem}
\usepackage{cancel}

\usepackage{hyperref}

\usepackage[top=1in, bottom=1.25in, left=1.1in, right=1.1in]{geometry}

\usepackage{mathtools} % for \DeclarePairedDelimiter{\ceil}{\lceil}{\rceil}

\usepackage{simplewick}

\newcommand{\msout}[1]{\text{\sout{\ensuremath{#1}}}}


%%%%%% My stuffs - Stef
\newcommand{\lsim}{\mathrel{\mathop{\kern 0pt \rlap
  {\raise.2ex\hbox{$<$}}}
  \lower.9ex\hbox{\kern-.190em $\sim$}}}
\newcommand{\gsim}{\mathrel{\mathop{\kern 0pt \rlap
  {\raise.2ex\hbox{$>$}}}
  \lower.9ex\hbox{\kern-.190em $\sim$}}}

%
% Key
%
\newcommand{\key}[1]{\medskip{\sffamily\bfseries\color{blue}#1}\par\medskip}
%\newcommand{\key}[1]{}
\newcommand{\q}[1] {\medskip{\sffamily\bfseries\color{red}#1}\par\medskip}
\newcommand{\comment}[2]{{\color{red}{{\bf #1:}  #2}}}


\newcommand{\ie}{{\it i.e.} }
\newcommand{\eg}{{\it e.g.} }

%
% Energy scales
%
\newcommand{\ev}{{\,{\rm eV}}}
\newcommand{\kev}{{\,{\rm keV}}}
\newcommand{\mev}{{\,{\rm MeV}}}
\newcommand{\gev}{{\,{\rm GeV}}}
\newcommand{\tev}{{\,{\rm TeV}}}
\newcommand{\fb}{{\,{\rm fb}}}
\newcommand{\ifb}{{\,{\rm fb}^{-1}}}

%
% SUSY notations
%
\newcommand{\neu}{\tilde{\chi}^0}
\newcommand{\neuo}{{\tilde{\chi}^0_1}}
\newcommand{\neut}{{\tilde{\chi}^0_2}}
\newcommand{\cha}{{\tilde{\chi}^\pm}}
\newcommand{\chao}{{\tilde{\chi}^\pm_1}}
\newcommand{\chaop}{{\tilde{\chi}^+_1}}
\newcommand{\chaom}{{\tilde{\chi}^-_1}}
\newcommand{\Wpm}{W^\pm}
\newcommand{\chat}{{\tilde{\chi}^\pm_2}}
\newcommand{\smu}{{\tilde{\mu}}}
\newcommand{\smur}{\tilde{\mu}_R}
\newcommand{\smul}{\tilde{\mu}_L}
\newcommand{\sel}{{\tilde{e}}}
\newcommand{\selr}{\tilde{e}_R}
\newcommand{\sell}{\tilde{e}_L}
\newcommand{\smurl}{\tilde{\mu}_{R,L}}

\newcommand{\casea}{\texttt{IA}}
\newcommand{\caseb}{\texttt{IB}}
\newcommand{\casec}{\texttt{II}}

\newcommand{\caseasix}{\texttt{IA-6}}

%
% Greek
%
\newcommand{\es}{{\epsilon}}
\newcommand{\sg}{{\sigma}}
\newcommand{\dt}{{\delta}}
\newcommand{\kp}{{\kappa}}
\newcommand{\lm}{{\lambda}}
\newcommand{\Lm}{{\Lambda}}
\newcommand{\gm}{{\gamma}}
\newcommand{\mn}{{\mu\nu}}
\newcommand{\Gm}{{\Gamma}}
\newcommand{\tho}{{\theta_1}}
\newcommand{\tht}{{\theta_2}}
\newcommand{\lmo}{{\lambda_1}}
\newcommand{\lmt}{{\lambda_2}}
%
% LaTeX equations
%
\newcommand{\beq}{\begin{equation}}
\newcommand{\eeq}{\end{equation}}
\newcommand{\bea}{\begin{eqnarray}}
\newcommand{\eea}{\end{eqnarray}}
\newcommand{\ba}{\begin{array}}
\newcommand{\ea}{\end{array}}
\newcommand{\bit}{\begin{itemize}}
\newcommand{\eit}{\end{itemize}}

\newcommand{\nbea}{\begin{eqnarray*}}
\newcommand{\neea}{\end{eqnarray*}}
\newcommand{\nbeq}{\begin{equation*}}
\newcommand{\neeq}{\end{equation*}}

\newcommand{\no}{{\nonumber}}
\newcommand{\td}[1]{{\widetilde{#1}}}
\newcommand{\sqt}{{\sqrt{2}}}
%
\newcommand{\me}{{\rlap/\!E}}
\newcommand{\met}{{\rlap/\!E_T}}
\newcommand{\rdmu}{{\partial^\mu}}
\newcommand{\gmm}{{\gamma^\mu}}
\newcommand{\gmb}{{\gamma^\beta}}
\newcommand{\gma}{{\gamma^\alpha}}
\newcommand{\gmn}{{\gamma^\nu}}
\newcommand{\gmf}{{\gamma^5}}
%
% Roman expressions
%
\newcommand{\br}{{\rm Br}}
\newcommand{\sign}{{\rm sign}}
\newcommand{\Lg}{{\mathcal{L}}}
\newcommand{\M}{{\mathcal{M}}}
\newcommand{\tr}{{\rm Tr}}

\newcommand{\msq}{{\overline{|\mathcal{M}|^2}}}

%
% kinematic variables
%
%\newcommand{\mc}{m^{\rm cusp}}
%\newcommand{\mmax}{m^{\rm max}}
%\newcommand{\mmin}{m^{\rm min}}
%\newcommand{\mll}{m_{\ell\ell}}
%\newcommand{\mllc}{m^{\rm cusp}_{\ell\ell}}
%\newcommand{\mllmax}{m^{\rm max}_{\ell\ell}}
%\newcommand{\mllmin}{m^{\rm min}_{\ell\ell}}
%\newcommand{\elmax} {E_\ell^{\rm max}}
%\newcommand{\elmin} {E_\ell^{\rm min}}
\newcommand{\mxx}{m_{\chi\chi}}
\newcommand{\mrec}{m_{\rm rec}}
\newcommand{\mrecmin}{m_{\rm rec}^{\rm min}}
\newcommand{\mrecc}{m_{\rm rec}^{\rm cusp}}
\newcommand{\mrecmax}{m_{\rm rec}^{\rm max}}
%\newcommand{\mpt}{\rlap/p_T}

%%%song
\newcommand{\cosmax}{|\cos\Theta|_{\rm max} }
\newcommand{\maa}{m_{aa}}
\newcommand{\maac}{m^{\rm cusp}_{aa}}
\newcommand{\maamax}{m^{\rm max}_{aa}}
\newcommand{\maamin}{m^{\rm min}_{aa}}
\newcommand{\eamax} {E_a^{\rm max}}
\newcommand{\eamin} {E_a^{\rm min}}
\newcommand{\eaamax} {E_{aa}^{\rm max}}
\newcommand{\eaacusp} {E_{aa}^{\rm cusp}}
\newcommand{\eaamin} {E_{aa}^{\rm min}}
\newcommand{\exxmax} {E_{\neuo \neuo}^{\rm max}}
\newcommand{\exxcusp} {E_{\neuo \neuo}^{\rm cusp}}
\newcommand{\exxmin} {E_{\neuo \neuo}^{\rm min}}
%\newcommand{\mxx}{m_{XX}}
%\newcommand{\mrec}{m_{\rm rec}}
\newcommand{\erec}{E_{\rm rec}}
%\newcommand{\mrecmin}{m_{\rm rec}^{\rm min}}
%\newcommand{\mrecc}{m_{\rm rec}^{\rm cusp}}
%\newcommand{\mrecmax}{m_{\rm rec}^{\rm max}}
%%%song

\newcommand{\mc}{m^{\rm cusp}}
\newcommand{\mmax}{m^{\rm max}}
\newcommand{\mmin}{m^{\rm min}}
\newcommand{\mll}{m_{\mu\mu}}
\newcommand{\mllc}{m^{\rm cusp}_{\mu\mu}}
\newcommand{\mllmax}{m^{\rm max}_{\mu\mu}}
\newcommand{\mllmin}{m^{\rm min}_{\mu\mu}}
\newcommand{\mllcusp}{m^{\rm cusp}_{\mu\mu}}
\newcommand{\elmax} {E_\mu^{\rm max}}
\newcommand{\elmin} {E_\mu^{\rm min}}
\newcommand{\elmaxw} {E_W^{\rm max}}
\newcommand{\elminw} {E_W^{\rm min}}
\newcommand{\R} {{\cal R}}

\newcommand{\ewmax} {E_W^{\rm max}}
\newcommand{\ewmin} {E_W^{\rm min}}
\newcommand{\mwrec}{m_{WW}}
\newcommand{\mwrecmin}{m_{WW}^{\rm min}}
\newcommand{\mwrecc}{m_{WW}^{\rm cusp}}
\newcommand{\mwrecmax}{m_{WW}^{\rm max}}

\newcommand{\mpt}{{\rlap/p}_T}

%%%%%% END My stuffs - Stef

\newcommand{\dunno}{$ {}^{\mbox {--}}\backslash(^{\rm o}{}\underline{\hspace{0.2cm}}{\rm o})/^{\mbox {--}}$}

\DeclarePairedDelimiter{\ceil}{\lceil}{\rceil}
\DeclarePairedDelimiter{\floor}{\lfloor}{\rfloor}

\DeclareMathOperator{\ord}{ord}
\DeclareMathOperator{\tor}{tor}





\begin{document}

\title{Apostol's Analytic Number Theory}
\bigskip
\author{Stefanus$^1$\\
$^1$ Samsung Semiconductor Inc\\ San Jose, CA 95134 USA\\
}
%
\date{\today}
%
\begin{abstract}
Just for fun :)

\end{abstract}
%
\maketitle

\renewcommand{\theequation}{A.\arabic{equation}}  % redefine the command that creates the equation no.
\setcounter{equation}{0}  % reset counter 

\underline{\textbf{\textit{Chapter 2}}}
\bigskip

{\bf Problem 2.1} Find all integers $n$ such that
%
\nbea
\begin{array}{l r c l c l r c l c l r c l}
{\rm(a)} & \varphi(n) & = & n/2 & ~~~~~~~~~ & {\rm(b)} & \varphi(n) & = & \varphi(2n) & ~~~~~~~~~ & {\rm(c)} & \varphi(n) & = & 12
\end{array}
\neea
%

For (a), using the definition of $\varphi(n)$, $\varphi(n) = n \prod_{p|n} \left ( 1 - \frac{1}{p} \right )$
%
\nbea
\frac{\bcancel{n}}{2} & = & \bcancel{n} \prod_{p|n} \left ( 1 - \frac{1}{p} \right ) \\
\frac{1}{2} & = & \frac{\prod_{p|n} \left ( p - 1 \right )}{\prod_{p|n} p} \\
\prod_{p|n} p & = & 2\prod_{p|n} \left ( p - 1 \right )
\neea
%
if $n$ is odd the LHS is odd while the RHS is even, so it can't be. If $n$ is even the LHS only has one factor of 2 while the RHS has many so it will only work if $n=2$.

For (b)
%
\nbea
\bcancel{n} \prod_{p|n} \left ( 1 - \frac{1}{p} \right )  & = & 2\bcancel{n} \prod_{p|2n} \left ( 1 - \frac{1}{p} \right )
\neea
%
If $n$ is even then
%
\nbea
\prod_{p|2n} \left ( 1 - \frac{1}{p} \right ) & = & \prod_{p|n} \left ( 1 - \frac{1}{p} \right )
\neea
%
and so
%
\nbea
\prod_{p|n} \left ( 1 - \frac{1}{p} \right )  & = & 2\prod_{p|n} \left ( 1 - \frac{1}{p} \right ) \\
\to 1 & = & 2
\neea
%
which is impossible, so $n$ has to be odd, in that case
%
\nbea
\prod_{p|2n} \left ( 1 - \frac{1}{p} \right ) & = & \left ( 1 - \frac{1}{2} \right ) \prod_{p|n} \left ( 1 - \frac{1}{p} \right ) \\
& = & \frac{1}{2} \prod_{p|n} \left ( 1 - \frac{1}{p} \right )
\neea
%
and therefore
%
\nbea
\prod_{p|n} \left ( 1 - \frac{1}{p} \right )  & = & 2 \frac{1}{2} \prod_{p|n} \left ( 1 - \frac{1}{p} \right ) \\
\to 1 & = & 1
\neea
%
and therefore $\varphi(n) = \varphi(2n)$ for all odd $n$.

For (c)
%
\nbea
\varphi(n) = 12 & = & 2 \cdot 2 \cdot 3 \\
& = & \prod_{p|n} p^{\alpha_p} - p^{\alpha_p-1} \\
\varphi\left (\prod_{p|n} p^{\alpha_p} \right ) & = & \prod_{p|n}p^{\alpha_p-1} (p - 1)
\neea
%
the only possible solution is $n=13$

{\bf Problem 2.2}. For each of the following statements either give a proof or exhibit a counter example.

(a) If $(m,n)=1$ then $(\varphi(m),\varphi(n)) = 1$

(b) If $n$ is composite, then $(n, \varphi(n)) > 1$

(c) If the same primes divide $m$ and $n$, then $n\varphi(m) = m\varphi(n)$

For (a) a counter example will be $(3,4) = 1$, while $\varphi(3) = 2, ~\varphi(4) = 2$

For (b) a counter example would be $n = 15$ which means that $\varphi(15) = 8$ and $(15,8) = 1$

For (c) I think what it means by ``the same primes divide $m$ and $n$'' is that $m = \prod p^{\alpha_p}$ and $n = \prod p^{\beta_p}$, so they both have the same primes but they might have different exponents for each prime, in this case $\prod_{p|n} = \prod_{p|m}$
%
\nbea
n\varphi(m) & = & n \left ( m \prod_{p|m} \left ( 1 - \frac{1}{p}\right ) \right ) \\
& = & m \left ( n \prod_{p|n} \left ( 1 - \frac{1}{p}\right ) \right ) \\
n\varphi(m) & = & m\varphi(n)
\neea
%

{\bf Problem 2.3}. Prove that
%
\nbea
\frac{n}{\varphi(n)} = \sum_{d|n} \frac{\mu^2(d)}{\varphi(d)}
\neea
%

Since $\mu(n)$ and $\varphi(n)$ are both multiplicative so is $\mu^2/\varphi$, in that case $g(n) = \sum_{d|n} \frac{\mu^2(d)}{\varphi(d)}$ is also multiplicative. To determine $g(n)$ we need only compute $g(p^\alpha)$ for prime powers
%
\nbea
g(p^\alpha) & = & \sum_{d|p^\alpha} \frac{\mu^2(d)}{\varphi(d)} \\
& = & \frac{\mu^2(1)}{\varphi(1)} + \frac{\mu^2(p)}{\varphi(p)} + \ldots + \frac{\mu^2(p^\alpha)}{\varphi(p^\alpha)} \\
& = & 1 + \frac{1}{p - 1} \\
& = & \frac{p}{p - 1} \\
& = & p^\alpha \cdot \frac{p}{p^\alpha(p - 1)} \\
\to \sum_{d|p^\alpha} \frac{\mu^2(d)}{\varphi(d)}& = & \frac{p^\alpha}{\varphi(p^\alpha)}
\neea
%

We can also prove it the other way around by assuming the LHS, to do this it is easiest to use the Mobius inversion formula
%
\nbea
\frac{n}{\varphi(n)} = \sum_{d|n} g(d)
\neea
%
and we want to find out what this $g(d)$ is, which is
%
\nbea
g(n) & = & \sum_{d|n} \frac{d}{\varphi(d)} \mu\left ( \frac{n}{d}\right )
\neea
%
The RHS is multiplicative so like above we just need to evaluate $g(p^\alpha)$ for prime powers
%
\nbea
g(p^\alpha) & = & \sum_{d|p^\alpha} \frac{d}{\varphi(d)} \mu\left ( \frac{p^\alpha}{d}\right ) \\
& = & \frac{p^{\alpha-1}}{\varphi(p^{\alpha-1})} \mu\left ( \frac{p^\alpha}{p^{\alpha-1}}\right ) + \frac{p^\alpha}{\varphi(p^\alpha)} \mu\left ( \frac{p^\alpha}{p^\alpha}\right ) \\
& = & -\frac{p^{\alpha-1}}{\varphi(p^{\alpha-1})} + \frac{p^\alpha}{\varphi(p^\alpha)} \\
& = & -\frac{p^\alpha}{\varphi(p^\alpha)}  + \frac{p^\alpha}{\varphi(p^\alpha)} \\
& = & 0
\neea
%
if $\alpha > 1$ and if $\alpha = 1$ we get
%
\nbea
g(p) & = & \sum_{d|p} \frac{d}{\varphi(d)} \mu\left ( \frac{p}{d}\right ) \\
& = & \frac{1}{\varphi(1)} \mu\left ( \frac{p}{1}\right ) + \frac{p}{\varphi(p)} \mu\left ( \frac{p}{p}\right ) \\
& = & -1 + \frac{p}{\varphi(p)} \\
& = & -1 + \frac{p}{p - 1} \\
& = & \frac{1}{p - 1} \\
g(p) & = & \frac{1}{\varphi(p)}
\neea
%
This means that $g(p^\alpha) = 1/\varphi(p^\alpha)$ is $\alpha = 1$ and $g(p^\alpha) = 0$ if $\alpha > 1$, in other words $g(p^\alpha) = \mu^2(p^\alpha)/\varphi(p^\alpha)$








\end{document}