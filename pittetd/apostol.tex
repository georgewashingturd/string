\documentclass[aps,preprint,preprintnumbers,nofootinbib,showpacs,prd]{revtex4-1}
\usepackage{graphicx,color}
\usepackage{caption}
\usepackage{subcaption}
\usepackage{amsmath,amssymb}
\usepackage{multirow}
\usepackage{amsthm}%        But you can't use \usewithpatch for several packages as in this line. The search 

\usepackage{cancel}

%%% for SLE
\usepackage{dcolumn}   % needed for some tables
\usepackage{bm}        % for math
\usepackage{amssymb}   % for math
\usepackage{multirow}
%%% for SLE -End

\usepackage{ulem}
\usepackage{cancel}

\usepackage{hyperref}

\usepackage[top=1in, bottom=1.25in, left=1.1in, right=1.1in]{geometry}

\usepackage{mathtools} % for \DeclarePairedDelimiter{\ceil}{\lceil}{\rceil}

\usepackage{simplewick}

\newcommand{\msout}[1]{\text{\sout{\ensuremath{#1}}}}


%%%%%% My stuffs - Stef
\newcommand{\lsim}{\mathrel{\mathop{\kern 0pt \rlap
  {\raise.2ex\hbox{$<$}}}
  \lower.9ex\hbox{\kern-.190em $\sim$}}}
\newcommand{\gsim}{\mathrel{\mathop{\kern 0pt \rlap
  {\raise.2ex\hbox{$>$}}}
  \lower.9ex\hbox{\kern-.190em $\sim$}}}

%
% Key
%
\newcommand{\key}[1]{\medskip{\sffamily\bfseries\color{blue}#1}\par\medskip}
%\newcommand{\key}[1]{}
\newcommand{\q}[1] {\medskip{\sffamily\bfseries\color{red}#1}\par\medskip}
\newcommand{\comment}[2]{{\color{red}{{\bf #1:}  #2}}}


\newcommand{\ie}{{\it i.e.} }
\newcommand{\eg}{{\it e.g.} }

%
% Energy scales
%
\newcommand{\ev}{{\,{\rm eV}}}
\newcommand{\kev}{{\,{\rm keV}}}
\newcommand{\mev}{{\,{\rm MeV}}}
\newcommand{\gev}{{\,{\rm GeV}}}
\newcommand{\tev}{{\,{\rm TeV}}}
\newcommand{\fb}{{\,{\rm fb}}}
\newcommand{\ifb}{{\,{\rm fb}^{-1}}}

%
% SUSY notations
%
\newcommand{\neu}{\tilde{\chi}^0}
\newcommand{\neuo}{{\tilde{\chi}^0_1}}
\newcommand{\neut}{{\tilde{\chi}^0_2}}
\newcommand{\cha}{{\tilde{\chi}^\pm}}
\newcommand{\chao}{{\tilde{\chi}^\pm_1}}
\newcommand{\chaop}{{\tilde{\chi}^+_1}}
\newcommand{\chaom}{{\tilde{\chi}^-_1}}
\newcommand{\Wpm}{W^\pm}
\newcommand{\chat}{{\tilde{\chi}^\pm_2}}
\newcommand{\smu}{{\tilde{\mu}}}
\newcommand{\smur}{\tilde{\mu}_R}
\newcommand{\smul}{\tilde{\mu}_L}
\newcommand{\sel}{{\tilde{e}}}
\newcommand{\selr}{\tilde{e}_R}
\newcommand{\sell}{\tilde{e}_L}
\newcommand{\smurl}{\tilde{\mu}_{R,L}}

\newcommand{\casea}{\texttt{IA}}
\newcommand{\caseb}{\texttt{IB}}
\newcommand{\casec}{\texttt{II}}

\newcommand{\caseasix}{\texttt{IA-6}}

%
% Greek
%
\newcommand{\es}{{\epsilon}}
\newcommand{\sg}{{\sigma}}
\newcommand{\dt}{{\delta}}
\newcommand{\kp}{{\kappa}}
\newcommand{\lm}{{\lambda}}
\newcommand{\Lm}{{\Lambda}}
\newcommand{\gm}{{\gamma}}
\newcommand{\mn}{{\mu\nu}}
\newcommand{\Gm}{{\Gamma}}
\newcommand{\tho}{{\theta_1}}
\newcommand{\tht}{{\theta_2}}
\newcommand{\lmo}{{\lambda_1}}
\newcommand{\lmt}{{\lambda_2}}
%
% LaTeX equations
%
\newcommand{\beq}{\begin{equation}}
\newcommand{\eeq}{\end{equation}}
\newcommand{\bea}{\begin{eqnarray}}
\newcommand{\eea}{\end{eqnarray}}
\newcommand{\ba}{\begin{array}}
\newcommand{\ea}{\end{array}}
\newcommand{\bit}{\begin{itemize}}
\newcommand{\eit}{\end{itemize}}

\newcommand{\nbea}{\begin{eqnarray*}}
\newcommand{\neea}{\end{eqnarray*}}
\newcommand{\nbeq}{\begin{equation*}}
\newcommand{\neeq}{\end{equation*}}

\newcommand{\no}{{\nonumber}}
\newcommand{\td}[1]{{\widetilde{#1}}}
\newcommand{\sqt}{{\sqrt{2}}}
%
\newcommand{\me}{{\rlap/\!E}}
\newcommand{\met}{{\rlap/\!E_T}}
\newcommand{\rdmu}{{\partial^\mu}}
\newcommand{\gmm}{{\gamma^\mu}}
\newcommand{\gmb}{{\gamma^\beta}}
\newcommand{\gma}{{\gamma^\alpha}}
\newcommand{\gmn}{{\gamma^\nu}}
\newcommand{\gmf}{{\gamma^5}}
%
% Roman expressions
%
\newcommand{\br}{{\rm Br}}
\newcommand{\sign}{{\rm sign}}
\newcommand{\Lg}{{\mathcal{L}}}
\newcommand{\M}{{\mathcal{M}}}
\newcommand{\tr}{{\rm Tr}}

\newcommand{\msq}{{\overline{|\mathcal{M}|^2}}}

%
% kinematic variables
%
%\newcommand{\mc}{m^{\rm cusp}}
%\newcommand{\mmax}{m^{\rm max}}
%\newcommand{\mmin}{m^{\rm min}}
%\newcommand{\mll}{m_{\ell\ell}}
%\newcommand{\mllc}{m^{\rm cusp}_{\ell\ell}}
%\newcommand{\mllmax}{m^{\rm max}_{\ell\ell}}
%\newcommand{\mllmin}{m^{\rm min}_{\ell\ell}}
%\newcommand{\elmax} {E_\ell^{\rm max}}
%\newcommand{\elmin} {E_\ell^{\rm min}}
\newcommand{\mxx}{m_{\chi\chi}}
\newcommand{\mrec}{m_{\rm rec}}
\newcommand{\mrecmin}{m_{\rm rec}^{\rm min}}
\newcommand{\mrecc}{m_{\rm rec}^{\rm cusp}}
\newcommand{\mrecmax}{m_{\rm rec}^{\rm max}}
%\newcommand{\mpt}{\rlap/p_T}

%%%song
\newcommand{\cosmax}{|\cos\Theta|_{\rm max} }
\newcommand{\maa}{m_{aa}}
\newcommand{\maac}{m^{\rm cusp}_{aa}}
\newcommand{\maamax}{m^{\rm max}_{aa}}
\newcommand{\maamin}{m^{\rm min}_{aa}}
\newcommand{\eamax} {E_a^{\rm max}}
\newcommand{\eamin} {E_a^{\rm min}}
\newcommand{\eaamax} {E_{aa}^{\rm max}}
\newcommand{\eaacusp} {E_{aa}^{\rm cusp}}
\newcommand{\eaamin} {E_{aa}^{\rm min}}
\newcommand{\exxmax} {E_{\neuo \neuo}^{\rm max}}
\newcommand{\exxcusp} {E_{\neuo \neuo}^{\rm cusp}}
\newcommand{\exxmin} {E_{\neuo \neuo}^{\rm min}}
%\newcommand{\mxx}{m_{XX}}
%\newcommand{\mrec}{m_{\rm rec}}
\newcommand{\erec}{E_{\rm rec}}
%\newcommand{\mrecmin}{m_{\rm rec}^{\rm min}}
%\newcommand{\mrecc}{m_{\rm rec}^{\rm cusp}}
%\newcommand{\mrecmax}{m_{\rm rec}^{\rm max}}
%%%song

\newcommand{\mc}{m^{\rm cusp}}
\newcommand{\mmax}{m^{\rm max}}
\newcommand{\mmin}{m^{\rm min}}
\newcommand{\mll}{m_{\mu\mu}}
\newcommand{\mllc}{m^{\rm cusp}_{\mu\mu}}
\newcommand{\mllmax}{m^{\rm max}_{\mu\mu}}
\newcommand{\mllmin}{m^{\rm min}_{\mu\mu}}
\newcommand{\mllcusp}{m^{\rm cusp}_{\mu\mu}}
\newcommand{\elmax} {E_\mu^{\rm max}}
\newcommand{\elmin} {E_\mu^{\rm min}}
\newcommand{\elmaxw} {E_W^{\rm max}}
\newcommand{\elminw} {E_W^{\rm min}}
\newcommand{\R} {{\cal R}}

\newcommand{\ewmax} {E_W^{\rm max}}
\newcommand{\ewmin} {E_W^{\rm min}}
\newcommand{\mwrec}{m_{WW}}
\newcommand{\mwrecmin}{m_{WW}^{\rm min}}
\newcommand{\mwrecc}{m_{WW}^{\rm cusp}}
\newcommand{\mwrecmax}{m_{WW}^{\rm max}}

\newcommand{\mpt}{{\rlap/p}_T}

%%%%%% END My stuffs - Stef

\newcommand{\dunno}{$ {}^{\mbox {--}}\backslash(^{\rm o}{}\underline{\hspace{0.2cm}}{\rm o})/^{\mbox {--}}$}

\DeclarePairedDelimiter{\ceil}{\lceil}{\rceil}
\DeclarePairedDelimiter{\floor}{\lfloor}{\rfloor}

\DeclareMathOperator{\ord}{ord}
\DeclareMathOperator{\tor}{tor}





\begin{document}

\title{Apostol's Analytic Number Theory}
\bigskip
\author{Stefanus$^1$\\
$^1$ Samsung Semiconductor Inc\\ San Jose, CA 95134 USA\\
}
%
\date{\today}
%
\begin{abstract}
Just for fun :)

\end{abstract}
%
\maketitle

\renewcommand{\theequation}{A.\arabic{equation}}  % redefine the command that creates the equation no.
\setcounter{equation}{0}  % reset counter 

\underline{\textbf{\textit{Chapter 2}}}
\bigskip

{\bf Problem 2.1} Find all integers $n$ such that
%
\nbea
\begin{array}{l r c l c l r c l c l r c l}
{\rm(a)} & \varphi(n) & = & n/2 & ~~~~~~~~~ & {\rm(b)} & \varphi(n) & = & \varphi(2n) & ~~~~~~~~~ & {\rm(c)} & \varphi(n) & = & 12
\end{array}
\neea
%

For (a), using the definition of $\varphi(n)$, $\varphi(n) = n \prod_{p|n} \left ( 1 - \frac{1}{p} \right )$
%
\nbea
\frac{\bcancel{n}}{2} & = & \bcancel{n} \prod_{p|n} \left ( 1 - \frac{1}{p} \right ) \\
\frac{1}{2} & = & \frac{\prod_{p|n} \left ( p - 1 \right )}{\prod_{p|n} p} \\
\prod_{p|n} p & = & 2\prod_{p|n} \left ( p - 1 \right )
\neea
%
if $n$ is odd the LHS is odd while the RHS is even, so it can't be. If $n$ is even the LHS only has one factor of 2 while the RHS has many so it will only work if $n=2$.

For (b)
%
\nbea
\bcancel{n} \prod_{p|n} \left ( 1 - \frac{1}{p} \right )  & = & 2\bcancel{n} \prod_{p|2n} \left ( 1 - \frac{1}{p} \right )
\neea
%
If $n$ is even then
%
\nbea
\prod_{p|2n} \left ( 1 - \frac{1}{p} \right ) & = & \prod_{p|n} \left ( 1 - \frac{1}{p} \right )
\neea
%
and so
%
\nbea
\prod_{p|n} \left ( 1 - \frac{1}{p} \right )  & = & 2\prod_{p|n} \left ( 1 - \frac{1}{p} \right ) \\
\to 1 & = & 2
\neea
%
which is impossible, so $n$ has to be odd, in that case
%
\nbea
\prod_{p|2n} \left ( 1 - \frac{1}{p} \right ) & = & \left ( 1 - \frac{1}{2} \right ) \prod_{p|n} \left ( 1 - \frac{1}{p} \right ) \\
& = & \frac{1}{2} \prod_{p|n} \left ( 1 - \frac{1}{p} \right )
\neea
%
and therefore
%
\nbea
\prod_{p|n} \left ( 1 - \frac{1}{p} \right )  & = & 2 \frac{1}{2} \prod_{p|n} \left ( 1 - \frac{1}{p} \right ) \\
\to 1 & = & 1
\neea
%
and therefore $\varphi(n) = \varphi(2n)$ for all odd $n$.

For (c)
%
\nbea
\varphi(n) = 12 & = & 2 \cdot 2 \cdot 3 \\
& = & \prod_{p|n} p^{\alpha_p} - p^{\alpha_p-1} \\
\varphi\left (\prod_{p|n} p^{\alpha_p} \right ) & = & \prod_{p|n}p^{\alpha_p-1} (p - 1)
\neea
%
the only possible solution is $n=13$

{\bf Problem 2.2}. For each of the following statements either give a proof or exhibit a counter example.

(a) If $(m,n)=1$ then $(\varphi(m),\varphi(n)) = 1$

(b) If $n$ is composite, then $(n, \varphi(n)) > 1$

(c) If the same primes divide $m$ and $n$, then $n\varphi(m) = m\varphi(n)$

For (a) a counter example will be $(3,4) = 1$, while $\varphi(3) = 2, ~\varphi(4) = 2$

For (b) a counter example would be $n = 15$ which means that $\varphi(15) = 8$ and $(15,8) = 1$

For (c) I think what it means by ``the same primes divide $m$ and $n$'' is that $m = \prod p^{\alpha_p}$ and $n = \prod p^{\beta_p}$, so they both have the same primes but they might have different exponents for each prime, in this case $\prod_{p|n} = \prod_{p|m}$
%
\nbea
n\varphi(m) & = & n \left ( m \prod_{p|m} \left ( 1 - \frac{1}{p}\right ) \right ) \\
& = & m \left ( n \prod_{p|n} \left ( 1 - \frac{1}{p}\right ) \right ) \\
n\varphi(m) & = & m\varphi(n)
\neea
%

{\bf Problem 2.3}. Prove that
%
\nbea
\frac{n}{\varphi(n)} = \sum_{d|n} \frac{\mu^2(d)}{\varphi(d)}
\neea
%

Since $\mu(n)$ and $\varphi(n)$ are both multiplicative so is $\mu^2/\varphi$, in that case $g(n) = \sum_{d|n} \frac{\mu^2(d)}{\varphi(d)}$ is also multiplicative. To determine $g(n)$ we need only compute $g(p^\alpha)$ for prime powers
%
\nbea
g(p^\alpha) & = & \sum_{d|p^\alpha} \frac{\mu^2(d)}{\varphi(d)} \\
& = & \frac{\mu^2(1)}{\varphi(1)} + \frac{\mu^2(p)}{\varphi(p)} + \ldots + \frac{\mu^2(p^\alpha)}{\varphi(p^\alpha)} \\
& = & 1 + \frac{1}{p - 1} \\
& = & \frac{p}{p - 1} \\
& = & p^\alpha \cdot \frac{p}{p^\alpha(p - 1)} \\
\to \sum_{d|p^\alpha} \frac{\mu^2(d)}{\varphi(d)}& = & \frac{p^\alpha}{\varphi(p^\alpha)}
\neea
%

We can also prove it the other way around by assuming the LHS, to do this it is easiest to use the Mobius inversion formula
%
\nbea
\frac{n}{\varphi(n)} = \sum_{d|n} g(d)
\neea
%
and we want to find out what this $g(d)$ is, which is
%
\nbea
g(n) & = & \sum_{d|n} \frac{d}{\varphi(d)} \mu\left ( \frac{n}{d}\right )
\neea
%
The RHS is multiplicative so like above we just need to evaluate $g(p^\alpha)$ for prime powers
%
\nbea
g(p^\alpha) & = & \sum_{d|p^\alpha} \frac{d}{\varphi(d)} \mu\left ( \frac{p^\alpha}{d}\right ) \\
& = & \frac{p^{\alpha-1}}{\varphi(p^{\alpha-1})} \mu\left ( \frac{p^\alpha}{p^{\alpha-1}}\right ) + \frac{p^\alpha}{\varphi(p^\alpha)} \mu\left ( \frac{p^\alpha}{p^\alpha}\right ) \\
& = & -\frac{p^{\alpha-1}}{\varphi(p^{\alpha-1})} + \frac{p^\alpha}{\varphi(p^\alpha)} \\
& = & -\frac{p^\alpha}{\varphi(p^\alpha)}  + \frac{p^\alpha}{\varphi(p^\alpha)} \\
& = & 0
\neea
%
if $\alpha > 1$ and if $\alpha = 1$ we get
%
\nbea
g(p) & = & \sum_{d|p} \frac{d}{\varphi(d)} \mu\left ( \frac{p}{d}\right ) \\
& = & \frac{1}{\varphi(1)} \mu\left ( \frac{p}{1}\right ) + \frac{p}{\varphi(p)} \mu\left ( \frac{p}{p}\right ) \\
& = & -1 + \frac{p}{\varphi(p)} \\
& = & -1 + \frac{p}{p - 1} \\
& = & \frac{1}{p - 1} \\
g(p) & = & \frac{1}{\varphi(p)}
\neea
%
This means that $g(p^\alpha) = 1/\varphi(p^\alpha)$ is $\alpha = 1$ and $g(p^\alpha) = 0$ if $\alpha > 1$, in other words $g(p^\alpha) = \mu^2(p^\alpha)/\varphi(p^\alpha)$

{\bf Problem 2.4}. Prove that $\varphi(n) > n/6$ for all $n$ with at most $8$ distinct prime factors.

First, let's demystify this number $8$, the reason $8$ is involved is because if you multiply out $(p-1)/p$ for the first eight primes we get
%
\nbea
\frac{1}{2}\cdot\frac{2}{3}\cdot\frac{4}{5}\cdot\frac{6}{7}\cdot\frac{10}{11}\cdot\frac{12}{13}\cdot\frac{16}{17}\cdot\frac{18}{19} & = & \frac{55296}{323323} \sim 0.171  > \frac{1}{6}
\neea
%
but if we multiply the first nine
%
\nbea
\frac{1}{2}\cdot\frac{2}{3}\cdot\frac{4}{5}\cdot\frac{6}{7}\cdot\frac{10}{11}\cdot\frac{12}{13}\cdot\frac{16}{17}\cdot\frac{18}{19}\cdot\frac{22}{23} & = & \frac{110592}{676039} \sim 0.164  < \frac{1}{6}
\neea
%
So that's how we got the eight and of course if we chose any other eight primes we will get something bigger than $55296/323323>1/6$ because $n/(n+1)$ converges to 1 as $n \to \infty$, \ie $n/(n+1)$ gets bigger as $n$ gets bigger.

Another reason we have to limit it to eight is because $n/(n+1) < 1$ and if we keep multiplying them we'll get a smaller and smaller number and after some point we will reach $< 1/6$.

The rest is straightforward,
%
\nbea
\frac{\varphi(n)}{n} & = & \prod_{p|n}\frac{p - 1}{p}
\neea
%
so the argument above holds

{\bf Problem 2.5}. Define $\nu(1) = 0$, and for $n > 1$ let $\nu(n)$ be the number of distinct prime factors of $n$. Let $f = \mu * \nu$ and prove that $f(n)$ is either 0 or 1.

As the inverse of $\mu$ is $\mu^{-1} = u$, this means that
%
\nbea
u * f & = & (u * \mu) * \nu \\
& = & I * \nu \\
u * f & = & \nu \\
\to \nu(n) & = & \sum_{d|n} f(d)
\neea
%
$\nu$ is obviously not multiplicative since $\nu(1) \neq 1,~\nu(pq) \neq \nu(p)\nu(q)$ but it is actually additive since $\nu(p^\alpha q^\beta) = \nu(p^\alpha) + \nu(q^\beta) = \nu(p) + \nu(q)$ where $p \neq q$ are distinct primes, so let's decompose $n$ into its primal constituents, $n = \prod_i p_i^{\alpha_i}$
%
\nbea
\nu\left(\prod_i p_i^{\alpha_i}\right) & = & \sum_{d|n} f(d) \\
\sum_i \nu\left(p_i^{\alpha_i}\right) & = & \sum_{d|n} f(d) \\
\sum_i \nu\left(p_i\right) & = & \sum_{d|n} f(d)
\neea
%
from here we can immediately see that $f(n)$ is given by
%
\nbea
f(n) = \left\{
\begin{array}{l}
1 {\rm~if~} n {\rm ~is~prime} \\
0 {\rm~otherwise}
\end{array} \right.
\neea
%

{\bf Problem 2.6}. Prove that
%
\nbea
\sum_{d^2|n} \mu(d) & = & \mu^2(n)
\neea
%
and, more generally,
%
\nbea
\sum_{d^k|n} \mu(d) = \left \{
\begin{array}{l c l}
0 & & {\rm if~} m^k|n {\rm~for~some~} m > 1 \\
1 & & {\rm otherwise}
\end{array}\right.
\neea
%
The last sum is extended over all positive divisors $d$ of $n$ whose $k$th power also divide $n$.

The key point here is again ``multiplicative'', since $\mu(d)$ is multiplicative so is $\sum_{d^2|n} \mu(d)$ so we need to only consider $g(p^\alpha) = \sum_{d^2|p^\alpha} \mu(d)$ but note that even though the sum is over $d^2 \to \sum_{d^2|n}$, $\mu$ is only taking $d$, $\mu(d)$ and not $\mu(d^2)$
%
\nbea
\sum_{d^2|p^\alpha} \mu(d) & = & \mu(1) + \mu(p) \\
& = & 1 - 1 \\
& = & 0
\neea
%
The above holds if $\alpha > 1$ otherwise for $0 \le \alpha \le 1 \to \sum_{d^2|p^\alpha} \mu(d) = \mu(1) = +1$, in short
%
\nbea
g(p^\alpha) = \sum_{d^2|p^\alpha} \mu(d) & = & \left \{
\begin{array}{r c  l}
0 && {\rm if ~} \alpha > 1 \\
1 && {\rm if ~} 0 \le \alpha \le 1 \\
\end{array}\right. \\
& = & \mu^2(p^\alpha)
\neea
%
The second part follows closely, again since it is multiplicative and again note that even though the sum is over $d^k \to \sum_{d^k|n}$, $\mu$ is only taking $d$, $\mu(d)$ and not $\mu(d^k)$
%
\nbea
\sum_{d^k|p^\alpha} \mu(d) & = & \mu(1) + \mu(p) \\
& = & 1 - 1 \\
& = & 0
\neea
%
if $\alpha > k$ otherwise for $0 \le \alpha \le k \to \sum_{d^k|p^\alpha} \mu(d) = \mu(1) = +1$, the only difference now is that we can't say it is equal to $\mu^2(p^\alpha)$ because say $\alpha = k-1 > 0 \to \mu(p^{k-1}) = 0$ but $\sum_{d^k|p^{k-1}} \mu(d) = \mu(1) = +1$

{\bf Problem 2.7}. Let $\mu(p,d)$ denote the value of the Mobius function at the gcd of $p$ and $d$. Prove that for every prime $p$ we have
%
\nbea
\sum_{d|n} \mu(d)\mu(p,d) = \left \{
\begin{array}{l c l}
1 && {\rm if~} n = 1 \\
2 && {\rm if~} n = p^a, a \ge 1 \\
0 && {\rm otherwise}.
\end{array}\right.
\neea
%

The thing is the gcd $(p,mn)$ is multiplicative as long as $(m,n)=1$ because $p$ is prime and once we expand $m$ and $n$ in their primal constituents it is evident, \ie $(p,mn) = (p,m)(p,n)$, therefore $\mu(p,mn) = \mu(p,m)\mu(p,n)$

The first case is obvious $\sum_{d|1} \mu(d)\mu(p,d) = \mu(1)\mu(1) = 1$.

The second case
%
\nbea
\sum_{d|p^a} \mu(d)\mu(p,d) & = & \mu(1)\mu(p,1) + \mu(p)\mu(p,p) \\
& = & \mu(1)\mu(1) + \mu(p)\mu(p) \\
& = & (1)(1) + (-1)(-1) \\
& = & 2
\neea
%

To show the last case it's easiest to utilize the fact that $g(n) = \sum_{d|n}\mu(d)\mu(p,d)$ is multiplicative and now we just need to show $g(q^b), ~q\neq p$ as $g(p^a)$ is already covered above
%
\nbea
g(q^b) = \sum_{d|q^b}\mu(d)\mu(p,d) & = & \mu(1)\mu(p,1) + \mu(q)\mu(p,q) \\
& = & \mu(1)\mu(1) + \mu(q)\mu(1) \\
& = & (1)(1) + (-1)(1) \\
& = & 0
\neea
%

{\bf Problem 2.8}. Prove that
%
\nbea
\sum_{d|n} \mu(d) \log^m d = 0
\neea
%
if $m \ge 1$ and $n$ has more than $m$ distinct prime factors. [{\it Hint:} Induction.]

To use induction we need to prove the base case, the thing is that $\log$ is not multiplicative, so that's a bit hard. The base case should be $m = 1$ and then we go up from there to bigger $m$ ?!? \dunno

But one thing I notice is that we only need to consider numbers with one power of distinct primes, \ie $n = p_1p_2\ldots p_k$ because $\mu(d)$ is zero if the powers of the primes are not zero that is
%
\nbea
\sum_{d|n} \mu(d) \log^m d & = & \cancel{\mu(1) \log^m(1)} + \mu(p_1)\log^m(p_1) + \ldots + \mu(p_k)\log^m(p_k) + \\
&& \mu(p_1p_2)\log^m(p_1p_2) + \ldots + \mu(p_{k-1}p_k)\log^m(p_{k-1}p_k) + \ldots + \\
&& \mu(p_1p_2\ldots p_k)\log^m(p_1p_2\ldots p_k)
\neea
%
and from the definition of $\mu(d)$ we know that if it has odd number of primes it's negative and it there are an even number of distinct primes $\mu$ is positive, therefore
%
\nbea
\sum_{d|n} \mu(d) \log^m d & = & -(\log^m(p_1) + \ldots + \log^m(p_k)) \\
&& +(\log^m(p_1p_2) + \ldots + \log^m(p_{k-1}p_k)) + \\
&& -(\log^m(p_1p_2p_3)+ \ldots + \log^m(p_{k-2}p_{k-1}p_k)) +\\
&& (-1)^k\log^m(p_1p_2\ldots p_k)
\neea
%
Since log is additive we can expand them but before we do that let's denote $\log(p_k) = l_k$
%
\nbea
\sum_{d|n} \mu(d) \log^m d & = & -\sum_{i_1 = (k|1)} l_{i_1}^m + \sum_{i_1, i_2 = (k|2)} (l_{i_1} + l_{i_2})^m - \sum_{i_1, i_2, i_3 = (k|3)} (l_{i_1} + l_{i_2} + l_{i_3})^m + \\
&& \ldots + (-1)^k \sum_{i_1, i_2, \ldots , i_k = (k|k)} (l_{i_1} + l_{i_2} + \ldots + l_{i_k})
\neea
%
where the notation $(k|j)$ means that all combinations of $k$ choose $j$, as a concrete example, say $m=4, ~k=5$ which is the minimum $k$ required
%
\nbea
\sum_{d|n} \mu(d) \log^m d & = & -(l_1^4 + l_2^4 + l_3^4 + l_4^4 + l_5^4) + \\
&& + ((l_1+l_2)^4 + (l_1+l_3)^4 + (l_1+l_4)^4 + (l_1+l_5)^4 + (l_2+l_3)^4 + (l_2+l_4)^4 +  \\ 
&& ~~~~(l_2+l_5)^4 + (l_3+l_4)^4 + (l_3+l_5)^4 + (l_4+l_5)^4) + \\
&& - ((l_1 + l_2 + l_3)^4 + (l_1 + l_2 + l_4)^4 + (l_1 + l_2 + l_5)^4 + (l_1 + l_3 + l_4)^4 + \\
&& ~~~~ (l_1 + l_3 + l_5)^4  + (l_1 + l_4 + l_5)^4  + (l_2 + l_3 + l_4)^4 + (l_2 + l_3 + l_5)^4 + \\
&& ~~~~ (l_2 + l_4 + l_5)^4 + (l_3 + l_4 + l_5)^4) \\
&& + ((l_1 + l_2 + l_3 + l_4)^4 + (l_1 + l_2 + l_3 + l_5)^4 + (l_1 + l_2 + l_4 + l_5)^4 + \\
&& ~~~~ (l_1 + l_3 + l_4 + l_5)^4 + (l_2 + l_3 + l_4 + l_5)^4) + \\
&& - ((l_1 + l_2 + l_3 + l_4 + l_5)^4)
\neea
%
Now we gather coefficients of same powers, say we collect all $l_1^4$, 
%
\nbea
(5|1) & \to & (-1)l_1^4 \\
(5|2) & \to & (+4)l_1^4 \\
(5|3) & \to & (-6)l_1^4 \\
(5|4) & \to & (+4)l_1^4 \\
(5|5) & \to & (-1)l_1^4
\neea
%
so it's basically the Pascal triangle coefficients, why is this? Well, for example, for $(5|1)$, first we fix {\bf one} $l$ and then choose a partner for it from the remaining {\bf four}, however in this case we only need one $l$ and we already fixed it, so we will just need {\bf zero} partner, \ie ${4 \choose 0} = 1$.

For $(5|2)$ we first pick an $l$ and then choose a partner (again because $(5|${\bf 2}$)$ means we need {\bf 2} $l$'s in total) for it from 4 available choices, which is ${4 \choose 1}$, \ie this $l$ will appear ${4 \choose 1} = 4$ times, for $(5|3)$ it's the same thing we first pick an $l$ and then choose {\it two} partners for it, \ie  this $l$ will then appear ${4 \choose 2} = 6$ times, and for $(5|3)$, it's pick an $l$ and choose ${4 \choose 3} = 4$ partners and so on and therefore the coefficients of $l_1$ is just those of Pascal triangle's but with the signs alternating between plus and minus. And this is true for other $l$'s not just $l_1$.

We now need to tackle the cross terms say $l_1^3l_2$, first thing to note that this cross product is always preceded by a constant (which again is from Pascal triangle), for $(l_1 + l_2)^4$ it is $4l_1^3l_2$, note that this coefficient is the same no matter how many terms are being exponentiated, \ie even for $(l_1 + l_2 + l_3 + \ldots + l_w)^4$, the coefficient for $l_1^3l_2$ is still 4 because it is still ${4 \choose 3}$ no matter what, this is because
%
\nbea
(l_1 + l_2 + \ldots)^4 & = & \underbrace{(l_1 + l_2 + \ldots)}_\text{bin \#1}\underbrace{(l_1 + l_2 + \ldots)}_\text{bin \#2}\underbrace{(l_1 + l_2 + \ldots)}_\text{bin \#3}\underbrace{(l_1 + l_2 + \ldots)}_\text{bin \#4}
\neea
%
To get $l_1^3l_2$ we need to gather {\bf three} $l_1$'s and we have {\bf four} bins to choose for as shown above that's why we have 4 choose 3, ${4 \choose 3} = 4$ possibilities. And as the number of bins are the same no matter how many $l$'s we have the number of possibilities is still the same.

We also have other cross terms like $l_1^2l_3l_4$, in this case, we need to gather {\bf two} $l_1$'s from {\bf four} bins so it's ${4 \choose 2} = 6$, next we need to choose {\bf one} $l_3$ from the remaining {\bf two} bins which is ${2 \choose 1} = 2$ and once we've chosen the bin for $l_2$, the other bin will definitely contain $l_3$, so in total there are
%
\nbea
{4 \choose 2}\times{2 \choose 1} & = & 6 \times 2 = 12
\neea
%
and since the number of bins is constant no matter what this coefficient remains the same no matter how many $l$'s we have.

So now for $4l_1^3l_2$ we have
%
\nbea
(5|1) & \to & (0) \\
(5|2) & \to & (+1)4l_1^3l_2 \\
(5|3) & \to & (-3)4l_1^3l_2 \\
(5|4) & \to & (+3)4l_1^3l_2 \\
(5|5) & \to & (-1)4l_1^3l_2
\neea
%
again Pascal triangle, why is this? This time we fix {\bf two} $l$'s (instead of just one for $l^4$ above), and then calculate how many partners this couple might have, for $(5|2)$, we only need {\bf two} in total so because we already fixed two of them we just need {\bf zero} partner from the three remaining ones, \ie ${3 \choose 0} = 1$. For $(5|3)$, again we fix {\bf two} $l$'s and choose one more partner (because in total we need 3) from the remaining three, \ie ${3 \choose 1} = 3$ and so on. This is also true for any two-term cross terms.

And this pattern continues for higher cross terms like $l_1^2l_2l_3$, \eg for $(5|4)$ we fix {\bf three} $l$'s and then choose one partner from the remaining {\bf two}, which means ${2 \choose 1} = 2$.

The pattern continues for any $k$, say we now have $k=6$ while $m$ stays the same, $m=4$, in this case we have $(6|1),(6|2),(6|3),(6|4),(6|5),(6|6)$, and the number of say $l_1^4$ is given by the recipe above, fix an $l$ and then choose a partner for it depending on which combination $(6|j)$ you're on, therefore the general formula is


\bigskip
In Exercises 10, 11, and 12, $d(n)$ denotes the number of positive divisors of $n$.

{\bf Problem 2.10}. Prove that $\prod_{t|n} t = n^{d(n)/2}$.

Again, let's decompose $n$ into its primal constituents $n = \prod_i^N p_i^{\alpha_i}$ then $d(n)$ is given by
%
\nbea
d(n) = d\left(\prod_i^N p_i^{\alpha_i}\right) & = & \prod_i^N (\alpha_i + 1)
\neea
%
To see why this is we just need to recall that the number of combinations an $N$-digit (base-10) number has is 
%
\nbea
\# {\rm~of~combo} = \underbrace{10 \times 10 \times 10 \times \ldots \times 10}_\text{$N$ {\rm of~them}}
\neea
%
because each digit can take 10 possible different values. For our case, each prime factor plays the role of a digit, however, each has different possible values, which is $(\alpha_i+1)$ because we can have $p_i^0,p_i^1,p_i^2,\ldots,p_N^{\alpha_N}$ so the total number of combinations for $\prod_i^N p_i^{\alpha_i}$ is
%
\nbea
\# {\rm~of~combo} = \underbrace{(\alpha_1 + 1) (\alpha_2 + 1) (\alpha_3 + 1) \ldots (\alpha_N + 1)}_\text{$N$ {\rm prime~factors}}
\neea
%

Next, we can decompose $\prod_{t|n} t$ in terms of its primal constituents as well, say we focus on $p_1$ of $\prod_i^N p_i^{\alpha_i}$, the divisors of $p_1^{\alpha_1}$ are $p_1^0, p_1^1, \ldots, p_1^{\alpha_1}$, so if we multiply all of them we have $p_1^{1 + 2 + 3 + \ldots + \alpha_1} = p_1^{\frac{\alpha_1(\alpha_1 + 1)}{2}} = \left ( p_1^{\alpha_1}\right )^{\frac{\alpha_1+1}{2}}$.

But here $p_1^{\alpha_1}$ is not alone, each divisor of $p_1^{\alpha_1}$, \ie $p_1^{j}, ~0 \le j \le \alpha_1$, occurs $(\alpha_2+1)(\alpha_3+1)\ldots(\alpha_N+1)$ times, so the final exponent for $p_1$ in $\prod_{t|n} t$ is
%
\nbea
\left (p_1^{\alpha_1}\right )^{\frac{(\alpha_1+1)}{2}(\alpha_2+1)(\alpha_3+1)\ldots(\alpha_N+1)} & = & \left (p_1^{\alpha_1}\right )^{d(n)/2}
\neea
%
the same case goes for any other $p_i$, thus $\prod_{t|n}t = n^{d(n)/2}$. As a concrete example, take $n = p_1^2p_2^3$, the divisors of $n$ are
%
\nbea
\begin{array}{c c c c c c c}
p_1^0 ~~ p_2^0 & ~~~~~~~~ & p_1^0 ~~ p_2^1 & ~~~~~~~~ & p_1^0 ~~ p_2^2 & ~~~~~~~~ & p_1^0 ~~ p_2^3 \\
p_1^1 ~~ p_2^0 & ~~~~~~~~ & p_1^1 ~~ p_2^1 & ~~~~~~~~ & p_1^1 ~~ p_2^2 & ~~~~~~~~ & p_1^1 ~~ p_2^3 \\
p_1^2 ~~ p_2^0 & ~~~~~~~~ & p_1^2 ~~ p_2^1 & ~~~~~~~~ & p_1^2 ~~ p_2^2 & ~~~~~~~~ & p_1^2 ~~ p_2^3
\end{array}
\neea
%
so you can see that $(p_1^0~p_1^1~p_1^2)$ occurs $4=(\alpha_2+1)$ times $\to (p_1^0~p_1^1~p_1^2)^{\alpha_2+1}$.

{\bf Problem 2.11}. Prove that $d(n)$ is odd if, and only if, $n$ is square.

As shown above for $n = \prod_i^N p_i^{\alpha_i}$, $d(n) = \prod_i^N (\alpha_i + 1)$, so to get $d(n)$ to be odd we need {\it all} of $\alpha_i$ to be even so that $(\alpha_i + 1)$ is odd, therefore $n$ must be even

{\bf Problem 2.12}. Prove that $\sum_{t|n} d(t)^3 = \left (\sum_{t|n} d(t)\right )^2$.

The above relationship is evidently not true in general, we therefore need to utilize the properties of $d(t)$ to derive it. One thing to note is that $g(n) = \sum_{t|n} d(t)^3$ is multiplicative as $d(t)$ is. Therefore we just need to consider $g(p^\alpha) = \sum_{t|p^\alpha} d(t)^3$.

My strategy would be to utilize induction. Assume that $\sum_{t|p^\alpha} d(t)^3 = \left (\sum_{t|p^\alpha} d(t)\right )^2$ is true up to some $p^\alpha$, we now want to know what happens with $p^{\alpha+1}$
%
\nbea
\sum_{t|p^{\alpha+1}} d(t)^3 & = & d(p^{\alpha+1})^3 + \sum_{t|p^\alpha} d(t)^3
\neea
%
and $d(p^{\alpha+1}) = \alpha+2$ thus
%
\nbea
d(p^{\alpha+1})^3 + \sum_{t|p^\alpha} d(t)^3 & = & (\alpha + 2)^3 + \left ( \sum_{t|p^\alpha} d(t)\right )^2 \\
& = & (\alpha + 2)^2(\alpha + 2) + \left ( \sum_{t|p^\alpha} d(t)\right )^2 \\
& = & (\alpha + 2)^2 + (\alpha + 2)^2(\alpha + 1) + \left ( \sum_{t|p^\alpha} d(t)\right )^2 \\
& = & d(p^{\alpha + 1})^2 + (\alpha + 2)\cdot 2 \frac{(\alpha + 2)(\alpha + 1)}{2} + \left ( \sum_{t|p^\alpha} d(t)\right )^2 \\
& = & d(p^{\alpha + 1})^2 + 2 d(p^{\alpha+1})\left (\sum_{t|p^\alpha} d(t) \right ) + \left ( \sum_{t|p^\alpha} d(t)\right )^2 \\
& = & \left ( d(p^{\alpha + 1}) + \sum_{t|p^\alpha} d(t) \right )^2 \\
\sum_{t|p^{\alpha+1}} d(t)^3 & = & \left ( \sum_{t|p^{\alpha+1}} d(t) \right )^2
\neea
%
Going to line 5 we have used the fact that $\sum_{t|p^\alpha} d(t) = \sum_{i=1}^{\alpha+1} i = \frac{(\alpha+1)(\alpha + 2)}{2} $ since $d(p^j) = j+1$. We can of course dispel induction for a bruter force approach by expanding $\sum_{t|p^\alpha+1} d(t)^3 = \sum_{i=1}^{\alpha+1} i^3$ but this requires us to know the formula for a sum of consecutive cubes \dunno









\end{document}