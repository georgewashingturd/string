\documentclass[aps,preprint,preprintnumbers,nofootinbib,showpacs,prd]{revtex4-1}
\usepackage{graphicx,color}
\usepackage{caption}
\usepackage{subcaption}
\usepackage{amsmath,amssymb}
\usepackage{multirow}
\usepackage{amsthm}%        But you can't use \usewithpatch for several packages as in this line. The search 

\usepackage{cancel}

%%% for SLE
\usepackage{dcolumn}   % needed for some tables
\usepackage{bm}        % for math
\usepackage{amssymb}   % for math
\usepackage{multirow}
%%% for SLE -End

\usepackage{ulem}
\usepackage{cancel}

\usepackage{hyperref}

\usepackage[top=1in, bottom=1.25in, left=1.1in, right=1.1in]{geometry}

\usepackage{mathtools} % for \DeclarePairedDelimiter{\ceil}{\lceil}{\rceil}

\usepackage{simplewick}

\newcommand{\msout}[1]{\text{\sout{\ensuremath{#1}}}}


%%%%%% My stuffs - Stef
\newcommand{\lsim}{\mathrel{\mathop{\kern 0pt \rlap
  {\raise.2ex\hbox{$<$}}}
  \lower.9ex\hbox{\kern-.190em $\sim$}}}
\newcommand{\gsim}{\mathrel{\mathop{\kern 0pt \rlap
  {\raise.2ex\hbox{$>$}}}
  \lower.9ex\hbox{\kern-.190em $\sim$}}}

%
% Key
%
\newcommand{\key}[1]{\medskip{\sffamily\bfseries\color{blue}#1}\par\medskip}
%\newcommand{\key}[1]{}
\newcommand{\q}[1] {\medskip{\sffamily\bfseries\color{red}#1}\par\medskip}
\newcommand{\comment}[2]{{\color{red}{{\bf #1:}  #2}}}


\newcommand{\ie}{{\it i.e.} }
\newcommand{\eg}{{\it e.g.} }

%
% Energy scales
%
\newcommand{\ev}{{\,{\rm eV}}}
\newcommand{\kev}{{\,{\rm keV}}}
\newcommand{\mev}{{\,{\rm MeV}}}
\newcommand{\gev}{{\,{\rm GeV}}}
\newcommand{\tev}{{\,{\rm TeV}}}
\newcommand{\fb}{{\,{\rm fb}}}
\newcommand{\ifb}{{\,{\rm fb}^{-1}}}

%
% SUSY notations
%
\newcommand{\neu}{\tilde{\chi}^0}
\newcommand{\neuo}{{\tilde{\chi}^0_1}}
\newcommand{\neut}{{\tilde{\chi}^0_2}}
\newcommand{\cha}{{\tilde{\chi}^\pm}}
\newcommand{\chao}{{\tilde{\chi}^\pm_1}}
\newcommand{\chaop}{{\tilde{\chi}^+_1}}
\newcommand{\chaom}{{\tilde{\chi}^-_1}}
\newcommand{\Wpm}{W^\pm}
\newcommand{\chat}{{\tilde{\chi}^\pm_2}}
\newcommand{\smu}{{\tilde{\mu}}}
\newcommand{\smur}{\tilde{\mu}_R}
\newcommand{\smul}{\tilde{\mu}_L}
\newcommand{\sel}{{\tilde{e}}}
\newcommand{\selr}{\tilde{e}_R}
\newcommand{\sell}{\tilde{e}_L}
\newcommand{\smurl}{\tilde{\mu}_{R,L}}

\newcommand{\casea}{\texttt{IA}}
\newcommand{\caseb}{\texttt{IB}}
\newcommand{\casec}{\texttt{II}}

\newcommand{\caseasix}{\texttt{IA-6}}

%
% Greek
%
\newcommand{\es}{{\epsilon}}
\newcommand{\sg}{{\sigma}}
\newcommand{\dt}{{\delta}}
\newcommand{\kp}{{\kappa}}
\newcommand{\lm}{{\lambda}}
\newcommand{\Lm}{{\Lambda}}
\newcommand{\gm}{{\gamma}}
\newcommand{\mn}{{\mu\nu}}
\newcommand{\Gm}{{\Gamma}}
\newcommand{\tho}{{\theta_1}}
\newcommand{\tht}{{\theta_2}}
\newcommand{\lmo}{{\lambda_1}}
\newcommand{\lmt}{{\lambda_2}}
%
% LaTeX equations
%
\newcommand{\beq}{\begin{equation}}
\newcommand{\eeq}{\end{equation}}
\newcommand{\bea}{\begin{eqnarray}}
\newcommand{\eea}{\end{eqnarray}}
\newcommand{\ba}{\begin{array}}
\newcommand{\ea}{\end{array}}
\newcommand{\bit}{\begin{itemize}}
\newcommand{\eit}{\end{itemize}}

\newcommand{\nbea}{\begin{eqnarray*}}
\newcommand{\neea}{\end{eqnarray*}}
\newcommand{\nbeq}{\begin{equation*}}
\newcommand{\neeq}{\end{equation*}}

\newcommand{\no}{{\nonumber}}
\newcommand{\td}[1]{{\widetilde{#1}}}
\newcommand{\sqt}{{\sqrt{2}}}
%
\newcommand{\me}{{\rlap/\!E}}
\newcommand{\met}{{\rlap/\!E_T}}
\newcommand{\rdmu}{{\partial^\mu}}
\newcommand{\gmm}{{\gamma^\mu}}
\newcommand{\gmb}{{\gamma^\beta}}
\newcommand{\gma}{{\gamma^\alpha}}
\newcommand{\gmn}{{\gamma^\nu}}
\newcommand{\gmf}{{\gamma^5}}
%
% Roman expressions
%
\newcommand{\br}{{\rm Br}}
\newcommand{\sign}{{\rm sign}}
\newcommand{\Lg}{{\mathcal{L}}}
\newcommand{\M}{{\mathcal{M}}}
\newcommand{\tr}{{\rm Tr}}

\newcommand{\msq}{{\overline{|\mathcal{M}|^2}}}

%
% kinematic variables
%
%\newcommand{\mc}{m^{\rm cusp}}
%\newcommand{\mmax}{m^{\rm max}}
%\newcommand{\mmin}{m^{\rm min}}
%\newcommand{\mll}{m_{\ell\ell}}
%\newcommand{\mllc}{m^{\rm cusp}_{\ell\ell}}
%\newcommand{\mllmax}{m^{\rm max}_{\ell\ell}}
%\newcommand{\mllmin}{m^{\rm min}_{\ell\ell}}
%\newcommand{\elmax} {E_\ell^{\rm max}}
%\newcommand{\elmin} {E_\ell^{\rm min}}
\newcommand{\mxx}{m_{\chi\chi}}
\newcommand{\mrec}{m_{\rm rec}}
\newcommand{\mrecmin}{m_{\rm rec}^{\rm min}}
\newcommand{\mrecc}{m_{\rm rec}^{\rm cusp}}
\newcommand{\mrecmax}{m_{\rm rec}^{\rm max}}
%\newcommand{\mpt}{\rlap/p_T}

%%%song
\newcommand{\cosmax}{|\cos\Theta|_{\rm max} }
\newcommand{\maa}{m_{aa}}
\newcommand{\maac}{m^{\rm cusp}_{aa}}
\newcommand{\maamax}{m^{\rm max}_{aa}}
\newcommand{\maamin}{m^{\rm min}_{aa}}
\newcommand{\eamax} {E_a^{\rm max}}
\newcommand{\eamin} {E_a^{\rm min}}
\newcommand{\eaamax} {E_{aa}^{\rm max}}
\newcommand{\eaacusp} {E_{aa}^{\rm cusp}}
\newcommand{\eaamin} {E_{aa}^{\rm min}}
\newcommand{\exxmax} {E_{\neuo \neuo}^{\rm max}}
\newcommand{\exxcusp} {E_{\neuo \neuo}^{\rm cusp}}
\newcommand{\exxmin} {E_{\neuo \neuo}^{\rm min}}
%\newcommand{\mxx}{m_{XX}}
%\newcommand{\mrec}{m_{\rm rec}}
\newcommand{\erec}{E_{\rm rec}}
%\newcommand{\mrecmin}{m_{\rm rec}^{\rm min}}
%\newcommand{\mrecc}{m_{\rm rec}^{\rm cusp}}
%\newcommand{\mrecmax}{m_{\rm rec}^{\rm max}}
%%%song

\newcommand{\mc}{m^{\rm cusp}}
\newcommand{\mmax}{m^{\rm max}}
\newcommand{\mmin}{m^{\rm min}}
\newcommand{\mll}{m_{\mu\mu}}
\newcommand{\mllc}{m^{\rm cusp}_{\mu\mu}}
\newcommand{\mllmax}{m^{\rm max}_{\mu\mu}}
\newcommand{\mllmin}{m^{\rm min}_{\mu\mu}}
\newcommand{\mllcusp}{m^{\rm cusp}_{\mu\mu}}
\newcommand{\elmax} {E_\mu^{\rm max}}
\newcommand{\elmin} {E_\mu^{\rm min}}
\newcommand{\elmaxw} {E_W^{\rm max}}
\newcommand{\elminw} {E_W^{\rm min}}
\newcommand{\R} {{\cal R}}

\newcommand{\ewmax} {E_W^{\rm max}}
\newcommand{\ewmin} {E_W^{\rm min}}
\newcommand{\mwrec}{m_{WW}}
\newcommand{\mwrecmin}{m_{WW}^{\rm min}}
\newcommand{\mwrecc}{m_{WW}^{\rm cusp}}
\newcommand{\mwrecmax}{m_{WW}^{\rm max}}

\newcommand{\mpt}{{\rlap/p}_T}

%%%%%% END My stuffs - Stef

\newcommand{\dunno}{$ {}^{\mbox {--}}\backslash(^{\rm o}{}\underline{\hspace{0.2cm}}{\rm o})/^{\mbox {--}}$}

\DeclarePairedDelimiter{\ceil}{\lceil}{\rceil}
\DeclarePairedDelimiter{\floor}{\lfloor}{\rfloor}

\DeclareMathOperator{\ord}{ord}
\DeclareMathOperator{\tor}{tor}





\begin{document}

\title{Apostol's Analytic Number Theory}
\bigskip
\author{Stefanus$^1$\\
$^1$ Samsung Semiconductor Inc\\ San Jose, CA 95134 USA\\
}
%
\date{\today}
%
\begin{abstract}
Just for fun :)

\end{abstract}
%
\maketitle

\renewcommand{\theequation}{A.\arabic{equation}}  % redefine the command that creates the equation no.
\setcounter{equation}{0}  % reset counter 

\underline{\textbf{\textit{Chapter 2}}}
\bigskip

{\bf Problem 2.1} Find all integers $n$ such that
%
\nbea
\begin{array}{l r c l c l r c l c l r c l}
{\rm(a)} & \varphi(n) & = & n/2 & ~~~~~~~~~ & {\rm(b)} & \varphi(n) & = & \varphi(2n) & ~~~~~~~~~ & {\rm(c)} & \varphi(n) & = & 12
\end{array}
\neea
%

For (a), using the definition of $\varphi(n)$, $\varphi(n) = n \prod_{p|n} \left ( 1 - \frac{1}{p} \right )$
%
\nbea
\frac{\bcancel{n}}{2} & = & \bcancel{n} \prod_{p|n} \left ( 1 - \frac{1}{p} \right ) \\
\frac{1}{2} & = & \frac{\prod_{p|n} \left ( p - 1 \right )}{\prod_{p|n} p} \\
\prod_{p|n} p & = & 2\prod_{p|n} \left ( p - 1 \right )
\neea
%
if $n$ is odd the LHS is odd while the RHS is even, so it can't be. If $n$ is even the LHS only has one factor of 2 while the RHS has many so it will only work if $n=2$.

For (b)
%
\nbea
\bcancel{n} \prod_{p|n} \left ( 1 - \frac{1}{p} \right )  & = & 2\bcancel{n} \prod_{p|2n} \left ( 1 - \frac{1}{p} \right )
\neea
%
If $n$ is even then
%
\nbea
\prod_{p|2n} \left ( 1 - \frac{1}{p} \right ) & = & \prod_{p|n} \left ( 1 - \frac{1}{p} \right )
\neea
%
and so
%
\nbea
\prod_{p|n} \left ( 1 - \frac{1}{p} \right )  & = & 2\prod_{p|n} \left ( 1 - \frac{1}{p} \right ) \\
\to 1 & = & 2
\neea
%
which is impossible, so $n$ has to be odd, in that case
%
\nbea
\prod_{p|2n} \left ( 1 - \frac{1}{p} \right ) & = & \left ( 1 - \frac{1}{2} \right ) \prod_{p|n} \left ( 1 - \frac{1}{p} \right ) \\
& = & \frac{1}{2} \prod_{p|n} \left ( 1 - \frac{1}{p} \right )
\neea
%
and therefore
%
\nbea
\prod_{p|n} \left ( 1 - \frac{1}{p} \right )  & = & 2 \frac{1}{2} \prod_{p|n} \left ( 1 - \frac{1}{p} \right ) \\
\to 1 & = & 1
\neea
%
and therefore $\varphi(n) = \varphi(2n)$ for all odd $n$.

For (c)
%
\nbea
\varphi(n) = 12 & = & 2 \cdot 2 \cdot 3 \\
& = & \prod_{p|n} p^{\alpha_p} - p^{\alpha_p-1} \\
\varphi\left (\prod_{p|n} p^{\alpha_p} \right ) & = & \prod_{p|n}p^{\alpha_p-1} (p - 1)
\neea
%
the only possible solution is $n=13$

{\bf Problem 2.2}. For each of the following statements either give a proof or exhibit a counter example.

(a) If $(m,n)=1$ then $(\varphi(m),\varphi(n)) = 1$

(b) If $n$ is composite, then $(n, \varphi(n)) > 1$

(c) If the same primes divide $m$ and $n$, then $n\varphi(m) = m\varphi(n)$

For (a) a counter example will be $(3,4) = 1$, while $\varphi(3) = 2, ~\varphi(4) = 2$

For (b) a counter example would be $n = 15$ which means that $\varphi(15) = 8$ and $(15,8) = 1$

For (c) I think what it means by ``the same primes divide $m$ and $n$'' is that $m = \prod p^{\alpha_p}$ and $n = \prod p^{\beta_p}$, so they both have the same primes but they might have different exponents for each prime, in this case $\prod_{p|n} = \prod_{p|m}$
%
\nbea
n\varphi(m) & = & n \left ( m \prod_{p|m} \left ( 1 - \frac{1}{p}\right ) \right ) \\
& = & m \left ( n \prod_{p|n} \left ( 1 - \frac{1}{p}\right ) \right ) \\
n\varphi(m) & = & m\varphi(n)
\neea
%

{\bf Problem 2.3}. Prove that
%
\nbea
\frac{n}{\varphi(n)} = \sum_{d|n} \frac{\mu^2(d)}{\varphi(d)}
\neea
%

Since $\mu(n)$ and $\varphi(n)$ are both multiplicative so is $\mu^2/\varphi$, in that case $g(n) = \sum_{d|n} \frac{\mu^2(d)}{\varphi(d)}$ is also multiplicative. To determine $g(n)$ we need only compute $g(p^\alpha)$ for prime powers
%
\nbea
g(p^\alpha) & = & \sum_{d|p^\alpha} \frac{\mu^2(d)}{\varphi(d)} \\
& = & \frac{\mu^2(1)}{\varphi(1)} + \frac{\mu^2(p)}{\varphi(p)} + \ldots + \frac{\mu^2(p^\alpha)}{\varphi(p^\alpha)} \\
& = & 1 + \frac{1}{p - 1} \\
& = & \frac{p}{p - 1} \\
& = & p^\alpha \cdot \frac{p}{p^\alpha(p - 1)} \\
\to \sum_{d|p^\alpha} \frac{\mu^2(d)}{\varphi(d)}& = & \frac{p^\alpha}{\varphi(p^\alpha)}
\neea
%

We can also prove it the other way around by assuming the LHS, to do this it is easiest to use the Mobius inversion formula
%
\nbea
\frac{n}{\varphi(n)} = \sum_{d|n} g(d)
\neea
%
and we want to find out what this $g(d)$ is, which is
%
\nbea
g(n) & = & \sum_{d|n} \frac{d}{\varphi(d)} \mu\left ( \frac{n}{d}\right )
\neea
%
The RHS is multiplicative so like above we just need to evaluate $g(p^\alpha)$ for prime powers
%
\nbea
g(p^\alpha) & = & \sum_{d|p^\alpha} \frac{d}{\varphi(d)} \mu\left ( \frac{p^\alpha}{d}\right ) \\
& = & \frac{p^{\alpha-1}}{\varphi(p^{\alpha-1})} \mu\left ( \frac{p^\alpha}{p^{\alpha-1}}\right ) + \frac{p^\alpha}{\varphi(p^\alpha)} \mu\left ( \frac{p^\alpha}{p^\alpha}\right ) \\
& = & -\frac{p^{\alpha-1}}{\varphi(p^{\alpha-1})} + \frac{p^\alpha}{\varphi(p^\alpha)} \\
& = & -\frac{p^\alpha}{\varphi(p^\alpha)}  + \frac{p^\alpha}{\varphi(p^\alpha)} \\
& = & 0
\neea
%
if $\alpha > 1$ and if $\alpha = 1$ we get
%
\nbea
g(p) & = & \sum_{d|p} \frac{d}{\varphi(d)} \mu\left ( \frac{p}{d}\right ) \\
& = & \frac{1}{\varphi(1)} \mu\left ( \frac{p}{1}\right ) + \frac{p}{\varphi(p)} \mu\left ( \frac{p}{p}\right ) \\
& = & -1 + \frac{p}{\varphi(p)} \\
& = & -1 + \frac{p}{p - 1} \\
& = & \frac{1}{p - 1} \\
g(p) & = & \frac{1}{\varphi(p)}
\neea
%
This means that $g(p^\alpha) = 1/\varphi(p^\alpha)$ is $\alpha = 1$ and $g(p^\alpha) = 0$ if $\alpha > 1$, in other words $g(p^\alpha) = \mu^2(p^\alpha)/\varphi(p^\alpha)$

{\bf Problem 2.4}. Prove that $\varphi(n) > n/6$ for all $n$ with at most $8$ distinct prime factors.

First, let's demystify this number $8$, the reason $8$ is involved is because if you multiply out $(p-1)/p$ for the first eight primes we get
%
\nbea
\frac{1}{2}\cdot\frac{2}{3}\cdot\frac{4}{5}\cdot\frac{6}{7}\cdot\frac{10}{11}\cdot\frac{12}{13}\cdot\frac{16}{17}\cdot\frac{18}{19} & = & \frac{55296}{323323} \sim 0.171  > \frac{1}{6}
\neea
%
but if we multiply the first nine
%
\nbea
\frac{1}{2}\cdot\frac{2}{3}\cdot\frac{4}{5}\cdot\frac{6}{7}\cdot\frac{10}{11}\cdot\frac{12}{13}\cdot\frac{16}{17}\cdot\frac{18}{19}\cdot\frac{22}{23} & = & \frac{110592}{676039} \sim 0.164  < \frac{1}{6}
\neea
%
So that's how we got the eight and of course if we chose any other eight primes we will get something bigger than $55296/323323>1/6$ because $n/(n+1)$ converges to 1 as $n \to \infty$, \ie $n/(n+1)$ gets bigger as $n$ gets bigger.

Another reason we have to limit it to eight is because $n/(n+1) < 1$ and if we keep multiplying them we'll get a smaller and smaller number and after some point we will reach $< 1/6$.

The rest is straightforward,
%
\nbea
\frac{\varphi(n)}{n} & = & \prod_{p|n}\frac{p - 1}{p}
\neea
%
so the argument above holds

{\bf Problem 2.5}. Define $\nu(1) = 0$, and for $n > 1$ let $\nu(n)$ be the number of distinct prime factors of $n$. Let $f = \mu * \nu$ and prove that $f(n)$ is either 0 or 1.

As the inverse of $\mu$ is $\mu^{-1} = u$, this means that
%
\nbea
u * f & = & (u * \mu) * \nu \\
& = & I * \nu \\
u * f & = & \nu \\
\to \nu(n) & = & \sum_{d|n} f(d)
\neea
%
$\nu$ is obviously not multiplicative since $\nu(1) \neq 1,~\nu(pq) \neq \nu(p)\nu(q)$ but it is actually additive since $\nu(p^\alpha q^\beta) = \nu(p^\alpha) + \nu(q^\beta) = \nu(p) + \nu(q)$ where $p \neq q$ are distinct primes, so let's decompose $n$ into its primal constituents, $n = \prod_i p_i^{\alpha_i}$
%
\nbea
\nu\left(\prod_i p_i^{\alpha_i}\right) & = & \sum_{d|n} f(d) \\
\sum_i \nu\left(p_i^{\alpha_i}\right) & = & \sum_{d|n} f(d) \\
\sum_i \nu\left(p_i\right) & = & \sum_{d|n} f(d)
\neea
%
from here we can immediately see that $f(n)$ is given by
%
\nbea
f(n) = \left\{
\begin{array}{l}
1 {\rm~if~} n {\rm ~is~prime} \\
0 {\rm~otherwise}
\end{array} \right.
\neea
%

{\bf Problem 2.6}. Prove that
%
\nbea
\sum_{d^2|n} \mu(d) & = & \mu^2(n)
\neea
%
and, more generally,
%
\nbea
\sum_{d^k|n} \mu(d) = \left \{
\begin{array}{l c l}
0 & & {\rm if~} m^k|n {\rm~for~some~} m > 1 \\
1 & & {\rm otherwise}
\end{array}\right.
\neea
%
The last sum is extended over all positive divisors $d$ of $n$ whose $k$th power also divide $n$.

The key point here is again ``multiplicative'', since $\mu(d)$ is multiplicative so is $\sum_{d^2|n} \mu(d)$ so we need to only consider $g(p^\alpha) = \sum_{d^2|p^\alpha} \mu(d)$ but note that even though the sum is over $d^2 \to \sum_{d^2|n}$, $\mu$ is only taking $d$, $\mu(d)$ and not $\mu(d^2)$
%
\nbea
\sum_{d^2|p^\alpha} \mu(d) & = & \mu(1) + \mu(p) \\
& = & 1 - 1 \\
& = & 0
\neea
%
The above holds if $\alpha > 1$ otherwise for $0 \le \alpha \le 1 \to \sum_{d^2|p^\alpha} \mu(d) = \mu(1) = +1$, in short
%
\nbea
g(p^\alpha) = \sum_{d^2|p^\alpha} \mu(d) & = & \left \{
\begin{array}{r c  l}
0 && {\rm if ~} \alpha > 1 \\
1 && {\rm if ~} 0 \le \alpha \le 1 \\
\end{array}\right. \\
& = & \mu^2(p^\alpha)
\neea
%
The second part follows closely, again since it is multiplicative and again note that even though the sum is over $d^k \to \sum_{d^k|n}$, $\mu$ is only taking $d$, $\mu(d)$ and not $\mu(d^k)$
%
\nbea
\sum_{d^k|p^\alpha} \mu(d) & = & \mu(1) + \mu(p) \\
& = & 1 - 1 \\
& = & 0
\neea
%
if $\alpha > k$ otherwise for $0 \le \alpha \le k \to \sum_{d^k|p^\alpha} \mu(d) = \mu(1) = +1$, the only difference now is that we can't say it is equal to $\mu^2(p^\alpha)$ because say $\alpha = k-1 > 0 \to \mu(p^{k-1}) = 0$ but $\sum_{d^k|p^{k-1}} \mu(d) = \mu(1) = +1$

{\bf Problem 2.7}. Let $\mu(p,d)$ denote the value of the Mobius function at the gcd of $p$ and $d$. Prove that for every prime $p$ we have
%
\nbea
\sum_{d|n} \mu(d)\mu(p,d) = \left \{
\begin{array}{l c l}
1 && {\rm if~} n = 1 \\
2 && {\rm if~} n = p^a, a \ge 1 \\
0 && {\rm otherwise}.
\end{array}\right.
\neea
%

The thing is the gcd $(p,mn)$ is multiplicative as long as $(m,n)=1$ because $p$ is prime and once we expand $m$ and $n$ in their primal constituents it is evident, \ie $(p,mn) = (p,m)(p,n)$, therefore $\mu(p,mn) = \mu(p,m)\mu(p,n)$

The first case is obvious $\sum_{d|1} \mu(d)\mu(p,d) = \mu(1)\mu(1) = 1$.

The second case
%
\nbea
\sum_{d|p^a} \mu(d)\mu(p,d) & = & \mu(1)\mu(p,1) + \mu(p)\mu(p,p) \\
& = & \mu(1)\mu(1) + \mu(p)\mu(p) \\
& = & (1)(1) + (-1)(-1) \\
& = & 2
\neea
%

To show the last case it's easiest to utilize the fact that $g(n) = \sum_{d|n}\mu(d)\mu(p,d)$ is multiplicative and now we just need to show $g(q^b), ~q\neq p$ as $g(p^a)$ is already covered above
%
\nbea
g(q^b) = \sum_{d|q^b}\mu(d)\mu(p,d) & = & \mu(1)\mu(p,1) + \mu(q)\mu(p,q) \\
& = & \mu(1)\mu(1) + \mu(q)\mu(1) \\
& = & (1)(1) + (-1)(1) \\
& = & 0
\neea
%

{\bf Problem 2.8}. Prove that
%
\nbea
\sum_{d|n} \mu(d) \log^m d = 0
\neea
%
if $m \ge 1$ and $n$ has more than $m$ distinct prime factors. [{\it Hint:} Induction.]

To use induction we need to prove the base case, the thing is that $\log$ is not multiplicative, so that's a bit hard. The base case should be $m = 1$ and then we go up from there to bigger $m$ ?!? \dunno

But one thing I notice is that we only need to consider numbers with one power of distinct primes, \ie $n = p_1p_2\ldots p_k$ because $\mu(d)$ is zero if the powers of the primes are not zero that is
%
\nbea
\sum_{d|n} \mu(d) \log^m d & = & \cancel{\mu(1) \log^m(1)} + \mu(p_1)\log^m(p_1) + \ldots + \mu(p_k)\log^m(p_k) + \\
&& \mu(p_1p_2)\log^m(p_1p_2) + \ldots + \mu(p_{k-1}p_k)\log^m(p_{k-1}p_k) + \ldots + \\
&& \mu(p_1p_2\ldots p_k)\log^m(p_1p_2\ldots p_k)
\neea
%
and from the definition of $\mu(d)$ we know that if it has odd number of primes it's negative and it there are an even number of distinct primes $\mu$ is positive, therefore
%
\nbea
\sum_{d|n} \mu(d) \log^m d & = & -(\log^m(p_1) + \ldots + \log^m(p_k)) \\
&& +(\log^m(p_1p_2) + \ldots + \log^m(p_{k-1}p_k)) + \\
&& -(\log^m(p_1p_2p_3)+ \ldots + \log^m(p_{k-2}p_{k-1}p_k)) +\\
&& (-1)^k\log^m(p_1p_2\ldots p_k)
\neea
%
Since log is additive we can expand them but before we do that let's denote $\log(p_k) = l_k$
%
\nbea
\sum_{d|n} \mu(d) \log^m d & = & -\sum_{i_1 = (k|1)} l_{i_1}^m + \sum_{i_1, i_2 = (k|2)} (l_{i_1} + l_{i_2})^m - \sum_{i_1, i_2, i_3 = (k|3)} (l_{i_1} + l_{i_2} + l_{i_3})^m + \\
&& \ldots + (-1)^k \sum_{i_1, i_2, \ldots , i_k = (k|k)} (l_{i_1} + l_{i_2} + \ldots + l_{i_k})
\neea
%
where the notation $(k|j)$ means that all combinations of $k$ choose $j$, as a concrete example, say $m=4, ~k=5$ which is the minimum $k$ required
%
\nbea
\sum_{d|n} \mu(d) \log^m d & = & -(l_1^4 + l_2^4 + l_3^4 + l_4^4 + l_5^4) + \\
&& + ((l_1+l_2)^4 + (l_1+l_3)^4 + (l_1+l_4)^4 + (l_1+l_5)^4 + (l_2+l_3)^4 + (l_2+l_4)^4 +  \\ 
&& ~~~~(l_2+l_5)^4 + (l_3+l_4)^4 + (l_3+l_5)^4 + (l_4+l_5)^4) + \\
&& - ((l_1 + l_2 + l_3)^4 + (l_1 + l_2 + l_4)^4 + (l_1 + l_2 + l_5)^4 + (l_1 + l_3 + l_4)^4 + \\
&& ~~~~ (l_1 + l_3 + l_5)^4  + (l_1 + l_4 + l_5)^4  + (l_2 + l_3 + l_4)^4 + (l_2 + l_3 + l_5)^4 + \\
&& ~~~~ (l_2 + l_4 + l_5)^4 + (l_3 + l_4 + l_5)^4) \\
&& + ((l_1 + l_2 + l_3 + l_4)^4 + (l_1 + l_2 + l_3 + l_5)^4 + (l_1 + l_2 + l_4 + l_5)^4 + \\
&& ~~~~ (l_1 + l_3 + l_4 + l_5)^4 + (l_2 + l_3 + l_4 + l_5)^4) + \\
&& - ((l_1 + l_2 + l_3 + l_4 + l_5)^4)
\neea
%
Now we gather coefficients of same powers, say we collect all $l_1^4$, 
%
\nbea
(5|1) & \to & (-1)l_1^4 \\
(5|2) & \to & (+4)l_1^4 \\
(5|3) & \to & (-6)l_1^4 \\
(5|4) & \to & (+4)l_1^4 \\
(5|5) & \to & (-1)l_1^4
\neea
%
so they're basically the Pascal triangle coefficients, why is this? Well, for example, for $(5|1)$, first we fix {\bf one} $l$ and then choose a partner for it from the remaining {\bf four}, however in this case we only need one $l$ and we already fixed it, so we will just need {\bf zero} partner, \ie ${4 \choose 0} = 1$.

For $(5|2)$ we first pick an $l$ and then choose a partner (again because $(5|${\bf 2}$)$ means we need {\bf 2} $l$'s in total) for it from 4 available choices, which is ${4 \choose 1}$, \ie this $l$ will appear ${4 \choose 1} = 4$ times, for $(5|3)$ it's the same thing we first pick an $l$ and then choose {\it two} partners for it, \ie  this $l$ will then appear ${4 \choose 2} = 6$ times, and for $(5|3)$, it's pick an $l$ and choose ${4 \choose 3} = 4$ partners and so on and therefore the coefficients of $l_1$ is just those of Pascal triangle's but with the signs alternating between plus and minus. And this is true for other $l$'s not just $l_1$.

We now need to tackle the cross terms say $l_1^3l_2$, first thing to note that this cross product is always preceded by a constant (which again is from Pascal triangle), for $(l_1 + l_2)^4$ it is $4l_1^3l_2$, note that this coefficient is the same no matter how many terms are being exponentiated, \ie even for $(l_1 + l_2 + l_3 + \ldots + l_w)^4$, the coefficient for $l_1^3l_2$ is still 4 because it is still ${4 \choose 3}$ no matter what, this is because
%
\nbea
(l_1 + l_2 + \ldots)^4 & = & \underbrace{(l_1 + l_2 + \ldots)}_\text{bin \#1}\underbrace{(l_1 + l_2 + \ldots)}_\text{bin \#2}\underbrace{(l_1 + l_2 + \ldots)}_\text{bin \#3}\underbrace{(l_1 + l_2 + \ldots)}_\text{bin \#4}
\neea
%
To get $l_1^3l_2$ we need to gather {\bf three} $l_1$'s and we have {\bf four} bins to choose for as shown above that's why we have 4 choose 3, ${4 \choose 3} = 4$ possibilities. And as the number of bins are the same no matter how many $l$'s we have the number of possibilities is still the same.

We also have other cross terms like $l_1^2l_3l_4$, in this case, we need to gather {\bf two} $l_1$'s from {\bf four} bins so it's ${4 \choose 2} = 6$, next we need to choose {\bf one} $l_3$ from the remaining {\bf two} bins which is ${2 \choose 1} = 2$ and once we've chosen the bin for $l_2$, the other bin will definitely contain $l_3$, so in total there are
%
\nbea
{4 \choose 2}\times{2 \choose 1} & = & 6 \times 2 = 12
\neea
%
and since the number of bins is constant no matter what this coefficient remains the same no matter how many $l$'s we have.

So now for $4l_1^3l_2$ we have
%
\nbea
(5|1) & \to & (0) \\
(5|2) & \to & (+1)4l_1^3l_2 \\
(5|3) & \to & (-3)4l_1^3l_2 \\
(5|4) & \to & (+3)4l_1^3l_2 \\
(5|5) & \to & (-1)4l_1^3l_2
\neea
%
again Pascal triangle, why is this? This time we fix {\bf two} $l$'s (instead of just one for $l^4$ above), and then calculate how many partners this couple might have, for $(5|2)$, we only need {\bf two} in total so because we already fixed two of them we just need {\bf zero} partner from the three remaining ones, \ie ${3 \choose 0} = 1$. For $(5|3)$, again we fix {\bf two} $l$'s and choose one more partner (because in total we need 3) from the remaining three, \ie ${3 \choose 1} = 3$ and so on. This is also true for any two-term cross terms.

And this pattern continues for higher cross terms like $l_1^2l_2l_3$, \eg for $(5|4)$ we fix {\bf three} $l$'s and then choose one partner from the remaining {\bf two}, which means ${2 \choose 1} = 2$.

This pattern continues for any $k$, say we now have $k=6$ while $m$ stays the same, $m=4$, in this case we have $(6|1),(6|2),(6|3),(6|4),(6|5),(6|6)$, and to get the coefficients for different $l$ powers we use the same method as described above.

Say you want to know the coefficient $l_1^4$ for each $(6|1),(6|2),(6|3),(6|4),(6|5),(6|6)$, then fix an $l$ and choose a partner for it depending on which combination $(6|j)$ you're on; for just a single $l$ the combination is ${6-1\choose j-1}$ and for three $l$'s like $l_1l_2l_3^2$ we fix three and then choose a partner resulting in ${6-3 \choose j-3}$ combo.

And here we immediately see why the number of distinct primes $k$ must be larger than $m$, the exponent of $\log$, it's because if $k = m$ then on the last combo $(k=m|j=m)$ we will have ${m-m\choose m-m} = 1$ but these $l$'s, $l_1^{a_1}l_2^{a_2}\ldots l_m^{a_m}$ can only be found once and there'll be nothing to cancel it, the same is true if $k < m$, the longest $l$ combo $l_1^{a_1}l_2^{a_2}\ldots l_k^{a_k}$ is only generated once and there's nothing to cancel it to zero.

As a concrete example take $m=3$ and $k=2$, we will then have
%
\nbea
-(l_1^3 + l_2^3) + (l_1+l_2)^3 & = & 3l_1^2l_2 + 3l_1l_2^2
\neea
%
and for $m=3$, $k=3$
%
\nbea
-(l_1^3 + l_2^3 + l_3^3) + ((l_1+l_2)^3 + (l_1+l_3)^3 + (l_2+l_3)^3) - (l_1 + l_2 + l_3)^3 & = & -6l_1l_2l_3
\neea
%
so you see the longest $l$ combo is not canceled whenever $k \le m$. But if $k > m$ then the longest $l$ combo is still $m$ and for every combo of the form $(k|m \le j \le k)$ we have a coefficient of ${k-m\choose j - m}$ which is just the Pascal triangle for $(1-1)^{k-m} = 0$.

In summary, there are three numbers involved, the exponent $m$, the number of distinct primes $k$, and lastly the dummy index $j$ as indicated below
%
\nbea
\sum_{j=1}^{k} \sum_{(k|j)} \left ( l_{i_1} + \ldots + l_{i_j} \right )^m
\neea
%
and the pattern of these different combinations of $l$'s can be seen in Table~\ref{Tab:1} and we immediately see why they all sum to zero.

%
\begin{table}[]
\centering
\caption{Coefficients of various combo of $l$'s for different $j$'s with a given $k$ and $m$, note that the sum of the exponents of $l$'s is always $m$, $\sum_i a_i = m$}
\label{Tab:1}
\begin{tabular}{c | c c c c c }
~~~ ~ ~~~ & ~~~$\underbrace{l^m}_\text{one $l$}$~~~ & ~~$\underbrace{l_{b_1}^{a_1}l_{b_2}^{a_2}}_\text{2 $l$'s}$~~ & ~~$\underbrace{l_{b_1}^{a_1}l_{b_2}^{a_2}l_{b_3}^{a_3}}_\text{3 $l$'s}$~~ & ~~~ $\ldots$ ~~~ & $\underbrace{l_{b_1}^{a_1}l_{b_2}^{a_2}\ldots l_{b_m}^{a_m}}_\text{$m$ $l$'s}$ \\ \hline
$(k|1)$ & ${k-1 \choose 1-1}$ & & & & \\ 
$(k|2)$ & ${k-1 \choose 2-1}$ & ${k-2 \choose 2-2}$ & & & \\ 
$(k|3)$ & ${k-1 \choose 3-1}$ & ${k-2 \choose 3-2}$ & ${k-3 \choose 3-3}$ & & \\ 
$\vdots$ & $\vdots$ & $\vdots$ & $\vdots$ & &\\ 
$(k|m)$ & ${k-1 \choose m-1}$ & ${k-2 \choose m-2}$ & ${k-3 \choose m-3}$ & & ${k-m \choose m-m}$ \\ 
$(k|m+1)$ & ${k-1 \choose (m+1)-1}$ & ${k-2 \choose (m+1)-2}$ & ${k-3 \choose (m+1)-3}$ & & ${k-m \choose (m+1)-m}$ \\ 
$\vdots$ & $\vdots$ & $\vdots$ & $\vdots$ & & $\vdots$ \\ 
$(k|k)$ & ${k-1 \choose k-1}$ & ${k-2 \choose k-2}$ & ${k-3 \choose k-3}$ & & ${k-m \choose m-m}$ \\ 
\end{tabular}
\end{table}
%

















\bigskip
In Exercises 10, 11, and 12, $d(n)$ denotes the number of positive divisors of $n$.

{\bf Problem 2.10}. Prove that $\prod_{t|n} t = n^{d(n)/2}$.

Again, let's decompose $n$ into its primal constituents $n = \prod_i^N p_i^{\alpha_i}$ then $d(n)$ is given by
%
\nbea
d(n) = d\left(\prod_i^N p_i^{\alpha_i}\right) & = & \prod_i^N (\alpha_i + 1)
\neea
%
To see why this is we just need to recall that the number of combinations an $N$-digit (base-10) number has is 
%
\nbea
\# {\rm~of~combo} = \underbrace{10 \times 10 \times 10 \times \ldots \times 10}_\text{$N$ {\rm of~them}}
\neea
%
because each digit can take 10 possible different values. For our case, each prime factor plays the role of a digit, however, each has different possible values, which is $(\alpha_i+1)$ because we can have $p_i^0,p_i^1,p_i^2,\ldots,p_N^{\alpha_N}$ so the total number of combinations for $\prod_i^N p_i^{\alpha_i}$ is
%
\nbea
\# {\rm~of~combo} = \underbrace{(\alpha_1 + 1) (\alpha_2 + 1) (\alpha_3 + 1) \ldots (\alpha_N + 1)}_\text{$N$ {\rm prime~factors}}
\neea
%

Next, we can decompose $\prod_{t|n} t$ in terms of its primal constituents as well, say we focus on $p_1$ of $\prod_i^N p_i^{\alpha_i}$, the divisors of $p_1^{\alpha_1}$ are $p_1^0, p_1^1, \ldots, p_1^{\alpha_1}$, so if we multiply all of them we have $p_1^{1 + 2 + 3 + \ldots + \alpha_1} = p_1^{\frac{\alpha_1(\alpha_1 + 1)}{2}} = \left ( p_1^{\alpha_1}\right )^{\frac{\alpha_1+1}{2}}$.

But here $p_1^{\alpha_1}$ is not alone, each divisor of $p_1^{\alpha_1}$, \ie $p_1^{j}, ~0 \le j \le \alpha_1$, occurs $(\alpha_2+1)(\alpha_3+1)\ldots(\alpha_N+1)$ times, so the final exponent for $p_1$ in $\prod_{t|n} t$ is
%
\nbea
\left (p_1^{\alpha_1}\right )^{\frac{(\alpha_1+1)}{2}(\alpha_2+1)(\alpha_3+1)\ldots(\alpha_N+1)} & = & \left (p_1^{\alpha_1}\right )^{d(n)/2}
\neea
%
the same case goes for any other $p_i$, thus $\prod_{t|n}t = n^{d(n)/2}$. As a concrete example, take $n = p_1^2p_2^3$, the divisors of $n$ are
%
\nbea
\begin{array}{c c c c c c c}
p_1^0 ~~ p_2^0 & ~~~~~~~~ & p_1^0 ~~ p_2^1 & ~~~~~~~~ & p_1^0 ~~ p_2^2 & ~~~~~~~~ & p_1^0 ~~ p_2^3 \\
p_1^1 ~~ p_2^0 & ~~~~~~~~ & p_1^1 ~~ p_2^1 & ~~~~~~~~ & p_1^1 ~~ p_2^2 & ~~~~~~~~ & p_1^1 ~~ p_2^3 \\
p_1^2 ~~ p_2^0 & ~~~~~~~~ & p_1^2 ~~ p_2^1 & ~~~~~~~~ & p_1^2 ~~ p_2^2 & ~~~~~~~~ & p_1^2 ~~ p_2^3
\end{array}
\neea
%
so you can see that $(p_1^0~p_1^1~p_1^2)$ occurs $4=(\alpha_2+1)$ times $\to (p_1^0~p_1^1~p_1^2)^{\alpha_2+1}$.

{\bf Problem 2.11}. Prove that $d(n)$ is odd if, and only if, $n$ is square.

As shown above for $n = \prod_i^N p_i^{\alpha_i}$, $d(n) = \prod_i^N (\alpha_i + 1)$, so to get $d(n)$ to be odd we need {\it all} of $\alpha_i$ to be even so that $(\alpha_i + 1)$ is odd, therefore $n$ must be even

{\bf Problem 2.12}. Prove that $\sum_{t|n} d(t)^3 = \left (\sum_{t|n} d(t)\right )^2$.

The above relationship is evidently not true in general, we therefore need to utilize the properties of $d(t)$ to derive it. One thing to note is that $g(n) = \sum_{t|n} d(t)^3$ is multiplicative as $d(t)$ is. Therefore we just need to consider $g(p^\alpha) = \sum_{t|p^\alpha} d(t)^3$.

My strategy would be to utilize induction. Assume that $\sum_{t|p^\alpha} d(t)^3 = \left (\sum_{t|p^\alpha} d(t)\right )^2$ is true up to some $p^\alpha$, we now want to know what happens with $p^{\alpha+1}$
%
\nbea
\sum_{t|p^{\alpha+1}} d(t)^3 & = & d(p^{\alpha+1})^3 + \sum_{t|p^\alpha} d(t)^3
\neea
%
and $d(p^{\alpha+1}) = \alpha+2$ thus
%
\nbea
d(p^{\alpha+1})^3 + \sum_{t|p^\alpha} d(t)^3 & = & (\alpha + 2)^3 + \left ( \sum_{t|p^\alpha} d(t)\right )^2 \\
& = & (\alpha + 2)^2(\alpha + 2) + \left ( \sum_{t|p^\alpha} d(t)\right )^2 \\
& = & (\alpha + 2)^2 + (\alpha + 2)^2(\alpha + 1) + \left ( \sum_{t|p^\alpha} d(t)\right )^2 \\
& = & d(p^{\alpha + 1})^2 + (\alpha + 2)\cdot 2 \frac{(\alpha + 2)(\alpha + 1)}{2} + \left ( \sum_{t|p^\alpha} d(t)\right )^2 \\
& = & d(p^{\alpha + 1})^2 + 2 d(p^{\alpha+1})\left (\sum_{t|p^\alpha} d(t) \right ) + \left ( \sum_{t|p^\alpha} d(t)\right )^2 \\
& = & \left ( d(p^{\alpha + 1}) + \sum_{t|p^\alpha} d(t) \right )^2 \\
\sum_{t|p^{\alpha+1}} d(t)^3 & = & \left ( \sum_{t|p^{\alpha+1}} d(t) \right )^2
\neea
%
Going to line 5 we have used the fact that $\sum_{t|p^\alpha} d(t) = \sum_{i=1}^{\alpha+1} i = \frac{(\alpha+1)(\alpha + 2)}{2} $ since $d(p^j) = j+1$. We can of course dispel induction for a bruter force approach by expanding $\sum_{t|p^\alpha+1} d(t)^3 = \sum_{i=1}^{\alpha+1} i^3$ but this requires us to know the formula for a sum of consecutive cubes \dunno

\bigskip
\underline{\textbf{\textit{Chapter 3}}}
\smallskip

{\bf Section 3.3, Page 54}. ``Euler's summation formula, Theorem 3.1'', if you look carefully enough at the definition the lower limit of the sum, $\sum_{y<n\le x}$, is just a less than sign and {\bf not} and less than {\bf equal} sign, this is crucial as we go to the proof of Theorem 3.3

{\bf PROOF of Theorem 3.1, Page 55}, there's some ``fudging'' going on here and not just once :) in the first line
%
\nbea
\int_{n-1}^{n}[t] f'(t)dt & \to & \int_{n-1}^n (n-1)f'(t)dt
\neea
%
so we assume that $[t] = n-1$ for the whole interval $[n-1,n]$, well this is true for if the interval excludes $n$, \ie $[n-1,n)$. So here's where the ``fudging'' comes in, in the definition of Riemann integral we can always remove a point since the area under a curve of just one point is zero $\to dt = 0$ so in this case we remove the end point $n$.

The second ``fudging'' happens in Eq (6) when we go from $-\int_m^k \to -\int_y^x$, the area under the curve is definitely different for those two integrals unless $m=[y]=y$ and $k=[x]=x$ , let's see this with a concrete example say, $f(t) = t \to f'(t) = 1$ with $y = 1.5$ and $x = 3.1$, the integral $\int_y^x[t]f'(t)dt$ is given by
%
\nbea
\int_{1.5}^{3.1}[t]f'(t)dt & = & \int_{1.5}^{2}dt + \int_{2}^{3}2dt+  \int_{3}^{3.1}3dt \\
& = & (2-1.5) + 2(3-2) + 3(3.1-3) \\
& = & 0.5 + 2 + 0.3 \\
& = & 2.8
\neea
%
while the integral $\int_{[y]}^{[x]}[t]f'(t)dt$ is given by
%
\nbea
\int_{[1.5]}^{[3.1]}[t]f'(t)dt & = & \int_{1}^{2}dt + \int_{2}^{3}2dt \\
& = & (2-1) + 2(3-2) \\
& = & 3
\neea
%
so ........ \dunno

{\bf Page 55}, ``Integration by parts gives us $\dots$ and when this is combined with (6) we obtain (5)'', what we want to do here is to move everything to the LHS
%
\nbea
\int_y^x f(t)dt = xf(x) - yf(y) - \int_y^xtf'(t)dt \\
\to \int_y^x f(t)dt - xf(x) + yf(y) + \int_y^xtf'(t)dt & = & 0
\neea
%
and we add this zero to (6) to get (5) sans the fudging discussed above :)

{\bf PROOF of Theorem 3.2, Page 55} The peculiar thing here is the contant 1 which we get from $-f(y)([y]-y)$. The problem here is that the lower limit $y = 1$ and so $[y]-y = 1-1=0$, however, as explained above, the lower limit must not be an integer as $y < n$ and $n$ starts with $n=1$, the less than sign is crucial here.

This means that $0 < y < 1$ and in this case $y$ has to be $y = 1^{-}$ such that we can take the limit $y \to 1$, so even though the lower limit of the integrals is $y=1$ we cannot just simply choose $y=1$, this manifests more strongly in the proof of Theorem 3.2 part (b) where in this case $f(y) = x^{-s}$
%
\nbea
-f(y)([y]-y) & = & -\frac{1}{y^s} (0 - y) \\
& = & \frac{1}{y^{s-1}}
\neea
%
it won't equal 1 unless we take the limit $y \to 1$, we can however, choose any value of $y$ in the range $0 <y < 1$ say $y = 1/w$ where $0 < w < \infty$
%
\nbea
\sum_{n\le x} \frac{1}{n^s} & = & \int_{1/w}^x \frac{dt}{t^s} - s\int_{1/w}^x\frac{t-[t]}{t^{s+1}} + \frac{[x]-x}{x^s} - \frac{0 - (1/w)}{(1/w)^s} \\
& = & \frac{x^{1-s}}{1-s} - \frac{(1/w)^{1-s}}{1-s} - s\int_{1/w}^1\frac{t-[t]}{t^{s+1}} - s\int_{1}^x\frac{t-[t]}{t^{s+1}} - \frac{x-[x]}{x^s} + (1/w)^{1-s} \\
& = & \frac{x^{1-s}}{1-s} - \frac{(1/w)^{1-s}}{1-s} - s\int_{1/w}^1\frac{t-[t]}{t^{s+1}} - s\int_{1}^x\frac{t-[t]}{t^{s+1}} - \frac{x-[x]}{x^s} + \frac{(1-s)(1/w)^{1-s}}{1-s}
\neea
%
we now focus on the first integral on the RHS
%
\nbea
-s\int_{1/w}^1\frac{t-[t]}{t^{s+1}} & = & -s\int_{1/w}^1\frac{t-[1/w]}{t^{s+1}} \\
& = & -s\int_{1/w}^1\frac{t}{t^{s+1}} \\
& = & -s\int_{1/w}^1\frac{1}{t^{s}} \\
& = & -\frac{s}{1-s} + \frac{s (1/w)^{1-s}}{1-s}
\neea
%
Combining everything we get
%
\nbea
\sum_{n\le x} \frac{1}{n^s} & = & \frac{x^{1-s}}{1-s} - \bcancel{\frac{(1/w)^{1-s}}{1-s}} -\frac{s}{1-s} + \cancel{\frac{s(1/w)^{1-s}}{1-s}} - s\int_{1}^x\frac{t-[t]}{t^{s+1}} - \frac{x-[x]}{x^s} + \frac{(\bcancel{1}-\cancel{s})(1/w)^{1-s}}{1-s} \\
& = & \frac{x^{1-s}}{1-s} +\frac{-1 + (1 - s)}{1-s} - s\int_{1}^x\frac{t-[t]}{t^{s+1}} - \frac{x-[x]}{x^s} \\
& = & \frac{x^{1-s}}{1-s} - \frac{1}{1-s} + 1 - s\int_{1}^x\frac{t-[t]}{t^{s+1}} - \frac{x-[x]}{x^s}
\neea
%
and we get the same result.

The lesson here is that we need to be very careful in choosing and processing these limits otherwise we'll get the wrong result.

{\bf Page 56}, first equation, the key here is that $t-[t] < 1$ and so the inequality holds

{\bf Page 56}, the value of $C$ (and $C(s)$), in the roughly middle section of the page we have
%
\nbea
\lim_{x\to\infty}\left ( \sum_{n \le x} \frac{1}{n} - \log x\right) = 1 - \int_1^\infty \frac{t - [t]}{t^2} dt,
\neea
%
and the RHS is $C$ and therefore $C$ equals the Euler's constant since the LHS is Euler's constant but it seems like this is only true when $x \to \infty$, which is to say that $C$ is independent of $x$, but recall that from earlier in the page
%
\nbea
\sum_{n\le x} \frac{1}{n} & = & \log x + C + O\left ( \frac{1}{x}\right ) \\
\to C & = & \sum_{n\le x} \frac{1}{n} - \log x - O\left ( \frac{1}{x}\right ) 
\neea
%
It therefore seems like $C$ depends on $x$ after all, how can this be? The answer is that
%
\nbea
\sum_{n\le x} \frac{1}{n} - \log x - O\left ( \frac{1}{x}\right ) & = & \lim_{x\to\infty}\left ( \sum_{n \le x} \frac{1}{n} - \log x\right)
\neea
%
Thus the LHS does {\bf not} depend on $x$ after all even though it looks like it, this argument also applies to $C(s)$ lower down in the page

{\bf Page 58}, Figure 3.1, note that this figure is a bit deceptive, to calculate $\sum_{qd \le 10}$ we need to draw one hyperbola for every $qd = \{1,2,3,4,5,6,7,8,9,10\}$, in the figure there are only 3 hyperbolas drawn. Also, the way the sum is calculated is by counting {\bf all} lattice points underneath the largest hyperbola, true that we have to count only points that lie on the hyperbolas but if we draw {\bf all} hyperbolas with $qd \le x$ we will cover all lattice points under the largest hyperbolic curve.

{\bf Page 60}, ``It can be shown that $\zeta(2) = \pi^2/6$'', an elementary proof of this fact can be obtained from the Fourier series ({\it not} Fourier transform) of $x^2$
%
\nbea
x^2 = \frac{a_0}{2} + \sum_{n=1}^\infty a_n\cos(nx) + \sum_{n=1}^\infty b_n\sin(nx)
\neea
%
$b_n = 0$ for all $n$ because $x^2$ is an even function while
%
\nbea
a_0 & = & \int_{-\pi}^{+\pi} x^2 \frac{dx}{\pi} \\
& = & \frac{1}{3\pi} \left (\pi^3 - (-\pi)^3 \right ) \\
& = & \frac{2\pi^2}{3}
\neea
%
and
%
\nbea
a_n & = & \int_{-\pi}^{+\pi} x^2 \cos(nx) \frac{dx}{\pi} \\
& = & \left.\frac{2x}{n\pi}\sin(nx)\right|_{-\pi}^{+\pi} - 2\int_{-\pi}^{+\pi} x \sin(nx) \frac{dx}{n\pi}
\neea
%
the boundary terms are zero since $\sin(\pm\pi) = 0$, we now do another integration by parts on the remaining integral
%
\nbea
- 2\int_{-\pi}^{+\pi} x \sin(nx) \frac{dx}{n\pi} & = & \left.\frac{2x}{n^2\pi}\cos(nx)\right|_{-\pi}^{+\pi} - 2\int_{-\pi}^{+\pi}  \cos(nx) \frac{dx}{n^2\pi} \\
& = & \left.\frac{2x}{n^2\pi}\cos(nx)\right|_{-\pi}^{+\pi}  + \left.\frac{-2}{n^3\pi}\sin(nx)\right|_{-\pi}^{+\pi} \\
a_n & = & (-1)^n\frac{4}{n^2}
\neea
%
where $\cos(\pm n\pi) = \cos(n\pi) = (-1)^n$ and of course $\sin(\pm n\pi) = 0$, thus
%
\nbea
x^2 & = & \frac{\pi^2}{3} + \sum_{n=1}^\infty (-1)^n\frac{4}{n^2} \cos(nx)
\neea
%
setting $x=\pi$ we get
%
\nbea
\pi^2 & = & \frac{\pi^2}{3} + \sum_{n=1}^\infty (-1)^n\frac{4}{n^2} \cos(n\pi) \\
\frac{2\pi^2}{3} & = & \sum_{n=1}^\infty (-1)^{2n}\frac{4}{n^2} \\
\to \frac{\pi^2}{6} & = & \sum_{n=1}^\infty \frac{1}{n^2}
\neea
%

{\bf Page 61}, PROOF of Theorem 3.6,
%
\nbea
\sum_{n\le x} \sigma_{-\beta}(n) & = & \sum_{n\le x} \sum_{d|n}\frac{1}{d^\beta} = \sum_{d\le x} \sum_{q\le x/d}\frac{1}{q^\beta}
\neea
%

if $\beta \neq 1$ we have
%
\nbea
\sum_{d\le x} \sum_{q\le x/d}\frac{1}{q^\beta} & = & \sum_{d\le x} \frac{(x/d)^{1-\beta}}{1-\beta} + \zeta(\beta) + O\left ((x/d)^{-\beta}\right ) \\
& = & x\zeta(\beta) + \frac{x^{1-\beta}}{1-\beta} \sum_{d\le x} d^{\beta-1} + O\left (\frac{1}{x^\beta}\sum_{d\le x} d^{\beta}\right ) \\
& = & x\zeta(\beta) + \frac{x^{1-\beta}}{1-\beta} \left ( \frac{x^\beta}{\beta} + O(x^{\beta-1}) \right ) + O\left (\frac{1}{x^\beta}\left (\frac{x^{\beta+1}}{\beta+1} + O(x^{\beta}) \right )\right ) \\
& = & x\zeta(\beta) + O(x)  + O(1)  + O(x) + O(1) \\
& = & x\zeta(\beta) + O(x)
\neea
%




%
\nbea
\sum_{n\le x} \log n & = & \int_1^x \log t ~dt + \int _1^x \frac{(t - [t]) }{t} dt + \log x ([x] - x) - \lim_{y\to 1} \log y ([y] - y) \\
& = & \bcancel{ x \log x} - x + 1 + \int _1^x \frac{(t - [t]) }{t} dt - \bcancel{x \log x}  + [x]\log x \\
& = & [x]\log x - x + 1 + O(\log x) \\
& = & O(x \log x)
\neea
%

if $\beta = 1$ we have
%
\nbea
\sum_{d\le x} \sum_{q\le x/d}\frac{1}{q^\beta} & = & \sum_{d\le x} \log \left ( \frac{x}{d}\right ) + C + O\left ( \frac{d}{x}\right ) \\
& = & xC + x \log x - \sum_{d\le x} \log d + O\left ( \frac{1}{x} \sum_{d\le x} d\right ) \\
& = & xC + x \log x - O(x \log x) + O\left ( \frac{1}{x} \left (\frac{x^2}{2} + O(x)\right )\right ) \\
& = & xC + x \log x - O(x \log x) + O(x) + O(1)\\
& = & xC + O(x \log x)
\neea
%

{\bf Problem 20}. If $n$ is a positive , $[\sqrt{n} + \sqrt{n+1}] = [\sqrt{4n+2}]$

%
\nbea
\sqrt{n + 1} & = & \sqrt{n}\sqrt{1 + \frac{1}{n}} \\
& = & \sqrt{n} \left ( 1 + \frac{1}{2}\left (\frac{1}{n}\right ) - \frac{1}{8}\left (\frac{1}{n}\right )^2 + \frac{1}{16}\left (\frac{1}{n}\right )^3 + \ldots  \right ) \\
& = & \sqrt{n} \left ( 1 + \Delta \right )
\neea
%
with $\Delta < 1$
%
\nbea
\sqrt{4n + 2} & = & 2\sqrt{n}\sqrt{1 + \frac{1}{2n}} \\
& = & 2\sqrt{n} \left ( 1 + \frac{1}{2}\left (\frac{1}{2n}\right ) - \frac{1}{8}\left (\frac{1}{2n}\right )^2 + \frac{1}{16}\left (\frac{1}{2n}\right )^3 + \ldots  \right ) \\
& = & 2\sqrt{n} \left ( 1 + \tilde\Delta \right )
\neea
%

{\bf Problem 21}. Determine all $n$ where $[\sqrt{n}]$ divides $n$

First, fix $n = a^2$ this means that for all integers $m \in \{n+1,n+2,\ldots,(a+1)^2-1\}$, $[\sqrt{m}] = a$ and since $a|n$ any number $w = n+aq, ~ n \le w < (a+1)^2$ is divisible by $a = [\sqrt{n}]$ so they are
%
\nbea
\{1,2,3,4,6,8,9,12,15,16,20,24,25,30,35,36,42,48,49,\ldots\}
\neea
%


{\bf Problem 22}. If $n$ is a positive integer, prove that
%
\nbea
\left \lbrack \frac{8n + 13}{25} \right \rbrack - \left \lbrack \frac{n - 12 - \left \lbrack \frac{n-17}{25}\right \rbrack}{3} \right \rbrack
\neea
%
is independent of $n$

%
\nbea
8n + 13 & = & 25w + r \\
8(n+1) + 13 & = & 25(w+1) + r' \\
8 & = & 25 + r' - r \\
\to r' - r & = & -17
\neea
%


%
\nbea
n-12 - l & = & 3u + z \\
(n+1) -12 - l' & = & 3(u+1) + z' \\
1 - (l' - l) & = & 3 + z' - z \\
(l'-l) & = & -2 + z - z' \\
(l'-l) +(z'-z) & = & -2
\neea
%


but since $(25,8) = 1$ it must mean that $25|a$ and the smallest $a$ is $a=25$.

%
\nbea
\frac{\frac{25j - 13}{8} - 17}{25} & = & 
\neea
%

Using Bezout's lemma, since $\gcd(25,8) = 1$ we have
%
\nbea
25x + 8y & = & 1 \\
25x' + 8y' & = & 13
\neea
%



OK, the strategy here is simple, although initially I wanted to see if I can utilize something from chapter 3 to solve this problem, maybe like putting a sum in front of those two terms and see what I get but let's just stay simple.

The strategy here is to show that when we increase $n$ the change in the first square bracket is the same as the change in the second square bracket, let's now look at the first square bracket more closely, if we remove the constant 13 for now we have
%
\nbea
\left \lbrack \frac{8n}{25} \right \rbrack
\neea
%
the key here is that $25 = 8\cdot3 + 1$, that means everytime $n$ increases from $n = 3m$ to $n = 3m+1$, the square bracket also increases from $\left \lbrack \frac{8n}{25} \right \rbrack \to \left \lbrack \frac{8n}{25} \right \rbrack + 1$

This is because in the language of mod what we have is
%
\nbea
8\cdot 3 & \equiv & -1 \pmod{25}
\neea
%
and thus $8\cdot(3m) \equiv -m \pmod{25}$ and once we have $8\cdot(3\cdot 8 + 1) \equiv 0 \pmod{25}$ therefore $8\cdot(25 + 3f)$

%
\nbea
%\begin{array}{}
8\cdot 3 + 13 & \equiv & -13 \pmod{25}
%\end{array}
\neea
%

%
\nbea
%\begin{array}{}
8\cdot 1 + 13 & \equiv & -4 \pmod{25} \\
8\cdot 4 + 13 & \equiv & -5 \pmod{25} \\
8\cdot 7 + 13 & \equiv & -6 \pmod{25} \\
8\cdot 10 + 13 & \equiv & -7 \pmod{25} \\
8\cdot 13 + 13 & \equiv & -8 \pmod{25} \\
%\end{array}
\neea
%
and thus $8\cdot 14 + 13 \equiv 0 \pmod{25}$ meaning $8\cdot (13 + 3) + 13 \not\equiv -1 \pmod{25}$ instead $8\cdot(14 + 3) + 13 \equiv -1 \pmod{25}$

At the same time $\left \lbrack \frac{n-17}{25} \right \rbrack$ will change value at transition from $n=16 \to n=17$ and this is in sync with $\left \lbrack \frac{8\cdot 16 + 13}{25} \right \rbrack \to \left \lbrack \frac{8\cdot 17 + 13}{25} \right \rbrack$

the pattern repeats every 25 counts $n \to n + 25$ as 
%
\nbea
8m + 13 & = & 25j \\
8(m + 25) + 13 & = & 8m + 13 + 8\cdot 25 \\
& = & 25(j + 8)
\neea
%

%
\nbea
8(m + a) + 13 & = & 8m + 13 + 8\cdot a \\
25h & = & 25j + 8a \\
25(h-j) & = & 8a \\
25 &|& 8a
\neea
%
ditto with $\frac{n-17}{25}$





















\end{document}