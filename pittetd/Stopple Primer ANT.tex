\documentclass[aps,preprint,preprintnumbers,nofootinbib,showpacs,prd]{revtex4-1}
\usepackage{graphicx,color}
\usepackage{caption}
\usepackage{subcaption}
\usepackage{amsmath,amssymb}
\usepackage{multirow}
\usepackage{amsthm}%        But you can't use \usewithpatch for several packages as in this line. The search 

\usepackage{cancel}

%%% for SLE
\usepackage{dcolumn}   % needed for some tables
\usepackage{bm}        % for math
\usepackage{amssymb}   % for math
\usepackage{multirow}
%%% for SLE -End

\usepackage{ulem}
\usepackage{cancel}

\usepackage{hyperref}
\usepackage{mathrsfs}
\usepackage[top=1in, bottom=1.25in, left=1.1in, right=1.1in]{geometry}

\usepackage{mathtools} % for \DeclarePairedDelimiter{\ceil}{\lceil}{\rceil}


\newcommand{\msout}[1]{\text{\sout{\ensuremath{#1}}}}


%%%%%% My stuffs - Stef
\newcommand{\lsim}{\mathrel{\mathop{\kern 0pt \rlap
  {\raise.2ex\hbox{$<$}}}
  \lower.9ex\hbox{\kern-.190em $\sim$}}}
\newcommand{\gsim}{\mathrel{\mathop{\kern 0pt \rlap
  {\raise.2ex\hbox{$>$}}}
  \lower.9ex\hbox{\kern-.190em $\sim$}}}

%
% Key
%
\newcommand{\key}[1]{\medskip{\sffamily\bfseries\color{blue}#1}\par\medskip}
%\newcommand{\key}[1]{}
\newcommand{\q}[1] {\medskip{\sffamily\bfseries\color{red}#1}\par\medskip}
\newcommand{\comment}[2]{{\color{red}{{\bf #1:}  #2}}}


\newcommand{\ie}{{\it i.e.} }
\newcommand{\eg}{{\it e.g.} }

%
% Energy scales
%
\newcommand{\ev}{{\,{\rm eV}}}
\newcommand{\kev}{{\,{\rm keV}}}
\newcommand{\mev}{{\,{\rm MeV}}}
\newcommand{\gev}{{\,{\rm GeV}}}
\newcommand{\tev}{{\,{\rm TeV}}}
\newcommand{\fb}{{\,{\rm fb}}}
\newcommand{\ifb}{{\,{\rm fb}^{-1}}}

%
% SUSY notations
%
\newcommand{\neu}{\tilde{\chi}^0}
\newcommand{\neuo}{{\tilde{\chi}^0_1}}
\newcommand{\neut}{{\tilde{\chi}^0_2}}
\newcommand{\cha}{{\tilde{\chi}^\pm}}
\newcommand{\chao}{{\tilde{\chi}^\pm_1}}
\newcommand{\chaop}{{\tilde{\chi}^+_1}}
\newcommand{\chaom}{{\tilde{\chi}^-_1}}
\newcommand{\Wpm}{W^\pm}
\newcommand{\chat}{{\tilde{\chi}^\pm_2}}
\newcommand{\smu}{{\tilde{\mu}}}
\newcommand{\smur}{\tilde{\mu}_R}
\newcommand{\smul}{\tilde{\mu}_L}
\newcommand{\sel}{{\tilde{e}}}
\newcommand{\selr}{\tilde{e}_R}
\newcommand{\sell}{\tilde{e}_L}
\newcommand{\smurl}{\tilde{\mu}_{R,L}}

\newcommand{\casea}{\texttt{IA}}
\newcommand{\caseb}{\texttt{IB}}
\newcommand{\casec}{\texttt{II}}

\newcommand{\caseasix}{\texttt{IA-6}}

%
% Greek
%
\newcommand{\es}{{\epsilon}}
\newcommand{\sg}{{\sigma}}
\newcommand{\dt}{{\delta}}
\newcommand{\kp}{{\kappa}}
\newcommand{\lm}{{\lambda}}
\newcommand{\Lm}{{\Lambda}}
\newcommand{\gm}{{\gamma}}
\newcommand{\mn}{{\mu\nu}}
\newcommand{\Gm}{{\Gamma}}
\newcommand{\tho}{{\theta_1}}
\newcommand{\tht}{{\theta_2}}
\newcommand{\lmo}{{\lambda_1}}
\newcommand{\lmt}{{\lambda_2}}
%
% LaTeX equations
%
\newcommand{\beq}{\begin{equation}}
\newcommand{\eeq}{\end{equation}}
\newcommand{\bea}{\begin{eqnarray}}
\newcommand{\eea}{\end{eqnarray}}
\newcommand{\ba}{\begin{array}}
\newcommand{\ea}{\end{array}}
\newcommand{\bit}{\begin{itemize}}
\newcommand{\eit}{\end{itemize}}

\newcommand{\nbea}{\begin{eqnarray*}}
\newcommand{\neea}{\end{eqnarray*}}
\newcommand{\nbeq}{\begin{equation*}}
\newcommand{\neeq}{\end{equation*}}

\newcommand{\no}{{\nonumber}}
\newcommand{\td}[1]{{\widetilde{#1}}}
\newcommand{\sqt}{{\sqrt{2}}}
%
\newcommand{\me}{{\rlap/\!E}}
\newcommand{\met}{{\rlap/\!E_T}}
\newcommand{\rdmu}{{\partial^\mu}}
\newcommand{\gmm}{{\gamma^\mu}}
\newcommand{\gmb}{{\gamma^\beta}}
\newcommand{\gma}{{\gamma^\alpha}}
\newcommand{\gmn}{{\gamma^\nu}}
\newcommand{\gmf}{{\gamma^5}}
%
% Roman expressions
%
\newcommand{\br}{{\rm Br}}
\newcommand{\sign}{{\rm sign}}
\newcommand{\Lg}{{\mathcal{L}}}
\newcommand{\M}{{\mathcal{M}}}
\newcommand{\tr}{{\rm Tr}}

\newcommand{\msq}{{\overline{|\mathcal{M}|^2}}}

%
% kinematic variables
%
%\newcommand{\mc}{m^{\rm cusp}}
%\newcommand{\mmax}{m^{\rm max}}
%\newcommand{\mmin}{m^{\rm min}}
%\newcommand{\mll}{m_{\ell\ell}}
%\newcommand{\mllc}{m^{\rm cusp}_{\ell\ell}}
%\newcommand{\mllmax}{m^{\rm max}_{\ell\ell}}
%\newcommand{\mllmin}{m^{\rm min}_{\ell\ell}}
%\newcommand{\elmax} {E_\ell^{\rm max}}
%\newcommand{\elmin} {E_\ell^{\rm min}}
\newcommand{\mxx}{m_{\chi\chi}}
\newcommand{\mrec}{m_{\rm rec}}
\newcommand{\mrecmin}{m_{\rm rec}^{\rm min}}
\newcommand{\mrecc}{m_{\rm rec}^{\rm cusp}}
\newcommand{\mrecmax}{m_{\rm rec}^{\rm max}}
%\newcommand{\mpt}{\rlap/p_T}

%%%song
\newcommand{\cosmax}{|\cos\Theta|_{\rm max} }
\newcommand{\maa}{m_{aa}}
\newcommand{\maac}{m^{\rm cusp}_{aa}}
\newcommand{\maamax}{m^{\rm max}_{aa}}
\newcommand{\maamin}{m^{\rm min}_{aa}}
\newcommand{\eamax} {E_a^{\rm max}}
\newcommand{\eamin} {E_a^{\rm min}}
\newcommand{\eaamax} {E_{aa}^{\rm max}}
\newcommand{\eaacusp} {E_{aa}^{\rm cusp}}
\newcommand{\eaamin} {E_{aa}^{\rm min}}
\newcommand{\exxmax} {E_{\neuo \neuo}^{\rm max}}
\newcommand{\exxcusp} {E_{\neuo \neuo}^{\rm cusp}}
\newcommand{\exxmin} {E_{\neuo \neuo}^{\rm min}}
%\newcommand{\mxx}{m_{XX}}
%\newcommand{\mrec}{m_{\rm rec}}
\newcommand{\erec}{E_{\rm rec}}
%\newcommand{\mrecmin}{m_{\rm rec}^{\rm min}}
%\newcommand{\mrecc}{m_{\rm rec}^{\rm cusp}}
%\newcommand{\mrecmax}{m_{\rm rec}^{\rm max}}
%%%song

\newcommand{\mc}{m^{\rm cusp}}
\newcommand{\mmax}{m^{\rm max}}
\newcommand{\mmin}{m^{\rm min}}
\newcommand{\mll}{m_{\mu\mu}}
\newcommand{\mllc}{m^{\rm cusp}_{\mu\mu}}
\newcommand{\mllmax}{m^{\rm max}_{\mu\mu}}
\newcommand{\mllmin}{m^{\rm min}_{\mu\mu}}
\newcommand{\mllcusp}{m^{\rm cusp}_{\mu\mu}}
\newcommand{\elmax} {E_\mu^{\rm max}}
\newcommand{\elmin} {E_\mu^{\rm min}}
\newcommand{\elmaxw} {E_W^{\rm max}}
\newcommand{\elminw} {E_W^{\rm min}}
\newcommand{\R} {{\cal R}}

\newcommand{\ewmax} {E_W^{\rm max}}
\newcommand{\ewmin} {E_W^{\rm min}}
\newcommand{\mwrec}{m_{WW}}
\newcommand{\mwrecmin}{m_{WW}^{\rm min}}
\newcommand{\mwrecc}{m_{WW}^{\rm cusp}}
\newcommand{\mwrecmax}{m_{WW}^{\rm max}}

\newcommand{\mpt}{{\rlap/p}_T}

%%%%%% END My stuffs - Stef

\newcommand{\dunno}{$ {}^{\mbox {--}}\backslash(^{\rm o}{}\underline{\hspace{0.2cm}}{\rm o})/^{\mbox {--}}$}

\DeclarePairedDelimiter{\ceil}{\lceil}{\rceil}
\DeclarePairedDelimiter{\floor}{\lfloor}{\rfloor}

\DeclareMathOperator{\re}{Re}


\begin{document}

\title{Jeffrey Stopple's Primer of Analytic Number Theory Book}
\bigskip
\author{Stefanus Koesno$^1$\\
$^1$ Somewhere in California\\ San Jose, CA 95134 USA\\
}
%
\date{\today}
%
\begin{abstract}
Very good book, had a lot of fun

\end{abstract}
%
\maketitle

\renewcommand{\theequation}{A.\arabic{equation}}  % redefine the command that creates the equation no.
\setcounter{equation}{0}  % reset counter 

Saw this somewhere, show that
%
\nbea
\sum_{n=1}^\infty \frac{2}{n(n+1)} & = & 1+\frac{1}{3}+\frac{1}{6}+\frac{1}{10}+\ldots = 2
\neea
%
historically this was first proved by Pietro Mengoli in 1650, the real gist of the story is that Mengoli couldn't deduce the value of the sum $\sum_n 1/n^2$ which is similar to the one above (this quadratic harmonic series was finally solved by Euler, neither Leibnitz nor Bernoulli could do it). So back to the original problem, my approach was to split $\frac{2}{n(n+1)}$ into a sum of two fractions
%
\nbea
\frac{2}{n(n+1)} & = & \frac{A}{n}+\frac{B}{n+1} \\
& = & \frac{An+A+Bn}{n(n+1)}
\neea
%
this means that $A = 2$ and $B=-2$ and thus the series can be written as
%
\nbea
\sum_{n=1}^\infty \frac{2}{n(n+1)} & = & \sum_{n=1}^\infty \left(\frac{2}{n} - \frac{2}{n+1}\right) \\
& = & \left(\frac{2}{1} - \frac{2}{2}\right) + \left(\frac{2}{2} - \frac{2}{3}\right) + \left(\frac{2}{3} - \frac{2}{4}\right) + \left(\frac{2}{4} - \right. \ldots \\
& = & \frac{2}{1} + \left( - \frac{2}{2} + \frac{2}{2} \right) + \left( - \frac{2}{3}+ \frac{2}{3}\right) + \left( - \frac{2}{4}+ \frac{2}{4}\right) + \ldots \\
& = & 2
\neea
%
so the series more or less ``telescopes'' and we are left with just the first term although I believe the method above is not legit as we turned an (absolutely?) convergent series into a conditionally convergent series? \dunno ~~~but to then again it might still be ok because if we denote
%
\nbea
H_n & = & \frac{2}{n} -\frac{2}{n+1} \\
\to S(k) & = & \sum_{n=1}^k H_n \\
& = & \frac{2}{1} - \frac{2}{k+1}
\neea
%
and if we take the limit $k\to\infty$
%
\nbea
\lim_{k\to\infty} S(k) & = & \frac{2}{1} - \frac{2}{\infty} \\
& = & 2
\neea
%
the only trick remaining is showing that
%
\nbea
\sum_{n=1}^\infty \frac{2}{n(n+1)} & = & \lim_{k\to\infty} S(k)
\neea
%
which can be shown by noting that if the sum is not infinite the two agree, \ie
%
\nbea
\sum_{n=1}^k \frac{2}{n(n+1)} & = & S(k)
\neea
%
so if we take both sides to infinity we get
%
\nbea
\to \lim_{k\to\infty}\sum_{n=1}^k \frac{2}{n(n+1)} & = & \lim_{k\to\infty} S(k) \\
\sum_{n=1}^\infty \frac{2}{n(n+1)} & = & 2
\neea
%
sounds a lot like a roundabout way of saying the same thing LOL \dunno

-=-=-=-=-=-=-=-=--=-=-=-=-=-=-=-=--=-=-=-=-=-=-=-=--=-=-=-=-=-=-=-=--=-=-=-=-=-=-=-=--=-=-=-=-=

{\bf Exercise 1.2.1}, Page 12, Imitate this argument to get a formula for the hexagonal numbers $h(n)$.

A triangular number, $t_n$ is a number where $t_1=1$ and $t_2 = 3$ and because it's 3 it's a triangular number, the explicit formula is
%
\nbea
t_n & = & \frac{n(n+1)}{2}
\neea
%
which is just $1 + 2 + \ldots + n$.

For the hexagonal number we have $\Delta^2h(n) = 6-2 = 4$ and so $\Delta h(n) = 4n + C$ such that
%
\nbea
\Delta^2h(n) = \Delta(\Delta h(n)) = \Delta(4n+C) = (4(n+1)+C) - (4n+C) = 4
\neea
%
This means that $h(n)$ should contain $Cn+D$ such that $\Delta h(n) = (C(n+1) + D) - (Cn+D) = C$ but this should not be the only term $h(n)$ contains, it should also contain a term whose difference is $4n$ because $\Delta h(n)$ contains $4n$ and that is provided by
%
\nbea
\Delta(4t(n-1)) = 4t(n) - 4t(n-1) = 4\frac{n(n+1)}{2} - 4\frac{n(n-1)}{2} = 4n
\neea
%
so
%
\nbea
h(n) = 4t(n-1) + Cn + D = 2n(n-1) + Cn + D
\neea
%
applying initial conditions $h(1) = 1$ and $h(2) = 6$
%
\nbea
0 + C + D & = & 1 \\
4 + 2C + D & = & 6
\neea
%
giving $C=1$ and $D=0$, so
%
\nbea
h(n) & = & 2n(n-1) + n = n(2n-1)
\neea
%
This seems to be correct since it gives
%
\nbea
\begin{array}{r c c c c c c c c c}
h(n): & 1 & 6 & 15 & 28 & 45 & 66 & 91 & 120 & \ldots, \\
\Delta h(n): & 5 & 9 & 13 & 17 & 21 & 25 & 29 & 33 & \ldots, \\
\Delta^2h(n): & 4 & 4 & 4 & 4 & 4 & 4 & 4 & 4 & \ldots .
\end{array}
\neea
%
and in general $\ldots$

{\bf Exercise 1.2.4}, Page 14, find a formula for polygonal numbers with $a$ sides, for any $a$, \ie a function $f(n)$ with
%
\nbea
\Delta^2f(n) = a - 2, & ~~~~~~~~~~~~ & {\rm with~} f(1) = 1 {\rm ~and~} f(2) = a.
\neea
%

Using the result from Exercise 1.2.1 the generic formula we have is
%
\nbea
f(n) = (a-2)t(n-1) + Cn + D = (a-2)n(n-1)/2 + Cn + D
\neea
%
applying initial conditions
%
\nbea
0 + C + D & = & 1 \\
(a-2) + 2C + D & = & a
\neea
%
which means it's actually independent of $a$, interesting, and therefore $C=1$ and $D=0$ all the time $\ldots$ and therefore
%
\nbea
f(n) & = & n\left\lbrack\left(\frac{a}{2}-1\right)n - \frac{a}{2} + 2\right\rbrack
\neea
%
the good thing about this is that it works for {\bf any} $a$, zero, positive and negative.

{\bf Exercise 1.2.5}, Page 16, verify that
%
\nbea
n^{\underline 1} + 3n^{\underline 2} + n^{\underline 3} & = & n^3
\neea
%
Now use this fact to find formulas for
%
\nbea
\sum_{0\le k <n+1} k^3.
\neea
%

%
\nbea
n^{\underline 1} & = & n\\
3n^{\underline 2} & = & 3n(n-1) \\
& = & 3n^2 - 3n \\
n^{\underline 3} & = & n(n-1)(n-2) \\
& = & n^3 - 3n^2 + 2n
\neea
%
so it's obvious if we sum them we get $n^3$, now we need to ``integrate'' these things
%
\nbea
\Sigma (n^{\underline 1} + 3n^{\underline 2} + n^{\underline 3}) & = & \frac{n^{\underline 2}}{2} + \bcancel{3}\frac{n^{\underline 3}}{\bcancel{3}} + \frac{n^{\underline 4}}{4}
\neea
%
using this in the big sum we get
%
\nbea
\sum_{0\le k <n+1} k^3 & = & \left.\Sigma(n^{\underline 1} + 3n^{\underline 2} + n^{\underline 3})\right|_0^{n+1} \\
& = & \left.\frac{n^{\underline 2}}{2} + n^{\underline 3} + \frac{n^{\underline 4}}{4}\right|_0^{n+1} \\
& = & \frac{(n+1)^{\underline 2}}{2} + (n+1)^{\underline 3} + \frac{(n+1)^{\underline 4}}{4} \\
& = & \frac{(n+1)n}{2} + (n+1)n(n-1) + \frac{(n+1)n(n-1)(n-2)}{4} \\
& = & \frac{n(n+1)\{2 + 4(n-1) + (n-1)(n-2)\}}{4} \\
& = & \frac{n^2(n+1)^2}{4}
\neea
%

{\bf Exercise 1.2.12}, Page 21, Use the fact that if $\Delta f(k) = g(k)$ and $a$ is any constant, then $\Delta (f(k+a)) = g(k+a)$, and the fact that $2(k-1)^{\underline {-2}} = 1/t_k$ to find the sum of the reciprocals of the first $n$ triangular numbers
%
\nbea
\frac{1}{t_1} + \frac{1}{t_2} + \ldots + \frac{1}{t_n}.
\neea
%
Next compute
%
\nbea
\frac{1}{T_1} + \frac{1}{T_2} + \ldots + \frac{1}{T_n},
\neea
%
the sum of the reciprocals of the first $n$ tetrahedral numbers.

So what we want is
%
\nbea
\sum_{1\le k < n+1} \frac{1}{t_k} & = & \left.\Sigma \frac{1}{t_k} \right |_{1}^{n+1} \\
& = & \left.\Sigma 2(k-1)^{\underline{-2}} \right |_{1}^{n+1} \\
& = & \left.\Sigma \frac{2(k-1)^{\underline{-1}}}{-2+1} \right |_{1}^{n+1} \\
& = & -2 \left \lbrack (n)^{\underline{-1}} - 0^{\underline{-1}}\right\rbrack \\
& = & -2\left\lbrack\frac{1}{(n+1)} - \frac{1}{0+1}\right\rbrack \\
& = & \frac{2n}{n+1}
\neea
%
the thing to note here is that $0^{\underline{-1}}$ is {\bf not} zero it is actually $1/(0+1) = 1$. For the tetrahedral numbers we know that it is given in Eq. 1.4 and 1.5
%
\nbea
T_n & = & \frac{n(n+1)(n+2)}{6} \\
\to \frac{1}{T_n} & = & 6~\frac{1}{n(n+1)(n+2)}\\
& = & 6 (n-1)^{\underline{-3}}
\neea
%
so the sum is also pretty much the same
%
\nbea
\sum_{1\le k < n+1} \frac{1}{T_k} & = & \left.\Sigma \frac{1}{T_k} \right |_{1}^{n+1} \\
& = & \left.\Sigma 6(k-1)^{\underline{-3}} \right |_{1}^{n+1} \\
& = & \left.\Sigma \frac{6(k-1)^{\underline{-2}}}{-3+1} \right |_{1}^{n+1} \\
& = & -3 \left \lbrack (n)^{\underline{-2}} - 0^{\underline{-2}}\right\rbrack \\
& = & -3\left\lbrack\frac{1}{(n+1)(n+2)} - \frac{1}{(0+1)(0+2)}\right\rbrack \\
& = & \frac{3(n^2+3n)}{2(n+1)(n+2)}
\neea
%

{\bf Exercise 1.2.13}, Page 23, Use Summation by Parts and the Fundamental Theorem to compute $\sum_{0\le k<n}H_k$. (Hint: You can write $H_k = H_k \cdot 1 = H_k\cdot k^{\underline 0}$.) Your answer will have Harmonic numbers in it of course.

%
\nbea
\Sigma H_k & = & \Sigma (H_k \cdot 1) = \Sigma (H_k\cdot k^{\underline 0})
\neea
%
I don't know how to integrate $H_k$ (since that is the question we are trying to answer here) but I know how to differentiate it, the derivative is just $k^{\underline{-1}}$ so we'll take $u = H_k$ such that $\Delta u = k^{\underline{-1}}$ and so we'll take $\Delta v = k^{\underline 0}$ such that $v = k^{\underline 1}$.

Next we need $\Sigma(\Delta u\cdot Ev)$
%
\nbea
\Sigma(\Delta u\cdot Ev) & = & \Sigma(k^{\underline{-1}} \cdot (k+1)^{\underline 1}) \\
& = & \Sigma\left(\frac{1}{\bcancel{(k+1)}}\cdot \bcancel{(k+1)}k \right) \\
& = & \frac{k^{\underline 2}}{2}
\neea
%
so we have
%
\nbea
\Sigma (u\cdot\Delta v) & = & uv - \Sigma(\Delta u\cdot Ev) \\
\to \Sigma (H_k\cdot k^{\underline 0}) & = & H_k k^{\underline 1} - \frac{k^{\underline 2}}{2}
\neea
%
to verify let's differentiate it
%
\nbea
\Delta\left(H_k k^{\underline 1} - \frac{k^{\underline 2}}{2}\right) & = & k^{\underline {-1}}(k+1)^{\underline 1} + H_k k^{\underline 0} - k^{\underline 1} \\
& = & \frac{(k+1)k}{k+1} + H_k - k \\
& = & H_k
\neea
%
note that the product rule here is different, there's a shift involved, see Page 22, and therefore
%
\nbea
\sum_{0 \le k < n} H_k\cdot k^{\underline 0} & = & \left.\Sigma(u\cdot\Delta v)\right|_0^{n} \\
& = & \left. \left( H_k k^{\underline 1} - \frac{k^{\underline 2}}{2}\right)\right |_0^n\\
& = & n\left(H_n - \frac{n-1}{2}\right)
\neea
%

{\bf Exercise 1.2.14}, Page 23, Use Summation by Parts and the Fundamental Theorem to compute $\sum_{0\le k <n}k2^k$. (Hint: You need the first part of Exercise 1.2.9.)

The thing to note here is that $2^k$ is like $e$ its derivative is itself
%
\nbea
\Delta 2^k & = & 2^{k+1} - 2^k \\
& = & 2^k
\neea
%
so we will want this to be $\Delta v = 2^k$ such that $v = 2^k$ and $u = k = k^{\underline 1}$ such that $\Delta u = k^{\underline 0} = 1$ and so
%
\nbea
\Sigma(u\cdot \Delta v) & = & uv - \Sigma(\Delta u\cdot Ev) \\
& = & k^{\underline 1} 2^k - \Sigma\left(k^{\underline 1} 2^{k+1}\right) \\
& = & k2^k - \Sigma(2^{k+1}) \\
& = & k2^k - 2^{k+1} \\
& = & 2^k(k - 2)
\neea
%
to verify let's differentiate the above
%
\nbea
\Delta\left\{ 2^k(k - 2) \right\} & = & 2^k ((k+1)-2) + 2^k \\
& = & k2^k
\neea
%
so it's correct and again the thing to note is that the product rule here now involves a shift, moving on
%
\nbea
\sum_{0\le k <n} k2^k & = & \left. \left( 2^k(k - 2) \right) \right |_0^n \\
& = & 2^n(n-2) - (-2) \\
& = & 2^n(n-2) + 2
\neea
%

-=-=-=-=-=-=-=-=--=-=-=-=-=-=-=-=--=-=-=-=-=-=-=-=--=-=-=-=-=-=-=-=--=-=-=-=-=-=-=-=--=-=-=-=-=

{\bf Exercise 2.1.7}, Page 30, we know that $6=2^1\cdot3$ and $28=2^2\cdot 7$ are perfect numbers. The next ones are $496=2^4\cdot31$ and $8128=2^6\cdot127$ (check this!). What is the pattern in $3,7,31,127,\ldots$ ? What is the pattern in the exponents $1,2,4,6,\ldots$ ? Try to make a conjecture about perfect numbers. Euclid, in his {\it Elements}, proved a general theorem about perfect numbers around the year 300 B.C.

%
\nbea
3 & = & 2^2 - 1 \\
7 & = & 2^3 - 1 \\
31 & = & 2^5 - 1\\
127 & = & 2^7 - 1
\neea
%
so the exponents are $2,3,5,7$ which are all primes and are all one more that the $1,2,4,6$ so it looks like the formula for even perfect numbers are
%
\nbea
2^{p-1}\cdot (2^p - 1)
\neea
%
where $p$ is prime.

{\bf Exercise 2.1.8}, Page 30, \#2 and \#3 are the interesting ones since the exponents of the $2^p-1$ are 11 and 13 and they are both prime so $2^{10}(2^{11}-1)$ and $2^{12}(2^{13}-1)$ are also prime, let's start with \#2

%
\nbea
2096128 & = & 2^{10}(2^11-1) = 2^{10}\cdot 23 \cdot 89 \\
\to \sigma(2096128) & = & \sigma(2^{10}) \sigma(23) \sigma(89) \\
& = & (2^11-1) (23+1)(89+1) \\
& = & 4421520 \\
\to s(2096128) & = & \sigma(2096128) - 2096128  = 2325392\\
& \neq & 2096128
\neea
%
so even though 11, the exponent of $2^{11}-1$ is prime, $2^{10}(2^{11}-1)$ is not perfect, while
%
\nbea
33550336 & = & 2^{12}(2^{13}-1) = 2^{12} \cdot 8191 \\
\to \sigma(33550336) & = & \sigma(2^{12})\sigma(8191) \\
& = & (2^{13}-1)(8191+1) \\
& = & 8191\cdot(8191+1) \\
& = & 67100672 \\
\to s(33550336) & = & \sigma(33550336) - 33550336 = 33550336
\neea
%
so it is a perfect number, thus the formula has to be correctec to only include $2^p-1$ that is prime (and $p$ itself is also prime)

{\bf Exercise 2.1.12}, Page 31, there is an interesting theorem about perfect numbers and sums of cubes. For example,
%
\nbea
28 = 1^3 + 3^3
\neea
%
Try to make a conjecture about what is true. Ignore the first perfect number, 6; it doesn't fit the pattern.
%
\nbea
28 & = & 1^3 + 3^3\\
496 & = & 1^3 + 3^3 + 5^3 + 7^3\\
8128 & = & 1^3 + 3^3 + 5^3 + 7^3 + 9^3 + 11^3 + 13^3 + 15^3
\neea
%
the pattern is given by the next exercise

{\bf Exercise 2.1.12}, Page 31,
%
\nbea
1^3 + 3^3 + 5^3 + \ldots + (2N-1)^3 & = & \left(1^3 + 2^3 + 3^3 + \ldots + (2N)^3\right) - \left(2^3+4^3+6^3 + \ldots+(2N)^3\right) \\
& = & \frac{(2N)^2(2N+1)^2}{4} - 2^3\left(1^3 + 2^3 + 3^3 + \ldots+N^3\right) \\
& = & \frac{4N^2(2N+1)^2}{4} - 2^3 \frac{N^2(N+1)^2}{4} \\
& = & N^2(2N+1)^2 - 2N^2(N+1)^2 \\
& = & N^2 \left( 4N^2 + 4N + 1 - 2N^2 - 4N - 2\right) \\
& = & N^2(2N^2-1)
\neea
%
so $N=2^{(p-1)/2}$ where $p$ is prime and $2^p-1$ is also prime

{\bf Exercise 2.1.16}, Page 33, Suppose $m$ and $n$ are integers such that $\sigma(m) = m + n = \sigma(n)$. What can you say is true about $s(m)$ and $s(n)$?
%
\nbea
s(m) & = & \sigma(m) - m \\
& = & m+n-m \\
s(m)& = & n
\neea
%
and
%
\nbea
s(n) & = & \sigma(n) - n\\
& = & m+n-n \\
s(n)& = & m
\neea
%
so a pair of amicable numbers $m$ and $n$ are those such that $s(m)=n$ and $s(n) = m$ and therefore $\sigma(m)=\sigma(n) = m+n$.

\bigskip
\underline{\textbf{\textit{Chapter 3}}}
\bigskip

Lessons from Chapter 3
%
\bit
\item Area is equal to length if the width is one, this is the trick used over and over, once you morphed length into area you can use integrals
\eit

\bigskip
\underline{\textbf{\textit{Chapter 4}}}
\bigskip

His notion of average order is slightly different than that of Apostol's

{\bf Exercise 4.2.4}, Page 76, the differences are
%
\nbea
\frac{1}{3} - \frac{27}{82} & = & 0.0040650406504065040650406504065 \\
\frac{27}{82} - \frac{15}{46} & = & 0.00318133616118769883351007423118 \\
\frac{15}{46} - \frac{12}{37} & = & 0.00176263219741480611045828437133
\neea
%
very small indeed, roughly one part in a thousand

{\bf Page 80}, {\bf Corollary}. {\it The average order of $\sigma(n)$ is $\zeta(2)n$.} Note that here he's talking about average order quite differently from Apostol, his reasoning here is since
%
\nbea
\sum_{k=1}^n \sigma(k) &\sim& \zeta(2)\frac{n^2}{2} \\
\sum_{k=1}^n \zeta(2) k &\sim& \zeta(2)\frac{n^2}{2} \\
\to \sum_{k=1}^n \sigma(k) &\sim& \sum_{k=1}^n \zeta(2) k
\neea
%
so since both sides are summed exactly the same each term as the same average order? \dunno

\bigskip
\underline{\textbf{\textit{Interlude 1}}}
\bigskip

{\bf Exercise I1.3.4}, Page 95, from the {\it definition} of $\log(x)$ as an integral over $1/t$
%
\nbea
\log(xy) & = & \int_1^{xy} \frac{1}{t} dt \\
& = & \int_1^x \frac{1}{t} dt + \int_x^{xy} \frac{1}{t} dt \\
& = & \log(x) + \int_x^{xy} \frac{1}{t} dt
\neea
%
The second term in the RHS can be massaged using the usual change of variable $s = xt$ such that $t = s/x$ and $dt = ds/x$ and so
%
\nbea
\int_x^{xy} \frac{1}{t} dt & = & \int_1^{y} \frac{\bcancel{x}}{s} \frac{ds}{\bcancel{x}} \\
& = & \log(y)
\neea
%
and so $\log(xy) = \log(x) + \log(y)$, we can actually do the change of variable in a slightly different way
%
\nbea
\int_x^{xy} \frac{1}{t} dt & = & \int_1^{y} \frac{x}{x}\times\frac{1}{t} dt \\
& = & \int_x^{xy} \frac{1}{(xt)} d(xt), ~~~~~ s=xt \\
& = & \int_1^{y} \frac{1}{s} ds \\
\neea
%
it was OK to do that because here $x$ is constant w.r.t the integration. A more interesting thing would be $\log(x/y)$ in this case
%
\nbea
\log(x/y) & = & \int_1^{x/y} \frac{1}{t} dt \\
& = & \int_1^{x} \frac{1}{t} dt - \int_{x/y}^{x} \frac{1}{t} dt
\neea
%
here is a subtraction because $x/y < x$, the next steps are the same as previous case's.

\bigskip
\underline{\textbf{\textit{Chapter 5}}}
\bigskip

{\bf Page 107}, lower half of page, nearing the funny trick to prove the Theorem we have
%
\nbea
\pi(2n+1) & < & 2\frac{2n+1}{\log(2n)} \\
& < & 2\frac{2n+1}{\log(2n+1)}\frac{\log(2n+1)}{\log(2n)}
\neea
%
and for $n=30$, $\frac{\log(2n+1)}{\log(2n)} = 1.00403710574428$ so we can just adjust the constant 2 to $2.008\ldots$ or something :) and we can certainly do approximation this way as well
%
\nbea
\frac{\log(2n+1)}{\log(2n)} & = & \frac{\log(2n)+\log(1+1/2n)}{\log(2n)} \\
& \approx & 1
\neea
%
But actually if you numerically calculate $3.39 \frac{n}{\log(n)}+1$ and $2\frac{2n+1}{\log(2n+1)}$ starting at $n=67$, $3.39 \frac{n}{\log(n)}+1 < 2\frac{2n+1}{\log(2n+1)}$ and we can sort of prove this. Let's do a change of variables, $y=\log(2n+1)$, such that $2n+1 = e^y$ and $n = (e^y-1)/2$
%
\nbea
\neea
%

%
\nbea
4\log^2(n)(\log(2n+1)-1) - 3.39 \log^2(2n+1)(\log(n)-1) \\
4\log^2(n)\log(2n+1)+3.39 \log^2(2n+1) - 4\log^2(n) - 3.39 \log^2(2n+1)\log(n)
\neea
%


{\bf Page 108}, Theorem that says
%
\nbea
\frac{1}{2}\frac{x}{\log(x)} < \pi(x)
\neea
%
for $x\ge15$. In fact we can say that this is true for $x\ge3$, 15 numbers are not too much to check so if you just calculate all 15 of them you'll see that this inequality holds for $x\ge3$ \dunno

{\bf Page 108}, Lemma, this is actually Theorem 3.14 of Apostol, so here's I replicate my own explanation of why this is true, visually we can see it below with an example for  $\alpha(p)$ with $x=10$ and $p=2$, each column shows the number of 2's in each number (we only consider even numbers here obviously)
%
\nbea
\begin{array}{c c c c c l}
   &    &   & 2  &  & \longrightarrow x/p^3 = 10/2^3 = 1 {\rm~factor~of~} 2\\
   & 2 &    & 2 &   & \longrightarrow x/p^2 = 10/2^2 = 2 {\rm~factor~of~} 2\\
2 & 2 & 2 & 2 & 2 & \longrightarrow x/p^1 = 10/2^1 = 5 {\rm~factor~of~} 2\\
\uparrow & \uparrow & \uparrow & \uparrow & \uparrow \\
2 & 4 & 6 & 8 & 10 & \longrightarrow {\rm a~total~of~}5+2+1=8{\rm~factors~of~}2
\end{array}
\neea
%

























\end{document}