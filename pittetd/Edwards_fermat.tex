\documentclass[aps,preprint,preprintnumbers,nofootinbib,showpacs,prd]{revtex4-1}
\usepackage{graphicx,color}
\usepackage{caption}
\usepackage{subcaption}
\usepackage{amsmath,amssymb}
\usepackage{multirow}
\usepackage{amsthm}%        But you can't use \usewithpatch for several packages as in this line. The search 

\usepackage{cancel}

%%% for SLE
\usepackage{dcolumn}   % needed for some tables
\usepackage{bm}        % for math
\usepackage{amssymb}   % for math
\usepackage{multirow}
%%% for SLE -End

\usepackage{ulem}
\usepackage{cancel}

\usepackage{hyperref}
\usepackage{mathrsfs}
\usepackage[top=1in, bottom=1.25in, left=1.1in, right=1.1in]{geometry}

\usepackage{mathtools} % for \DeclarePairedDelimiter{\ceil}{\lceil}{\rceil}

%\usepackage{xeCJK}
%\setCJKmainfont{SimSun}

\newcommand{\msout}[1]{\text{\sout{\ensuremath{#1}}}}


%%%%%% My stuffs - Stef
\newcommand{\lsim}{\mathrel{\mathop{\kern 0pt \rlap
  {\raise.2ex\hbox{$<$}}}
  \lower.9ex\hbox{\kern-.190em $\sim$}}}
\newcommand{\gsim}{\mathrel{\mathop{\kern 0pt \rlap
  {\raise.2ex\hbox{$>$}}}
  \lower.9ex\hbox{\kern-.190em $\sim$}}}

%
% Key
%
\newcommand{\key}[1]{\medskip{\sffamily\bfseries\color{blue}#1}\par\medskip}
%\newcommand{\key}[1]{}
\newcommand{\q}[1] {\medskip{\sffamily\bfseries\color{red}#1}\par\medskip}
\newcommand{\comment}[2]{{\color{red}{{\bf #1:}  #2}}}


\newcommand{\ie}{{\it i.e.} }
\newcommand{\eg}{{\it e.g.} }

%
% Energy scales
%
\newcommand{\ev}{{\,{\rm eV}}}
\newcommand{\kev}{{\,{\rm keV}}}
\newcommand{\mev}{{\,{\rm MeV}}}
\newcommand{\gev}{{\,{\rm GeV}}}
\newcommand{\tev}{{\,{\rm TeV}}}
\newcommand{\fb}{{\,{\rm fb}}}
\newcommand{\ifb}{{\,{\rm fb}^{-1}}}

%
% SUSY notations
%
\newcommand{\neu}{\tilde{\chi}^0}
\newcommand{\neuo}{{\tilde{\chi}^0_1}}
\newcommand{\neut}{{\tilde{\chi}^0_2}}
\newcommand{\cha}{{\tilde{\chi}^\pm}}
\newcommand{\chao}{{\tilde{\chi}^\pm_1}}
\newcommand{\chaop}{{\tilde{\chi}^+_1}}
\newcommand{\chaom}{{\tilde{\chi}^-_1}}
\newcommand{\Wpm}{W^\pm}
\newcommand{\chat}{{\tilde{\chi}^\pm_2}}
\newcommand{\smu}{{\tilde{\mu}}}
\newcommand{\smur}{\tilde{\mu}_R}
\newcommand{\smul}{\tilde{\mu}_L}
\newcommand{\sel}{{\tilde{e}}}
\newcommand{\selr}{\tilde{e}_R}
\newcommand{\sell}{\tilde{e}_L}
\newcommand{\smurl}{\tilde{\mu}_{R,L}}

\newcommand{\casea}{\texttt{IA}}
\newcommand{\caseb}{\texttt{IB}}
\newcommand{\casec}{\texttt{II}}

\newcommand{\caseasix}{\texttt{IA-6}}

%
% Greek
%
\newcommand{\es}{{\epsilon}}
\newcommand{\sg}{{\sigma}}
\newcommand{\dt}{{\delta}}
\newcommand{\kp}{{\kappa}}
\newcommand{\lm}{{\lambda}}
\newcommand{\Lm}{{\Lambda}}
\newcommand{\gm}{{\gamma}}
\newcommand{\mn}{{\mu\nu}}
\newcommand{\Gm}{{\Gamma}}
\newcommand{\tho}{{\theta_1}}
\newcommand{\tht}{{\theta_2}}
\newcommand{\lmo}{{\lambda_1}}
\newcommand{\lmt}{{\lambda_2}}
%
% LaTeX equations
%
\newcommand{\beq}{\begin{equation}}
\newcommand{\eeq}{\end{equation}}
\newcommand{\bea}{\begin{eqnarray}}
\newcommand{\eea}{\end{eqnarray}}
\newcommand{\ba}{\begin{array}}
\newcommand{\ea}{\end{array}}
\newcommand{\bit}{\begin{itemize}}
\newcommand{\eit}{\end{itemize}}

\newcommand{\nbea}{\begin{eqnarray*}}
\newcommand{\neea}{\end{eqnarray*}}
\newcommand{\nbeq}{\begin{equation*}}
\newcommand{\neeq}{\end{equation*}}

\newcommand{\no}{{\nonumber}}
\newcommand{\td}[1]{{\widetilde{#1}}}
\newcommand{\sqt}{{\sqrt{2}}}
%
\newcommand{\me}{{\rlap/\!E}}
\newcommand{\met}{{\rlap/\!E_T}}
\newcommand{\rdmu}{{\partial^\mu}}
\newcommand{\gmm}{{\gamma^\mu}}
\newcommand{\gmb}{{\gamma^\beta}}
\newcommand{\gma}{{\gamma^\alpha}}
\newcommand{\gmn}{{\gamma^\nu}}
\newcommand{\gmf}{{\gamma^5}}
%
% Roman expressions
%
\newcommand{\br}{{\rm Br}}
\newcommand{\sign}{{\rm sign}}
\newcommand{\Lg}{{\mathcal{L}}}
\newcommand{\M}{{\mathcal{M}}}
\newcommand{\tr}{{\rm Tr}}

\newcommand{\msq}{{\overline{|\mathcal{M}|^2}}}

%
% kinematic variables
%
%\newcommand{\mc}{m^{\rm cusp}}
%\newcommand{\mmax}{m^{\rm max}}
%\newcommand{\mmin}{m^{\rm min}}
%\newcommand{\mll}{m_{\ell\ell}}
%\newcommand{\mllc}{m^{\rm cusp}_{\ell\ell}}
%\newcommand{\mllmax}{m^{\rm max}_{\ell\ell}}
%\newcommand{\mllmin}{m^{\rm min}_{\ell\ell}}
%\newcommand{\elmax} {E_\ell^{\rm max}}
%\newcommand{\elmin} {E_\ell^{\rm min}}
\newcommand{\mxx}{m_{\chi\chi}}
\newcommand{\mrec}{m_{\rm rec}}
\newcommand{\mrecmin}{m_{\rm rec}^{\rm min}}
\newcommand{\mrecc}{m_{\rm rec}^{\rm cusp}}
\newcommand{\mrecmax}{m_{\rm rec}^{\rm max}}
%\newcommand{\mpt}{\rlap/p_T}

%%%song
\newcommand{\cosmax}{|\cos\Theta|_{\rm max} }
\newcommand{\maa}{m_{aa}}
\newcommand{\maac}{m^{\rm cusp}_{aa}}
\newcommand{\maamax}{m^{\rm max}_{aa}}
\newcommand{\maamin}{m^{\rm min}_{aa}}
\newcommand{\eamax} {E_a^{\rm max}}
\newcommand{\eamin} {E_a^{\rm min}}
\newcommand{\eaamax} {E_{aa}^{\rm max}}
\newcommand{\eaacusp} {E_{aa}^{\rm cusp}}
\newcommand{\eaamin} {E_{aa}^{\rm min}}
\newcommand{\exxmax} {E_{\neuo \neuo}^{\rm max}}
\newcommand{\exxcusp} {E_{\neuo \neuo}^{\rm cusp}}
\newcommand{\exxmin} {E_{\neuo \neuo}^{\rm min}}
%\newcommand{\mxx}{m_{XX}}
%\newcommand{\mrec}{m_{\rm rec}}
\newcommand{\erec}{E_{\rm rec}}
%\newcommand{\mrecmin}{m_{\rm rec}^{\rm min}}
%\newcommand{\mrecc}{m_{\rm rec}^{\rm cusp}}
%\newcommand{\mrecmax}{m_{\rm rec}^{\rm max}}
%%%song

\newcommand{\mc}{m^{\rm cusp}}
\newcommand{\mmax}{m^{\rm max}}
\newcommand{\mmin}{m^{\rm min}}
\newcommand{\mll}{m_{\mu\mu}}
\newcommand{\mllc}{m^{\rm cusp}_{\mu\mu}}
\newcommand{\mllmax}{m^{\rm max}_{\mu\mu}}
\newcommand{\mllmin}{m^{\rm min}_{\mu\mu}}
\newcommand{\mllcusp}{m^{\rm cusp}_{\mu\mu}}
\newcommand{\elmax} {E_\mu^{\rm max}}
\newcommand{\elmin} {E_\mu^{\rm min}}
\newcommand{\elmaxw} {E_W^{\rm max}}
\newcommand{\elminw} {E_W^{\rm min}}
\newcommand{\R} {{\cal R}}

\newcommand{\ewmax} {E_W^{\rm max}}
\newcommand{\ewmin} {E_W^{\rm min}}
\newcommand{\mwrec}{m_{WW}}
\newcommand{\mwrecmin}{m_{WW}^{\rm min}}
\newcommand{\mwrecc}{m_{WW}^{\rm cusp}}
\newcommand{\mwrecmax}{m_{WW}^{\rm max}}

\newcommand{\mpt}{{\rlap/p}_T}

%%%%%% END My stuffs - Stef

\newcommand{\dunno}{$ {}^{\mbox {--}}\backslash(^{\rm o}{}\underline{\hspace{0.2cm}}{\rm o})/^{\mbox {--}}$}

\DeclarePairedDelimiter{\ceil}{\lceil}{\rceil}
\DeclarePairedDelimiter{\floor}{\lfloor}{\rfloor}

\DeclareMathOperator{\re}{Re}


\begin{document}

\title{Harold Edwards Fermat's last theorem}
\bigskip
\author{Stefanus Koesno$^1$\\
$^1$ Somewhere in California\\ San Jose, CA 95134 USA\\
}
%
\date{\today}
%
\begin{abstract}

\end{abstract}
%
\maketitle

\renewcommand{\theequation}{A.\arabic{equation}}  % redefine the command that creates the equation no.
\setcounter{equation}{0}  % reset counter 

{\bf Page 8, Problem 3}. Show that if $d^2|z^2$ then $d|z$, start with $\gcd(d,z) = c$
%
\nbea
d^2 k & = & z^2 \\
c^2 D^2 k & = & c^2 Z^2 \\
D^2 k & = & Z^2
\neea
%
we know that $\gcd(D,Z) = 1$ and so $\gcd(D^2,Z^2) = 1$, since $k|Z^2$, if $\gcd(D^2,k) > 1$ then $\gcd(D^2,Z^2) > 1$ as well, therefore $\gcd(D^2,k) = 1$, this means that $k = K^2$ by our assumption that if $vw = u^2$ and $v$ and $w$ are co-prime then both $v$ and $w$ are squares.

Substituting $k=K^2$ back into our first equation
%
\nbea
d^2K^2 & = & z^2 \\
\to dK & = & z
\neea
%
thus $d|z$ although I'm not sure about this particular line of reasoning, since $\gcd(D^2,Z^2) = 1$ and $D^2|Z^2$ then $D^2 = 1$ and we immediately get $k=Z^2$ and we don't need our assumption about $vw=u^2$ at all. But I'm not sure how else to show that $\gcd(D^2,k) = 1$ except by showing that $\gcd(D^2,Z^2) = 1$.

Now, the one step I stil need to prove, is that $\gcd(D,Z) = 1$ means $\gcd(D^2,Z^2) = 1$, again, without using the fundamental theorem. What I need is Bezout, assume that $\gcd(D^2,Z^2) = g > 1$ then Bezout tells us that
%
\nbea
D^2 a + Z^2 b & = & g
\neea
%
but from $\gcd(D,Z)=1$ we also get
%
\nbea
D A + Z B & = & 1 \\
\to DgA + ZgB & = & g
\neea
%
Now there might be some other numbers such that
%
\nbea
D M + Z N & = & g
\neea
%
but this means either that $\gcd(M,N) = g$ or $\gcd(M,N) = y | g$, let's dicuss the latter first
%
\nbea
D y M' + Z y N' & = & yg' \\
D M' + Z N' & = & g'
\neea
%
with $\gcd(M',N') = 1$ but this means that $\gcd(D,Z) = g'$, the only way this works is that $g' = 1$ and $\gcd(M,N) = g$, but if this is the case then
%
\nbea
D M' + Z N' & = & 1
\neea
%
but Bezout also tells us that all solutions to $D A' + Z B' = 1$ are of the form see ent.pdf Problem 2.5
%
\nbea
A' & = & A + l D \\
B' & = & B - lD
\neea
%
Equating the two Bezouts $g = D^2 a + Z^2 b = DgA' + ZgB'$
%
\nbea
\to gA' & = & gA + gl D = Da\\
\to gB' & = & gB - gl Z = Zb 
\neea
%
Now, equating the two Bezouts again
%
\nbea
D^2 a + Z^2 b & = & DgA + ZgB \\
D(Da - gA) & = & Z(gB - Zb) \\
\to D(glD) & = & Z(glZ) \\
D^2 & = & Z^2
\neea
%
which is a contradiction, therefore if $\gcd(D,Z) = 1$ then $\gcd(D^2,Z^2) = 1$ as well. The above proof is easily generalizable to $\gcd(D^n,Z^m)$
%
\nbea
\to gA' & = & gA + gl D = D^{n-1}a\\
\to gB' & = & gB - gl Z = Z^{m-1}b 
\neea
%
Now, equating the two Bezouts again
%
\nbea
D^n a + Z^m b & = & DgA + ZgB \\
D(D^{n-1}a - gA) & = & Z(gB - Z^{m-1}b) \\
\to D(glD) & = & Z(glZ) \\
D^2 & = & Z^2
\neea
%

{\bf Page 14, Problem 1}. Prove that if $Ad^2$ is a square then $A$ is a square, using the result from previous Problem, say $Ad^2 = z^2$ since $d^2|z^2$ this also means that $d|z$ so we can write $z = dk$, therefore
%
\nbea
Ad^2 & = & z^2 = d^2k^2 \\
A & = & k^2
\neea
%
and we are done.

{\bf Page 14, Problem 2}. Show that $x^4 - y^4 = z^2$ has no non-zero integer solutions. One thing we should not do is to blindly apply the Pythagorean formula ($m^2-n^2, 2mn, m^2+n^2$) over and over again, what we should do is follow what was done in the preceding section.

In that section we have $p,q,$ and $p^2 - q^2$ are all squares due to $t^2 = pq(p^2-q^2)$ so if we designate
%
\nbea
p & = & x^2 \\
q & = & y^2 \\
p^2 - q^2 & = & z^2 \\
\to x^4 - y^4 & = & z^2
\neea
%
and we have our current problem. Following what was done in the book
%
\nbea
z^2 & = & p^2 - q^2 \\
& = & (p-q)(p+q)
\neea
%
since $p$ and $q$ are co-prime so are $p-q$ and $p+q$. From $x^4 - y^4 = z^2$ we know that $x$ and thus $p$ is odd but here we can have $y$ even or odd, for simplicity let's start with $y$ and thus $q$ even so that we have the case in the book, so
%
\nbea
p + q = r^2  ~~~~~~~~~~~~~~~~ p - q = s^2
\neea
%
with both $r$ and $s$ odd and co-prime and
%
\nbea
u = \frac{r - s}{2}   ~~~~~~~~~~~~~~~~ v = \frac{r + s}{2}
\neea
%
with $u$ and $v$ integers and co-prime and so
%
\nbea
uv & = & \frac{r^2 - s^2}{4} = \frac{(p+q)-(p-q)}{4} = \frac{q}{2} = \frac{y^2}{2}
\neea
%
since $uv$ integer $y^2/2$ must also be an integer, thus $y = 2k$, $y^2/2 = 2k^2$ therefore
%
\nbea
\frac{uv}{2} & = & \frac{y^2}{4} = k^2 \\
\to uv & = & 2k^2
\neea
%
since $u$ and $v$ are co-prime this means that one of them is even and the other odd, let's take $u$ odd and $v=2v'$ even, then $u(2v') = 2k^2$ and therefore
%
\nbea
u = U^2   ~~~~~~~~~~~~~~~~ v = 2 V^2
\neea
%
since $u$ and $v$ are coprime. Thus
%
\nbea
r & = & u + v = U^2 + 2V^2
\neea
%
and 
%
\nbea
u^2 + v^2 & = & \frac{(r-s)^2 + (r+s)^2}{4} \\
& = & \frac{2r^2 + 2s^2}{4} = \frac{r^2 + s^2}{2} \\
& = & \frac{(p+q) + (p-q)}{2} = \frac{2p}{2} \\
& = & p \\
u^2 + v^2 & = & x^2
\neea
%
Thus we have a primitive triple $u,v,x$ (because $u,v$ are co-prime), thus we have $P^2-Q^2,2PQ,P^2+Q^2$ (note that above we have designated $u$ as the odd one) and so
%
\nbea
\frac{uv}{2} = k^2 = (P^2 - Q^2)PQ
\neea
%
so we have the same situation as $t^2 = pq(p^2-q^2)$ but $uv/2 = q/4 < t^2$ and our infinite descent begins.

The above was when $q$ even such that $p-q$ and $p+q$ are odd. Now we deal with the case of $q$ odd such that $p-q = 2r^2$ and $p+q = 2s^2$ are both even, this is because now $p-q$ and $p+q$ are no longer co-prime so we cannot follow the same steps above.

What we have is (from the pythagorean triple formula)
%
\nbea
p = x^2 & = & m^2 + n ^2 \\
q = y^2 & = & m^2 - n^2
\neea
%
thus we can use them directly, $m^2$ plays the role of $p$ and $n^2$ play the role of $q$ as they are already co-prime and of opposite parities and of course $x$ plays the role of $p+q$ and $y$, $p-q$. Thus in this case
%
\nbea
u = \frac{x-y}{2} ~~~~~~~~~~~~~~~~ v = \frac{x+y}{2}
\neea
%
Thus, just like above
%
\nbea
uv = \frac{x^2 - y^2}{4} = \frac{n^2}{2} ~~~~~~~~~~~~~~~~ u = U^2 ~~~~~~ v = 2V^2
\neea
%
and therefore
%
\nbea
u^2 + v^2 = m^2
\neea
%
Thus we have our infinite descent all over again.












{\bf page 25, Problem 1}. Prove that $2^{37}-1$ is not prime














{\bf Page 25, Problem 2}. If $p = 4n+3$ divides $x^2 + y^2$ then
%
\nbea
(x^2)^{2n+1} + (y^2)^{2n+1} & = & \left\lbrack (x^2) + (y^2) \right\rbrack \left\lbrack (x^2)^{2n} - (x^2)^{2n-1}(y^2)  +(x^2)^{2n-2}(y^2)^2 \cdots + (y^2)^{2n} \right\rbrack
\neea
%
and hence $p|(x^2)^{2n+1} + (y^2)^{2n+1}$ as well. But
%
\nbea
(x^2)^{2n+1} & = & x^{4n+2} = x^{p-1}
\neea
%
and so if $p\nmid x$ and $p\nmid y$ then because $p|x^{p-1} - 1$ and $p|y^{p-1} - 1$ we have
%
\nbea
(x^2)^{2n+1} + (y^2)^{2n+1} & = & (x^{p-1} - 1) + (y^{p-1} - 1) + 2 \\
& = & pm_x + pm_y + 2 \\
& = & pm_{xy} + 2
\neea
%
therefore we have a contradiction as now $p$ no longer divides $(x^2)^{2n+1} + (y^2)^{2n+1}$ as it differs from a multiple of $p$ by 2. But if one of them, either $x$ or $y$ is divisible by $p$ then (for simplicity let's assume it's $x$)
%
\nbea
(x^2)^{2n+1} + (y^2)^{2n+1} & = & pm_x + (y^{p-1} - 1) + 1 \\
& = & pm_x + pm_y + 1 \\
& = & pm_{xy} + 1
\neea
%
and this time it differs from a multiple of $p$ by 1.

{\bf Page 33, Problem 2}. This is quite a fun one to do, let's do the one for $A = 13$, we start with
%
\nbea
1^2 - A \cdot 0^2 & = & 1
\neea
%
multiplying it with $r^2 - A = s$ we get
%
\nbea
r^2 - A(1+r)^2 & = & 1\cdot s
\neea
%
here $k=1$, so now we need to find an $r$ such that $r^2 < A$ but $r^2 - A$ is a negative number, here since $k=1$ we do not need to care if $k|(1+r)$ or not. The answer is $r=3$ such that $s = r^2 - A = -4$ and our next equation is
%
\nbea
3^2 - A \cdot 1^2 = -4
\neea
%
multiplying it by $r^2 - A = s$ we get
%
\nbea
(3r + A)^2 - A(3 + r)^2 & = & -4 \cdot s
\neea
%
but now we need to make sure $k = -4$ divides $(3+r)$, an $r$ that works is $r=1$ this way $r^2 < 13$ and $r^2 - A = -12$ is negative and so we get
%
\nbea
4^2 - A\cdot1^2 = 3
\neea
%
next, multiplying it with $r^2 - A = s$ again
%
\nbea
(4r + A)^2 - A(4+r)^2 & = & 3\cdot s
\neea
%
here $k = 3$, to make sure $3|(4+r)$ and since $r^2 < 13$ we get $r=2$ and thus
%
\nbea
7^2 - A\cdot 2^2 = -3
\neea
%
and I got tired after this :) so the $s$ we recovered so far are $1,-4,3,-3, \dots$

{\bf Page 33, Problem 3}. The first part is straightforward, since $p^2 - Aq^2 = k$ and $P^2 - AQ^2 = K$ with $P = (pr + qA)/|k|$ and $Q = (p + qr)/|k|$
%
\nbea
pQ & = & \frac{p^2 + pqr}{|k|} \\
Pq & = & \frac{pqr + q^2A}{|k|} \\
\to pQ - Pq & = & \frac{p^2 - Aq^2}{|k|} \\
& = & \pm 1
\neea
%
as $|k| = |p^2-Aq^2|$ by definition. Now for the more fun part, since $pQ - Pq = \pm1$, Bezout tells us that $\gcd(Q, P) = 1$, but from $P^2 - AQ^2 = K$, if $\gcd(Q,K) > 1$ then it will also divide $P$ and vice versa, therefore these three are co-prime.

We need this for the next step, we want a new number $R$ such that $QR + P$ is divisible by $K$ or in other words
%
\nbea
QR + P & \equiv & 0 \pmod{K} \\
R & \equiv & Q^{-1}(-P) \pmod{K}
\neea
%
we are guaranteed to have such a $Q^{-1}$ because $\gcd(Q,K) = 1$ and the final step is also straightforward, say
%
\nbea
W \equiv QA + PR & \equiv & QA + P(Q^{-1}(-P)) \pmod{K} \\
W & \equiv & QA - Q^{-1} P^2 \pmod{K} \\
QW & \equiv & Q^2A - P^2 \equiv 0 \pmod{K}
\neea
%
and by definition $K|Q^2A - P^2$ but since $\gcd(Q,K) = 1$ this means that $W \equiv QA + PR \equiv 0 \pmod{K}$ and we are done.

{\bf Page , Problem 1}. Show that the only integral solutions to $1 + x + x^2 + x^3$ being a square is $x = -1,0,1,7$. First some factorization
%
\nbea
1 + x + x^2 + x^3 & = & (1 + x + x^2) + x^3 \\
& = & (1 + x)^2 - x + x^3 = (1 + x)^2 - x(1-x^2) \\
& = & (1 + x)^2 - x(1+x)(1-x)\\
& = & (1 + x)\lbrack(1 + x) - x(1-x)\rbrack = (1 + x)\lbrack1 + x - x + x^2\rbrack \\
& = & (1+x)(1+x^2)
\neea
%
From here we can conclude that $x$ cannot be even unless $x=0$, here's how, we know that $A^2 = (1+x)(1+x^2)$ 
and suppose that $d$ is the common factor of $(1+x)$ and $(1+x^2)$, therefore $d$ also divides
%
\nbea
(1+x)^2 - (1+x^2) & = & 2x
\neea
% 
thus $d|2$ or $d|x$. But here $x$ is even, thus $1+x$ is odd, do $d$ can't divide 2, so $d|x$ but since $d|(1+x)$ and $x$ and $1+x$ are co-prime, $d=1$. This means that 
%
\nbea
1 + x & = & y^2 \\
1 + x^2 & = & z^2
\neea
%
the last equation means $z^2 - x^2 = 1$ but the difference between two squares cannot be one unless $x=0$. Thus is $x$ is even then $x = 0$.

Next is $x$ odd. Let's recast $x = 2m+1$, we then have
%
\nbea
A^2 = 1 + x + x^3 + x^3 = (1+x)(1+x^2) & = & (1 + 2m + 1)(1 + (2m + 1)^2) \\
& = & 2(m+1)(4m^2 + 4m + 2) \\
& = & 4(m+1)(2m^2 + 2m + 1) \\
\to A^2 & = & 4(m + 1)(m^2 + (m+1^2))
\neea
%
Let's divide out the factor of $4$ and we have
%
\nbea
A'^2 & = & (m + 1)(m^2 + (m+1)^2)
\neea
%
any factor of $m+1$ and $m^2 + (m+1)^2$ would also divide $m^2$ but $m+1$ and $m^2$ are co-prime thus $(m+1)$ and $m^2 + (m+1)^2$ are co-prime thus each of them is square
%
\nbea
m + 1 & = & w^2 \\
m^2 + (m+1)^2 & = & y^2
\neea
%
since $m$ and $m+1$ are co-prime, $m^2 + (m+1)^2 = y^2$ forms a primitive Pythagorean triple, therefore we can cast it in the usual form, but here we have two choices, either $m$ is even or $m$ is odd. First, let's tackle the $m$ even
%
\nbea
m & = & 2ab \\
m + 1 & = & a^2 - b^2 = w^2
\neea
%
since from above we know that $m+1$ is square and 
%
\nbea
(m + 1) - m = 1 & = & a^2 - b^2 - 2ab \\
1 & = & (a-b)^2 - 2b^2
\neea
%
This is a form of Pell's equation and I was messing around with Pell's equation for roughly two days until I found out that I don't need to mess with Pell's equation at all. We'll talk about Pell's equation later :)

What we need here is to note that since $m + 1 = w^2 = a^2 - b^2$, so it forms another Pythagorean triple, since $a$ and $b$ are co-prime, they are also primitive
%
\nbea
a & = & c^2 + d^2 \\
b & = & 2cd
\neea
%
therefore the Pell's equation we had earlier becomes
%
\nbea
1 & = & (a-b)^2 - 2b^2 \\
& = & (c^2 + d^2 - 2cd)^2 - 2(2cd)^2 \\
& = & (c-d)^4 - 8(cd)^2
\neea
%
we now do the oldest trick in the book, substitutions
%
\nbea
s = c + d ~~~~~~~~~~~~~~ t = c - d
\neea
%
therefore $st = c^2 - d^2$ and
%
\nbea
s + t = 2c ~~~~~~~~~~~~~~~ s - t = 2d
\neea
%
and
%
\nbea
(s + t)(s - t) = s^2 - t^2 & = & 4cd \\
\to \frac{s^2 - t^2}{4} & = & cd
\neea
%
making the substitution
%
\nbea
1 = (c-d)^4 - 8(cd)^2 & = & t^4 - 8\left(\frac{s^2 - t^2}{4}\right )^2 \\
1 & = &  t^4 - \frac{1}{2}\left(s^2 - t^2\right )^2 \\
\to 2 & = & 2t^4 - \left(s^2 - t^2\right )^2
\neea
%
at this point I was quite stuck until I tried the simplest solution, the quadratic formula, solving for $s$ we get
%
\nbea
s & = & \pm \sqrt{t^2 \pm\sqrt{2}\sqrt{t^4 - 1}}
\neea
%
for $s$ to have a chance to be an integer we need $\sqrt{2}\sqrt{t^4 - 1}$ to be an integer, therefore
%
\nbea
t^4 - 1 & = & 2Q^2 \\
(t^2 - 1)(t^2 + 1) & = & 2Q^2
\neea
%
but $t = c - d$ and $c$ and $d$ are of opposite parities, thus $t$ is odd and $\gcd(t^2 - 1, t^2 + 1) = 2$ thus one of $t^2 - 1$ and $t^2 + 1$ is a square and the other is 2 times a square but either way we cannot have $t^2 \pm 1$ equal a square unless $t = 0$ or $t = \pm1$. But from the original Pell's equation $2 = 2t^4 - (s^2 - t^2)$ if $t = 0$ we have no solution for $s$. If $t=\pm1$ then $s = \pm1$ as well (or $\mp 1$).

But from $s = c + d$ it has to be positive (remember that $c$ and $d$ are part of a Pythagorean triple), thus $s = 1$, so one of $c$ or $d$ must be zero. From $b = 2cd$ we have $b = 0$ and the original Pell's equation gets to
%
\nbea
1 & = & (a - b)^2 - 2b^2 \\
& = & (a - 0)^2 - 2 \cdot 0^2 \\
\to 1 & = & a
\neea
%
and from $m + 1 = w^2 = a^2 - b^2$ we have $m + 1 = 1$ and $m = 0$ and from $x = 2m + 1$ we have $x = 1$. Thus we so far had $x = 0$ and $x = 1$ as solutions.

Next, we tackle $x = 2m + 1$ odd with $m$ odd, thus like the previous case
%
\nbea
m & = & a^2 - b^2 \\
m + 1 & = & 2ab = w^2
\neea
%
and they obey a (negative) Pell's equation just like above
%
\nbea
m + 1 - m = 1 & = & 2ab - (a^2 - b^2) \\
& = & 2b^2 - (a - b)^2
\neea
%
since $a$ and $b$ are co-prime we have either $a = 2A^2$ and $b = B^2$ or $a = A^2$ and $b = 2B^2$ but if it is the latter then from the Pell's equations
%
\nbea
1 & = & 2b^2 - (a - b)^2 \\
& = & (2B)^2 - (A^2-2B^2)^2
\neea
%
but again, the difference of two squares cannot be one unless $2B = 1$ and $A^2 - 2B^2 = 0$ but this is impossible since $A$ and $B$ are integers, therefore $b = B^2$ and $a = 2A^2$ and we have
%
\nbea
1 & = & 2B^4 - (A^2 - 2B^2)^2
\neea
%
again I was messing with Pell's equations again but again it was not needed, quadratic formula to the rescue! solving for $B$ we get
%
\nbea
B & = & \pm \sqrt{-2A^2 \pm\sqrt{8A^4 + 1}}
\neea
%
for $B$ to be an integer we must have $8A^4 + 1$ to be a square, on the outset it is nothing more than just a Pell's equation but we can do better, borrowing the trick I used in solving $y^2 = x^3 + 1$, we do the following
%
\nbea
8A^4 + 1 & = & k^2 \\
8A^4 & = & k^2 - 1 \\
& = & (k - 1)(k + 1) \\
\to 8A^4 & = & n(n + 2)
\neea
%
where $n = k - 1$. We know that $\gcd(n, n+2)$ is at most $2$ but since the LHS is $8A^4$ it must be 2 :) So we need to distribute the prime factors of $8A^4$ into $n$ and $n + 2$ and since $\gcd(n,n+2) = 2$ we must have either
%
\nbea
n = 2C^4 ~~~~~~ n + 2 = 2^{4r + 2}D^4 ~~~~~~~~{\rm or} ~~~~~~~~n = 2^{4r + 2}D^4  ~~~~~~ n + 2 = 2C^4
\neea
%
the $2^{4r + 2}$ is to make sure that when we multiply $n$ and $n + 2$ we get an overall factor $8$, also from their difference $n + 2 - n = 2$ we get (including the two cases)
%
\nbea
\pm2 & = & 2C^4 - 2^{4r + 2}D^4 \\
\to \pm 1 & = & C^4 - 2E^4, ~~~~~ E^4 = 2^{4r}D^4
\neea
%
at this point we will generalize things, since we can think of $1 = 1^4$, I wanted to see if there are solutions to the following equations
%
\nbea
2E^4 = C^4 + F^4 ~~~~~~~~~~{\rm and} ~~~~~~~~~~~2E^4 = C^4 - F^4 
\neea
%
Now, if $C$ and $F$ are both even, then we can cancel an overall factor of $2^4$ throughout (the same with any common factor between them), and if after canceling they still contain $2$ we can do the same until we reach a point where $C$ and $F$ are both odd and co-prime. They must be both odd because the LHS is even.

So we can just consider co-prime solutions with $C$ and $F$ odd. Since they are both odd
%
\nbea
C + F = 2u  ~~~~~~~~~~~~~~~~ C - F = 2v \\
C = u + v    ~~~~~~~~~~~~~~~~ F = u - v
\neea
%
with $u$, $v$ of opposite parities and co-prime since their sum (and difference) is odd and $C$ and $F$ are co-prime. Now tackling the first case $2E^4 = F^4 + C^4$ we get
%
\nbea
2E^4 & = & (u + v)^4 + (u - v)^4 \\
\bcancel{2}E^4& = & \bcancel{2}(u^4 + 6u^2v^2 + v^4) \\
E^4 & = & (u^2 + v^2)^2 + (2uv)^2
\neea
%
but since $u$ and $v$ are co-prime and of opposite parities they form a Pythagorean triple
%
\nbea
(u^2 - v^2)^2 + (2uv)^2 & = & (u^2 + v^2)^2 \\
(u^2 - v^2)^2 & = & (u^2 + v^2)^2 - (2uv)^2
\neea
%
multiplying both of them
%
\nbea
(E^2)^2 (u^2 - v^2)^2 & = & \left \lbrack (u^2 + v^2)^2 + (2uv)^2 \right \rbrack \left \lbrack (u^2 + v^2)^2 - (2uv)^2 \right \rbrack \\
\left \lbrack (E^2) (u^2 - v^2) \right \rbrack^2 & = & (u^2 + v^2)^4 - (2uv)^4
\neea
%
but we know that $Z^2 = Y^4 - X^4$ has no non-trivial solutions, the only trivial solutions are all zeroes (which we cannot have here since $Y = u^2 + v^2$), the other trivial solution $Z = 1$ and $Y=1$ with $X = 0$.

This means that $2uv = 0$ so either $u=0$ or $v = 0$ or both (but we can't have both zero because we need $Y = u^2 + v^2$ to be 1). But from $C = u + v$ and $F = u - v$, if one of them is zero we have $C = \pm F = \pm 1$, either way, from $2E^4 = C^4 + F^4$ we have $E = 1$.

And from $n = 2C^4 = 2, n + 2 = 4E^4 = 4$ (or the other way round) we get $8A^4 = n(n+2) = 8$, meaning $A = 1$. And from $B = \pm \sqrt{-2A^2 \pm\sqrt{8A^4 + 1}}$, we get $B = \pm 1$ which in turn means $m = a^2 - b^2 = 4A^4 - B^2 = 3$ and $m + 1 = 2ab = 4A^2B^2 = 4$, this translates to $x = 2m+1 = 7$.

The other case is $2E^4 = C^4 - F^4$, again subtituting $C = u + v$ and $F = u-v$
%
\nbea
2E^4 & = & (u + v)^4 - (u - v)^4 \\
& = & 8uv(u^2 + v^2) \\
\to E^4 & = & 4uv(u^2 + v^2)
\neea
%
now $u$ is co-prime to $u^2 + v^2$ and $v$ is also co-prime to $u^2 + v^2$ also $u^2 + v^2$ is odd so it is also co-prime to $4$, therefore $4uv$ is co-prime to $u^2 + v^2$, thus
%
\nbea
4uv & = & H^4 \\
u^2 + v^2 & = & I^4
\neea
%
but $u$ and $v$ are co-prime and of opposite parities, say $v$ is odd, from $4uv = H^4$ we must have $v = V^4$ and from the second equation we therefore have $u^2 + (V^2)^4 = I^4$ but again $Z^2 = Y^4 - X^4$ has only trivial solutions. Again the all zero solution is out because then $u = v = 0$ but they must be of opposite parities.

The other trivial solution is $Y = I = 1 \to v = 1$ and $Z = v = 0$ and $X = u = 1$, this means $C = u + v = 1$ and $F = u - v = 1$ and $2E^4 = C^4 - F^4 = 1 - 1 = 0 \to E = 0$, which means one of $n$ or $n + 2$ is zero, which also means $8A^4 = n(n + 2) = 0 \to A = 0$ and $B = \pm \sqrt{-2A^2 \pm\sqrt{8A^4 + 1}} = \pm 1$. This means that $m = a^2 - b^2 = 4A^4 - B^2 = -1$ and $m + 1 = 2ab = 4A^2B^2 = 0$ and $x = 2m+1 = -1$ and this is FINALLY our last solution LOL.



\underline{\textbf{\textit{Pell's Equation Permutation}}}. Obviously we encountered Pell's equations along the way of the form $x^2 - 2y^2 = \pm 1$ in this problem. So I learned a thing or two about solving Pell's equation, well not really in solving it because for that you'll need the Indian cyclic method or the English method but here is how we generate more solutions once we get the first one, note that
%
\nbea
1 = x^2 - n y^2 & = & (x + \sqrt{n}y)(x - \sqrt{n}y)
\neea
%
so say we get a first solution $x_0, y_0$ we can exponentiate it to get the $k^{\rm th}$ one, $(x_0 + \sqrt{n}y_0)^k = x_k + \sqrt{n}y_k$ (we can also choose to do it with a negative sign $(x_0 - \sqrt{n}y_0)^k = x_k - \sqrt{n}y_k$), this is why, from Pell's equation we know that
%
\nbea
1 & = & (x_0 + \sqrt{n}y_0)(x_0 - \sqrt{n}y_0) \\
\to 1^k & = & (x_0 + \sqrt{n}y_0)^k(x_0 - \sqrt{n}y_0)^k
\neea
%
the two brackets on the RHS will yield the same $x_k, y_k$ because
%
\nbea
(x_0 \pm \sqrt{n}y_0)^k & = & \sum_{i=0}^k (\pm 1)^i \binom{k}{i} x_0^{k-i} y_0^i n^{\frac{i}{2}} \\
& = & \sum_{i=0}^{\floor{k/2}} (\pm 1)^{2i} \binom{k}{2i} x_0^{k-2i} y_0^{2i} n^{\frac{2i}{2}} + \sum_{i=0}^{\floor{(k-1)/2}} (\pm 1)^{2i+1} \binom{k}{2i+1} x_0^{k-2i-1} y_0^{2i+1} n^{\frac{2i+1}{2}} \\
& = & \left\{\sum_{i=0}^{\floor{k/2}} \binom{k}{2i} x_0^{k-2i} y_0^{2i} n^{i}\right\} \pm \sqrt{n} \left\{\sum_{i=0}^{\floor{(k-1)/2}} \binom{k}{2i+1} x_0^{k-2i-1} y_0^{2i+1} n^{i}\right\} \\
& = & x_k \pm \sqrt{n} y_k 
\neea
%
we are just basically splitting the sum to even and odd indices, and we saw above that $(x_0 + \sqrt{n}y_0)^k$ and $(x_0 - \sqrt{n}y_0)^k$ generate the same $x_k, y_k$ such that when we multiply them
%
\nbea
1^k & = & (x_0 + \sqrt{n}y_0)^k(x_0 - \sqrt{n}y_0)^k \\
1 & = & (x_k + \sqrt{n}y_k)(x_k - \sqrt{n}y_k) \\
& = & x_k^2 - n y_k^2
\neea
%
For $n = 2$ we can use the above formula but more than that we can actually derive a recurrence relation $x_{n+2} = 6x_{n+1} - x_n$, the same applies to $y_{n+2} = 6y_{n+1} - y_n$. Although we also had the negative Pell's equation as well $x^2 - 2y^2 = \pm1$, the first 9 solutions can be found on Table.~\ref{Tab:1}. How do we prove this recurrence relation, say with induction?
%
\nbea
x_{n+2}^2 - 2 y_{n+2}^2 & = & (6x_{n+1} - x_n)^2 - 2 (6y_{n+1} - y_n)^2 \\
& = & (6^2 x_{n+1}^2 + x_n^2 - 2x_{n+1}x_n) - 2 (6^2 y_{n+1}^2 + y_n^2 - 2y_{n+1}y_n)
\neea
%
The somewhat problematic terms here are the crossed ones
%
\nbea
x_{n+1} x_n & = & (6 x_n - x_{n-1})x_n \\
& = & 6x_n^2 - x_{n-1}x_n \\
& = & 6 x_n^2 - x_{n-1}(6x_{n-1} - x_{n-2}) \\
& = & 6 x_n^2 - 6 x_{n-1}^2 + \dots + (-1)^{n} x_1 x_0
\neea
%
The same of course goes with $y_{n+2}$, going back to $x_{n+2}^2 - 2 y_{n+2}^2$ we get
%
\nbea
x_{n+2}^2 - 2 y_{n+2}^2 & = & (6^2 x_{n+1}^2 + x_n^2 - 2\cdot 6 x_{n+1}x_n) - 2 (6^2 y_{n+1}^2 + y_n^2 - 2\cdot 6y_{n+1}y_n) \\
& = & 6^2 (x_{n+1}^2 - 2 y_{n+1}^2) + (x_n^2 - 2y_n^2) - 12 (-1)^n(x_1 x_0 - 2 y_1 y_0) - 12 \sum_{i = 1}^n (-1)^{i+n} 6(x^2 - 2y^2) \\
& = & \pm6^2 \pm 1 - 12 (-1)^n(x_1x_0 -2 y_1y_0) - 12 \sum_{i = 1}^n (-1)^{i+n} (\pm6)
\neea
%
the $\pm6^2$, $\pm1$ in the middle and the $\pm6$ in the sum depend on $x^2 -2y^2 = \pm1$, the last sum is either $0$ or $\pm6$ depending on whether $n$ is even or odd respectively. So if $n$ is even we have
%
\nbea
x_{n+2}^2 - 2 y_{n+2}^2 & = & \left \{
\begin{array}{l l}
6^2 + 1 - 12(1\cdot3 -2 0\cdot 2) = 1 &~~~~~{\rm for}~ x^2 - 2y^2 = 1 \\
-6^2 - 1 - 12(1\cdot7 -2 \cdot1\cdot 5) = -1 &~~~~~{\rm for}~ x^2 - 2y^2 = -1
\end{array}
\right .
\neea
%
if $n$ is odd we have
%
\nbea
x_{n+2}^2 - 2 y_{n+2}^2 & = & \left \{
\begin{array}{l l}
6^2 + 1 + 12(1\cdot3 -2 0\cdot 2) -12\cdot6= 1 &~~~~~{\rm for}~ x^2 - 2y^2 = 1 \\
-6^2 - 1 + 12(1\cdot7 -2 \cdot1\cdot 5) + 12 \cdot 6 = -1 &~~~~~{\rm for}~ x^2 - 2y^2 = -1
\end{array}
\right .
\neea
%
and the induction is complete (note that we get the values for $x_0,x_1,y_0,y_1$ from Table.~\ref{Tab:1} above).

%
\begin{table}[]
\centering
\caption{Solutions to left $x^2 - 2y^2 = 1$ and right $x^2 - 2y^2 = -1$}
\label{Tab:1}
\begin{tabular}{|c|c|c|c|}
\hline
~~~$x$~~~ & ~~~$y$~~~ & ~~~$x$~~~ & ~~~$y$~~~  \\ \hline 
1 & 0 & 1 & 1 \\
3 & 2 & 7 & 5 \\
17 & 12 & 41 & 29 \\ 
99 & 70 & 239 & 169 \\
577 & 408 & 1393 & 985 \\
3363 & 2378 & 8119 & 5741 \\
19601 & 13860 & 47321 & 33461 \\
114243 & 80782 & 275807 & 195025 \\
665857 & 470832 & 1607521 & 1136689 \\
\hline
\end{tabular}
\end{table}
%

I initially wanted to show that the solutions to $x^2 - 2y^2 = \pm1$ are actually not biquadratic, \ie I wanted to show that $x$ (or maybe $y$ can't remember which one) cannot be a square. I tried using any odd methods, like listing all the values and then modding them mod 10, 4, 6, 16, etc because for a number to be square it must be $0,1,4,9$ mode 16, $1,3$ mod 6, etc, but there's always a number that satisfy all of them mods LOL. I also tried showing that $x^2 - 2(y^2)^2 = -1$ has no solution but I was wrong since from Table.~\ref{Tab:1} we see that we have $y^2 = 169$, the thing is that we had more constraints, \ie $x = 2A^2 - B^2$ and this is the one not satisfied by this particular solution.

Another thing I noticed was that the rightmost $y$ in Table.~\ref{Tab:1} is also the hypotenuse of a right triangle whose legs differ by one. Note that we initially had $m^2 + (m+1)^2 = Y^2$ and $m=2ab$, $m+1 = a^2 - b^2$ or the other way round, and it generates the Pell's equations $(a-b)^2 - 2b^2 = \pm1$. But the funny thing is that the hypotenuse itself, as a result of these Pell's equation, has the values of the solutions of the Pell's equations themselves, it looked mysterious but it is actually not, because the hypotenuse itself also obeys the same Pell's equation, because
%
\nbea
x^2 + (x+1)^2 & = & y^2 \\
2x^2 + 2x + 1 & = & y^2 \\
(2x + 1)^2 - 2x^2 - 2x & = & y^2 \\
(2x + 1)^2 - 2x^2 - 2x -1 + 1 & = & y^2 \\
(2x + 1)^2 - y^2 + 1 & = & y^2 \\
\to (2x + 1)^2 - 2y^2 & = & -1
\neea
%
so the lesson is if some variable has the same values as a solution to a certain equation, then it {\it must} also be a solution :)

Another thing I tried was to reapply Pell's equation to $\sqrt{8A^4 + 1} = \sqrt{2(2A^2)^2 + 1}$ above. And it got nowhere fast, the results I got was that this also means that half of the solution to $y$ in $x^2 - 2y^2 = 1$ must also be a square, \ie $\frac{1}{2}y = h^2$. But again it got nowhere even faster. But on the flip side the result above (as long as I don't miss any assumptions in applying it to this case) shows that $\frac{1}{2}y$ with $y$ on the left of Table.~\ref{Tab:1} cannot be square :)

I tried other approaches as well. My very first attempt was actually to use geometric series $1 + x + x^2 + x^3 = (x^4 - 1)/(x - 1)$ but then I wasn't sure what to do next. And others are not even worth mentioning :)

\bigskip
\underline{\textbf{\textit{Basic and Extra Ingredients}}}

The knowledge needed to solve this problem (my way) is
\bit
\item Factorization, $1 + x + x^2 + x^3 = (1 + x)(1 + x^2)$
%
\item Even odd classifications, what happens if $x$ is odd, what happens if $x$ is even
%
\item Pythagorean triples, $m^2 - n^2, 2mn, m^2 + n^2$
%
\item Quadratic formula, $\frac{-b \pm \sqrt{b^2 - 4ac}}{2a}$
%
\item $Z^2 = Y^4 - X^4$ has only trivial solutions
%
\item Prime factorization implication, $8A^4 = n(n+2) \to n = 2C^4, n + 2 = 2^{4n+2}D^4$
%
\item Generalization, $2E^4 = C^4 - 1 \to 2E^4 = C^4 - F^4$ and find that the latter has no solution
\eit
%
Note that I already knew all these, the trick is what to use and when :) and of course Pell's equation is not even needed


\bigskip
\underline{\textbf{\textit{Sinister problems :)}}}

Sometimes I wonder, without knowing the above solution, what happens if the following problems are presented, would someone be able to solve them?
\begin{enumerate}
\item Prove that the following hypotenuses are all the hypotenuses of right triangles whose legs differ by one, 1, 5, 29, 169, 985, 5741, 33461, 195025, 1136689, etc or in other words they obey the following recurrence relation $h_{n+2} = 6h_{n+1} - h_n$
%
\item Prove that each number in the following sequence, 0, 1, 6, 35, 204, 1189, 6930, 40391, 235416, etc, is not a square, except for 0 and 1.
\end{enumerate}

I guess the lesson is that it's harder to prove that a sequence of numbers is not squares than to prove that certain equation $2E^4 = C^4 + F^4$ has no non-trivial integer solutions.

















Since $t = c - d$















\end{document}