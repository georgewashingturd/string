\documentclass[aps,preprint,preprintnumbers,nofootinbib,showpacs,prd]{revtex4-1}
\usepackage{graphicx,color}
\usepackage{caption}
\usepackage{subcaption}
\usepackage{amsmath,amssymb}
\usepackage{multirow}
\usepackage{amsthm}%        But you can't use \usewithpatch for several packages as in this line. The search 

\usepackage{cancel}

%%% for SLE
\usepackage{dcolumn}   % needed for some tables
\usepackage{bm}        % for math
\usepackage{amssymb}   % for math
\usepackage{multirow}
%%% for SLE -End

\usepackage{ulem}
\usepackage{cancel}

\usepackage{hyperref}
\usepackage{mathrsfs}
\usepackage[top=1in, bottom=1.25in, left=1.1in, right=1.1in]{geometry}

\usepackage{mathtools} % for \DeclarePairedDelimiter{\ceil}{\lceil}{\rceil}

%\usepackage{xeCJK}
%\setCJKmainfont{SimSun}

\newcommand{\msout}[1]{\text{\sout{\ensuremath{#1}}}}


%%%%%% My stuffs - Stef
\newcommand{\lsim}{\mathrel{\mathop{\kern 0pt \rlap
  {\raise.2ex\hbox{$<$}}}
  \lower.9ex\hbox{\kern-.190em $\sim$}}}
\newcommand{\gsim}{\mathrel{\mathop{\kern 0pt \rlap
  {\raise.2ex\hbox{$>$}}}
  \lower.9ex\hbox{\kern-.190em $\sim$}}}

%
% Key
%
\newcommand{\key}[1]{\medskip{\sffamily\bfseries\color{blue}#1}\par\medskip}
%\newcommand{\key}[1]{}
\newcommand{\q}[1] {\medskip{\sffamily\bfseries\color{red}#1}\par\medskip}
\newcommand{\comment}[2]{{\color{red}{{\bf #1:}  #2}}}


\newcommand{\ie}{{\it i.e.} }
\newcommand{\eg}{{\it e.g.} }

%
% Energy scales
%
\newcommand{\ev}{{\,{\rm eV}}}
\newcommand{\kev}{{\,{\rm keV}}}
\newcommand{\mev}{{\,{\rm MeV}}}
\newcommand{\gev}{{\,{\rm GeV}}}
\newcommand{\tev}{{\,{\rm TeV}}}
\newcommand{\fb}{{\,{\rm fb}}}
\newcommand{\ifb}{{\,{\rm fb}^{-1}}}

%
% SUSY notations
%
\newcommand{\neu}{\tilde{\chi}^0}
\newcommand{\neuo}{{\tilde{\chi}^0_1}}
\newcommand{\neut}{{\tilde{\chi}^0_2}}
\newcommand{\cha}{{\tilde{\chi}^\pm}}
\newcommand{\chao}{{\tilde{\chi}^\pm_1}}
\newcommand{\chaop}{{\tilde{\chi}^+_1}}
\newcommand{\chaom}{{\tilde{\chi}^-_1}}
\newcommand{\Wpm}{W^\pm}
\newcommand{\chat}{{\tilde{\chi}^\pm_2}}
\newcommand{\smu}{{\tilde{\mu}}}
\newcommand{\smur}{\tilde{\mu}_R}
\newcommand{\smul}{\tilde{\mu}_L}
\newcommand{\sel}{{\tilde{e}}}
\newcommand{\selr}{\tilde{e}_R}
\newcommand{\sell}{\tilde{e}_L}
\newcommand{\smurl}{\tilde{\mu}_{R,L}}

\newcommand{\casea}{\texttt{IA}}
\newcommand{\caseb}{\texttt{IB}}
\newcommand{\casec}{\texttt{II}}

\newcommand{\caseasix}{\texttt{IA-6}}

%
% Greek
%
\newcommand{\es}{{\epsilon}}
\newcommand{\sg}{{\sigma}}
\newcommand{\dt}{{\delta}}
\newcommand{\kp}{{\kappa}}
\newcommand{\lm}{{\lambda}}
\newcommand{\Lm}{{\Lambda}}
\newcommand{\gm}{{\gamma}}
\newcommand{\mn}{{\mu\nu}}
\newcommand{\Gm}{{\Gamma}}
\newcommand{\tho}{{\theta_1}}
\newcommand{\tht}{{\theta_2}}
\newcommand{\lmo}{{\lambda_1}}
\newcommand{\lmt}{{\lambda_2}}
%
% LaTeX equations
%
\newcommand{\beq}{\begin{equation}}
\newcommand{\eeq}{\end{equation}}
\newcommand{\bea}{\begin{eqnarray}}
\newcommand{\eea}{\end{eqnarray}}
\newcommand{\ba}{\begin{array}}
\newcommand{\ea}{\end{array}}
\newcommand{\bit}{\begin{itemize}}
\newcommand{\eit}{\end{itemize}}

\newcommand{\nbea}{\begin{eqnarray*}}
\newcommand{\neea}{\end{eqnarray*}}
\newcommand{\nbeq}{\begin{equation*}}
\newcommand{\neeq}{\end{equation*}}

\newcommand{\no}{{\nonumber}}
\newcommand{\td}[1]{{\widetilde{#1}}}
\newcommand{\sqt}{{\sqrt{2}}}
%
\newcommand{\me}{{\rlap/\!E}}
\newcommand{\met}{{\rlap/\!E_T}}
\newcommand{\rdmu}{{\partial^\mu}}
\newcommand{\gmm}{{\gamma^\mu}}
\newcommand{\gmb}{{\gamma^\beta}}
\newcommand{\gma}{{\gamma^\alpha}}
\newcommand{\gmn}{{\gamma^\nu}}
\newcommand{\gmf}{{\gamma^5}}
%
% Roman expressions
%
\newcommand{\br}{{\rm Br}}
\newcommand{\sign}{{\rm sign}}
\newcommand{\Lg}{{\mathcal{L}}}
\newcommand{\M}{{\mathcal{M}}}
\newcommand{\tr}{{\rm Tr}}

\newcommand{\msq}{{\overline{|\mathcal{M}|^2}}}

%
% kinematic variables
%
%\newcommand{\mc}{m^{\rm cusp}}
%\newcommand{\mmax}{m^{\rm max}}
%\newcommand{\mmin}{m^{\rm min}}
%\newcommand{\mll}{m_{\ell\ell}}
%\newcommand{\mllc}{m^{\rm cusp}_{\ell\ell}}
%\newcommand{\mllmax}{m^{\rm max}_{\ell\ell}}
%\newcommand{\mllmin}{m^{\rm min}_{\ell\ell}}
%\newcommand{\elmax} {E_\ell^{\rm max}}
%\newcommand{\elmin} {E_\ell^{\rm min}}
\newcommand{\mxx}{m_{\chi\chi}}
\newcommand{\mrec}{m_{\rm rec}}
\newcommand{\mrecmin}{m_{\rm rec}^{\rm min}}
\newcommand{\mrecc}{m_{\rm rec}^{\rm cusp}}
\newcommand{\mrecmax}{m_{\rm rec}^{\rm max}}
%\newcommand{\mpt}{\rlap/p_T}

%%%song
\newcommand{\cosmax}{|\cos\Theta|_{\rm max} }
\newcommand{\maa}{m_{aa}}
\newcommand{\maac}{m^{\rm cusp}_{aa}}
\newcommand{\maamax}{m^{\rm max}_{aa}}
\newcommand{\maamin}{m^{\rm min}_{aa}}
\newcommand{\eamax} {E_a^{\rm max}}
\newcommand{\eamin} {E_a^{\rm min}}
\newcommand{\eaamax} {E_{aa}^{\rm max}}
\newcommand{\eaacusp} {E_{aa}^{\rm cusp}}
\newcommand{\eaamin} {E_{aa}^{\rm min}}
\newcommand{\exxmax} {E_{\neuo \neuo}^{\rm max}}
\newcommand{\exxcusp} {E_{\neuo \neuo}^{\rm cusp}}
\newcommand{\exxmin} {E_{\neuo \neuo}^{\rm min}}
%\newcommand{\mxx}{m_{XX}}
%\newcommand{\mrec}{m_{\rm rec}}
\newcommand{\erec}{E_{\rm rec}}
%\newcommand{\mrecmin}{m_{\rm rec}^{\rm min}}
%\newcommand{\mrecc}{m_{\rm rec}^{\rm cusp}}
%\newcommand{\mrecmax}{m_{\rm rec}^{\rm max}}
%%%song

\newcommand{\mc}{m^{\rm cusp}}
\newcommand{\mmax}{m^{\rm max}}
\newcommand{\mmin}{m^{\rm min}}
\newcommand{\mll}{m_{\mu\mu}}
\newcommand{\mllc}{m^{\rm cusp}_{\mu\mu}}
\newcommand{\mllmax}{m^{\rm max}_{\mu\mu}}
\newcommand{\mllmin}{m^{\rm min}_{\mu\mu}}
\newcommand{\mllcusp}{m^{\rm cusp}_{\mu\mu}}
\newcommand{\elmax} {E_\mu^{\rm max}}
\newcommand{\elmin} {E_\mu^{\rm min}}
\newcommand{\elmaxw} {E_W^{\rm max}}
\newcommand{\elminw} {E_W^{\rm min}}
\newcommand{\R} {{\cal R}}

\newcommand{\ewmax} {E_W^{\rm max}}
\newcommand{\ewmin} {E_W^{\rm min}}
\newcommand{\mwrec}{m_{WW}}
\newcommand{\mwrecmin}{m_{WW}^{\rm min}}
\newcommand{\mwrecc}{m_{WW}^{\rm cusp}}
\newcommand{\mwrecmax}{m_{WW}^{\rm max}}

\newcommand{\mpt}{{\rlap/p}_T}

%%%%%% END My stuffs - Stef

\newcommand{\dunno}{$ {}^{\mbox {--}}\backslash(^{\rm o}{}\underline{\hspace{0.2cm}}{\rm o})/^{\mbox {--}}$}

\DeclarePairedDelimiter{\ceil}{\lceil}{\rceil}
\DeclarePairedDelimiter{\floor}{\lfloor}{\rfloor}

\DeclareMathOperator{\re}{Re}


\begin{document}

\title{Harold Edwards Fermat's last theorem}
\bigskip
\author{Stefanus Koesno$^1$\\
$^1$ Somewhere in California\\ San Jose, CA 95134 USA\\
}
%
\date{\today}
%
\begin{abstract}

\end{abstract}
%
\maketitle

\renewcommand{\theequation}{A.\arabic{equation}}  % redefine the command that creates the equation no.
\setcounter{equation}{0}  % reset counter 

{\bf Page 8, Problem 3}. Show that if $d^2|z^2$ then $d|z$, start with $\gcd(d,z) = c$
%
\nbea
d^2 k & = & z^2 \\
c^2 D^2 k & = & c^2 Z^2 \\
D^2 k & = & Z^2
\neea
%
we know that $\gcd(D,Z) = 1$ and so $\gcd(D^2,Z^2) = 1$, since $k|Z^2$, if $\gcd(D^2,k) > 1$ then $\gcd(D^2,Z^2) > 1$ as well, therefore $\gcd(D^2,k) = 1$, this means that $k = K^2$ by our assumption that if $vw = u^2$ and $v$ and $w$ are co-prime then both $v$ and $w$ are squares.

Substituting $k=K^2$ back into our first equation
%
\nbea
d^2K^2 & = & z^2 \\
\to dK & = & z
\neea
%
thus $d|z$ although I'm not sure about this particular line of reasoning, since $\gcd(D^2,Z^2) = 1$ and $D^2|Z^2$ then $D^2 = 1$ and we immediately get $k=Z^2$ and we don't need our assumption about $vw=u^2$ at all. But I'm not sure how else to show that $\gcd(D^2,k) = 1$ except by showing that $\gcd(D^2,Z^2) = 1$.

Now, the one step I stil need to prove, is that $\gcd(D,Z) = 1$ means $\gcd(D^2,Z^2) = 1$, again, without using the fundamental theorem. What I need is Bezout, assume that $\gcd(D^2,Z^2) = g > 1$ then Bezout tells us that
%
\nbea
D^2 a + Z^2 b & = & g
\neea
%
but from $\gcd(D,Z)=1$ we also get
%
\nbea
D A + Z B & = & 1 \\
\to DgA + ZgB & = & g
\neea
%
Now there might be some other numbers such that
%
\nbea
D M + Z N & = & g
\neea
%
but this means either that $\gcd(M,N) = g$ or $\gcd(M,N) = y | g$, let's dicuss the latter first
%
\nbea
D y M' + Z y N' & = & yg' \\
D M' + Z N' & = & g'
\neea
%
with $\gcd(M',N') = 1$ but this means that $\gcd(D,Z) = g'$, the only way this works is that $g' = 1$ and $\gcd(M,N) = g$, but if this is the case then
%
\nbea
D M' + Z N' & = & 1
\neea
%
but Bezout also tells us that all solutions to $D A' + Z B' = 1$ are of the form see ent.pdf Problem 2.5
%
\nbea
A' & = & A + l D \\
B' & = & B - lD
\neea
%
Equating the two Bezouts $g = D^2 a + Z^2 b = DgA' + ZgB'$
%
\nbea
\to gA' & = & gA + gl D = Da\\
\to gB' & = & gB - gl Z = Zb 
\neea
%
Now, equating the two Bezouts again
%
\nbea
D^2 a + Z^2 b & = & DgA + ZgB \\
D(Da - gA) & = & Z(gB - Zb) \\
\to D(glD) & = & Z(glZ) \\
D^2 & = & Z^2
\neea
%
which is a contradiction, therefore if $\gcd(D,Z) = 1$ then $\gcd(D^2,Z^2) = 1$ as well. The above proof is easily generalizable to $\gcd(D^n,Z^m)$
%
\nbea
\to gA' & = & gA + gl D = D^{n-1}a\\
\to gB' & = & gB - gl Z = Z^{m-1}b 
\neea
%
Now, equating the two Bezouts again
%
\nbea
D^n a + Z^m b & = & DgA + ZgB \\
D(D^{n-1}a - gA) & = & Z(gB - Z^{m-1}b) \\
\to D(glD) & = & Z(glZ) \\
D^2 & = & Z^2
\neea
%

{\bf Page 14, Problem 1}. Prove that if $Ad^2$ is a square then $A$ is a square, using the result from previous Problem, say $Ad^2 = z^2$ since $d^2|z^2$ this also means that $d|z$ so we can write $z = dk$, therefore
%
\nbea
Ad^2 & = & z^2 = d^2k^2 \\
A & = & k^2
\neea
%
and we are done.

{\bf Page 14, Problem 2}. Show that $x^4 - y^4 = z^2$ has no non-zero integer solutions. One thing we should not do is to blindly apply the Pythagorean formula ($m^2-n^2, 2mn, m^2+n^2$) over and over again, what we should do is follow what was done in the preceding section.

In that section we have $p,q,$ and $p^2 - q^2$ are all squares due to $t^2 = pq(p^2-q^2)$ so if we designate
%
\nbea
p & = & x^2 \\
q & = & y^2 \\
p^2 - q^2 & = & z^2 \\
\to x^4 - y^4 & = & z^2
\neea
%
and we have our current problem. Following what was done in the book
%
\nbea
z^2 & = & p^2 - q^2 \\
& = & (p-q)(p+q)
\neea
%
since $p$ and $q$ are co-prime so are $p-q$ and $p+q$. From $x^4 - y^4 = z^2$ we know that $x$ and thus $p$ is odd but here we can have $y$ even or odd, for simplicity let's start with $y$ and thus $q$ even so that we have the case in the book, so
%
\nbea
p + q = r^2  ~~~~~~~~~~~~~~~~ p - q = s^2
\neea
%
with both $r$ and $s$ odd and co-prime and
%
\nbea
u = \frac{r - s}{2}   ~~~~~~~~~~~~~~~~ v = \frac{r + s}{2}
\neea
%
with $u$ and $v$ integers and co-prime and so
%
\nbea
uv & = & \frac{r^2 - s^2}{4} = \frac{(p+q)-(p-q)}{4} = \frac{q}{2} = \frac{y^2}{2}
\neea
%
since $uv$ integer $y^2/2$ must also be an integer, thus $y = 2k$, $y^2/2 = 2k^2$ therefore
%
\nbea
\frac{uv}{2} & = & \frac{y^2}{4} = k^2 \\
\to uv & = & 2k^2
\neea
%
since $u$ and $v$ are co-prime this means that one of them is even and the other odd, let's take $u$ odd and $v=2v'$ even, then $u(2v') = 2k^2$ and therefore
%
\nbea
u = U^2   ~~~~~~~~~~~~~~~~ v = 2 V^2
\neea
%
since $u$ and $v$ are coprime. Thus
%
\nbea
r & = & u + v = U^2 + 2V^2
\neea
%
and 
%
\nbea
u^2 + v^2 & = & \frac{(r-s)^2 + (r+s)^2}{4} \\
& = & \frac{2r^2 + 2s^2}{4} = \frac{r^2 + s^2}{2} \\
& = & \frac{(p+q) + (p-q)}{2} = \frac{2p}{2} \\
& = & p \\
u^2 + v^2 & = & x^2
\neea
%
Thus we have a primitive triple $u,v,x$ (because $u,v$ are co-prime), thus we have $P^2-Q^2,2PQ,P^2+Q^2$ (note that above we have designated $u$ as the odd one) and so
%
\nbea
\frac{uv}{2} = k^2 = (P^2 - Q^2)PQ
\neea
%
so we have the same situation as $t^2 = pq(p^2-q^2)$ but $uv/2 = q/4 < t^2$ and our infinite descent begins.

The above was when $q$ even such that $p-q$ and $p+q$ are odd. Now we deal with the case of $q$ odd such that $p-q = 2r^2$ and $p+q = 2s^2$ are both even, this is because now $p-q$ and $p+q$ are no longer co-prime so we cannot follow the same steps above.

What we have is (from the pythagorean triple formula)
%
\nbea
p = x^2 & = & m^2 + n ^2 \\
q = y^2 & = & m^2 - n^2
\neea
%
thus we can use them directly, $m^2$ plays the role of $p$ and $n^2$ play the role of $q$ as they are already co-prime and of opposite parities and of course $x$ plays the role of $p+q$ and $y$, $p-q$. Thus in this case
%
\nbea
u = \frac{x-y}{2} ~~~~~~~~~~~~~~~~ v = \frac{x+y}{2}
\neea
%
Thus, just like above
%
\nbea
uv = \frac{x^2 - y^2}{4} = \frac{n^2}{2} ~~~~~~~~~~~~~~~~ u = U^2 ~~~~~~ v = 2V^2
\neea
%
and therefore
%
\nbea
u^2 + v^2 = m^2
\neea
%
Thus we have our infinite descent all over again.












{\bf page 25, Problem 1}. Prove that $2^{37}-1$ is not prime














{\bf Page 25, Problem 2}. If $p = 4n+3$ divides $x^2 + y^2$ then
%
\nbea
(x^2)^{2n+1} + (y^2)^{2n+1} & = & \left\lbrack (x^2) + (y^2) \right\rbrack \left\lbrack (x^2)^{2n} - (x^2)^{2n-1}(y^2)  +(x^2)^{2n-2}(y^2)^2 \cdots + (y^2)^{2n} \right\rbrack
\neea
%
and hence $p|(x^2)^{2n+1} + (y^2)^{2n+1}$ as well. But
%
\nbea
(x^2)^{2n+1} & = & x^{4n+2} = x^{p-1}
\neea
%
and so if $p\nmid x$ and $p\nmid y$ then because $p|x^{p-1} - 1$ and $p|y^{p-1} - 1$ we have
%
\nbea
(x^2)^{2n+1} + (y^2)^{2n+1} & = & (x^{p-1} - 1) + (y^{p-1} - 1) + 2 \\
& = & pm_x + pm_y + 2 \\
& = & pm_{xy} + 2
\neea
%
therefore we have a contradiction as now $p$ no longer divides $(x^2)^{2n+1} + (y^2)^{2n+1}$ as it differs from a multiple of $p$ by 2. But if one of them, either $x$ or $y$ is divisible by $p$ then (for simplicity let's assume it's $x$)
%
\nbea
(x^2)^{2n+1} + (y^2)^{2n+1} & = & pm_x + (y^{p-1} - 1) + 1 \\
& = & pm_x + pm_y + 1 \\
& = & pm_{xy} + 1
\neea
%
and this time it differs from a multiple of $p$ by 1.

{\bf Page 33, Problem 2}. This is quite a fun one to do, let's do the one for $A = 13$, we start with
%
\nbea
1^2 - A 0^2 & = & 1
\neea
%
multiplying it with $r^2 - A = s$ we get
%
\nbea
r^2 - A(1+r)^2 & = & 1\cdot s
\neea
%
here $k=1$, so now we need to find an $r$ such that $r^2 < A$ but $r^2 - A$ is a negative number, here since $k=1$ we do not need to care if $k|(1+r)$ or not. The answer is $r=3$ such that $s = r^2 - A = -4$ and our next equation is
%
\nbea
3^2 - A 1^2 = -4
\neea
%
multiplying it by $r^2 - A = s$ we get
%
\nbea
(3r + A)^2 - A(3 + r)^2 & = & -4 \cdot s
\neea
%
but now we need to make sure $k = -4$ divides $(3+r)$, an $r$ that works is $r=1$ this way $r^2 < 13$ and $r^2 - A = -12$ is negative and so we get
%
\nbea
4^2 - A1^2 = 3
\neea
%
next, multiplying it with $r^2 - A = s$ again
%
\nbea
(4r + A)^2 - A(4+r)^2 & = & 3\cdot s
\neea
%
here $k = 3$, to make sure $3|(4+r)$ and since $r^2 < 13$ we get $r=2$ and thus
%
\nbea
7^2 - A 2^2 = -3
\neea
%
and I got tired after this :) so the $s$ we recovered so far are $1,-4,3,-3, \dots$

{\bf Page 33, Problem 3}. The first part is straightforward, since $p^2 - Aq^2 = k$ and $P^2 - AQ^2 = K$ with $P = (pr + qA)/|k|$ and $Q = (p + qr)/|k|$
%
\nbea
pQ & = & \frac{p^2 + pqr}{|k|} \\
Pq & = & \frac{pqr + q^2A}{|k|} \\
\to pQ - Pq & = & \frac{p^2 - Aq^2}{|k|} \\
& = & \pm 1
\neea
%
as $|k| = |p^2-Aq^2|$ by definition. Now for the more fun part, since $pQ - Pq = \pm1$, Bezout tells us that $\gcd(Q, P) = 1$, but from $P^2 - AQ^2 = K$, if $\gcd(Q,K) > 1$ then it will also divide $P$ and vice versa, therefore these three are co-prime.

We need this for the next step, we want a new number $R$ such that $QR + P$ is divisible by $K$ or in other words
%
\nbea
QR + P & \equiv & 0 \pmod{K} \\
R & \equiv & Q^{-1}(-P) \pmod{K}
\neea
%
we are guaranteed to have such a $Q^{-1}$ because $\gcd(Q,K) = 1$ and the final step is also straightforward, say
%
\nbea
W \equiv QA + PR & \equiv & QA + P(Q^{-1}(-P)) \pmod{K} \\
W & \equiv & QA - Q^{-1} P^2 \pmod{K} \\
QW & \equiv & Q^2A - P^2 \equiv 0 \pmod{K}
\neea
%
and by definition $K|Q^2A - P^2$ but since $\gcd(Q,K) = 1$ this means that $W \equiv QA + PR \equiv 0 \pmod{K}$ and we are done.



















\end{document}