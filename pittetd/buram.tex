\documentclass[aps,preprint,preprintnumbers,nofootinbib,showpacs,prd]{revtex4-1}
\usepackage{graphicx,color}
\usepackage{caption}
\usepackage{subcaption}
\usepackage{amsmath,amssymb}
\usepackage{multirow}
\usepackage{amsthm}%        But you can't use \usewithpatch for several packages as in this line. The search 

\usepackage{cancel}

%%% for SLE
\usepackage{dcolumn}   % needed for some tables
\usepackage{bm}        % for math
\usepackage{amssymb}   % for math
\usepackage{multirow}
%%% for SLE -End

\usepackage{ulem}
\usepackage{cancel}

\usepackage{hyperref}
\usepackage{mathrsfs}
\usepackage[top=1in, bottom=1.25in, left=1.1in, right=1.1in]{geometry}

\usepackage{mathtools} % for \DeclarePairedDelimiter{\ceil}{\lceil}{\rceil}

%\usepackage{xeCJK}
%\setCJKmainfont{SimSun}

\newcommand{\msout}[1]{\text{\sout{\ensuremath{#1}}}}


%%%%%% My stuffs - Stef
\newcommand{\lsim}{\mathrel{\mathop{\kern 0pt \rlap
  {\raise.2ex\hbox{$<$}}}
  \lower.9ex\hbox{\kern-.190em $\sim$}}}
\newcommand{\gsim}{\mathrel{\mathop{\kern 0pt \rlap
  {\raise.2ex\hbox{$>$}}}
  \lower.9ex\hbox{\kern-.190em $\sim$}}}

%
% Key
%
\newcommand{\key}[1]{\medskip{\sffamily\bfseries\color{blue}#1}\par\medskip}
%\newcommand{\key}[1]{}
\newcommand{\q}[1] {\medskip{\sffamily\bfseries\color{red}#1}\par\medskip}
\newcommand{\comment}[2]{{\color{red}{{\bf #1:}  #2}}}


\newcommand{\ie}{{\it i.e.} }
\newcommand{\eg}{{\it e.g.} }

%
% Energy scales
%
\newcommand{\ev}{{\,{\rm eV}}}
\newcommand{\kev}{{\,{\rm keV}}}
\newcommand{\mev}{{\,{\rm MeV}}}
\newcommand{\gev}{{\,{\rm GeV}}}
\newcommand{\tev}{{\,{\rm TeV}}}
\newcommand{\fb}{{\,{\rm fb}}}
\newcommand{\ifb}{{\,{\rm fb}^{-1}}}

%
% SUSY notations
%
\newcommand{\neu}{\tilde{\chi}^0}
\newcommand{\neuo}{{\tilde{\chi}^0_1}}
\newcommand{\neut}{{\tilde{\chi}^0_2}}
\newcommand{\cha}{{\tilde{\chi}^\pm}}
\newcommand{\chao}{{\tilde{\chi}^\pm_1}}
\newcommand{\chaop}{{\tilde{\chi}^+_1}}
\newcommand{\chaom}{{\tilde{\chi}^-_1}}
\newcommand{\Wpm}{W^\pm}
\newcommand{\chat}{{\tilde{\chi}^\pm_2}}
\newcommand{\smu}{{\tilde{\mu}}}
\newcommand{\smur}{\tilde{\mu}_R}
\newcommand{\smul}{\tilde{\mu}_L}
\newcommand{\sel}{{\tilde{e}}}
\newcommand{\selr}{\tilde{e}_R}
\newcommand{\sell}{\tilde{e}_L}
\newcommand{\smurl}{\tilde{\mu}_{R,L}}

\newcommand{\casea}{\texttt{IA}}
\newcommand{\caseb}{\texttt{IB}}
\newcommand{\casec}{\texttt{II}}

\newcommand{\caseasix}{\texttt{IA-6}}

%
% Greek
%
\newcommand{\es}{{\epsilon}}
\newcommand{\sg}{{\sigma}}
\newcommand{\dt}{{\delta}}
\newcommand{\kp}{{\kappa}}
\newcommand{\lm}{{\lambda}}
\newcommand{\Lm}{{\Lambda}}
\newcommand{\gm}{{\gamma}}
\newcommand{\mn}{{\mu\nu}}
\newcommand{\Gm}{{\Gamma}}
\newcommand{\tho}{{\theta_1}}
\newcommand{\tht}{{\theta_2}}
\newcommand{\lmo}{{\lambda_1}}
\newcommand{\lmt}{{\lambda_2}}
%
% LaTeX equations
%
\newcommand{\beq}{\begin{equation}}
\newcommand{\eeq}{\end{equation}}
\newcommand{\bea}{\begin{eqnarray}}
\newcommand{\eea}{\end{eqnarray}}
\newcommand{\ba}{\begin{array}}
\newcommand{\ea}{\end{array}}
\newcommand{\bit}{\begin{itemize}}
\newcommand{\eit}{\end{itemize}}

\newcommand{\nbea}{\begin{eqnarray*}}
\newcommand{\neea}{\end{eqnarray*}}
\newcommand{\nbeq}{\begin{equation*}}
\newcommand{\neeq}{\end{equation*}}

\newcommand{\no}{{\nonumber}}
\newcommand{\td}[1]{{\widetilde{#1}}}
\newcommand{\sqt}{{\sqrt{2}}}
%
\newcommand{\me}{{\rlap/\!E}}
\newcommand{\met}{{\rlap/\!E_T}}
\newcommand{\rdmu}{{\partial^\mu}}
\newcommand{\gmm}{{\gamma^\mu}}
\newcommand{\gmb}{{\gamma^\beta}}
\newcommand{\gma}{{\gamma^\alpha}}
\newcommand{\gmn}{{\gamma^\nu}}
\newcommand{\gmf}{{\gamma^5}}
%
% Roman expressions
%
\newcommand{\br}{{\rm Br}}
\newcommand{\sign}{{\rm sign}}
\newcommand{\Lg}{{\mathcal{L}}}
\newcommand{\M}{{\mathcal{M}}}
\newcommand{\tr}{{\rm Tr}}

\newcommand{\msq}{{\overline{|\mathcal{M}|^2}}}

%
% kinematic variables
%
%\newcommand{\mc}{m^{\rm cusp}}
%\newcommand{\mmax}{m^{\rm max}}
%\newcommand{\mmin}{m^{\rm min}}
%\newcommand{\mll}{m_{\ell\ell}}
%\newcommand{\mllc}{m^{\rm cusp}_{\ell\ell}}
%\newcommand{\mllmax}{m^{\rm max}_{\ell\ell}}
%\newcommand{\mllmin}{m^{\rm min}_{\ell\ell}}
%\newcommand{\elmax} {E_\ell^{\rm max}}
%\newcommand{\elmin} {E_\ell^{\rm min}}
\newcommand{\mxx}{m_{\chi\chi}}
\newcommand{\mrec}{m_{\rm rec}}
\newcommand{\mrecmin}{m_{\rm rec}^{\rm min}}
\newcommand{\mrecc}{m_{\rm rec}^{\rm cusp}}
\newcommand{\mrecmax}{m_{\rm rec}^{\rm max}}
%\newcommand{\mpt}{\rlap/p_T}

%%%song
\newcommand{\cosmax}{|\cos\Theta|_{\rm max} }
\newcommand{\maa}{m_{aa}}
\newcommand{\maac}{m^{\rm cusp}_{aa}}
\newcommand{\maamax}{m^{\rm max}_{aa}}
\newcommand{\maamin}{m^{\rm min}_{aa}}
\newcommand{\eamax} {E_a^{\rm max}}
\newcommand{\eamin} {E_a^{\rm min}}
\newcommand{\eaamax} {E_{aa}^{\rm max}}
\newcommand{\eaacusp} {E_{aa}^{\rm cusp}}
\newcommand{\eaamin} {E_{aa}^{\rm min}}
\newcommand{\exxmax} {E_{\neuo \neuo}^{\rm max}}
\newcommand{\exxcusp} {E_{\neuo \neuo}^{\rm cusp}}
\newcommand{\exxmin} {E_{\neuo \neuo}^{\rm min}}
%\newcommand{\mxx}{m_{XX}}
%\newcommand{\mrec}{m_{\rm rec}}
\newcommand{\erec}{E_{\rm rec}}
%\newcommand{\mrecmin}{m_{\rm rec}^{\rm min}}
%\newcommand{\mrecc}{m_{\rm rec}^{\rm cusp}}
%\newcommand{\mrecmax}{m_{\rm rec}^{\rm max}}
%%%song

\newcommand{\mc}{m^{\rm cusp}}
\newcommand{\mmax}{m^{\rm max}}
\newcommand{\mmin}{m^{\rm min}}
\newcommand{\mll}{m_{\mu\mu}}
\newcommand{\mllc}{m^{\rm cusp}_{\mu\mu}}
\newcommand{\mllmax}{m^{\rm max}_{\mu\mu}}
\newcommand{\mllmin}{m^{\rm min}_{\mu\mu}}
\newcommand{\mllcusp}{m^{\rm cusp}_{\mu\mu}}
\newcommand{\elmax} {E_\mu^{\rm max}}
\newcommand{\elmin} {E_\mu^{\rm min}}
\newcommand{\elmaxw} {E_W^{\rm max}}
\newcommand{\elminw} {E_W^{\rm min}}
\newcommand{\R} {{\cal R}}

\newcommand{\ewmax} {E_W^{\rm max}}
\newcommand{\ewmin} {E_W^{\rm min}}
\newcommand{\mwrec}{m_{WW}}
\newcommand{\mwrecmin}{m_{WW}^{\rm min}}
\newcommand{\mwrecc}{m_{WW}^{\rm cusp}}
\newcommand{\mwrecmax}{m_{WW}^{\rm max}}

\newcommand{\mpt}{{\rlap/p}_T}

%%%%%% END My stuffs - Stef

\newcommand{\dunno}{$ {}^{\mbox {--}}\backslash(^{\rm o}{}\underline{\hspace{0.2cm}}{\rm o})/^{\mbox {--}}$}

\DeclarePairedDelimiter{\ceil}{\lceil}{\rceil}
\DeclarePairedDelimiter{\floor}{\lfloor}{\rfloor}

\DeclareMathOperator{\re}{Re}


\begin{document}

\title{Jumbled up thoughts}
\bigskip
\author{Stefanus Koesno$^1$\\
$^1$ Somewhere in California\\ San Jose, CA 95134 USA\\
}
%
\date{\today}
%
\begin{abstract}

\end{abstract}
%
\maketitle

\renewcommand{\theequation}{A.\arabic{equation}}  % redefine the command that creates the equation no.
\setcounter{equation}{0}  % reset counter 


Prime numbers form like a set of bases for the integers, just like sinusoids form the bases for any function $f(x)$, the interesting thing is that this (primes as bases) is true for primes under the operation of additions, if we define some other operations then prime numbers won't be bases anymore.

Not sure if any of the techniques we have for spectral analysis can be derived from properties of primes or maybe the other around ~~~~~ \dunno

-=-=-=-=-=-=-=-=-=-=-=-=-=-=-=-=-=-=-=-=-=-=-=-=-=-=-=-=-=-=-=-=-=-=-


See if we can find $k,m > 1$ such that
%
\nbea
(a\pm1)^k & = & a^m \pm 1
\neea
%
where the min of $a \pm 1$ and $a$ is an integer $\ge 2$ (the answer is no and we'll see why). There are four cases total but two of them are equivalent
%
\nbea
(a + 1)^k = a^m + 1 & \longleftrightarrow & (a - 1)^k = a^m - 1 \\
(a + 1)^k = a^m - 1 & \longleftrightarrow & (a - 1)^k = a^m + 1
\neea
%
Let's denote $b = a + 1$ and tackle the case of $(a+1)^k = a^m + 1$ first which is equivalent to $(a - 1)^k = a^m - 1$.

Let's start with $b$ odd and therefore $a$ even
%
\nbea
b^k - 1 & = & a^m \\
(b-1)(b^{k-1} + b^{k-2} + b^{k-3} + \dots + b + 1) & = & a^m, ~~~~~ b-1 = a \\
b^{k-1} + b^{k-2} + b^{k-3} + \dots + b + 1 & = & a^{m-1}
\neea
%
Now if $k$ is odd then on the LHS there will be an even number of terms containing $b$ grouping them two by two
%
\nbea
(b^{k-1} + b^{k-2}) + (b^{k-3} + b^{k-4}) + \dots ( b^2+ b) + 1 & = & a^{m-1}
\neea
%
Each bracket is an even number therefore the total of the LHS is odd due to the $+1$ at the end but the RHS is even since $a$ is even so the above is impossible.

If $k$ is even
%
\nbea
b^{k-2}(b + 1) + b^{k-4}(b+1) + \dots + (b + 1) & = & a^{m-1} \\
(b+1)(b^{k-2} + b^{k-4} + \dots + 1) & = & a^{m-1}, ~~~~~ b+1 = a + 2 \\
\to (a + 2) &|& a^{m-1}
\neea
%
we know that $\gcd(a+2, a) \le 2$ so unless $a + 2 = 4 \to a = 2$, there's a divisor of $a+2$ that doesn't divide $a^{m-1}$ and so $(a+2)\nmid a^{m-1}$, the only thing left is to show that when $a = 2$ we have no solution either.

In the case of $a = 2$, $a + 2 = a^2$ and
%
\nbea
(b^{k-2} + b^{k-4} + \dots + 1) & = & a^{m-3}
\neea
%
but the situation repeats if there are only odd number of terms in the LHS then just like above the sum of the LHS is odd while the RHS is even, if there are an even number of terms in the LHS then we have a similar situation again but this time
%
\nbea
(b^2 + 1)(b^{k-4} + b^{k-8} + \dots + 1) & = & a^{m-3}
\neea
%
but this time since $b^2 + 1 = 10$, the LHS contains 5 as a prime factor while the RHS is a product of 2's only so there are no solutions $k,m>1$ if $a$ is even.

The next case is $a$ odd and $b$ even, in this case just like before we have
%
\nbea
b^{k-1} + b^{k-2} + b^{k-3} + \dots + b + 1 & = & a^{m-1}
\neea
%
Again we split it into two cases, if there are an odd number of terms in the LHS, \ie $k$ is odd, then we can do the following
%
\nbea
b^{k-1} + b^{k-2} + b^{k-3} + \dots + b^2 + b & = & a^{m-1} - 1 \\
b(b^{k-2} + b^{k-3} + \dots + b + 1 )& = & a^{m-1} - 1
\neea
%
Since initially we have an odd number of terms in the LHS after moving the constant 1 to the RHS we now have an even number of terms in the LHS and we can group them two by two like above
%
\nbea
b(b^{k-3}(b + 1) + b^{k-5}(b + 1) + \dots + (b + 1) )& = & a^{m-1} - 1 \\
b(b+1)(b^{k-3} + b^{k-5} + \dots + 1) & = & (a - 1)(a^{m-2} + a^{m-3} + \dots + 1)
\neea
%
while we also expand the RHS like usual. 

We now need more info to move on. Since $k > 1$ and $b$ is even, if we do modulo 4 the LHS is $0 \equiv \pmod{4}$, so to have a chance of a solution $a \equiv -1 \pmod{4}$ and not only that but $m$ has to be odd as well, this means $m-1$ is even and after extracting $(a-1)$ the second bracket in the RHS has an even number of terms as well.

This means we can group the terms two by two as well
%
\nbea
b(b+1)(b^{k-3} + b^{k-5} + \dots + 1) & = & (a - 1)(a^{m-3}(a + 1) + a^{m-5}(a+1) + \dots  + (a+ 1)) \\
\bcancel{b}(b+1)(b^{k-3} + b^{k-5} + \dots + 1) & = & (a - 1)\bcancel{(a + 1)}(a^{m-3} + a^{m-5} + \dots + 1),~~~~~a+1=b
\neea
%
Now, $(b + 1)$ is odd since $b$ is even, also $(b+1)(b^{k-3} + b^{k-5} + \dots + 1)$ is odd since again $b$ is even so the LHS is odd times odd which is odd while in the RHS $(a - 1)$ is even since $a$ is odd so the RHS is even, but we can't have an odd LHS equal to an even RHS so there's no solution for $k$ odd.

If $k$ is even then
%
\nbea
b^{k-1} + b^{k-2} + b^{k-3} + \dots + b + 1 & = & a^{m-1} \\
(b + 1)(b^{k-2} + b^{k-3} + \dots + 1 )& = & a^{m-1}, ~~~~~ b + 1 = a + 2 \\
\to (a + 2) & | & a^{m-1}
\neea
%
But this is the same case as before with $a$ even, $\gcd(a+2,a) \le 2$ but this time since $a$ is odd we don't even have the situation where $a + 2 = 4$ and so there is a divisor of $(a + 2)$ that doesn't divide $a^{m - 1}$ and so $(a+2)\nmid a^{m-1}$.

Now we tackle the case of $(a+1)^k = a^m - 1$ first which is equivalent to $(a - 1)^k = a^m + 1$
%
\nbea
b^k & = & (a-1)(a^{m-1} + a^{m-2} + \dots + a + 1), ~~~~~ a - 1 = b - 2 \\
\to (b-2) & | & b^k
\neea
%
so again, this case can't work because $\gcd(b-2,b) \le 2$ except for $b-2=2, a = 3$ and $b = 4$, in this case $m$ is even thanks to modulo 4 so then
%
\nbea
4^k & = & (3-1)(3^{m-2}(3 + 1) + 3^{m-4}(3 + 1) + \dots + (3 + 1)) \\
2 \cdot 4^{k-1} & = & 4 (3^{m-2} + 3^{m-4} + \dots + 1) \\
2 \cdot 4^{k-2} & = & 3^{m-2} + 3^{m-4} + \dots + 1
\neea
%
Now if there is an odd number of terms in the RHS then we are done because the sum of the RHS is then odd but the LHS is even, but if there are an even number of terms in the RHS we can group them two by two
%
\nbea
2 \cdot 4^{k-2} & = & 3^{m-4}(3^2 + 1) + 3^{m-8}(3^2 + 1) + \dots + (3^2 + 1) \\
& = & 10(3^{m-4} + 3^{m-8} + \dots + 1)
\neea
%
so the RHS contains 5 as a divisor while the LHS doesn't so it won't work.

In conclusion, there is no such $k,m > 1$ such that $(a \pm 1)^k = a^m \pm 1$, for all integers $a > 1$.

Just a side note, we can make this a little more complicated by noticing that
%
\nbea
2^m  = (1 + 1)^m & = & \sum_{l = 0}^m \binom{m}{l}
\neea
%
Also
%
\nbea
(2 + 1)^k & = & \sum_{j = 0}^k\binom{k}{j} 2^j = \sum_{j = 0}^k\binom{k}{j} \sum_{n=0}^j\binom{j}{n}
\neea
%
And also
%
\nbea
(3 + 1)^k & = & \sum_{j = 0}^k\binom{k}{j} 3^j = \sum_{j = 0}^k\binom{k}{j} (2 + 1)^j \\
& = & \sum_{j = 0}^k\binom{k}{j} \sum_{n = 0}^j\binom{j}{n} \sum_{i=0}^n\binom{n}{i}
\neea
%
So for the purpose of wasting space any number $n^k$ can be expressed as a nested binomial expansion
%
\nbea
n^k = ((n-1) + 1)^k = \sum_{j_1 = 0}^k\binom{k}{j_1} \sum_{j_2=0}^{j_1}\binom{j_1}{j_2} \dots \sum_{j_{n-1}=0}^{j_{n-2}}\binom{j_{n-2}}{j_{n-1}}
\neea
%
So the question of whether there are $k,m > 1$ such that $(a+1)^k = a^m + 1$ can be restated as a statement about nested binomial expansions
%
\nbea
\sum_{j_1 = 0}^k\binom{k}{j_1} \sum_{j_2=0}^{j_1}\binom{j_1}{j_2} \dots \sum_{j_{n-1}=0}^{j_{n-2}}\binom{j_{n-2}}{j_{n-1}} = \sum_{i_1 = 0}^m\binom{k}{i_1} \sum_{i_2=0}^{i_1}\binom{i_1}{i_2} \dots \sum_{i_{n-2}=0}^{i_{n-3}}\binom{i_{n-3}}{i_{n-2}} \pm 1
\neea
%

Another side note, any number is a product of power of primes, in that case any number can be expressed as a product of nested binomial sums and in this case we can use binomial sums as some sort of bases ~~~ \dunno

But one other thing, does this mean that $0^0 = 1$? because according to the binomial expansion
%
\nbea
(2 + 0)^3 & = & \binom{3}{0}2^3 \cdot0^0 + \binom{3}{1}2^2 \cdot 0^1 + \binom{3}{2}2^1 \cdot 0^2 + \binom{3}{3}2^0 \cdot 0^3  \\
2^3 & = & 2^3 \cdot0^0 \\
2^3/2^3 & = & 0^0
\neea
%
and so according to binomial expansion $0^0 = 1$ :)


-=-=-=-=-=-=-=-=-=-=-=-=-=-=-=-=-=-=-=-=-=-=-=-=-=-=-=-=-=-=-=-=-=-=-=-

Given that we can factorize the following
%
\nbea
x^k - 1 & = & (x-1)(x^{k-1} + x^{k-2} + \dots + x + 1) \\
x^k + 1 & = & (x+1)(x^{k-1} - x^{k-2} + \dots - x + 1), ~~~~~ k = 2m+1
\neea
%
Let $d = \gcd(x\mp1,x^{k-1} \pm x^{k-2} \pm \dots \pm x + 1)$ then $d|k$

{\it Proof}. Let's first tackle the case of $x^k - 1$. Note that the following is a sum of some constant multiplied by some power of $(x-1)$
%
\nbea
\frac{((x-1) + 1)^k - k(x-1)- 1}{x-1} & = & (x-1)^{k-1} + \binom{k}{1}(x-1)^{k-2} + \dots + \binom{k}{k-2}(x-1)
\neea
%
we need $-k(x-1) - 1$ so that each term in the expansion multiplies some power of $(x-1)$ and so the following produces
%
\nbea
x^{k-1} + x^{k-2} + \dots + 1 - \frac{((x-1) + 1)^k - k(x-1)- 1}{x-1} & = & \frac{x^k - 1}{x-1} - \frac{((x-1) + 1)^k - k(x-1)- 1}{x-1} \\
& = & \frac{x^k - 1 - x^k + k(x-1) + 1}{x-1} \\
& = & k
\neea
%
and so $d|k$ since $d$ divides the LHS.

By the same token, for $x^k + 1$, note that $k$ is odd for it to be factorisable 
%
\nbea
\frac{((x+1) - 1)^k - k(x-1) + 1}{x+1} & = & (x+1)^{k-1} - \binom{k}{1}(x+1)^{k-2} + \dots - \binom{k}{k-2}(x+1)
\neea
%
and
%
\nbea
&& x^{k-1} - x^{k-2} + \dots - x + 1 - \frac{((x+1) - 1)^k - k(x-1) + 1}{x+1} \\
&& ~~~~~ = \frac{(-x)^k - 1}{-x-1} - \frac{((x+1) + 1)^k - k(x-1)+ 1}{x+1}, ~~~~~ k {\rm ~is~odd} \\
&& ~~~~~ = \frac{x^k + 1 - x^k + k(x+1) - 1}{x+1} \\
&& ~~~~~ = k
\neea
%
and so in this case $d|k$ like before

-=-=-=-=-=-=-=-=-=-=-=-=-=-=-=-=-=-=-=-=-=-=-=-=-=-=-=-=-=-=-=-=-

Now back to some physics, last night I was thinking of what it means by combining relativity into quantum theory. We know that one of the main postulates of relativity is that nothing exceeds $c$, the speed of light in vacuum, we'll come back to this in a minute, but what does this mean for quantum mechanics?

The usual explanation is that only observables are physical and therefore their eigenvalues are physical. But does this mean that if we define a velocity operator, then it should not have eigenvalues exceeding $c$? This is usually not the case, as they say amplitudes (or eigenfunctions) are not physical, only the modulo square is.

But somehow it doesn't feel right. If the velocity operator does have eigenvalues exceeding $c$ then if I choose eigenfunctions corresponding to those eigenvalues, I will have non-zero probability of exceeding the speed of light.

But of course, the first question one should ask is, does it make sense to define a velocity operator?

Going back to QFT, we don't have a velocity operator but we do have a momentum operator and as far as I recall there's no limit to maximum eigenvalue this momentum operator can have and I don't see anybody did this in any textbook.

Now, the question is, are wave function amplitudes really of no physical consequence? For Aharanov-Bohm it is, if it is of physical consequence then we must not be so cavalier about saying that as long as the modulo square is zero we are ok. Two, we might need to define a new operator that will cut off at $c$. Or three, we might need to rework the framework of quantum mechanics so that this will be included from the beginning.

Back to the aforementioned ground states, why do we need ground states even in quantum theory, if particles can go through energy barriers why can't they go below ground states? because as long as the modulo square is zero it should be ok right? \ie there's no physical observable of going below the ground state. This might be the reason why quantum gravity is so hard, because a particle can have an amplitude inside singularity, \ie we can have a superposition out of singular and non-singular states.

As I look deeper into this, it reminds me of how different relativity is from Newtonian mechanics. The framework is completely different while the equations are just the consequences of the change in framework. We didn't cut off maximum speed or something like that to get relativity but the concept of space time was completely changed.

In arriving at QFT we didn't do that, the framework of quantum mechanics was not radically altered, we only salted and peppered it to allow Lorentz invariants and transformations, I believe there are things we should modify to get relativistic quantum mechanics, seems like the concept of amplitudes, probabilities, superposition, they all have to change somehow instead of just fiddling around with a new hamiltonian like Dirac did or imposing Lorentz compliant commutator algebra, maybe the algebra itself has got to change

-=-=-=-=-=-=-=-=-=-=-=-=-=-=-=-=-=-=-=-=-=-=-=-=-=-=-=-=-=-=-=-=-

First fact, if $a \equiv b \pmod{c}$ then $(b,c)=(a,c)$, this follows directly from the property of gcd
%
\nbea
(b,c) = (b + cn)
\neea
%
What we have is $d|k$, $(n,d)=1$, what we want is $m \equiv n \pmod{d}$ and $(m,k) = 1$. For simplicity assume that $n \le k$.

The goal here is to find an $x$ such that $m = n + dx$ is co-prime to $k$, since $d|k$, we have $k/d$ distinct values for $x$, \ie $0 \le x < k/d$, before  we get $m = n + k$. If $(n,k)=1$ then $(m = n+k,k)=1$ as well and of course $m \equiv n \pmod{d}$ so we have found our $m$. If $(n,k) > 1$ we have to do more work.

By adding $d$ to $n$ we are actually cycling through different members of the residue class of $k$ and not only that but we are actually cycling through different members, so we are cycling through $k/d$ different members because say two of them are equivalent modulo $k$ then
%
\nbea
n + dx_1 & \equiv & n + dx_2 \pmod{k} \\
\to k &|& d (x_1 - x_2)
\neea
%
but since $0 \le x_{1,2} < k/d$, $d(x_1 - x_2) < k$ so the above is a contradiction.

Let's recap, what we have is a choice of $k/d$ numbers $n_x$ in the form $n_x = n + dx$ with $0 \le x < k/d$, all of which are unique modulo $k$. We want to see if one of them is co-prime to $k$, note also that $(n,k) > 1$.

Now assume that none of the $n + dx$ is co-prime to $k$ so it means that each must have a common divisor with $k$. Let's write $k$ in its primal constituents
%
\nbea
k = p_1^{e_1}p_2^{e_2} \dots p_z^{e_z}
\neea
%
denote the set of distinct prime divisors of $k$ and $d$ as $\mathbb{K} = \{p_1, p_2, \dots, p_z\}$ and $\mathbb{D} = \{p_{d_1}, p_{d_2}, \dots, p_{d_s}\}$ respectively with $\mathbb{D} \subset \mathbb{K}$ then the set of distinct prime divisors of the numbers $n_x$ must be in the set $\mathbb{N}_x = \mathbb{K} - \mathbb{D}$ since $(n,d) = (n_x = n + dx, d) = 1$.

Say $\#\mathbb{N}_x = t$ this means that $k/d \ge p_{j_1}^{e_{j_1}} p_{j_2}^{e_{j_2}} \dots p_{j_t}^{e_{j_t}} > 2^t > t$, \ie some of the $n_x$ will have the same common divisor.

Our task now is to distribute these $t$ primes in $\mathbb{N}_x$ into $k/d$ numbers $n_x = n + dx$. Each $n_x$ can have more than one prime as a divisor of course.

Some important observation, if some $n + dx_i$ is divisible by some $p_{j_u} \in \mathbb{N}_x$ then any $n + dx_i + dp_{j_u}y$ is also divisible by $p_{j_u}$, the converse is also true, if two $n_x$'s are divisible by $p_{j_u}$ then their difference has to be $dp_{j_u}y$.

So we can think of the numbers $n_x$ as a bitmap with $k/d$ number of bits where the each $n_x$ is represented by one bit. Once we choose a bit, say bit $u$ to contain a prime $p_{j_u}$ all other bits with a distance of multiples of $p_{j_u}$ from this bit will also contain $p_{j_u}$.

In this way the bits containing $p_{j_u}$ is $(u + sp_{j_u})$, \ie bit position $u \pmod{p_{j_u}}$ will all contain $p_{j_u}$. Once we assign each bit a prime from $\mathbb{N}_x$ we have the following assignments (we number our bits starting from bit 1 instead of 0)
%
\nbea
\begin{array}{c c c}
bit~position & & prime~assignment\\
b_1 \pm s_1 p_{j_1} \equiv c_1 \pmod{p_{j_1}} & \to & p_{j_1} \\
b_2 \pm s_2 p_{j_2} \equiv c_2 \pmod{p_{j_2}} & \to & p_{j_2} \\
& \vdots & \\
b_t \pm s_{t} p_{j_{t}} \equiv c_t \pmod{p_{j_t}} & \to & p_{j_{t}}
\end{array}
\neea
%
note that bit 1 is just $n$ and since $(n,k) > 1$, $n$ must contain one of the primes and $s_i$ is any integer (of course the total, $b_i \pm s_i p_{j_i}$, can't be less than zero of bigger than the total number of bits, $k/d$). The remaining question is whether there is a bit that wasn't represented by the above assignment and there is, thanks to Chinese Remainder Theorem there must be at least one. We can construct one quite easily, setup the Chinese remainders as follows
%
\nbea
c & \equiv & (c_1 + 1) \pmod{p_{j_1}} \\
c & \equiv & (c_2 + 1) \pmod{p_{j_2}} \\
& \vdots & \\
c & \equiv & (c_t + 1) \pmod{p_{j_t}}
\neea
%
Chinese remainder theorem guarantees that we have such $c$ and that such a $c$ is congruent modulo $p_{j_1} p_{j_2} \dots p_{j_t}$ so this means that $c \le p_{j_1} p_{j_2} \dots p_{j_t}$ which is less than the total number of bits $k/d \ge p_{j_1}^{e_{j_1}} p_{j_2}^{e_{j_2}} \dots p_{j_t}^{e_{j_t}}$. We can also construct other bits by changing the $ + 1$ into some other constant. Note that if the number of bits is less than $t$, the number of distinct primes the argument falls.

Thus we have shown that one of the $n_x$'s must be co-prime to $k$ and we can choose this one to be $m$.

-=-=-=-=- CHINESE REMAINDER POWER -=-=-=-=-

\underline{\textit{\textbf{First Case}}}

Believe it or not, the above proof about bit position and whatnot also serves as a proof that there are infinitely many primes. Suppose we only have finitely many primes $p_1,p_2, \dots, p_n$. I can show that among the first $p_1p_2 \cdots p_n$ natural numbers there are numbers that are not divisible by those primes as such there must be another prime besides those $n$ primes. On top of that I can tell you how many of those numbers there are.

How to see this? Easy, the Chinese to the rescue, set up the following system of congruences
%
\nbea
x & \equiv & a_1 \pmod{p_1} \\
x & \equiv & a_2 \pmod{p_2} \\
& \vdots & \\
x & \equiv & a_n \pmod{p_n}
\neea
%
as long as each $a_i \not\equiv 0 \pmod{p_i}$, $x$ is guaranteed to be not divisible by any of those $n$ primes, the solution of the above is guaranteed by our Chinese friend to exist, but not only that, it is also guaranteed to be less than $p_1p_2 \cdots p_n$.

Now, how many of those numbers are there? Each $a_i$ runs from $1,\dots,p_i-1$, so there are
%
\nbea
(p_1 - 1)(p_2 - 1) \cdots (p_n - 1) = \varphi(p_1p_2\cdots p_n)
\neea
%
which is consistent with what we already know about $\varphi$. The slight difference between this and Euclid's proof is that, this shows that even numbers less than $p_1p_2 \cdots p_n$ already contain new primes, Euclid showed that numbers bigger than $p_1p_2 \cdots p_n$ start requiring new primes but here we show that way before that we already need new primes.

\underline{\textit{\textbf{Second Case}}}

Talking about $\varphi$, our Chinese friend can also help us prove that if $(m,n)=1$ then $\varphi(mn) = \varphi(m)\varphi(n)$. Simply by setting up the following system of congruences
%
\nbea
x & \equiv & a_m \pmod{m} \\
x & \equiv & a_n \pmod{n}
\neea
%
as long as $a_{m,n}$ are co-prime to $m$ and $n$ then it will be co-prime to $mn$ as well. The question now is if there is another number that is co-prime to $mn$ but is not co-prime to $m$ or $n$, well the answer is obviously no because if a number $y$ is co-prime to $mn$ when we break it down to the system of congruences $y \equiv c_m \pmod{m}$ and $y \equiv c_n \pmod{n}$, $c_{m,n}$ will be co-prime to $m$ and $n$ respectively.

Since there are $\varphi(m)$ and $\varphi(n)$ numbers co-prime to $m$ and $n$, our Chinese friend guarantees there are only $\varphi(m)\varphi(n)$ unique solutions modulo $mn$.








-=-=-=-=-=-=-=-=-=-=-=-=-=-=-=-=-=-=-=-=-=-=-=-=-=-=-=-=-=-=-=-=-

Doing problem solving is in a essence doing a depth first search, starting from the root node (which is the problem node), we map out different approaches we can take to tackle the problem, we usually employ greedy algorithm to choose the most promising branch. We then traverse deeper into this particular branch to see if we can get to a solution.

The problem is sometimes we get too comfortable with a certain approach and gravitate towards that branch no matter what (note that greedy algorithm doesn't always pay off). Also, sometimes we need to go too deep into a branch to see if an approach works.

So if we can put some values into the edges of this tree maybe we can deploy Dijkstra or even dynamic programming approaches to choose the best approach in problem solving.

Of course the challenge is that most times we don't even know what branches (approaches) are available, we need to create new ones and that is where the real challenge lies.

-=-=-=-=-=-=-=-=-=-=-=-=-=-=-=-=-=-=-=-=-=-=-=-=-=-=-=-=-=-=-=-=-

How to proof that given two real numbers $a < b$, there is a rational number between them, in fact there are infinitely many rational numbers between them. My strategy is as follows, but first, let's see it with an example, take two numbers $0.1$ and $0.2$, we want to find a rational number, $m/n$, between them where $m$ and $n$ are integers,.

What do we do? what we do is ``stretch'' the interval by multiplying both numbers by some big number so that we can generate integers between them, for example, multiply both by 10 and we get $ [0.1, 0.2] \to [1, 2]$ we haven't inserted an integer between them yet, so we need a bigger number, let's try 20 and we get $ [0.1, 0.2] \to [2, 4]$, and we have an integer between them, \ie 3, and if we divide it by 20 we get our rational number in between 0.1 and 0.2, and if we check, $\frac{3}{20} = 0.15$ is indeed between $0.1$ and $0.2$.

We can generate more integers between them by multiplying them by an even bigger number, say 2000, then the interval becomes $[0.1,0.2] \to [200, 400]$ and this time we have 199 integers in between, if we divide each of those by 2000, we have 199 rational numbers between $[0.1,0.2]$.

So here goes the proof. Take the difference between $a$ and $b$
%
\nbea
\Delta = b - a
\neea
%
the goal is to make this interval to contain integer, so if we pick $n = \ceil{1/\Delta}$
%
\nbea
n(b-a) \ge 1
\neea
%
but here we are still not guaranteed to have an integer in between them, but the solution is obvious, if we want $N$ integers between them we just need to multiply $n$ by $N+1$
%
\nbea
(N+1)n(b-a) \ge N+1
\neea
%
so any integer in between $((N+1)na,(N+1)nb)$ divided by $(N+1)n$ will be the rational number between $(a,b)$ and there are at least $N$ of them and since we can set any $N$ we want there are infinitely many rational numbers between any two real numbers $a$ and $b$.

We can also use the above to show that there are infinitely many irrational numbers between two real numbers $a$ and $b$. The trick is to add an irrational number to both $a$ and $b$, \ie first we pick a rational number between two real numbers $a + \pi$ and $b + \pi$
%
\nbea
a + \pi < \frac{m}{n} < b+\pi
\neea
%
this means that
%
\nbea
a < \frac{m}{n} - \pi< b
\neea
%
since $m/n$ is rational adding $-\pi$ makes it irrational, so that means that there are infinitely many irrational numbers between any two real numbers. However, we can choose any irrational number we want besides $\pi$, something like $\sqrt{2}, \sqrt{3}, \dots$.

The usual proof for the existence of a rational between two real numbers is as follows, select a rational number such that
%
\nbea
\frac{1}{n} & < & b - a
\neea
%
choose $m$ such that
%
\nbea
m-1 \le na < m
\neea
%
so
%
\nbea
\frac{m}{n} - \frac{1}{n} \le a < \frac{m}{n}
\neea
%
therefore
%
\nbea
b = (b-a) + a & > & \frac{1}{n} + \left(\frac{m}{n} - \frac{1}{n}\right) \\
b & > &\frac{m}{n} > a
\neea
%

-=-=-=-=-=-=-=-=-=-=-=-=-=-=-=-=-=-=-=-=-=-=-=-=-=-=-=-=-=-=-=-=-=-=-=-

Proof that $p^{1/n}$ is irrational where $n > 1$ and $p$ is prime, assume it's rational $a/b$ with $(a,b)=1$ then
%
\nbea
\frac{a^n}{b^n} & = & p \\
\to p & | & a^n
\neea
%
which means that $p|a$ which also means $p^{n-1}|b^n$ which also means $p|b$, a contradiction since $(a,b)=1$, the interesting question now is how to design a new irrational number, maybe using continued fraction?

-=-=-=-=-=-=-=-=-=-=-=-=-=-=-=-=-=-=-=-=-=-=-=-=-=-=-=-=-=-=-=-=-=-=-=-

Completeness of the real numbers $\mathbb{R}$ means that $\mathbb{R}$ has no gaps, it is not so with the rationals, for example the set $\{r \in \mathbb{Q}, r^2 < 2\}$ does not have a least upper bound in $\mathbb{Q}$.

Say there is a least upper bound $a/b$ such that $a^2/b^2 > 2$ then
%
\nbea
\frac{a^2}{b^2} - 2& > & 0
\neea
%
we can show that there's another rational number such that it is smaller than $a/b$ but its square is still bigger than 2 rendering it smaller than the least upper bound, a contradiction.

Pick another number $\frac{a}{b+\Delta}$ where $\Delta$ is a rational number, we want this number to still be greater than $\sqrt{2}$
%
\nbea
\frac{a^2}{(b+\Delta)^2} & > & 2 \\
a^2 - 2b^2 - \Delta(4b + 2\Delta) & > & 0
\neea
%
since we can set $\Delta$ to be as small as possible we can always set the above inequality to be true, for example first choose $\Delta = 0.1$ which is a rational number, if this is too big we move the decimal point $\Delta = 0.01$ and so on.

Or we can just invoke the fact that there are infinitely many rational between any two real numbers and we are done.

-=-=-=-=-=-=-=-=-=-=-=-=-=-=-=-=-=-=-=-=-=-=-=-=-=-=-=-=-=-=-=-=-=-=-=-=-=-=-=-=-=-=-=-=-

We can also extend congruence to rational numbers
%
\nbea
\frac{a}{b} & \equiv & c \pmod{d}
\neea
%
as long as $(b,d) = 1$, we can think of it as $a b^{-1} \equiv c \pmod{d}$ and $b$ has an inverse as long as $(b,d) = 1$.

Because if $d$ divides
%
\nbea
\frac{a}{b} - c & = & \frac{a - cb}{b}
\neea
%
then $d|(a-cb)$ or $a - cb \equiv 0 \pmod{d}$ however, if $(b,d) \neq 1$ then we know from our experience that for $c$ to have a solution $(b,d)$ must divide $a$, so if $(b,d) = 1$ we are guaranteed to have a solution.

What happens if we add two fractions?
%
\nbea
\frac{a}{b} + \frac{a'}{b'} = \frac{ab' + a'b}{bb'} & \equiv & C \pmod{d}
\neea
%
suppose $C \not\equiv c + c' \pmod{d}$ that means $C \equiv c + c' + \Delta \pmod{d}$ and
%
\nbea
\frac{ab' + a'b - Cbb'}{bb'} & \equiv & \frac{ab' + a'b - (c + c' + \Delta)bb'}{bb'} \pmod{d} \\
& \equiv & \frac{(a - cb)b' + (a' - c'b')b - \Delta bb'}{bb'} \pmod{d} \\
& \equiv & -\Delta \pmod{d}
\neea
%
meaning $\Delta \equiv 0 \pmod{d}$ and therefore $C \equiv c + c' \pmod{d}$.

With this in mind we can show that
%
\nbea
\sum_{x=1}^{p-1} \frac{1}{x} \equiv 0 \pmod{p}
\neea
%
where $p$ is an odd prime. This is because $1/x \equiv x^{-1} \pmod{p}$ so the sum becomes
%
\nbea
\sum_{x=1}^{p-1} x^{-1} \pmod{p}
\neea
%
and we know that the inverse of every element of $(\mathbb{Z}/p\mathbb{Z})^*$ is unique therefore
%
\nbea
\sum_{x=1}^{p-1} x^{-1} \pmod{p} & \equiv & \sum_{x=1}^{p-1} x \pmod{p} \\
& \equiv & \frac{(p-1)p}{2} \pmod{p} \\
& \equiv & 0 \pmod{p}
\neea
%
another show of force of the power of congruence :)

-=-=-=-=-=-=-=-=-=-=-=-=-=-=-=-=-=-=-=-=-=-=-=-=-=-=-=-=-=-=-=-=-=-=-=-=-=-=-=-=-=-=-=-=-

Another fun thing to do with congruence, given
%
\nbea
2^4 + 5^4 = 2^7\cdot 5 + 1 = 641
\neea
%
show that $2^{32} + 1 \equiv 0 \pmod {641}$. The above two equations mean that
%
\nbea
2^7\cdot 5 + 1 & \equiv & 0 \pmod{641} \\
5 & \equiv & -(2^7)^{-1} \pmod{641}
\neea
%
so
%
\nbea
2^4 + 5^4 & \equiv & 2^4 + ( -(2^7)^{-1})^4 \pmod{641} \\
0 & \equiv &  2^4 + 2^{-28} \pmod{641} \\
\to 0 & \equiv &  2^{32} + 1 \pmod{641} 
\neea
%
since $(2,641)=1$, $2$ has an inverse.


-=-=-=-=-=- PRIME NUMBER THEOREM -=-=-=-=-=-

What do you need to analytically prove the prime number theorem? Here's a list of things
%
\bit
%
\item Equivalence between $\pi(x) \sim x/\log x$ and $\psi(x) \sim x$.
%
\item Integral of $\psi$, $\psi_1 = \int_1^\infty \psi(x) dx$ and how to express this integral in terms of a sum of $\Lambda(n)$, answer: Abel's identity.
%
\item The correspondence between the asymptotic form of $\psi_1$ and $\psi$, \ie the asymptotic form of $\psi$ is just the derivative of the asymptotic form of $\psi_1$, since $\psi_1$ is the integral of $\psi$.
%
\item Integral form of $\psi_1/x^2$. To do this we need the summation form of $\zeta'(s)/\zeta(s)$, this is gathered through the properties of Dirichlet series.
%
\item Various properties of $\zeta(s)$ and $\zeta'(s)$ along the line $\sigma = 1$ so that we can integrate the above along $\sigma=1$.
%
\item A very important one, Riemann-Lebesgue lemma to show that the integral above is zero.
%
\item And finally sum subtle technical things where we exchange sums and integrals.
\eit
%

-=-=-=-=-=-=-=-=-=-=-=-=-=-=-=-=-=-=-=-=-=-

%
\nbea
n! > \left(\frac{n}{e}\right)^n
\neea
%

Start with $k$ an integer starting from 2
%
\nbea
k & \ge & x, ~~~~~~~ x \ge 1 \\
\to \log k & \ge & \log x \\
\int_{k-1}^{k} \log k~ dx = \log k & \ge & \int_{k-1}^{k} \log x~ dx
\neea
%
so
%
\nbea
\sum_{k=2}^n \log k & \ge & \sum_{k=2}^n \int_{k-1}^{k} \log x~ dx \\
\to \log n! & \ge & \int_1^n \log x~dx \\
\log n! & \ge & n \log n - n + 1 \\
\log n! - (n\log n - n) & \ge & 1
\neea
%
exponentiating, we get the inequality we want.


-=-=-=-=-=-=-=-=-=-=-=-=-=-=-=-=-=-=-=-=-=-

Show that $y^2 = x^3 + 16$ only has $(x,y) = (0,\pm 4)$ as solutions.
%
\nbea
(y - 4)(y + 4) & = & x^3
\neea
%
For $y$ odd $\gcd(y-4,y+4) = 1$ since the common divisors must divide $8 = 2\cdot2\cdot2$ but $y \pm 4$ is odd so it doesn't have 2 as a factor, thus their gcd is 1. In this case $y-4$ and $y+4$ are both cubes but there are not cubes separated by 8.

This means $x$ and $y$ are both even, but if that's the case
%
\nbea
(2y_1)^2 & = & (2x_1)^3 + 16 \\
y_1^2 & = & 2x_1^3 + 4 
\neea
%
this means $2|y_1$ thus
%
\nbea
(2 y_2)^2 & = & 2x_1^3 + 4 \\
2y_2^2 & = & x_1^3 + 2
\neea
%
this means $2|x_1$
%
\nbea
2y_2^2 & = & (2x_2)^3 + 2 \\
y_2^2 & = & 4x_2^3 + 1
\neea
%
and
%
\nbea
(y_2 - 1)(y_2 + 1) & = & 4x_2^3
\neea
%
if $y_2$ is even we are done because the LHS is odd and the RHS is even, so $x_2 = 0$, if $y_2$ is odd then $(y_2\pm1)/2 = a_{\pm}$ and $a_+ - a_- = 1$
%
\nbea
a_+a_- & = & x_2^3 \\
a_+(a_+ - 1) & = & x_2^3
\neea
%
and $\gcd(a_+,a_+ - 1) = 1$ so each of them must be a cube but there are no two cubes separated by 1. Except for $\pm1$ and $0$ but that also means that $x_2 = 0$.

-=-=-=-=-=-=-=-=-=-=-=-=-=-=-=-=-=-=-=-=-=-

Here's my struggle in coming up with a proof of $p = a^2 + b^2$ for $p \equiv 1 \pmod{4}$, so, many twists and turns later here it is :)

The first thing I notice was that since $p \equiv 1 \pmod{4}$,
%
\nbea
(-1)^{\frac{p-1}{2}} & = & (-1)^{\frac{(4m+1)-1}{2}} \\
& = & (-1)^{2m} \\
& = & +1
\neea
%
and therefore $-1$ is a quadratic residue say $\equiv a^2 \pmod{p}$, therefore
%
\nbea
a^2 x^2 & \equiv & -x^2 \pmod{p}
\neea
%
this is true for {\bf any} $x \in (\mathbb{Z}/p\mathbb{Z})^*$. The trick now is to show that if we transform the above to equality rather than equivalence we get 
%
\nbea
y^2 + x^2 & = & p
\neea
%
But it was not too obvious yet how to get there, the only thing I can see is induction, like so
%
\nbea
a^2x^2 + x^2 & \equiv & 0 \pmod{p} \\
& = & pw
\neea
%
My strategy was to show that if all primes (1 mod 4) up to $p$ are of the form $a^2 + b^2$ we can show that $p$ must be a sum of two squares too. But still didn't know where to go yet.

Another thing I stumbled upon in this misadventure is Wilson's Theorem, for a prime $p$
%
\nbea
(p-1)! & \equiv & -1 \pmod{p} \\
1\cdot2\cdots \frac{p-1}{2} \cdot \left(-\frac{p-1}{2}\right) \cdots (-2)\cdot (-1) & \equiv & -1 \pmod{p} \\
\left \lbrack\left ( \frac{p-1}{2} \right ) ! \right \rbrack^2 (-1)^{\frac{p-1}{2}} & \equiv & -1 \pmod{p}
\neea
%
for $p = 4m+1$ we have
%
\nbea
\left \lbrack\left ( 2m \right ) ! \right \rbrack^2 & \equiv & -1 \pmod{p}
\neea
%
so the roots of $-1$ are
%
\nbea
\left ( 2m \right ) ! ~~~~~~~{\rm and} ~~~~~~~ -\left ( 2m \right ) !
\neea
%
I was hoping that an ``explicit'' form of the root of $-1$ would get me somewhere but it went nowhere quite fast :)

Back to the induction attempt, this shows that
%
\nbea
\left \lbrack\left ( 2m \right ) ! \right \rbrack^2 + 1^2 & \equiv & 0 \pmod{p} \\
& = & pw
\neea
%
If we inductively assume that all primes ($\equiv 1 \pmod{4}$) up to $p = 4m+1$ has the form $a^2 + b^2$ we can also deduce that $p$ is also of that form, the problem is that $w$ will certainly contain other prime factors much bigger than $p$ since $(2m)!$ is a very large number, so induction fails at this stage.

But worry not, we still have some tricks up our sleeves :) The good thing is that we work with equivalences, so we can use the smaller equivalence of $(2m)!$ to minimize the factorial, say $r \equiv (2m)! \pmod{p}$ with $r < p$, this is guaranteed, I even (unnecessarily) minimized it further by noting that we have two roots of $-1$ and we can choose the smaller one. But again, it's not necessary. So what we have now is
%
\nbea
\left \lbrack\left ( 2m \right ) ! \right \rbrack^2 + 1^2 \equiv r^2 + 1^2 & \equiv & 0 \pmod{p} \\
\to r^2 + 1^2 & = & pw
\neea
%
Another worry I had (again unnecessary) was that $w$ can contain primes bigger than $p$, and that would render the induction useless because instead of possible infinite descend we will get an infinite ascend :)

A break through came after I played around with the actual numbers using maxima, the code is shown below
%
\begin{verbatim}
t:0;
a:1277;
for j:1 thru 100 do
block(
    a:next_prime(a),
    if (mod(a,4) = 1) then 
    block(
        t:mod(factorial((a-1)/2),a),
        t:ifactors(t^2+1),
        display(t),
        display(a),
        for i:1 thru length(t) do
        block(
            if (mod(t[i][1],4)>2) then
            block(
                display(t[i][1])
            ) 
        )
    )  
);
display(t);
\end{verbatim}
%
I saw that $w$ only contains have primes less than $p$ and another thing, which I already observed previously, is that all prime factors of $w$ is either 2 or of the form $4u+1$.

Then I realized, of course!, $w$ can only contain prime factors less than (or equal to) $p$ because since $r < p$, we must have $r^2 + 1 \le p^2$ and so $w$ cannot have factors bigger than $p$ (if $r^2 + 1 = p$ then we are done so we need not care of this case). So our induction is on track.

The next thing we need to show is that $w$ can only contain prime factors of the form $4u+1$ or 2, it cannot contain a prime factor of the form $4u-1$ because assume the opposite, then
%
\nbea
r^2 + 1 & = & p s w', ~~~~~~~~ s{\rm~of~the~form~} 4u-1, ~~w = sw' \\
& \equiv & 0 \pmod{s} \\
r^2 & \equiv & - 1 \pmod{s}
\neea
%
but for a prime of the form $4u-1$, $(-1)$ cannot be a quadratic residue so the above is false and we show that $w$ cannot contain a prime factor of the form $4u-1$.

The next and final ingredient we need what I saw some time ago regarding the multiplication of two sums of squares
%
\nbea
(a^2 + b^2)(c^2 + d^2) & = & (ac + bd)^2 + (ad - bc)^2
\neea
%
\ie f we have two sums of squares we can multiply them together to get another sum of squares. It can be easily seen if we express them in complex numbers
%
\nbea
A\cdot C & = & (a+ib)(c+id) \\
\to |A||C| & = & |(a+ib)(c+id)| \\
(a^2 + b^2)(c^2 + d^2) & = & (ac + bd)^2 + (ad - bc)^2
\neea
%
another way of looking at it, which is my favorite is to write them down as matrices, write the complex numbers as follows
%
\nbea
A = \left (
\begin{array}{c c}
a & b \\
-b & a
\end{array} \right )
~~~
C = \left (
\begin{array}{c c}
c & d \\
-d & c
\end{array} \right ) \longrightarrow AC = \left (
\begin{array}{c c}
(ac + bd) & (ad - bc) \\
-(ad - bc) & (ac + bd)
\end{array} \right )
\neea
%
If we now take the determinant, noting that determinant of a product is the product of the determinants
%
\nbea
 |AC| = |A||C| & = & \left |
\begin{array}{c c}
(ac + bd) & (ad - bc) \\
-(ad - bc) & (ac + bd)
\end{array} \right | \\
\to (a^2 + b^2)(c^2 + d^2) & = & (ac + bd)^2 + (ad - bc)^2
\neea
%
Why do we need all this? From our induction we know that
%
\nbea
r^2 + 1^2 & = & pw
\neea
%
$w$ might only contain 2 or primes less than $p$ of the form $4u+1$ and by our hypothesis they are all sums of squares and from above we know that product of sums of squares is again a sum of squares (note that $2 = 1^2 + 1^2$) and so we peel of this equation one layer (one prime factor) at a time
%
\nbea
r^2 + 1^2 & = & pw \\
r^2 + 1^2 & = & pw_1w_2\cdots w_n
\neea
%
say $w_1 = a^2 + b^2 \to (r^2 + 1^2) = P(a^2 + b^2) = P w_1 $, if we write this in matrix form
%
\nbea
\left (
\begin{array}{c c}
r & 1 \\
-1 & r
\end{array} \right ) & = & P \cdot 
\left (
\begin{array}{c c}
a & b \\
-b & a
\end{array} \right ) \\
\to \frac{1}{(a^2 + b^2)}\left (
\begin{array}{c c}
a & -b \\
b & a
\end{array} \right )
\left (
\begin{array}{c c}
r & 1 \\
-1 & r
\end{array} \right ) & = & P \\
\frac{1}{(a^2 + b^2)}
\left (
\begin{array}{c c}
ar + b & a - br \\
-(a - br) & a + br
\end{array} \right ) & = & P
\neea
%
If we take the determinant of the LHS
%
\nbea
\frac{1}{(a^2 + b^2)^2} \lbrack(ar+b)^2 + (a-br)^2\rbrack & = & \frac{1}{(a^2 + b^2)^2} \lbrack(a^2+b^2)(r^2 + 1^2)\rbrack \\
\left(\frac{ar + b}{a^2 + b^2}\right)^2 + \left(\frac{a - br}{a^2 + b^2}\right)^2 & = & \frac{(r^2 + 1^2)}{(a^2+b^2)}
\neea
%
The RHS is just $P$ and it is an integer, we need to show that each term in the LHS is an integer to show that we can decompose $P$ in terms of sum of two squares. From
%
\nbea
(ar + b)(ar-b) & = & a^2r^2 - b^2 \\
& = & a^2r^2 + (b^2r^2 - b^2r^2) - b^2 \\
& = & r^2(a^2 + b^2) - b^2(r^2 + 1^2)
\neea
%
since $a^2 + b^2$ divides $r^2 + 1^2$ it divides the LHS and since it is prime it either divides $(ar + b)$ or $(ar-b)$, say it divides $(ar+b)$ we need to show that it divides $(a-br)$ but it follows immediately since
%
\nbea
(a^2 + b^2)(r^2 + 1) & = & (ar+b)^2 + (a-br)^2
\neea
%
since the LHS and the first term in the RHS are divisible by $a^2 + b^2$ it must divide the last term on the RHS as well.

Now suppose $a^2 + b^2$ divides $(ar-b)$ we need to rewrite the above as (from the fact that $(a^2 + b^2)(c^2 + d^2) = (a^2 + b^2)(d^2 + c^2)$, \ie the product of sum of squares formula is invariant under $a\leftrightarrow b, c\leftrightarrow d$)
%
\nbea
\left(\frac{ar + b}{a^2 + b^2}\right)^2 + \left(\frac{a - br}{a^2 + b^2}\right)^2 & = & \left(\frac{a + br}{a^2 + b^2}\right)^2 + \left(\frac{ar - b}{a^2 + b^2}\right)^2
\neea
%
and from
%
\nbea
(a^2 + b^2)(r^2 + 1)  = (a^2 + b^2)(1 + r^2) & = & (a+br)^2 + (ar-b)^2
\neea
%
using the same argument, since $a^2 + b^2$ divides the LHS and the last term on the RHS it must also divide the first term in the RHS. And so what we have is
%
\nbea
P & = & \left \{
\begin{array}{l }
\left(\frac{ar + b}{a^2 + b^2}\right)^2 + \left(\frac{a - br}{a^2 + b^2}\right)^2 ~~~~~{\rm if~} a^2 + b^2 ~{\rm divides~} (ar+b) \\ 
\left(\frac{a + br}{a^2 + b^2}\right)^2 + \left(\frac{ar - b}{a^2 + b^2}\right)^2 ~~~~~{\rm if~} a^2 + b^2 ~{\rm divides~} (a-br)
\end{array} \right.
\neea
%
so what we need is just repeat the process with $w_2, w_2, \dots$ until we only have $p$ on the RHS of $r^2 + 1^2 = pw_1w_2\cdots w_n$ and therefore $p$ is also a sum of two squares, doing this continually inductively we show that every prime $p = 4m+1$ can be expressed as a sum of two squares.

Another feature is that this sum is unique (up to ordering of course since $a^2 + b^2 = b^2 + a^2$), so let's choose a convention that if $p = a^2 + b^2$ then $a<b$ (since $p$ is prime $a \neq b$), so say we have two sums representing the same prime
%
\nbea
p & = & a^2 + b^2 \\
& = & c^2 + d^2 \\
\to p^2 & = & (ac + bd)^2 + (ad-bc)^2 \\
& = & (ad + bc)^2 + (ac-bd)^2
\neea
%
again using the product of sums of squares formula, since $p$ is prime, the three terms are all co-prime and and we get a primitive pythagorean triple
%
\nbea
p^2 & = & (m^2 + n^2)^2 = (2mn)^2  + (m^2 - n^2)^2 = (ac + bd)^2 + (ad-bc)^2 \\
& = & (k^2 + l^2)^2 ~~= (2kl)^2  + (k^2 - l^2)^2 ~~~~= (ad + bc)^2 + (ac-bd)^2 
\neea
%
at this point I got somewhat stuck, because even if we generate two more sums it's still OK as long as they don't generate infinitely many more, my initial strategy was to show there's an infinite descend but it might not exist. So, this endeavor turned out to be quite tricky, I thought this would be a quick proof by contradiction but then again :)

We can also try solutions to Diophantine equations of the form $m^2 + n^2 = k^2 + l^2$ but this is just over the top.

Another strategy was to show that all pythagorean triples are unique, \ie no two triples have the same hypotenuse, but this is evidently wrong as 65 serves as a counter example, $(16,63,65)$ and $(33,56,65)$, but this counter example also gave me an idea.

So $65 = 1^2 + 8^2 = 4^2 + 7^2$, since we have two sums of squares I tried combining them, but in doing so I found the new solutions are $(25,60,65)$ and $(39, 52, 65)$, it's no longer primitive, so this means that the new solutions above are actually divisible by $p$. So here it goes
%
\nbea
(ac + bd)(ac - bd) & = & a^2c^2 - b^2d^2 \\
& = & a^2c^2 + (b^2c^2 - b^2c^2)- b^2d^2 \\
& = & (a^2 + b^2)c^2 - b^2(c^2 + d^2) \\
& = & p(c^2 - b^2)
\neea
%
since $p$ is prime, it must either divides $(ac + bd)$ or $(ac - bd)$ ($c^2 - b^2$ cannot be zero by assumption). If it divides $(ac + bd)$ then since
%
\nbea
p^2 & = & (ac + bd)^2 + (ad-bc)^2 \longrightarrow (ac + bd)^2 \le p^2
\neea
%
we must have $(ac + bd)^2 = p^2$ but this means
%
\nbea
0 & = & (ad-bc)^2 \\
\to \frac{a}{b} & = & \frac{c}{d}
\neea
%
the only possible solution for this is that $a=c, b=d$. If $p$ divides $(ac - bd)$, from the other representation
%
\nbea
p^2 & = & (ad + bc)^2 + (ac-bd)^2 \longrightarrow (ac-bd)^2 \le p^2
\neea
%
we have 
%
\nbea
0 & = & (ad + bc)^2
\neea
%
which is just not possible, therefore there is only one possible sum of squares representation for $p$.

This uniqueness part can be proven mighty easily using the Gaussian Integer field, since this field is Euclidean (for any two numbers in this field $a, b$ we can express them as $a = bq + r$), every number can be factorized uniquely, therefore
%
\nbea
p & = & (a + ib)(a - ib)
\neea
%
since $a \pm ib$ are not related by units (the only units are $\pm1$ and $\pm i$) the above factorization of $p$ is unique and we are done.

\smallskip
\underline{\textit{\textbf{Bonus Featurette}}}
\smallskip

Another thing I observed during this misadventure was that any number (not just primes) of the form $n = 4m-1$ cannot be expressed as a sum of two squares, this is because $n$ must have a prime factor $p$ of the form $4u-1$, if $n$ can be expressed as a sum of two squares
%
\nbea
a^2 & \equiv & - b^2 \pmod{n} \\
a^2 & \equiv & - b^2 \pmod{p} \\
\left(\frac{a}{b}\right)^2 & \equiv & - 1 \pmod{p}
\neea
%
but $-1$ cannot be a quadratic residue mod $p = 4u-1$.

\underline{\textit{\textbf{The other lemma}}}, another thing is that a prime power of $p^k, p = 4u-1$ can be expressed a sum of two squares if $k$ is even and one of the number in the sum is zero, \ie $p^k = 0^2 + (p^{k/2})^2$ is the only sum of square representation for $p =4u-1$, this is due to the same reasoning above, say we assume the opposite
%
\nbea
a_0^2 & \equiv & - b_0^2 \pmod{p^k} \\
\left(\frac{a_0}{b_0}\right)^2 & \equiv & - 1 \pmod{p}
\neea
%
which is not allowed (since $-1$ is not a quadratic residue) unless both $a_0 \equiv b_0 \equiv 0 \pmod{p}$ but in this case, $p$ divides both $a_0$ and $b_0$ thus
%
\nbea
p^2 a_1^2 & = & -p^2 b_1^2 + p^{k} \\
a_1^2 & = & -b_1^2 + p^{k-2}
\neea
%
and we have the same situation again, repeating the process, if $k$ is even then we will get 
%
\nbea
a_n^2 & = & -b_n^2 + 1
\neea
%
which means that either $a_n$ or $b_n$ is zero, they can't both be zero because $a_0^2 + b_0^2 = p^k$, if $k$ is odd then
%
\nbea
a_n^2 & = & -b_n^2 + p
\neea
%
but this time we've got no solution, not even if $a_n$ or $b_n$ is zero, thus if $k$ is odd there's no solution while if $k$ is even then $p^k = (p^{k/2})^2 + 0^2$.

From the uniqueness claim, we can also say that if you draw a circle in the complex plane whose radius is a prime number, this circle will hit a lattice point exactly twelve or four times. Because if $p = 4m+1 = a^2 + b^2$, the points are $2ab\pm i(b^2-a^2),-2ab\pm i(b^2-a^2)$ and the same points with real and imaginary parts exchanged and of course the obvious ones $\pm p, \pm ip$, if on the other hand $p = 4m-1$ then the only points are just $\pm p, \pm ip$. Or if you like roots, then if $p = 4m+1$ , a circle with radius $\sqrt{p}$ has eight lattice points while a circle with radius $\sqrt{p}$ where $p=4m-1$ has none.

The reason I even got into this circular business is that because I wanted to prove the uniqueness part geometrically but it was quite troublesome to do so :)

\underline{\textit{\textbf{Pythagoras and friends}}}, let's start with the product of sums of squares, say we have three of them, are they associative? The answer is yes since we can represent them in terms of matrices and matrices are associative under multiplication.

Next question, what are the symmetries of this product of sums of squares? from $a\leftrightarrow b$, $c\leftrightarrow d$ and simultaneous $a\leftrightarrow b$ and $c\leftrightarrow d$ we get
%
\nbea
(ac + bd)^2 + (ad - bc)^2 \\
(bc + ad)^2 + (bd - ac)^2 \\
(ad + bc)^2 + (ac - bd)^2 \\
(bd + ac)^2 + (bc - ad)^2
\neea
%
we see that there are only two possibilities for the first term while there are four for the second although they come in plus minus pairs.

This plus minus thing is important when we multiply more than two sums of squares, so we need to see what happens when we apply a minus sign to $a,b,c,d$ but from the pattern above doesn't matter where we apply the minus sign or to how many of the elements we will just get the same things with just plus and minus cycled through different elements. Thus, if we ignore the plus minus sign, from two sums of squares we will get only two sets of sums of squares.

Next, how many sum of squares representation does $p^k$ have? $p = 4m+1$ of course. Let's start from $k=2$, in this case
%
\nbea
p^2 & = & (a^2 + b^2)^2 + 0^2 \\
p^2 & = & (2ab)^2 + (a^2 - b^2)^2
\neea
%
Let's call a sum of squares where both squares are non-zero and co-prime as a proper sum. Thus, $p^3$ 
%
\nbea
(a(2ab) + b(a^2 - b^2))^2 + (a(a^2 - b^2) - b(2ab))^2 \\
(a(a^2 - b^2) + b(2ab))^2 + (a(2ab) - b(a^2 - b^2))^2
\neea
%
has 2 proper reps, the pattern I saw was for $p^k$ with $k$ even has the same number of proper reps as $p^{k-1}$ and $p^{k-1}$ has twice the number of proper reps of $p^{k-2}$. This can be shown quite easily, the key is that multiplication is associative thus
%
\nbea
p^{k} & = & \left\{
\begin{array}{l l}
(p^2)(p^2)\cdots(p^2)p, & ~~~~~~~k~{\rm odd} \\
(p^2)(p^2)\cdots(p^2), & ~~~~~~~k~{\rm even}
\end{array}
\right.
\neea
%
each $p^2$ only generates one proper rep, the same with $p$, thus for $k$ odd we get in total $2^{\frac{k-1}{2}}$ proper reps, for $k$ even we get $2^{\frac{k}{2}-1}$ proper reps or in a fancier expression
%
\nbea
2^{\left\lfloor\frac{k-1}{2}\right\rfloor}
\neea
%
The reason of defining these proper reps is to see that for given a hypotenuse how many primitive triples we can build on it. First of a primitive hypotenuse cannot have prime factors of the form $4m-1$, that is quite obvious from our previous discussion. Second, from our preceding discussion, the number of primitive triples of a hypotenuse, $h$, will be of the form $2^{N}$ where
%
\nbea
2^N = 2^{M-1}\prod_{i=1}^M 2^{\left\lfloor\frac{k_i-1}{2}\right\rfloor} ~~~~~~~{\rm with} ~~~~~~~ h = \prod_{i=1}^M p^{k_i}
\neea
%
\underline{\textit{What this means}} is that given a hypotenuse $h$, there's only one primitive right triangle we can construct if $h$ is prime or a prime squared otherwise we get multiple primitive triangles out of it :)

\underline{\textit{\textbf{On Thue's Lemma}}}, this is one cool application of the pigeons :) let $0 \le x,y \le \floor{\sqrt{n}}$, $n$ need not be prime and given a number $a$, construct the following
%
\nbea
ax - y \pmod{n}
\neea
%
since there are possible $(\floor{\sqrt{n}} + 1)^2 > n^2$ values for $x$ and $y$ and there are only $n$ values $\pmod{n}$, there must be duplicates, this is where the pigeons come along :)

Assume we have two different sets of $(x,y)$ such that their construction above is the same then
%
\nbea
ax_1 - y_1 \equiv ax_2 - y_2 \pmod{n} \\
\to a(x_1 - x_2) \equiv (y_1 - y_2) \pmod{n}
\neea
%
let $w = x_1 - x_2$ and $z = y_1 - y_2$ then we claim that both $w$ and $z$ are non-zero because is $w$ is zero then $z$ is also zero but our assumption is that $(x_1,y_1)$ and $(x_2,y_2)$ are two distinct sets the second and more important (because this is what we want) thing is that $|w|$ and $|z|$ must be $\le \floor{\sqrt{n}}$ but this is guaranteed as $0 \le x,y \le \floor{\sqrt{n}}$. However, we must be careful that $x_1,x_2,y_1,y_2 \neq 0$ but again this is guaranteed as when $x_1 \equiv 0$ then $y_1 \equiv 0$ again this is fine because $(x_2,y_2)$ are a distinct set from $(x_1,y_1)$ so $w$ and $y$ won't be zero.

From this proof we can see that the solution to $p = a^2 + b^2$ comes from the difference $x_1-x_2$ rather than from $x_{1,2}$ themselves which explains my pathetic attempt in proving Thue's lemma without them pigeons :) because there I constructed a multiplication table for $ax \equiv y$ and tried to find a pattern I can use but the pattern is in the difference not the direct result of the multiplications.

\underline{\textit{\textbf{On 65}}}, now the fun part, if we combine the representations of sum of squares of $65$ repeatedly, will we get an infinite descend? The answer sadly is no because now since $65$ is not prime
%
\nbea
(ac + bd)(ac - bd) & = & (a^2 + b^2)c^2 - b^2(c^2 + d^2)
\neea
%
we can't claim that $(a^2 + b^2) = (c^2 + d^2) = 65$ divides $(ac + bd)$ or $(ac - bd)$, what we can say is that each prime factor of 65 divides either $(ac + bd)$ or $(ac - bd)$. Say one of its prime factor is $q$ and it divides $(ac + bd)$, this means
%
\nbea
65^2 & = & (ac + bd)^2 + (ad - bc)^2 \\
65^2 & = & (qq')^2 + (ad - bc)^2
\neea
%
but $q|65$ as well, this means what we get is no longer a primitive triple (the triples are no longer co-prime to one another), therefore this new representation is no longer $65 = u^2 + v^2$ instead it's $s\cdot(u^2 + v^2)$, thus we can't combine it we the previous two to get new ones. Thus

\underline{\textit{\textbf{65 Stupid Lemma}}}, if the same hypotenuse occurs in two primitive pythagorean triples, then combining these two using the product of sums of squares will not produce other primitive pythagorean triples. This doesn't mean that the same hypotenuse can only have two primitive triples but if you combine it this way it won't produce more primitive triples.

A hypotenuse can have more primitive triples because say it has three prime factors, $p_{1,2,3}$, all of the form $4m + 1$, then combining $p_1$ and $p_2$ alone (using the product of sums of squares) we already get two possible outputs, combining it with $p_3$ we get a total of four, for more details see previous subsection.

\smallskip
\underline{\textit{\textbf{Other failed attempts}}}
\smallskip

Throughout my meandering, the most painful attempt was trying to find an explicit pattern in the mutiplication table mod $p$, this is after I found out about Thue's Lemma which states that there must be $x,y$ with $0<|x|,|y|\le \floor{\sqrt{n}}$ such that $ax \equiv \pm y \pmod{n}$, why is this relevant? If both $|x|$ and $|ax|$ is $\le \floor{\sqrt{p}}$ then since
%
\nbea
a^2 & \equiv & -1 \pmod{p} \\
\to a^2x^2 & \equiv & -x^2 \pmod{p} \\
a^2x^2 + x^2 & \equiv & 0 \pmod{p} \\
& = & pw
\neea
%
$w$ cannot be 0 since $0 < |x|, |ax| \le \floor{\sqrt{p}}$, and $w$ cannot be $\ge 2$ since $2\floor{\sqrt{w}}^2 < 2p$, and it is a strict inequality since $p$ is prime and so $\floor{\sqrt{p}} < \sqrt{p}$. Thus $w$ must be 1 and we are done.

The challenge now is to show that we can find such $x$ such that the above is true. In other words, given any $a \in (\mathbb{Z}/p\mathbb{Z})^*$, there must be $x, y$ with $|x|,|y| \in \{1,2,\dots,\floor{\sqrt{p}}\}$ such that $ax \equiv y \pmod{p}$.

I tried proving Thue's Lemma without the pigeon hole principle and I failed, couldn't find a good way, thought about cross classification but nothing worked so far.

Since I also knew about the product of sums of squares thingy, I was also trying to find a condition where we can divide a sum of squares by another, but this turned out quite complicated as it is the subject of what is called Gaussian integers in algebraic number theory LOL

For the uniqueness part, I tried pigeon holing it to submission :) say from two representations of sum of squares we generate two more, then we have four, from this four we can generate up to 12 (4 choose 2 and each generates two), so some of them must be duplicates but then it was ok to have duplicates. Another thing I wanted to try was partitioning but was too lazy to work out all the details :)


-=-=-=-=-=-=-=-=-=-=-=-=-=-=-=-=-

Just a problem I saw ``somewhere'' :) Find all solutions to
%
\nbea
x^3 + y^3 + z^3 = (x + y + z)^3
\neea
%
well, I initally assumed they were all integers but it is not so, this is an ``implicit'' equation unlike Fermat's last theorem, we can't just randomly set two of them and the last will follow.

Expanding the RHS and canceling similar terms from LHS
%
\nbea
0 & = & 3yz^2 + 3xz^2 + 3y^2z + 6xyz + 3x^2z + 3xy^2 + 3x^2y \\
& = & 3 (yz^2 + xz^2 + y^2z + 2xyz + x^2z + xy^2 + x^2y) \\
& = & 3 ((y+x)z^2 + 2xyz + x^2(z+y) + (x+z)y^2 )
\neea
%
at this point we want to factorize the RHS, my initial idea was to do a quadratic formula for $y$, thus I massaged it into $ay^2 + by + c$, as to why I chose $y$ I'm not sure :)
%
\nbea
yz^2 + xz^2 + y^2z + 2xyz + x^2z + xy^2 + x^2y & = & y^2(z + x) + y(z^2 + 2xz + x^2)  + (xz^2 + x^2z) \\
& = & y^2(z + x) + y(z + x)^2  + xz(z + x) \\
& = & (z + x)(y^2 + y(z + x)  + xz) \\
& = & (z + x)(y + z)(y + x)
\neea
%
but as you can see I didn't even need to use the quadratic formula LOL

Thus the solutions to
%
\nbea
x^3 + y^3 + z^3 = (x + y + z)^3
\neea
%
is given by
%
\nbea
0 & = & (z + x)(y + z)(y + x)
\neea
%
any of them brackets equal zero and we're good which means that at least one of them has to be negative.



-=-=-=-=-=-=-=-=-=-=-=-=-=-=-=-=-=-=-

My reflection on the quadratic reciprocity formula (by Gauss of course who else). My first question is that how does one relate a number in mod $p$ to another congruence mod $q$ because it seems like this is what the reciprocity formula is all about.

But I was wrong, looking at different proofs what I found is this
%
\begin{enumerate}
%
\item We know that for a modulo prime $p$ the following numbers (and only the following numbers) are quadratic residues modulo $p$
%
\nbea
1^2,2^2,3^2,\dots,\left(\frac{p-1}{2}\right)^2
\neea
%
\item From Euler's formula we know that for a prime $p$
%
\nbea
\left(\frac{a}{p}\right) & = & a^{\frac{p-1}{2}} \pmod{p}
\neea
%
which can be shown fairly easily by expressing $a$ in terms of a primitive root of $p$.
%
\item So either for Euler's formula or by listing all the quadratic residues modulo $p$, we need to do $\frac{p-1}{2}$ operations to determine whether a number is a quadratic residue or not, no other way around it.
%
\item This minimum of $\frac{p-1}{2}$ operations also applies even if we use Gauss's lemma that starts with multiplying $a$ into a sequence of $\frac{p-1}{2}$ numbers
%
\nbea
x\in \{a,2a,3a,\dots, \frac{p-1}{2} a\}
\neea
%
such that the Legendre symbol is given by
%
\nbea
\left(\frac{a}{p}\right) & = & (-1)^n \pmod{p}
\neea
%
where $n$ is the number of numbers $ax$ that is congruent to $-x$ modulo $p$. We still need to do $\frac{p-1}{2}$ operations but this is a move in the right direction because calculating $(-1)^n$ is easy, we do not need to do $\frac{p-1}{2}$ operations we just need to know the parity of $n$.

Even calculating the parity of $n$ actually takes $\frac{p-1}{2}$ operations
%
\nbea
n \mod 2 & = & \sum_{x=1}^{\frac{p-1}{2}} \left \lfloor \frac{ax}{p} \right \rfloor
\neea
%
(note that this formula only works if $a$ is odd).
%
\item Finally when we come to the quadratic reciprocity formula itself
%
\nbea
\left(\frac{p}{q}\right)\left(\frac{q}{p}\right) & = & (-1)^{\frac{p-1}{2}\frac{q-1}{2}}
\neea
%
it seems that we need to only do half the work, once we know $\left(\frac{p}{q}\right)$ we know the other Legrende symbol as well (to get $\left(\frac{p}{q}\right)$ we still need to do $\frac{p-1}{2}$ operations), but this is not true, calculating $(-1)^{\frac{p-1}{2}\frac{q-1}{2}}$ still takes ${\frac{p-1}{2} + \frac{q-1}{2}}$ operations but the good thing is that since we are dealing with $-1$ we only need to know the parity of the exponent thus it saves us some time.
%
\end{enumerate}
%
So the moral of the story for me is that quadratic reciprocity formula is the tale of how to transform the same information into a more useful more efficient form,\ie just like the conservation of energy in physics, here the amount of work cannot be reduced, but efficiency can certainly be increased.

This is in direct contrast to the situation in $\mathbb{Z}$, to check if a number $n$ is a perfect square we sort of need to do only roughly $\sqrt{n}$ operations, one way to see this is to realize that a perfect square is a sum of consecutive odd numbers
%
\nbea
n^2 = 1 + 3 + 5 + \dots
\neea
%
which is clearer if you draw them out in terms of boxes (the benefit of this approach is that we need not do any multiplication).

P.S. of course my favorite proof of the quadratic reciprocity formula is the one utilizing Chinese Remainder Theorem, but even in that case, the $\frac{p-1}{2}$ operations cannot be avoided.

-=-=-=-=-=-=-=-=-=-=-=-=-=-=-=-=-=-=-=-=-=-

On the formula for $\left(\frac{2}{p}\right)$, we start with Wilson's formula
%
\nbea
(p-1)! & \equiv & -1 \pmod{p}
\neea
%
we then notice that, for example $p=5$
%
\nbea
1\cdot2\cdot3\cdot4 & \equiv & 2 (1\cdot 2) \cdot (1 \cdot 3) \\
& \equiv & 2 \left ( \frac{5-1}{2}\right)! \cdot (1\cdot-2) \\
& \equiv & 2^{\frac{5-1}{2}} \left ( \frac{5-1}{2}\right)! (-1)\cdot \left ( \frac{5-1}{2}\right)!
\neea
%
and for $p=11$
%
\nbea
1\cdot2\cdot3\cdot4 \cdot 5\cdot 6\cdot 7\cdot 8\cdot 9\cdot 10 & \equiv & 2 (1\cdot 2\cdot 3\cdot4 \cdot 5) \cdot (1 \cdot 3 \cdot 5\cdot 7\cdot 9) \\
& \equiv & 2^{\frac{11-1}{2}} \left ( \frac{11-1}{2}\right)! \cdot (1\cdot-2\cdot 3 \cdot-4 \cdot 5) \\
& \equiv & 2^{\frac{11-1}{2}} \left ( \frac{11-1}{2}\right)! (-1)^2\cdot \left ( \frac{11-1}{2}\right)! \\
& \equiv & 2^{\frac{11-1}{2}} (-1)^{\floor{(11-1)/4}} \left (\left ( \frac{11-1}{2}\right)!\right)^2
\neea
%
Therefore
%
\nbea
(p-1)! & \equiv & 2^{\frac{p-1}{2}} (-1)^{\floor{(p-1)/4}} \left (\left ( \frac{p-1}{2}\right)!\right)^2 \\
(-1)^{(p-1)/2} \left (\left ( \frac{p-1}{2}\right)!\right)^2 & \equiv & 2^{\frac{p-1}{2}} (-1)^{\floor{(p-1)/4}} \left (\left ( \frac{p-1}{2}\right)!\right)^2 \\
(-1)^{(p-1)/2 + \floor{(p-1)/4}} & \equiv & 2^{\frac{p-1}{2}}
\neea
%
so we didn't need Wilson's formula after all but as you see we need something to generate $2^{\frac{p-1}{2}}$, and this is the recurring theme throughout quadratic reciprocity, be it Gauss's lemma or in calculating the parity of the exponent in Gauss's lemma. I intially thought of Wilson's formula not for the factorial but for the $-1$ since Legendre's symbols are either $\pm1$ ot 0.

-=-=-=-=-=-=-=-=-=-=-=-=-=-=-=-=-=-=-=-=-=-

Why the minus sign? the integrand is $\frac{(-x)^s}{e^x - 1}$, there's a minus sign in front of $x$. This is due to the choice of contour. Note that we are going from $+\infty$ to left all the way to 0 and then to the right all the way to $+\infty$.

If there's no minus sign and this is the cut we choose, then
%
\nbea
x^s & = & e^{s\log x + 0\cdot\pi i} , ~~~~~~~~~~ {\rm going~from~} +\infty {\rm ~to~} 0 \\
x^s & = & e^{s\log x + 2\cdot\pi i} , ~~~~~~~~~~ {\rm going~from~} 0 {\rm ~to~} +\infty
\neea
% 
and when we add the two integrals we get zero
%
\nbea
e^{s\log x + 0\cdot\pi i} - e^{s\log x + 2\cdot\pi i} & = & 0
\neea
%
so if we don't want the minus sign we need to choose the other cut, coming from $-\infty$ to zero and back to $-\infty$ so that the phases will be $\pi i$ and $-\pi i$.

-=-=-=-=-=-=-=-=-=-=-=-=-=-=-=-=-

Compact here means closed and bounded.

The whole integral is zero thanks to a corollary of Goursat theorem which says that
%
\nbea
\oint_C f(z) dz & = & \sum_{k=1}^n \oint_{C_k} f(z) dz
\neea
%
where the contours $C_k$'s are all inside of $C$.


-=-=-=-=-=-=-=-=-=-=-=-=-=-=-=-=-

$\gcd(d,z) = c$

%
\nbea
d^2 k & = & z^2 \\
c^2 D^2 k & = & c^2 Z^2 \\
D^2 k & = & Z^2
\neea
%
we know that $\gcd(D,Z) = 1$ and so $\gcd(D^2,Z^2) = 1$, since $k|Z^2$, if $\gcd(D^2,k) > 1$ then $\gcd(D^2,Z^2) > 1$ as well, therefore $\gcd(D^2,k) = 1$, this means that $k = K^2$ by our assumption that if $vw = u^2$ and $v$ and $w$ are co-prime then both $v$ and $w$ are squares.

Substituting $k=K^2$ back into our first equation
%
\nbea
d^2K^2 & = & z^2 \\
\to dK & = & z
\neea
%
thus $d|z$ although I'm not sure about this particular line of reasoning, since $\gcd(D^2,Z^2) = 1$ and $D^2|Z^2$ then $D^2 = 1$ and we immediately get $k=Z^2$ and we don't need our assumption about $vw=u^2$ at all. But I'm not sure how else to show that $\gcd(D^2,k) = 1$ except by showing that $\gcd(D^2,Z^2) = 1$.

Now, the one step I stil need to prove, is that $\gcd(D,Z) = 1$ means $\gcd(D^2,Z^2) = 1$, again, without using the fundamental theorem. What I need is Bezout, assume that $\gcd(D^2,Z^2) = g > 1$ then Bezout tells us that
%
\nbea
D^2 a + Z^2 b & = & g
\neea
%
but from $\gcd(D,Z)=1$ we also get
%
\nbea
D A + Z B & = & 1 \\
\to DgA + ZgB & = & g
\neea
%
Now there might be some other numbers such that
%
\nbea
D M + Z N & = & g
\neea
%
but this means either that $\gcd(M,N) = g$ or $\gcd(M,N) = y | g$, let's dicuss the latter first
%
\nbea
D y M' + Z y N' & = & yg' \\
D M' + Z N' & = & g'
\neea
%
with $\gcd(M',N') = 1$ but this means that $\gcd(D,Z) = g'$, the only way this works is that $g' = 1$ and $\gcd(M,N) = g$, but if this is the case then
%
\nbea
D M' + Z N' & = & 1
\neea
%
but Bezout also tells us that all solutions to $D A' + Z B' = 1$ are of the form see ent.pdf Problem 2.5
%
\nbea
A' & = & A + l D \\
B' & = & B - lD
\neea
%
Equating the two Bezouts $g = D^2 a + Z^2 b = DgA' + ZgB'$
%
\nbea
\to gA' & = & gA + gl D = Da\\
\to gB' & = & gB - gl Z = Zb 
\neea
%
Now, equating the two Bezouts again
%
\nbea
D^2 a + Z^2 b & = & DgA + ZgB \\
D(Da - gA) & = & Z(gB - Zb) \\
\to D(glD) & = & Z(glZ) \\
D^2 & = & Z^2
\neea
%
which is a contradiction, therefore if $\gcd(D,Z) = 1$ then $\gcd(D^2,Z^2) = 1$ as well. The above proof is easily generalizable to $\gcd(D^n,Z^m)$
%
\nbea
\to gA' & = & gA + gl D = D^{n-1}a\\
\to gB' & = & gB - gl Z = Z^{m-1}b 
\neea
%
Now, equating the two Bezouts again
%
\nbea
D^n a + Z^m b & = & DgA + ZgB \\
D(D^{n-1}a - gA) & = & Z(gB - Z^{m-1}b) \\
\to D(glD) & = & Z(glZ) \\
D^2 & = & Z^2
\neea
%
















\end{document}