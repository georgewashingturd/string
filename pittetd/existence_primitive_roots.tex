\documentclass[aps,preprint,preprintnumbers,nofootinbib,showpacs,prd]{revtex4-1}
\usepackage{graphicx,color}
\usepackage{caption}
\usepackage{subcaption}
\usepackage{amsmath,amssymb}
\usepackage{multirow}
\usepackage{amsthm}%        But you can't use \usewithpatch for several packages as in this line. The search 

\usepackage{cancel}

%%% for SLE
\usepackage{dcolumn}   % needed for some tables
\usepackage{bm}        % for math
\usepackage{amssymb}   % for math
\usepackage{multirow}
%%% for SLE -End

\usepackage{ulem}
\usepackage{cancel}

\usepackage{hyperref}

\usepackage[top=1in, bottom=1.25in, left=1.1in, right=1.1in]{geometry}

\usepackage{mathtools} % for \DeclarePairedDelimiter{\ceil}{\lceil}{\rceil}

\usepackage{simplewick}

\newcommand{\msout}[1]{\text{\sout{\ensuremath{#1}}}}


%%%%%% My stuffs - Stef
\newcommand{\lsim}{\mathrel{\mathop{\kern 0pt \rlap
  {\raise.2ex\hbox{$<$}}}
  \lower.9ex\hbox{\kern-.190em $\sim$}}}
\newcommand{\gsim}{\mathrel{\mathop{\kern 0pt \rlap
  {\raise.2ex\hbox{$>$}}}
  \lower.9ex\hbox{\kern-.190em $\sim$}}}

%
% Key
%
\newcommand{\key}[1]{\medskip{\sffamily\bfseries\color{blue}#1}\par\medskip}
%\newcommand{\key}[1]{}
\newcommand{\q}[1] {\medskip{\sffamily\bfseries\color{red}#1}\par\medskip}
\newcommand{\comment}[2]{{\color{red}{{\bf #1:}  #2}}}


\newcommand{\ie}{{\it i.e.} }
\newcommand{\eg}{{\it e.g.} }

%
% Energy scales
%
\newcommand{\ev}{{\,{\rm eV}}}
\newcommand{\kev}{{\,{\rm keV}}}
\newcommand{\mev}{{\,{\rm MeV}}}
\newcommand{\gev}{{\,{\rm GeV}}}
\newcommand{\tev}{{\,{\rm TeV}}}
\newcommand{\fb}{{\,{\rm fb}}}
\newcommand{\ifb}{{\,{\rm fb}^{-1}}}

%
% SUSY notations
%
\newcommand{\neu}{\tilde{\chi}^0}
\newcommand{\neuo}{{\tilde{\chi}^0_1}}
\newcommand{\neut}{{\tilde{\chi}^0_2}}
\newcommand{\cha}{{\tilde{\chi}^\pm}}
\newcommand{\chao}{{\tilde{\chi}^\pm_1}}
\newcommand{\chaop}{{\tilde{\chi}^+_1}}
\newcommand{\chaom}{{\tilde{\chi}^-_1}}
\newcommand{\Wpm}{W^\pm}
\newcommand{\chat}{{\tilde{\chi}^\pm_2}}
\newcommand{\smu}{{\tilde{\mu}}}
\newcommand{\smur}{\tilde{\mu}_R}
\newcommand{\smul}{\tilde{\mu}_L}
\newcommand{\sel}{{\tilde{e}}}
\newcommand{\selr}{\tilde{e}_R}
\newcommand{\sell}{\tilde{e}_L}
\newcommand{\smurl}{\tilde{\mu}_{R,L}}

\newcommand{\casea}{\texttt{IA}}
\newcommand{\caseb}{\texttt{IB}}
\newcommand{\casec}{\texttt{II}}

\newcommand{\caseasix}{\texttt{IA-6}}

%
% Greek
%
\newcommand{\es}{{\epsilon}}
\newcommand{\sg}{{\sigma}}
\newcommand{\dt}{{\delta}}
\newcommand{\kp}{{\kappa}}
\newcommand{\lm}{{\lambda}}
\newcommand{\Lm}{{\Lambda}}
\newcommand{\gm}{{\gamma}}
\newcommand{\mn}{{\mu\nu}}
\newcommand{\Gm}{{\Gamma}}
\newcommand{\tho}{{\theta_1}}
\newcommand{\tht}{{\theta_2}}
\newcommand{\lmo}{{\lambda_1}}
\newcommand{\lmt}{{\lambda_2}}
%
% LaTeX equations
%
\newcommand{\beq}{\begin{equation}}
\newcommand{\eeq}{\end{equation}}
\newcommand{\bea}{\begin{eqnarray}}
\newcommand{\eea}{\end{eqnarray}}
\newcommand{\ba}{\begin{array}}
\newcommand{\ea}{\end{array}}
\newcommand{\bit}{\begin{itemize}}
\newcommand{\eit}{\end{itemize}}

\newcommand{\nbea}{\begin{eqnarray*}}
\newcommand{\neea}{\end{eqnarray*}}
\newcommand{\nbeq}{\begin{equation*}}
\newcommand{\neeq}{\end{equation*}}

\newcommand{\no}{{\nonumber}}
\newcommand{\td}[1]{{\widetilde{#1}}}
\newcommand{\sqt}{{\sqrt{2}}}
%
\newcommand{\me}{{\rlap/\!E}}
\newcommand{\met}{{\rlap/\!E_T}}
\newcommand{\rdmu}{{\partial^\mu}}
\newcommand{\gmm}{{\gamma^\mu}}
\newcommand{\gmb}{{\gamma^\beta}}
\newcommand{\gma}{{\gamma^\alpha}}
\newcommand{\gmn}{{\gamma^\nu}}
\newcommand{\gmf}{{\gamma^5}}
%
% Roman expressions
%
\newcommand{\br}{{\rm Br}}
\newcommand{\sign}{{\rm sign}}
\newcommand{\Lg}{{\mathcal{L}}}
\newcommand{\M}{{\mathcal{M}}}
\newcommand{\tr}{{\rm Tr}}

\newcommand{\msq}{{\overline{|\mathcal{M}|^2}}}

%
% kinematic variables
%
%\newcommand{\mc}{m^{\rm cusp}}
%\newcommand{\mmax}{m^{\rm max}}
%\newcommand{\mmin}{m^{\rm min}}
%\newcommand{\mll}{m_{\ell\ell}}
%\newcommand{\mllc}{m^{\rm cusp}_{\ell\ell}}
%\newcommand{\mllmax}{m^{\rm max}_{\ell\ell}}
%\newcommand{\mllmin}{m^{\rm min}_{\ell\ell}}
%\newcommand{\elmax} {E_\ell^{\rm max}}
%\newcommand{\elmin} {E_\ell^{\rm min}}
\newcommand{\mxx}{m_{\chi\chi}}
\newcommand{\mrec}{m_{\rm rec}}
\newcommand{\mrecmin}{m_{\rm rec}^{\rm min}}
\newcommand{\mrecc}{m_{\rm rec}^{\rm cusp}}
\newcommand{\mrecmax}{m_{\rm rec}^{\rm max}}
%\newcommand{\mpt}{\rlap/p_T}

%%%song
\newcommand{\cosmax}{|\cos\Theta|_{\rm max} }
\newcommand{\maa}{m_{aa}}
\newcommand{\maac}{m^{\rm cusp}_{aa}}
\newcommand{\maamax}{m^{\rm max}_{aa}}
\newcommand{\maamin}{m^{\rm min}_{aa}}
\newcommand{\eamax} {E_a^{\rm max}}
\newcommand{\eamin} {E_a^{\rm min}}
\newcommand{\eaamax} {E_{aa}^{\rm max}}
\newcommand{\eaacusp} {E_{aa}^{\rm cusp}}
\newcommand{\eaamin} {E_{aa}^{\rm min}}
\newcommand{\exxmax} {E_{\neuo \neuo}^{\rm max}}
\newcommand{\exxcusp} {E_{\neuo \neuo}^{\rm cusp}}
\newcommand{\exxmin} {E_{\neuo \neuo}^{\rm min}}
%\newcommand{\mxx}{m_{XX}}
%\newcommand{\mrec}{m_{\rm rec}}
\newcommand{\erec}{E_{\rm rec}}
%\newcommand{\mrecmin}{m_{\rm rec}^{\rm min}}
%\newcommand{\mrecc}{m_{\rm rec}^{\rm cusp}}
%\newcommand{\mrecmax}{m_{\rm rec}^{\rm max}}
%%%song

\newcommand{\mc}{m^{\rm cusp}}
\newcommand{\mmax}{m^{\rm max}}
\newcommand{\mmin}{m^{\rm min}}
\newcommand{\mll}{m_{\mu\mu}}
\newcommand{\mllc}{m^{\rm cusp}_{\mu\mu}}
\newcommand{\mllmax}{m^{\rm max}_{\mu\mu}}
\newcommand{\mllmin}{m^{\rm min}_{\mu\mu}}
\newcommand{\mllcusp}{m^{\rm cusp}_{\mu\mu}}
\newcommand{\elmax} {E_\mu^{\rm max}}
\newcommand{\elmin} {E_\mu^{\rm min}}
\newcommand{\elmaxw} {E_W^{\rm max}}
\newcommand{\elminw} {E_W^{\rm min}}
\newcommand{\R} {{\cal R}}

\newcommand{\ewmax} {E_W^{\rm max}}
\newcommand{\ewmin} {E_W^{\rm min}}
\newcommand{\mwrec}{m_{WW}}
\newcommand{\mwrecmin}{m_{WW}^{\rm min}}
\newcommand{\mwrecc}{m_{WW}^{\rm cusp}}
\newcommand{\mwrecmax}{m_{WW}^{\rm max}}

\newcommand{\mpt}{{\rlap/p}_T}

%%%%%% END My stuffs - Stef

\newcommand{\dunno}{$ {}^{\mbox {--}}\backslash(^{\rm o}{}\underline{\hspace{0.2cm}}{\rm o})/^{\mbox {--}}$}

\DeclarePairedDelimiter{\ceil}{\lceil}{\rceil}
\DeclarePairedDelimiter{\floor}{\lfloor}{\rfloor}

\DeclareMathOperator{\ord}{ord}
\DeclareMathOperator{\tor}{tor}





\begin{document}

\title{Primitive Existentialism by Root Counting}
\bigskip
\author{Stefanus$^1$\\
$^1$ Samsung Semiconductor Inc\\ San Jose, CA 95134 USA\\
}
%
\date{\today}
%
\begin{abstract}
Proving the existence of primitive roots modulo 2,4, $p$, $p^\alpha$ and $2p^\alpha$ and the non existence of any other modulos by explicitly counting the number of roots of the polynomial $x^d - 1\equiv 0$ where $d|\varphi(n)$.

\end{abstract}
%
\maketitle

\renewcommand{\theequation}{A.\arabic{equation}}  % redefine the command that creates the equation no.
\setcounter{equation}{0}  % reset counter 

%\textbf{\textit{*}}\underline{\textit{\textbf {Parental Advisory, Explicit Content}}}\textbf{\textit{*}}
*\hspace{0.5mm}\underline{\textit{\textbf {Parental Advisory, Explicit Content}}}\hspace{0.7mm}*

In the following discussion we will derive the number of primitive roots modulo $n$ (not necessarily prime) \underline{\textbf{\textit{Explicitly}}}\textbf{\textit{!}} and in doing so we will prove that only 2, 4, $p$, $p^\alpha$ and $2p^\alpha$ have primitive roots.

\bigskip
\underline{\textit{\textbf{Basic Ingredients}}}
\bigskip

To achieve our goal of counting primitive roots we need the following well known facts.
\bit
%
\item First, any polynomial of degree $f$ defined on a field has at most $f$ roots, see ent.pdf Proposition 2.5.3.
%
\item Next, the fact that $\mathbb{Z}/p\mathbb{Z}$ is a field as shown in Exercise 2.12 of ent.pdf.
%
\item Chinese Remainder Theorem and its useful cousin Chinese Remainder Map.
%
\item Lastly using an application of Proposition 2.5.5 of ent.pdf which states that for each divisor $d|(p-1)$, the polynomial of degree $d$ has exactly $d$ roots in $\mathbb{Z}/p\mathbb{Z}$.
%
\item And of course all other good stuffs from elementary number theory, like the Euler Totient function $\varphi(n)$, Fermat's Little Theorem, order of an element of $\mathbb{Z}/n\mathbb{Z}$, etc.
%
\eit
%
Although Proposition 2.5.5 can be automatically extended to $d|\varphi(n)$ with $n$ not prime as long as $\mathbb{Z}/n\mathbb{Z}$ forms a field, the problem is that for $n$ not prime $\mathbb{Z}/n\mathbb{Z}$ is guaranteed not to be a field. So for now we just concentrate on $n$ being prime.

\bigskip
\underline{\textit{\textbf{Counting Primal Roots}}}
\bigskip

With all these in mind we proceed to calculate the number of primitive roots in the unit group $(\mathbb{Z}/n\mathbb{Z})^*$. Fermat's Little Theorem tells us that every number in the unit group $(\mathbb{Z}/n\mathbb{Z})^*$ satisfy the following $x^{\varphi(n)} \equiv 1 \pmod{n}$, therefore the polynomial $x^{\varphi(n)} - 1 \equiv 0 \pmod{n}$ has exactly $\varphi(n)$ roots since that's the number of elements in the unit group.

To find the primitive roots we want to find elements of the unit groups that are \underline{{\bf not}} roots of $x^d - 1 \equiv 0$ where $d|\varphi(n)$ because these roots indicate that their order is less than $\varphi(n)$.

Now suppose that $\varphi(n)$ is just a product of two primes
%
\nbea
\varphi(n) & = & q_1q_2
\neea
%
From the $q_1q_2$ roots of $x^{\varphi(n)} \equiv 1$, how many of them are roots of $x^{q_1} \equiv 1$ and how many are roots of $x^{q_2} \equiv 1$? The roots that are not covered by $x^{q_{1}} \equiv 1$ and $x^{q_{2}} \equiv 1$ will be the primitive roots of $n$. So are there any?

From the $q_1q_2$ roots of $x^{\varphi(n)} \equiv 1$ we need to subtract $q_1$ roots that belong to $x^{q_1} \equiv 1$ and $q_2$ roots that belong to $x^{q_2} \equiv 1$, this is because we know that there are exactly $q_1$ and $q_2$ roots for those two polynomials as given by Proposition 2.5.5 of ent.pdf.

But in doing the subtractions we were double counting because 1 is always a root of $x^d - 1 \equiv 0$ whatever $d$ is, so when subtracting the roots of $x^{q_1} \equiv 1$ we already remove 1, thus we need to add 1 back, the number of primitive roots are then
%
\nbea
q_1q_2 - q_1 - q_2 + 1 & = & (q_1 - 1)(q_2 - 1) \\
& = & \varphi(q_1q_2) \\
& = & \varphi(\varphi(n))
\neea
%
Now what happens if $\varphi(n)$ has three distinct prime factors? We do the same, we start with $q_1q_2q_3$ as the total number of roots, we then remove the ones already covered by $q_1q_2$, followed by $q_1q_3$ and finally $q_2q_3$. But in doing so we are again over counting because while removing the roots of $x^{q_1q_2} \equiv 1$ we already removed the roots of $x^{q_1} \equiv 1$ (and of $x^{q_2} \equiv 1$) but when we removed the roots of $x^{q_1q_3} \equiv 1$ we again remove the roots of $x^{q_1} \equiv 1$, so we need to add them back in.
%
\nbea
q_1q_2q_3 - q_1q_2 - q_1q_3 - q_2q_3 + q_1 + q_2 + q_3
\neea
%
But as we are removing and adding back over counted roots, we also remove and add 1 (since 1 is always a root), here we removed it three times and then we added it back in three times, but 1 still should be removed, so the final tally is
%
\nbea
q_1q_2q_3 - q_1q_2 - q_1q_3 - q_2q_3 + q_1 + q_2 + q_3 - 1 & = & (q_1 - 1)(q_2 - 1)(q_3 - 1) \\
& = & \varphi(\varphi(n))
\neea
%
and the pattern continues. But what if we have a more generic
%
\nbea
\varphi(n) & = & q_1^{a_1}q_2^{a_2} \dots q_n^{a_n}
\neea
%
The pattern is still the same, let's limit $n=3$ to see a concrete example, again we start with $q_1^{a_1}q_2^{a_2}q_3^{a_3}$ we then remove the roots belonging to $q_1^{a_1}q_2^{a_2}q_3^{a_3-1}$ followed by the ones in $q_1^{a_1}q_2^{a_2-1}q_3^{a_3}$ and finally $q_1^{a_1-1}q_2^{a_2}q_3^{a_3}$.

Note that by removing the roots of $q_1^{a_1}q_2^{a_2}q_3^{a_3-1}$ we are already removing the roots of all its divisors. But just like before, we are over counting because we removed the roots of $q_1^{a_1}q_2^{a_2-1}q_3^{a_3-1}$ twice, once from $q_1^{a_1}q_2^{a_2}q_3^{a_3-1}$ and another time from $q_1^{a_1}q_2^{a_2-1}q_3^{a_3}$, so we need to add them back in
%
\nbea
q_1^{a_1}q_2^{a_2}q_3^{a_3} &&  - q_1^{a_1}q_2^{a_2}q_3^{a_3-1} - q_1^{a_1}q_2^{a_2-1}q_3^{a_3}- q_1^{a_1-1}q_2^{a_2}q_3^{a_3} \\
&& + q_1^{a_1}q_2^{a_2-1}q_3^{a_3-1} + q_1^{a_1-1}q_2^{a_2}q_3^{a_3-1} + q_1^{a_1-1}q_2^{a_2-1}q_3^{a_3}
\neea
%
but again, here we've removed the roots of $q_1^{a_1-1}q_2^{a_2-1}q_3^{a_3-1}$ three times and then added them back in three times, but we know that they should be removed, so the final tally is
%
\nbea
q_1^{a_1}q_2^{a_2}q_3^{a_3} &&  - q_1^{a_1}q_2^{a_2}q_3^{a_3-1} - q_1^{a_1}q_2^{a_2-1}q_3^{a_3}- q_1^{a_1-1}q_2^{a_2}q_3^{a_3} \\
&& + q_1^{a_1}q_2^{a_2-1}q_3^{a_3-1} + q_1^{a_1-1}q_2^{a_2}q_3^{a_3-1} + q_1^{a_1-1}q_2^{a_2-1}q_3^{a_3} \\
&& -q_1^{a_1-1}q_2^{a_2-1}q_3^{a_3-1}
\neea
%
which is just $(q_1^{a_1} - q_1^{a_1-1})(q_2^{a_2}-q_2^{a_2-1})(q_3^{a_3} - q_3^{a_3-1}) =\varphi(q_1^{a_1}q_2^{a_2}q_3^{a_3}) = \varphi(\varphi(n))$. Note that we don't need to mess with other divisors of $\varphi(n)$ because for example the roots of $x^s - 1$ are already covered by the roots of $x^t - 1$ as long as $s|t$.

So the generic strategy is to start with $\varphi(n)$ roots, express $\varphi(n)$ in terms of its primal constituents and then start removing the roots of the next highest divisor of $\varphi(n)$ and then take care of all the double counting until there's no more over counting and stop. This pattern persists for a generic prime $n$.

Since $\varphi(\varphi(n))$ can never be zero and as long as $n$ is prime, we have also not only proven that there are always primitive roots modulo a prime $p$ but also how many there are.

The key here is of course that the polynomial $x^d - 1 \equiv 0$ has {\it exactly} $d$ roots, and this rests on the ring $\mathbb{Z}/n\mathbb{Z}$ being a field, what happens if it's not a field?

\bigskip
\underline{\textit{\textbf{Composite Conundrum}}}
\bigskip

First we tackle the case of $n = p^\alpha$. Here $\varphi(n) = p^{\alpha-1}(p-1)$ and we tackle the problem of $x^d - 1\equiv 0 \pmod{p^\alpha}$ with $d|\varphi(n)$ in three steps, first, when $d|p^{\alpha-1}$, second $d|(p-1)$ and lastly the combination of both.

The goal here is to show that $x^d - 1 \equiv 0 \pmod{p^\alpha}$ has exactly $d$ roots if $d|\varphi(p^\alpha)$. First case is $d = p^\beta$, $\beta < \alpha$.

\bigskip
\underline{\textit{\textbf{First Case}}}, $d|p^{\alpha-1}$
\smallskip

The motivation for this is as follows, take for example $p^\alpha=3^2$, $\varphi(3^2) = 3\cdot 2$, and $d = 3$. If we cube each element of $\mathbb{Z}/3^2\mathbb{Z} = \{1,2,3,\dots,3^2 = 9\}$ we get
%
\nbea
\{1^3, 2^3, 3^3,~4^3, 5^3, 6^3,~7^3, 8^3, 9^3\} \equiv \{1, 8, 0,~1, 8, 0,~1, 8, 0\} \pmod{3^2}
\neea
%
So it looks like any number $a = 1 + x\cdot (3^2/3)$ will have $a^3 = \{1+x\cdot (3^2/3)\}^3 \equiv 1 \pmod{3^2}$, and the generic formula when $d = p^\beta$, $\beta \le \alpha - 1$ seems to be $a = 1 + x\cdot p^{\alpha-\beta}$. Exponentiating to the $d^{\rm th}$ power we get
%
\nbea
(1 + x\cdot p^{\alpha-\beta})^{p^\beta} & = & 1 + \sum_{j=1}^{p^\beta}\binom{p^\beta}{j} (x\cdot p^{\alpha-\beta})^j
\neea
%
We want to show that the sum is $\equiv 0 \pmod{p^\alpha}$. 

\smallskip
\underline{\textit{\textbf{Binomial Bifurcation}}}
\smallskip

Before we tackle the sum above, we will bifurcate our discussion into properties of binomial coefficients.

{\bf Proposition E.0}. (``E'' here stands for Explicit :) First, for $j \ge 1$ 
%
\nbea
\binom{p^n}{j} & = & \left\{
\begin{array}{l l}
k \cdot p^n & ~~~~~~~ {\rm if ~} \gcd(j,p^n) = 1 \\
l \cdot p^{n-w} & ~~~~~~~ {\rm if ~} \gcd(j,p^n) = p^w, ~ 1 \le w \le n
\end{array} \right.
\neea
%
where $\gcd(k,p) = \gcd(l,p) = 1$.

{\it Proof}. Let's rewrite the binomial as
%
\nbea
\binom{p^n}{j} & = & \frac{p^n!}{j!(p^n - j)!} \\
& = & \frac{p^n\cdot(p^n-1)\cdots (p^n - j + 1)}{j\cdot(j-1)\cdots 1}
\neea
%
Note that the goal here is to count the number of $p$ in the binomial coefficient.

One thing to note is that the numerator and denominator have the same number of terms. Now let $r$ be the highest exponent such that $p^r \le j$ and rearrange the fraction as
%
\nbea
&&\frac{1}{j} \cdot \frac{1}{(j-1)} \cdots \left(\frac{p^n}{p^r}\right) \cdots \left(\frac{p^{n}-p}{p^{r}-p}\right) \cdots \left(\frac{p^{n}-2p}{p^{r}-2p}\right) \cdots \left(\frac{p^{n}-(p^{r-1} - 1)p}{p}\right) \cdots \\
&& ~~~~~~~~~~~~~~~~~~~~~~~~~~~~~~~~~~~~~~~~~~~~~~~~~~~~~~~~~~~ \cdots \frac{p^{n}-p^{r} + 1}{1} \cdot \frac{(p^{n}-p^{r})^*}{1} \cdots \frac{p^{n} - j + 1}{1}
\neea
%
The terms in brackets are the only terms in the numerator (and denominator) that contain $p$, so basically we align the terms in the numerator and denominator so that those who contain $p$ are grouped together, and the last bracket with $()^*$ on it indicates that the numerator contains $p$ while the denominator doesn't, note that this term doesn't exist if $j = p^r$ since the binomial stops one term earlier, let's see this with a concrete example, take $p^n = 5^2$ and $j = 7$, we then get
%
\nbea
\frac{1}{7}~\frac{1}{6}~\left(\frac{25}{5}\right)~\frac{24}{4}~\frac{23}{3}~\frac{22}{2}~\frac{21}{1}~\frac{(20)^*}{1}
\neea
%
and if $j = 5^1$ we get
%
\nbea
\left(\frac{25}{5}\right)~\frac{24}{4}~\frac{23}{3}~\frac{22}{2}~\frac{21}{1}
\neea
%
the term with $()^*$ doesn't exist in this case.

The reason behind this rearrangement is that we want to count the number of $p$ in the fraction, by grouping the terms in the numerator and denominator that contain $p$ we reduce the problem into analyzing those terms only. These terms are of the form $(p^n - yp)/(p^r - yp)$ if $(y,p^n) = p^{t-1}$ we can express $yp = xp^t$ with $(x,p) = 1$ and 
%
\nbea
\frac{p^n - xp^t}{p^r - xp^t} & = & \frac{p^{n-t} - x}{p^{r-t} - x}
\neea
%
now the numerator and denominator no longer have any factor of $p$ since $x$ is co-prime to $p$, thus the only terms in brackets that contain $p$ are
%
\nbea
\left(\frac{p^n}{p^r}\right) ~~~~~{\rm and} ~~~~~ \frac{(p^{n}-p^{r})^*}{1}
\neea
%
thus the binomial coeffcient is just $k\cdot p^n$ since
%
\nbea
\left(\frac{p^n}{p^r}\right) \cdot \frac{(p^{n}-p^{r})}{1} & = & \frac{p^n (p^{n-r}-1)}{1}
\neea
%
and $\gcd(p, p^{n-r}-1) = 1$. Now if $j = p^r$ then we don't have the $()^*$ term, the binomial stops one term earlier, thus the binomial is equal to $l\cdot p^{n-r}$.

{\bf Proposition E.1}. For an odd prime $p$ and $m,n > 0$ we have
%
\nbea
(1 + xp^m)^{p^n} & \equiv & 1 + xp^{m+n} \pmod{p^{m+n+1}}
\neea
%

{\it Proof}. First we expand 
%
\nbea
(1 + xp^m)^{p^n} & = & 1 + \sum_{j=1}^{p^n} \binom{p^n}{j} (xp^m)^j
\neea
%
From Proposition E.0 we know that 
%
\nbea
\binom{p^n}{j} (xp^m)^j & = & \left\{
\begin{array}{l l}
k\cdot x^j p^{n+jm} & ~~~~~~~ {\rm if ~} \gcd(j,p^n) = 1 \\
l\cdot x^{yp^r} p^{n-r + m\cdot yp^r} & ~~~~~~~ {\rm if ~} \gcd(j,p^n) = p^r \to j = yp^r, ~\gcd(y,p) = 1, r > 0
\end{array}\right.
\neea
%
For the first case $n+jm > n+m$ if $j > 1$ and for the second case $myp^r > r$ for any $y$ and $r$ thus
%
\nbea
\binom{p^n}{j} (xp^m)^j & = & \left\{
\begin{array}{l l}
x\cdot p^{n+m} & ~~~~~~~ {\rm if ~} j = 1 \\
v\cdot p^{n + m + 1} & ~~~~~~~ {\rm if ~} j > 1
\end{array}\right.
\neea
%
therefore
%
\nbea
(1 + xp^m)^{p^n} & = & 1 + xp^{n+m} + v p^{n + m + 1} \\
& \equiv & 1 + xp^{n+m} \pmod{p^{n + m + 1}}
\neea
%

Going back to our proposed solution $(1 + x\cdot p^{\alpha-\beta})^{p^\beta}$, utilizing Proposition E.1 with $m = \alpha-\beta$ and $n=\beta$ we get
%
\nbea
(1 + x\cdot p^{\alpha-\beta})^{p^\beta} & = & 1 + x\cdot p^{\alpha-\beta +\beta} + v\cdot p^{\alpha-\beta+\beta + 1} \\
& = & 1 \pmod{p^\alpha}
\neea
%


The question now is how many of these numbers ($1 + x\cdot p^{\alpha-\beta}$ incongruent modulo $p^\alpha$) there are, this is equivalent to the number of unique values for $x$. The obvious answer is $0 \le x < p^\beta$, meaning we have $p^\beta$ solutions for $x^{p^\beta} - 1 \equiv 0$.

Are there more than that? What if we found a number $a^{p^\beta} - 1 \equiv 0$ besides the above? say there's the case then
%
\nbea
a^{p^\beta} \equiv 1 \pmod{p^\alpha} ~~\to~~ a^{p^\beta} & \equiv & 1 \pmod{p}
\neea
%
but by Fermat's Little Theorem this is forbidden because this means that either $p^\beta|(p-1)$ or $(p-1)|p^\beta$, except for $a \equiv 1 \pmod{p}$. So the only other possible roots is in the form $1 + xp$.

To prove that such roots do not exist we again use Proposition E.1, we start with
%
\nbea
(1 + x'\cdot p^{\alpha-\beta-1})^{p^\beta} & = & 1 + x'p^{\alpha - 1} + v p^{\alpha}
\neea
%
we want the RHS to be $\equiv 1 \pmod{p^\alpha}$ thus
%
\nbea
1 + x'p^{\alpha - 1} + v p^{\alpha} & = & 1 + w p^\alpha \\
x' & = & p(w - v) \\
p & | & x'
\neea
%
but this means that the necessary condition for $(1 + x'\cdot p^{\alpha-\beta-1})^{p^\beta} \equiv 1 \pmod{p^\alpha}$ is that $p | x'$. We now repeat the process with 
%
\nbea
(1 + x''\cdot p^{\alpha-\beta-2})^{p^\beta} & = & 1 + x''p^{\alpha - 2} + v' p^{\alpha-1}
\neea
%
and we still want the RHS to be $\equiv 1 \pmod{p^\alpha}$ therefore
%
\nbea
1 + x''p^{\alpha - 2} + v' p^{\alpha-1} & = & 1 + w' p^\alpha \\
x'' & = & p(pw' - v') \\
p & | & x''
\neea
%
so the necessary condition is still that $p|x''$ but this means that the solution is of the form $1 + x'\cdot p^{\alpha-\beta-1}$ but even in this case $p|x'$ so the solution is just $1 + x\cdot p^{\alpha-\beta}$, we can keep repeating with $p^{\alpha-\beta-3}$ and so on until we reach $p^{1}$ and working back up we will see that the solution must be of the form $1 + x\cdot p^{\alpha-\beta}$. So we have shown that $x^{p^\beta} - 1 \equiv 0$ has exactly $p^\beta$ solutions.

\bigskip
\underline{\textbf{\textit{Second Case}}}, $d|(p-1)$
\bigskip

Next case is when $d|(p-1)$, this one is a bit trickier and we need to utilize induction. By Proposition 2.5.5 of ent.pdf we know that $x^d - 1\equiv 0 \pmod{p}$ has exactly $d$ solutions, the problem we have now is that $\mathbb{Z}/p^\alpha\mathbb{Z}$ is no longer a field. But we can build the proof one power at a time.

Base case is $x^d - 1\equiv 0 \pmod{p}$, based on this how can we find the solutions to $x^d - 1\equiv 0 \pmod{p^2}$? Well, we know that if there is a solution, say $a^2 \equiv 1 \pmod{p^2}$, then $a$ has to also satisfy
%
\nbea
a^d & \equiv & 1 \pmod{p}
\neea
%
meaning $a$ is also a root $\pmod{p}$, \ie it is of the form $a + np$, our task is to see whether we can find such $n$ to get a solution mod $p^2$, (existence of $a$ mod $p$ is already guaranteed by Proposition 2.5.5). Let's see how this works
%
\nbea
(a + np)^d & = & a^d + da^{d-1}np + p^2 t, ~~~~~~~~ a^d = 1 + mp \\
& \equiv & 1 + mp + da^{d-1}np \pmod{p^2}
\neea
%
$a^d = 1 + mp$ since $a^d \equiv 1 \pmod{p}$. We want this whole thing to be $1 \pmod{p^2}$ so
%
\nbea
1 + mp + da^{d-1}np & \equiv & 1 \pmod{p^2} \\
\to m\bcancel{p} + da^{d-1}n\bcancel{p} & \equiv & 0 \pmod{p^{\bcancel{2}}} \\
da^{d-1}n\bcancel{p} & \equiv & -m \pmod{p} \\
n & \equiv & -m(da^{d-1})^{-1} \pmod{p}
\neea
%
the inverse is guaranteed since $da^{d-1}$ is co-prime to $p$ (this is a crux of the proof as we shall see soon) and $\mathbb{Z}/p\mathbb{Z}$ is a field since $p$ is prime, so $n$ is guaranteed to exist (modulo $p$). So we have the following solutions 
%
\nbea
a + (n + sp)p, ~~~~~ s \ge 0
\neea
%
however, for $s \ge 1$, $a + np + sp^2$ is bigger than $p^2$, so we only have one such $a + np < p^2$ (note that $n < p$). So for every root $a^d \equiv 1 \pmod{p}$ we have a {\it unique} root $a^d \equiv 1 \pmod{p^2}$. Using this info we can prove a unique root $\pmod{p^3}$, but this time we substitute $a = 1 +mp^2$ instead of $a = 1 + mp$, and in this way the induction goes.

The uniqueness part is crucial because we want to show that there are exactly $d$ roots and since we prove that there is a unique $a$ we are done.

\bigskip
\underline{\textbf{\textit{Non-existence of Primitive Roots modulo $2^\alpha$, $\alpha \ge 3$}}}
\smallskip

The proofs above give us an idea on how to prove that $2^\alpha$ with $\alpha \ge 3$ doesn't have primitive roots. Say we elevate the roots of $x^2 \equiv 1 \pmod{4}$ to modulo 8 just like before, we will get 4 roots, 1, 3, 5, 7, but $x^4 \equiv 1 \pmod{8}$ only has 4 roots and they are already covered by $x^2 \equiv 1$, that means that there are no primitive roots. Again, taking 1, 3, 5, 7, and elevating them to 9, 11, 13, 15, these eight are the roots of $x^4 \equiv 1 \pmod{16}$ and they are also all the roots of $x^8 \equiv 1\pmod{16}$ so 16 has no primitive roots and so on. Let's see this in detail. 

{\bf Proposition E.2}. Any number $1 + 2c$ with $0 \le c < 2^{\alpha-1}$ are all roots of $x^{\varphi(2^\alpha)/2} \equiv 1 \pmod{p^\alpha}$, so there are $2^{\alpha-1}$ roots and these roots are also roots of $x^{\varphi(2^\alpha)} \equiv 1 \pmod{2^\alpha}$, this is true for $\alpha \ge 3$.

{\it Proof}. We will use induction, base case is 8, with $a = 1, 3, 5, 7$ and from the assumption we know that $a^2 \equiv 1 \pmod{8} = 1 + 8m$ and so going from mod $8\to16$,
%
\nbea
a^2 & = & 1+8m \\
a^4 & = & (1 + 8m)^2 \\
& = & 1 + 16m + 64m^2 \\
& \equiv & 1 \pmod{16}
\neea
%
we now extend the solutions mod 8 from $1, 3, 5, 7$ to $9, 11, 13, 15$, so basically $a \to a + 8$, going to mod 16 we get
%
\nbea
(a + 8)^2 & = & a^2 + 16a + 64 ~~~~~~~ a^2 = 1+ 8m \\
\to & \equiv & 1 + 8m \pmod{16}
\neea
%
and so $(a + 8)^4 \equiv 1 \pmod{16}$ as well. We can thus repeat the inductive process to complete the proof which is a straightforward process by replacing 8 with $2^n$ and 16 with $2^{n+1}$ and therefore omitted here :)

In short, the roots of $x^{\varphi(2^\alpha)} \equiv 1 \pmod{2^\alpha}$ must satisfy $x^{\varphi(2^\alpha)} \equiv 1 \pmod{2}$, which means that $x$ must be an odd number. However, we have shown above that all odd numbers $< 2^\alpha$ are roots of $x^{2^{\alpha-2}} \equiv 1 \pmod{2^\alpha}$, thus we have covered all possible roots of $x^{\varphi(2^\alpha)} \equiv 1 \pmod{2^\alpha}$ with the roots of $x^{\varphi(2^\alpha)/2} \equiv 1 \pmod{2^\alpha}$.

{\bf Proposition E.3}. As a direct consequence of Proposition E.2, $2^\alpha$ with $\alpha > 2$ has no primitive roots :)

This also shows why there is primitive root modulo 4, because in this case $\varphi(4)/2 = 1$ and $x^1 \equiv 1$ can only have one root and not two.

\bigskip
\underline{\textbf{\textit{Last Case}}}, $d|p^{\alpha-1}(p-1)$
\bigskip

The last case we have is a combination of the above two, $d|p^{\alpha-1}(p-1)$, the proof is therefore also a combination of the two :) Let's denote $d = xy$ with $x|p^{\alpha-1} \to x = p^\beta$ with $\beta < \alpha$ and $y|(p-1)$ and by virtue of $(p,p-1) = 1$ we have $(x,y) = 1$ as well.

To tackle this we first find the solutions for $a^y - 1 \equiv 0 \pmod{p^{\alpha-\beta}}$ where $y|(p-1)$. But this is exactly the same as our previous case, the roots are the same as $a^y-1 \equiv 0 \pmod{p}$ elevated to $\pmod{p^{\alpha-\beta}}$ by the induction method above.

After finding all roots of $a^y - 1 \equiv 0 \pmod{p^{\alpha-\beta}}$, we then use these roots and extend them
%
\nbea
(a + w\cdot p^{\alpha-\beta})^{xy} & = & (a + w\cdot p^{\alpha-\beta})^{p^\beta y} \\
& = & (a^y)^{p^\beta} + \sum_{j=1}^{p^\beta} \binom{p^\beta}{j} (w\cdot p^{\alpha-\beta})^j \\
& = & (1 + s\cdot p^{\alpha-\beta})^{p^\beta} + \sum_{j=1}^{p^\beta} \binom{p^\beta}{j} (w\cdot p^{\alpha-\beta})^j \\
& = & 1 + \sum_{i=1}^{p^\beta} \binom{p^\beta}{i} (s\cdot p^{\alpha-\beta})^i + \sum_{j=1}^{p^\beta} \binom{p^\beta}{j} (w\cdot p^{\alpha-\beta})^j
\neea
%
just like before, using Proposition E.1, the sums are $0 \pmod{p^\alpha}$, the challenge now is to show that there are no other roots other than the ones shown above. Suppose there is another root, $b$, then
%
\nbea
b^{xy} & = & \left(b^{p^\beta}\right)^{y} \equiv 1 \pmod{p^\alpha} \\
\to \left(b^{p^\beta}\right)^y & \equiv & 1 \pmod{p^{\alpha-\beta}}
\neea
%
but when extending the roots $a^y \equiv 1 \pmod{p}$ to mod $p^{\alpha-\beta}$ the extension is unique mod $p^{\alpha-\beta}$, so if there's another root it must be of the form $a + w\cdot p^{\alpha-\beta}$ hence there can't be any other roots.

\bigskip
\underline{\textbf{\textit{The Case of $2p^\alpha$}}}
\bigskip

For $2p^\alpha$ we can repeat the whole process again (although we have to first show that there are unique roots modulo $2p$ and then extend it to mod $2p^\alpha$, this will be discussed later) or we can just utilize the following fact 

{\bf Proposition E.4}. If the order of $x$ modulo $a$ is $o_a$ and the order of $x$ mod $b$ is $o_b$ and $\gcd(a,b) = 1$ then the order of $x$ mod $ab$ is lcm$(o_a,o_b)$.

{\it Proof}. First suppose we pick a number $x$ and two other numbers $a,b$ with $\gcd(m,n) = 1$, the orders of $x$ are given by
%
\nbea
x^{o_a} & \equiv & 1 \pmod{a} \\
x^{o_b} & \equiv & 1 \pmod{b}
\neea
%
Denote ${\rm lcm}(o_a,o_b) = d$, it is then true that
%
\nbea
x^d &\equiv & 1 \pmod{a} \\
& = & 1 + au\\
x^d & \equiv & 1 \pmod{b} \\
& = & 1 + bv
\neea
%
Furthermore
%
\nbea
x^d & = & x^d \\
1 + au & = & 1 +bv \\
au & = & bv
\neea
%
Since $\gcd(a,b)=1$ this means that $b|u$ and $a|v \to au=bv = abs$, thus
%
\nbea
x^d & = & 1 + ab s \\
& \equiv & 1 \pmod{ab}
\neea
%
We know that the order of $x$ modulo $ab$ must be divisible by $o_a$ and $o_b$ and $d = {\rm lcm}(o_a,o_b)$ is the smallest number that is divisible by $o_a$ and $o_b$, therefore $x^d \equiv 1 \pmod{ab}$ means that the order of $x$ modulo $ab$ is indeed $d = {\rm lcm}(o_a,o_b)$.

Since we have proven that there are primitive roots, $g$, modulo $p^\alpha$, using Proposition E.4, the order of $g$ modulo $2p^\alpha$ is then given by lcm$(\varphi(2),\varphi(p^\alpha)) = \varphi(p^\alpha) = \varphi(2p^\alpha)$, thus the primitive roots modulo $p^\alpha$ are also primitive roots modulo $2p^\alpha$ and there are as many primitive roots for both as $\varphi(2p^\alpha) = \varphi(p^\alpha)$.

\bigskip
\underline{\textbf{\textit{All Other Cases}}}
\bigskip

We now use this result to characterize the primitive roots of a number $w$ based on its prime factorization. Thanks to Euclid, every number $w$ can be factorized into
%
\nbea
w & = & p_1^{n_1}p_2^{n_2}p_3^{n_3} \dots p_z^{n_z} \\
& = & \prod_{i=1}^{z} p_i^{n_i} \\
\to \varphi(n) & = & \prod_{i=1}^{z} p_i^{n_i-1}(p_i - 1)
\neea
%
We now list every number $1 < x < w$ for each distinct $\pmod{p_i^{n_i}}$, where $a_i$ is the order of each $x$ modulo $p_i^{n_i}$ where each $a_i$ is limited by Euler's theorem to be $a_i|(p_i^{n_i-1}(p_i-1))$, see Table.~\ref{Tab:2}
%
\begin{table}[]
\centering
\caption{}
\label{Tab:2}
\begin{tabular}{|c|c|c|c c c|}
\hline
$w=\prod_{1}^{z}$ & $p_1^{n_1}$ & $p_2^{n_2}$ & $p_3^{n_3}$ & $\dots$ & $p_z^{n_z}$ \\
$\varphi(p_i^{n_i}) = $ & $p_1^{n_1-1}(p_1-1)$ & $p_2^{n_2-1}(p_2-1)$ & $p_3^{n_3-1}(p_3-1)$ & $\dots$ & $p_z^{n_z-1}(p_z-1)$ \\ \hline 
 & $x^{a_1} \equiv 1 \pmod{p_1^{n_1}}$ & $x^{a_2} \equiv 1 \pmod{p_2^{n_2}}$ & $x^{a_3} \equiv 1 \pmod{p_3^{n_3}}$ & $\dots$ & $x^{a_z} \equiv 1 \pmod{p_z^{n_z}}$ \\
 & $a_1|(p_1^{n_1-1}(p_1-1))$ & $a_2|(p_2^{n_2-1}(p_2-1))$ & $a_3|(p_3^{n_3-1}(p_3-1))$ & $\dots$ & $a_z|(p_z^{n_z-1}(p_z-1))$ \\ \hline
\end{tabular}
\end{table}
%
As shown above, the order of $x$ modulo $w$ is given by
%
\nbea
x^{{\rm lcm}(a_1, a_2, \dots, a_z)} & \equiv & 1 \pmod{p_1^{n_1}p_2^{n_2} \dots p_z^{n_z}}
\neea
%
since $\gcd(p_i^{n_i},p_j^{n_j}) = 1$ for any $i, j$. For $x$ to be a primitive root of $n$ we need
%
\nbea
{\rm lcm}(a_1, a_2, \dots, a_z) & = & p_1^{n_1-1}(p_1 - 1)p_2^{n_2-1}(p_2 - 1) \dots p_z^{n_z-1}(p_z - 1)
\neea
%
Now since each $a_i|(p_i^{n_i-1}(p_i-1))$, for the above requirement to hold it has to be that each $a_i = (p_i^{n_i-1}(p_i-1))$. However, $2|(p_i-1)$ for every $p_i \neq 2$, therefore
%
\nbea
{\rm lcm}(a_1, a_2, \dots, a_z) & \le & p_1^{n_1-1}(p_1 - 1)p_2^{n_2-1}(p_2 - 1) \dots p_z^{n_z-1}(p_z - 1)
\neea
%
Thus except for $2, 4, p^\alpha$ and $2p^\alpha$, there is no primitive roots since the lcm is always smaller than $\varphi(w)$.

\bigskip
\underline{\textbf{\textit{Conclusion}}}
\bigskip

We have shown a method to prove the existence of primitive roots by counting their exact number through simple root counting of $x^\varphi(p) \equiv 1$ defined on $\mathbb{Z}/p\mathbb{Z}$, primitive roots are therefore the roots of aforementioned polynomial that are not roots of $x^d \equiv 1 \pmod{p}$ where $d|\varphi(p)$, special cases must be handled for odd primes $p^\alpha$ and $2^\alpha$, other cases can be derived from these. The conclusion is that only numbers of the form $2, 4, p, p^\alpha, 2p^\alpha$ have primitive roots.

\bigskip
\underline{\textbf{\textit{Some Afterthoughts}}}
\bigskip

The key ingredient here is of course the fact that a polynomial of degree $f$ over a field has only at most $f$ roots. Why doesn't it work for other cases? For one thing, if $\mathbb{Z}/n\mathbb{Z}$ is not a field, it implies that $n$ is composite. Take for example $x^2 \equiv 1 \pmod{15}$, there are actually four roots because the roots correspond to the Chinese Remainder map (see Shoup's book), they are
%
\nbea
x & \equiv & \pm 1 \pmod{3} \\
x & \equiv & \pm 1 \pmod{5} 
\neea
%
What happens if we restrict our polynomial to the unit group, $(\mathbb{Z}/n\mathbb{Z})^*$, instead? Well, for one thing the unit group doesn't have 0, so we can't set our polynomial to 0, fine, let's append 0 to the unit group, will that work? Using the same example above, it will still not work because $1$ is obviously co-prime to 3 and 5 and 15, thus the four $x$'s will still be co-prime to 15 and they are all roots.

So any odd composite number won't work, how about special cases like $p^\alpha$? Using our extension mechanism above we can show that a polynomial over $\mathbb{Z}/p^\alpha\mathbb{Z}$. Say we have a root $a$ on a polynomial of degree $f$ modulo $p$
%
\nbea
\sum_{i = 0}^f c_i x^i, ~~~~~ c_f \not\equiv 0
\neea
%
Let's do it one term at a time ($j > 0$)
%
\nbea
(a + np)^j & = & a^j + \sum_{k=1}^{j} \binom{j}{k} a^{j-k}(np)^k \\
& \equiv & a^j + ja^{j-1}np \pmod{p^2}
\neea
%
so
%
\nbea
\sum_{i = 0}^f c_i a^i & = & \sum_{i = 0}^f c_i (a^i + ia^{i-1}np) \pmod{p^2} \\
& = & \sum_{i = 0}^f c_i a^i + \sum_{i = 1}^f c_i~ i~ a^{i-1}np \pmod{p^2} \\
& = & wp + \sum_{i = 1}^f c_i~ i~ a^{i-1}np \pmod{p^2}
\neea
%
we want the whole thing to be $\equiv 0 \pmod{p^2}$
%
\nbea
w\bcancel{p} + \sum_{i = 1}^f c_i~ i~ a^{i-1}n\bcancel{p} & \equiv & 0 \pmod{p^{\bcancel{2}}} \\
\sum_{i = 1}^f c_i~ i~ a^{i-1}n & \equiv & -w \pmod{p} \\
n \cdot s & \equiv & -w \pmod{p} ~~~~~~~ s = \sum_{i = 1}^f c_i~ i~ a^{i-1}
\neea
%
Note that this is also true going from mod $p^n$ to $p^{n+1}$, as we will cancel $p^n$ from both sides instead of $p$ i the last step above. There are a few things to consider
%
\bit
%
\item If $p|w$ but $p\nmid s$ then $n|p$, in this case the root modulo $p^2$, $a'$, is the same as the one modulo $p$, $a$ because then $a' = a + p^2t$ and we want $a' < p^2$.
%
\item If $p\nmid w$ but $p\nmid s$ then there is a unique $n$ and therefore there is only one root modulo $p^2$ corresponding to a root modulo $p$
%
\item If $p\nmid w$ but $p|s$ then there is no solution for $n$ but this only means that there are less roots modulo $p^2$ compared to those of modulo $p$.
%
\item If $p|w$ but $p|s$ then $n$ can be anything but that also means that any $a' = a + np$ will be a root modulo $p^2$, this means that there are $p$ new roots $a'$ associated with the root $a$ mod $p$.
%
\eit
%
So as long as $p\nmid w$ or $p\nmid s$ we are good but looking closely, $s$ is just the derivative of the polynomial, so 

{\bf Proposition E.5}. For a polynomial of degree $f$ over a ring $\mathbb{Z}/p^\alpha\mathbb{Z}$, there are at most $f$ roots as long as the derivative of that polynomial has no roots other than zero mod $p$ or the roots modulo $p$ are not roots modulo $p^2$.

And for our case $x^d - 1 \equiv 0$, the derivative is just $d x^{d-1} \equiv 0$ and so the derivative is only zero if $x \equiv 0$ and thankfully that cannot be the case so using this method we could've proven the case of $p^\alpha$ rather quickly.


\bigskip
\underline{\textit{\textbf{Extending Roots from $p$ to $2p$}}}
\bigskip

So how do we extend the roots of $x^d \equiv 1 \pmod{p}$ to modulo $2p$? Well, for one thing if we have roots $x^d \equiv 1 \pmod{2p}$ then it is also true that each root satisfies
%
\nbea
x^d & \equiv & 1 \pmod{2} \\
x^d & \equiv & 1 \pmod{p}
\neea
%
from the first congruence it is guaranteed to be odd and from the second congruence the root mod $2p$ must also be a root modulo $p$. Say $a$ is a root modulo $p$ so by default $a < p$. If $a$ is even the root modulo $2p$ must then be of the form $a + p$ because, one it has to be odd and two it must be less than $2p$. If $a$ is already odd then $a$ is also a root modulo $2p$. 

The important question now is whether $a$ is also a root of $x^d \equiv 1 \pmod{2p}$, just because $a^d$ satisfy the above two congruences does it mean that $a^d \equiv 1 \pmod{2p}$? The answer is yes as it is guaranteed by the Chinese Remainder Map.

Here's how it works. Say we have a system of congruences (for simplicity assume there are only 2)
%
\nbea
y & \equiv & c_1 \pmod{p_1} \\
y & \equiv & c_2 \pmod{p_2}
\neea
%
Chinese Remainder Theorem guarantees that there is such a unique $y \equiv c \pmod{p_1p_2}$. Let's exponentiate $y \to y^u \equiv c^u \pmod{p_1p_2}$, then the congruences we have above become
%
\nbea
y^u & \equiv & c_1^u \pmod{p_1} \\
y^u & \equiv & c^u_2 \pmod{p_2}
\neea
%
but again, Chinese Remainder Theorem guarantees that solving $c_1^u \pmod{p_1}$ and $c^u_2 \pmod{p_2}$ generates a unique solution modulo $p_1p_2$. Therefore this solution must be the same as $c^u$ since expressing $c^u$ in terms of $\pmod{p_1}$ and $\pmod{p_2}$ generate the same exact congruences. This is called the Chinese Remainder Map, see Shoup's book for more details. But we are not done, for our case of $x^d \equiv 1 \pmod{2p}$ we have to show $c^u \equiv 1 \pmod{2p}$, lucky for us, $c_1^u \equiv 1 \pmod{2}$ and $c_2^u \equiv 1 \pmod{p}$ and we know that if we have a system of congruences
%
\nbea
x & \equiv & k \pmod{h_1} \\
x & \equiv & k \pmod{h_2} \\
& \vdots & \\
x & \equiv & k \pmod{h_m}
\neea
%
then $x$ is given by $x \equiv k \pmod{h_1h_2\cdots h_m}$. Therefore $c^u \equiv 1 \pmod{2p}$ and Chinese Remainder Map guarantees that the solution mod $p$ is also a solution mod $2p$.

Therefore the roots of $x^d \equiv 1 \pmod{2p}$ are given by
%
\nbea
x & = & \left \{
\begin{array}{l l}
a & ~~~~~~~ {\rm if~} a ~{\rm is~ odd} \\
a + p & ~~~~~~~ {\rm if~} a~ {\rm is~ even}
\end{array}
\right.
\neea
%
where $a$ is a root of $x^d \equiv 1 \pmod{p}$ and $a < p$. What this means is that every root modulo $p$ is mapped uniquely to a root modulo $2p$. Can there be any other roots? No because any root modulo $2p$ must be a root modulo $p$ and we have shown above that the extension to mod $2p$ is unique.

The awesome thing about the Chinese Remainder Map is that it works not only for multiplication (exponentiation is just a repeated multiplications) but also for addition (well, multiplication is just repeated additions). Therefore, if we have a polynomial of order $f$, $P_f(x)$ modulo $p$ and $a$ is a root mod $p$ then by the construction above we can extend it to mod $2p$, but instead of $1 \pmod{2}$ we get $c_0 \pmod{2}$ for the condition where $c_0$ is the coefficient of $x^0$ in $P_f(x)$
%
\nbea
x & = & \left \{
\begin{array}{l l}
a & ~~~~~~~ {\rm if~} a ~{\rm has~ the~same~parity~as~} -c_0 \\
a + p & ~~~~~~~ {\rm if~} a~ {\rm has~the~opposite~parity~of~} -c_0
\end{array}
\right.
\neea
%
Because we extend the roots mod $p$ to mod $2p$ uniquely, just like Proposition E.5, we have

{\bf Proposition E.5.1} as long as the derivative doesn't have a root other than 0 modulo $p$ or the roots mod $p$ are not root modulo $p^2$, a polynomial of degree $f$ modulo $2p^\alpha$ has at most $f$ roots.

Why doesn't it work for $2^\beta p^\alpha$? For mod $2^\beta p^\alpha$, we can have more roots than the degree of the polynomial, this is because we can extend the roots from $2p \to 4p \to \dots$. Once we have the roots mod $2p$, we just need to add $2p$ to this root to get another root, and it is also a root, again, guaranteed by the Chinese Remainder Map, say we have a root, $a + p$, mod $2p$ we set up the following system of congruences
%
\nbea
x & \equiv & a + p + 2p \pmod{4} \\
x & \equiv & a + p + 2p \pmod{p}
\neea
%
which gives us $x = a + p + 2p$ as a solution modulo $4p$ since it is less than $4p$, putting this into the polynomial we get
%
\nbea
P_f(x) - c_0 & \equiv & -c_0 \pmod{4} \\
P_f(x) - c_0 & \equiv & -c_0 \pmod{p}
\neea
%
and by the Chinese Remainder Map as explained above it must be a solution modulo $4p$.

Let's see it with a concrete example, $x^2 \equiv 1 \pmod{3}$. Extending it from $3 \to 2\cdot 3$, we see that the roots mod 3 are $1$ and $2$. Here, $c_0$ is odd, so $1 \to 1$ and $2 \to 2+3 = 5$ and the roots mod $2\cdot 3$ are 1 and 5. If we now go from $2\cdot 3 \to 4\cdot 3$, we just need to add $2\cdot 3$ to 1 and 5, so the roots mod $4\cdot 3$ are $1, 5, 7, 11$, they are all roots because they satisfy the congruences mod $p$ and mod $4$ and by the help of our Chinese friend they are also roots mod $4p$.

{\bf Proposition E.6}. We can use this as a generic strategy to find roots of polynomial over a generic ring $\mathbb{Z}/n\mathbb{Z}$, write $n$ in terms of its primal constituents 
%
\nbea
n & = & 2^\beta \prod_m q_m^{\alpha_m}
\neea
%
and the we find the roots over each {\it odd} prime ring, $\mathbb{Z}/q_m\mathbb{Z}$, we extend the roots from $q \to q^{\alpha}$ for each $q$, once we have all this we just use Chinese Remainder Theorem to get the combined solution modulo $\prod_m q_m^{\alpha_m}$. If we have a factor of $2^\beta$ we then extend the roots to modulo $2\prod_m q_m^{\alpha_m}$ by sufficiently adding $\prod_m q_m^{\alpha_m}$ depending on the parity of $-c_0$, once we've done this we add more roots by adding $2\prod_m q_m^{\alpha_m}$ followed by adding $2^2\prod_m q_m^{\alpha_m}$ and so on until we add $2^{\beta-1}\prod_m q_m^{\alpha_m}$.

So if there are $k$ roots modulo $\prod_m q_m^{\alpha_m}$ there will be $2^{\beta-1} k$ roots modulo $2\prod_m q_m^{\alpha_m}$.




























\end{document}