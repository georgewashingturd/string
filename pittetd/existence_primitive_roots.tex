\documentclass[aps,preprint,preprintnumbers,nofootinbib,showpacs,prd]{revtex4-1}
\usepackage{graphicx,color}
\usepackage{caption}
\usepackage{subcaption}
\usepackage{amsmath,amssymb}
\usepackage{multirow}
\usepackage{amsthm}%        But you can't use \usewithpatch for several packages as in this line. The search 

\usepackage{cancel}

%%% for SLE
\usepackage{dcolumn}   % needed for some tables
\usepackage{bm}        % for math
\usepackage{amssymb}   % for math
\usepackage{multirow}
%%% for SLE -End

\usepackage{ulem}
\usepackage{cancel}

\usepackage{hyperref}

\usepackage[top=1in, bottom=1.25in, left=1.1in, right=1.1in]{geometry}

\usepackage{mathtools} % for \DeclarePairedDelimiter{\ceil}{\lceil}{\rceil}

\usepackage{simplewick}

\newcommand{\msout}[1]{\text{\sout{\ensuremath{#1}}}}


%%%%%% My stuffs - Stef
\newcommand{\lsim}{\mathrel{\mathop{\kern 0pt \rlap
  {\raise.2ex\hbox{$<$}}}
  \lower.9ex\hbox{\kern-.190em $\sim$}}}
\newcommand{\gsim}{\mathrel{\mathop{\kern 0pt \rlap
  {\raise.2ex\hbox{$>$}}}
  \lower.9ex\hbox{\kern-.190em $\sim$}}}

%
% Key
%
\newcommand{\key}[1]{\medskip{\sffamily\bfseries\color{blue}#1}\par\medskip}
%\newcommand{\key}[1]{}
\newcommand{\q}[1] {\medskip{\sffamily\bfseries\color{red}#1}\par\medskip}
\newcommand{\comment}[2]{{\color{red}{{\bf #1:}  #2}}}


\newcommand{\ie}{{\it i.e.} }
\newcommand{\eg}{{\it e.g.} }

%
% Energy scales
%
\newcommand{\ev}{{\,{\rm eV}}}
\newcommand{\kev}{{\,{\rm keV}}}
\newcommand{\mev}{{\,{\rm MeV}}}
\newcommand{\gev}{{\,{\rm GeV}}}
\newcommand{\tev}{{\,{\rm TeV}}}
\newcommand{\fb}{{\,{\rm fb}}}
\newcommand{\ifb}{{\,{\rm fb}^{-1}}}

%
% SUSY notations
%
\newcommand{\neu}{\tilde{\chi}^0}
\newcommand{\neuo}{{\tilde{\chi}^0_1}}
\newcommand{\neut}{{\tilde{\chi}^0_2}}
\newcommand{\cha}{{\tilde{\chi}^\pm}}
\newcommand{\chao}{{\tilde{\chi}^\pm_1}}
\newcommand{\chaop}{{\tilde{\chi}^+_1}}
\newcommand{\chaom}{{\tilde{\chi}^-_1}}
\newcommand{\Wpm}{W^\pm}
\newcommand{\chat}{{\tilde{\chi}^\pm_2}}
\newcommand{\smu}{{\tilde{\mu}}}
\newcommand{\smur}{\tilde{\mu}_R}
\newcommand{\smul}{\tilde{\mu}_L}
\newcommand{\sel}{{\tilde{e}}}
\newcommand{\selr}{\tilde{e}_R}
\newcommand{\sell}{\tilde{e}_L}
\newcommand{\smurl}{\tilde{\mu}_{R,L}}

\newcommand{\casea}{\texttt{IA}}
\newcommand{\caseb}{\texttt{IB}}
\newcommand{\casec}{\texttt{II}}

\newcommand{\caseasix}{\texttt{IA-6}}

%
% Greek
%
\newcommand{\es}{{\epsilon}}
\newcommand{\sg}{{\sigma}}
\newcommand{\dt}{{\delta}}
\newcommand{\kp}{{\kappa}}
\newcommand{\lm}{{\lambda}}
\newcommand{\Lm}{{\Lambda}}
\newcommand{\gm}{{\gamma}}
\newcommand{\mn}{{\mu\nu}}
\newcommand{\Gm}{{\Gamma}}
\newcommand{\tho}{{\theta_1}}
\newcommand{\tht}{{\theta_2}}
\newcommand{\lmo}{{\lambda_1}}
\newcommand{\lmt}{{\lambda_2}}
%
% LaTeX equations
%
\newcommand{\beq}{\begin{equation}}
\newcommand{\eeq}{\end{equation}}
\newcommand{\bea}{\begin{eqnarray}}
\newcommand{\eea}{\end{eqnarray}}
\newcommand{\ba}{\begin{array}}
\newcommand{\ea}{\end{array}}
\newcommand{\bit}{\begin{itemize}}
\newcommand{\eit}{\end{itemize}}

\newcommand{\nbea}{\begin{eqnarray*}}
\newcommand{\neea}{\end{eqnarray*}}
\newcommand{\nbeq}{\begin{equation*}}
\newcommand{\neeq}{\end{equation*}}

\newcommand{\no}{{\nonumber}}
\newcommand{\td}[1]{{\widetilde{#1}}}
\newcommand{\sqt}{{\sqrt{2}}}
%
\newcommand{\me}{{\rlap/\!E}}
\newcommand{\met}{{\rlap/\!E_T}}
\newcommand{\rdmu}{{\partial^\mu}}
\newcommand{\gmm}{{\gamma^\mu}}
\newcommand{\gmb}{{\gamma^\beta}}
\newcommand{\gma}{{\gamma^\alpha}}
\newcommand{\gmn}{{\gamma^\nu}}
\newcommand{\gmf}{{\gamma^5}}
%
% Roman expressions
%
\newcommand{\br}{{\rm Br}}
\newcommand{\sign}{{\rm sign}}
\newcommand{\Lg}{{\mathcal{L}}}
\newcommand{\M}{{\mathcal{M}}}
\newcommand{\tr}{{\rm Tr}}

\newcommand{\msq}{{\overline{|\mathcal{M}|^2}}}

%
% kinematic variables
%
%\newcommand{\mc}{m^{\rm cusp}}
%\newcommand{\mmax}{m^{\rm max}}
%\newcommand{\mmin}{m^{\rm min}}
%\newcommand{\mll}{m_{\ell\ell}}
%\newcommand{\mllc}{m^{\rm cusp}_{\ell\ell}}
%\newcommand{\mllmax}{m^{\rm max}_{\ell\ell}}
%\newcommand{\mllmin}{m^{\rm min}_{\ell\ell}}
%\newcommand{\elmax} {E_\ell^{\rm max}}
%\newcommand{\elmin} {E_\ell^{\rm min}}
\newcommand{\mxx}{m_{\chi\chi}}
\newcommand{\mrec}{m_{\rm rec}}
\newcommand{\mrecmin}{m_{\rm rec}^{\rm min}}
\newcommand{\mrecc}{m_{\rm rec}^{\rm cusp}}
\newcommand{\mrecmax}{m_{\rm rec}^{\rm max}}
%\newcommand{\mpt}{\rlap/p_T}

%%%song
\newcommand{\cosmax}{|\cos\Theta|_{\rm max} }
\newcommand{\maa}{m_{aa}}
\newcommand{\maac}{m^{\rm cusp}_{aa}}
\newcommand{\maamax}{m^{\rm max}_{aa}}
\newcommand{\maamin}{m^{\rm min}_{aa}}
\newcommand{\eamax} {E_a^{\rm max}}
\newcommand{\eamin} {E_a^{\rm min}}
\newcommand{\eaamax} {E_{aa}^{\rm max}}
\newcommand{\eaacusp} {E_{aa}^{\rm cusp}}
\newcommand{\eaamin} {E_{aa}^{\rm min}}
\newcommand{\exxmax} {E_{\neuo \neuo}^{\rm max}}
\newcommand{\exxcusp} {E_{\neuo \neuo}^{\rm cusp}}
\newcommand{\exxmin} {E_{\neuo \neuo}^{\rm min}}
%\newcommand{\mxx}{m_{XX}}
%\newcommand{\mrec}{m_{\rm rec}}
\newcommand{\erec}{E_{\rm rec}}
%\newcommand{\mrecmin}{m_{\rm rec}^{\rm min}}
%\newcommand{\mrecc}{m_{\rm rec}^{\rm cusp}}
%\newcommand{\mrecmax}{m_{\rm rec}^{\rm max}}
%%%song

\newcommand{\mc}{m^{\rm cusp}}
\newcommand{\mmax}{m^{\rm max}}
\newcommand{\mmin}{m^{\rm min}}
\newcommand{\mll}{m_{\mu\mu}}
\newcommand{\mllc}{m^{\rm cusp}_{\mu\mu}}
\newcommand{\mllmax}{m^{\rm max}_{\mu\mu}}
\newcommand{\mllmin}{m^{\rm min}_{\mu\mu}}
\newcommand{\mllcusp}{m^{\rm cusp}_{\mu\mu}}
\newcommand{\elmax} {E_\mu^{\rm max}}
\newcommand{\elmin} {E_\mu^{\rm min}}
\newcommand{\elmaxw} {E_W^{\rm max}}
\newcommand{\elminw} {E_W^{\rm min}}
\newcommand{\R} {{\cal R}}

\newcommand{\ewmax} {E_W^{\rm max}}
\newcommand{\ewmin} {E_W^{\rm min}}
\newcommand{\mwrec}{m_{WW}}
\newcommand{\mwrecmin}{m_{WW}^{\rm min}}
\newcommand{\mwrecc}{m_{WW}^{\rm cusp}}
\newcommand{\mwrecmax}{m_{WW}^{\rm max}}

\newcommand{\mpt}{{\rlap/p}_T}

%%%%%% END My stuffs - Stef

\newcommand{\dunno}{$ {}^{\mbox {--}}\backslash(^{\rm o}{}\underline{\hspace{0.2cm}}{\rm o})/^{\mbox {--}}$}

\DeclarePairedDelimiter{\ceil}{\lceil}{\rceil}
\DeclarePairedDelimiter{\floor}{\lfloor}{\rfloor}

\DeclareMathOperator{\ord}{ord}
\DeclareMathOperator{\tor}{tor}





\begin{document}

\title{Primitive Existentialism in Counting Roots}
\bigskip
\author{Stefanus$^1$\\
$^1$ Samsung Semiconductor Inc\\ San Jose, CA 95134 USA\\
}
%
\date{\today}
%
\begin{abstract}
Proving the existence of primitive roots modulo 2,4, $p$ and $2p^\alpha$ by explicit counting the number of roots of the polynomial $x^d - 1\equiv 0$.

\end{abstract}
%
\maketitle

\renewcommand{\theequation}{A.\arabic{equation}}  % redefine the command that creates the equation no.
\setcounter{equation}{0}  % reset counter 

\textbf{\textit{*}}\underline{\textit{\textbf {Parental Advisory, Explicit Content}}}\textbf{\textit{*}}

In the following discussion we will derive the number of primitive roots modulo $n$ (not necessarily prime) \underline{\textbf{\textit{Explicitly}}}\textbf{\textit{!}} and in doing so we will prove that only 2, 4, $p$ and $2p^\alpha$ have primitive roots.

\bigskip
\underline{\textit{\textbf{Basic Ingredients}}}
\bigskip

We will achieve our goal of counting roots using a few facts.
\bit
%
\item First, any polynomial of degree $f$ defined on a field has at most $f$ roots, see ent.pdf Proposition 2.5.3.
%
\item Next, the fact that $\mathbb{Z}/p\mathbb{Z}$ is a field as shown in Exercise 2.12 of ent.pdf.
%
\item Lastly using an application of Proposition 2.5.5 of ent.pdf which states that for each divisor $d|(p-1)$, the polynomial of degree $d$ has exactly $d$ roots in $\mathbb{Z}/p\mathbb{Z}$
%
\eit
%
Although Proposition 2.5.5 can be automatically extended to $d|\varphi(n)$ when $n$ is not prime as long as $\mathbb{Z}/n\mathbb{Z}$ forms a field. The problem is that for $n$ not prime $\mathbb{Z}/n\mathbb{Z}$ is guaranteed not to be a field. So for now we just concentrate on $n$ being prime.

\bigskip
\underline{\textit{\textbf{Counting Primal Roots}}}
\bigskip

With all these in mind we proceed to calculate the number of primitive roots in the unit group $(\mathbb{Z}/n\mathbb{Z})^*$. As in above we know the polynomial $x^{\varphi(n)} - 1 \equiv 0$ has exactly $\varphi(n)$ roots. The primitive roots are then the roots that are \underline{{\bf not}} roots of $x^d - 1 \equiv 0$ where $d|\varphi(n)$.

Now suppose that $\varphi(n)$ is just a product of two primes
%
\nbea
\varphi(n) & = & q_1q_2
\neea
%
Now from the $q_1q_2$ roots of $x^{\varphi(n)} \equiv 1$, how many of them are roots of $x^{q_1} \equiv 1$ and how many are roots of $x^{q_2} \equiv 1$? The roots that are not covered by $x^{q_{1}} \equiv 1$ and $x^{q_{2}} \equiv 1$ will be the primitive roots of $n$. So are there any?

From the $q_1q_2$ roots of $x^{\varphi(n)} \equiv 1$ we need to subtract $q_1$ roots that belong to $x^{q_1} \equiv 1$ and $q_2$ roots that belong to $x^{q_2} \equiv 1$, this is because we know that there are exactly $q_1$ and $q_2$ roots for those two polynomials as given by Proposition 2.5.5 of ent.pdf.

But in doing the subtractions we were double counting because 1 is always a root of $x^d - 1 \equiv 0$ whatever $d$ is, so when subtracting the roots of $x^{q_1} \equiv 1$ we already remove 1, thus we need to add 1 back, thus the number of primitive roots are
%
\nbea
q_1q_2 - q_1 - q_2 + 1 & = & (q_1 - 1)(q_2 - 1) \\
& = & \varphi(q_1q_2) \\
& = & \varphi(\varphi(n))
\neea
%
Now what happens if $\varphi(n)$ has three distinct prime factors? we do the same, we start with $q_1q_2q_3$ as the total number of roots, we then remove the ones already covered by $q_1q_2$, followed by $q_1q_3$ and finally $q_2q_3$. But in doing so we are again over counting because while removing the roots of $x^{q_1q_2} \equiv 1$ we already removed the roots of $x^{q_1} \equiv 1$ (and of $x^{q_2} \equiv 1$) but when we removed the roots of $x^{q_1q_3} \equiv 1$ we again remove the roots of $x^{q_1} \equiv 1$, so we need to add them back in.
%
\nbea
q_1q_2q_3 - q_1q_2 - q_1q_3 - q_2q_3 + q_1 + q_2 + q_3
\neea
%
But as we are removing roots and adding back over counted ones, we also remove and add 1 (since 1 is always a root), here we removed it three times and then we added it back in three times, but 1 still should be removed, so the final tally is
%
\nbea
q_1q_2q_3 - q_1q_2 - q_1q_3 - q_2q_3 + q_1 + q_2 + q_3 - 1 & = & (q_1 - 1)(q_2 - 1)(q_3 - 1) \\
& = & \varphi(\varphi(n))
\neea
%
so the pattern continues. But what if we have a more generic
%
\nbea
\varphi(n) & = & q_1^{a_1}q_2^{a_2} \dots q_n^{a_n}
\neea
%
The pattern is still the same, let's limit $n=3$ to see a concrete example, again we start with $q_1^{a_1}q_2^{a_2}q_3^{a_3}$ we then remove the roots belonging to $q_1^{a_1}q_2^{a_2}q_3^{a_3-1}$ followed by the ones in $q_1^{a_1}q_2^{a_2-1}q_3^{a_3}$ and finally $q_1^{a_1-1}q_2^{a_2}q_3^{a_3}$.

Note that by removing the roots of $q_1^{a_1}q_2^{a_2}q_3^{a_3-1}$ we are already removing the roots of all its divisors. But just like before, we are over counting because we removed the roots of $q_1^{a_1}q_2^{a_2-1}q_3^{a_3-1}$ twice, once from $q_1^{a_1}q_2^{a_2}q_3^{a_3-1}$ and another time from $q_1^{a_1}q_2^{a_2-1}q_3^{a_3}$, so we need to add them back in
%
\nbea
q_1^{a_1}q_2^{a_2}q_3^{a_3} &&  - q_1^{a_1}q_2^{a_2}q_3^{a_3-1} - q_1^{a_1}q_2^{a_2-1}q_3^{a_3}- q_1^{a_1-1}q_2^{a_2}q_3^{a_3} \\
&& + q_1^{a_1}q_2^{a_2-1}q_3^{a_3-1} + q_1^{a_1-1}q_2^{a_2}q_3^{a_3-1} + q_1^{a_1-1}q_2^{a_2-1}q_3^{a_3}
\neea
%
but just like before here we've removed the roots of $q_1^{a_1-1}q_2^{a_2-1}q_3^{a_3-1}$ three times and then added them back in three times, but we know that they should be removed, so the final tally is again
%
\nbea
q_1^{a_1}q_2^{a_2}q_3^{a_3} &&  - q_1^{a_1}q_2^{a_2}q_3^{a_3-1} - q_1^{a_1}q_2^{a_2-1}q_3^{a_3}- q_1^{a_1-1}q_2^{a_2}q_3^{a_3} \\
&& + q_1^{a_1}q_2^{a_2-1}q_3^{a_3-1} + q_1^{a_1-1}q_2^{a_2}q_3^{a_3-1} + q_1^{a_1-1}q_2^{a_2-1}q_3^{a_3} \\
&& -q_1^{a_1-1}q_2^{a_2-1}q_3^{a_3-1}
\neea
%
which is just $(q_1^{a_1} - q_1^{a_1-1})(q_2^{a_2}-q_2^{a_2-1})(q_3^{a_3} - q_3^{a_3-1}) =\varphi(q_1^{a_1}q_2^{a_2}q_3^{a_3}) = \varphi(\varphi(n))$. Note that we don't need to mess with other divisors of $\varphi(n)$ because for example the roots of $x^3 - 1$ are already covered by the roots of $x^{3^2} - 1$ and so on.

So the generic strategy is to start with $\varphi(n)$ roots, express $\varphi(n)$ in terms of its primal constituents and then start removing the roots of the next highest divisor of $\varphi(n)$ and then take care of all the double counting until there's no more over counting and stop. We will then proceed through induction to prove it for any prime $n$.

Since $\varphi(\varphi(n))$ can never be zero and as long as $\mathbb{Z}/n\mathbb{Z}$ is a field, there's always exactly $\varphi(\varphi(n))$ primitive roots, since in this case $n$ is prime $\mathbb{Z}/n\mathbb{Z}$ is guaranteed to be a field, so we have also not only proven that there are always primitive roots modulo a prime $p$ but also how many there are.

\bigskip
\underline{\textit{\textbf{Composite Conundrum}}}
\bigskip

So we have proved that every off prime $p$ has $\varphi(\varphi(p))$ primitive roots, what about composite numbers? First we tackle the case of $n = p^\alpha$.

Here $\varphi(n) = p^{\alpha-1}(p-1)$, so we tackle the problem of $x^d - 1\equiv 0 \pmod{p^\alpha}$ with $d|\varphi(n)$ in three steps, first, when $d|p^{\alpha-1}$, second $d|(p-1)$ and lastly the combination of both.

The goal here is to show that $x^d - 1 \equiv 0 \pmod{p^\alpha}$ has exactly $d$ roots if $d|\varphi(p^\alpha)$. First case is $d = p^\beta$, $\beta < \alpha$.

\bigskip
\underline{\textit{\textbf{First Case}}}, $d|p^{\alpha-1}$
\bigskip

The motivation for this is as follows, take for example $p^\alpha=3^2$, $\varphi(3^2) = 3\cdot 2$, and therefore $d = 3$. If we cube all $a \in \mathbb{Z}/3^2\mathbb{Z} = \{1,2,3,\dots,3^2 = 9\}$ we get
%
\nbea
\{1^3, 2^3, 3^3,~4^3, 5^3, 6^3,~7^3, 8^3, 9^3\} \equiv \{1, 8, 0,~1, 8, 0,~1, 8, 0\} \pmod{3^2}
\neea
%
So it seems like any number $a = 1 + x\cdot (3^2/3)$ will have $a^3 = \{1+x\cdot (3^2/3)\}^3 \equiv 1 \pmod{3^2}$, so the generic formula when $d = p^\beta$, $\beta \le \alpha - 1$ is $a = 1 + x\cdot p^{\alpha-\beta}$. Exponentiating to the $d^{\rm th}$ power we get
%
\nbea
(1 + x\cdot p^{\alpha-\beta})^{p^\beta} & = & 1 + \sum_{j=1}^{p^\beta}\binom{p^\beta}{j} (x\cdot p^{\alpha-\beta})^j
\neea
%
We want to show that the sum is $\equiv 0 \pmod{p^\alpha}$. 

\smallskip
\underline{\textit{\textbf{Binomial Bifurcation}}}
\smallskip

Before we tackle the exponentiation above, we will bifurcate our discussion into binomial coefficients.

{\bf Proposition E.0}. First, for $j > 0$ 
%
\nbea
\binom{p^n}{j} & = & \left\{
\begin{array}{l l}
k \cdot p^n & ~~~~~~~ {\rm if ~} \gcd(j,p^n) = 1 \\
l \cdot p^{n-w} & ~~~~~~~ {\rm if ~} \gcd(j,p^n) = p^w, ~ 1 \le w \le n
\end{array} \right.
\neea
%
where $\gcd(k,p) = \gcd(l,p) = 1$.

{\it Proof}. Let's rewrite the binomial as
%
\nbea
\binom{p^n}{j} & = & \frac{p^n!}{j!(p^n - j)!} \\
& = & \frac{p^n\cdot(p^n-1)\cdots (p^n - j + 1)}{j\cdot(j-1)\cdots 1}
\neea
%
One thing to note is that the numerator and denominator have the same number of terms. Now let $r$ be the highest exponent such that $p^r \le j$ and rearrange the fraction as
%
\nbea
&&\frac{1}{j} \cdot \frac{1}{(j-1)} \cdots \left(\frac{p^n}{p^r}\right) \cdots \left(\frac{p^{n}-p}{p^{r}-p}\right) \cdots \left(\frac{p^{n}-2p}{p^{r}-2p}\right) \cdots \left(\frac{p^{n}-(p^{r-1} - 1)p}{p}\right) \cdots \\
&& ~~~~~~~~~~~~~~~~~~~~~~~~~~~~~~~~~~~~~~~~~~~~~~~~~~~~~~~~~~~~~~~~~~~~~~ \cdots \frac{p^{n}-p^{r} + 1}{1} \cdot \frac{(p^{n}-p^{r})^*}{1} \cdots \frac{p^{n} - j + 1}{1}
\neea
%
The terms in brackets are the only terms in the numerator (and denominator) that contain $p$, so basically we align the terms in the numerator and denominator so that those who contain $p$ are grouped together, and the last bracket with $()^*$ on it indicates that the numerator contains $p$ while the denominator doesn't, let's see this with a concrete example, take $p^n = 5^2$ and $j = 7$, we then get
%
\nbea
\frac{1}{7}~\frac{1}{6}~\left(\frac{25}{5}\right)~\frac{24}{4}~\frac{23}{3}~\frac{22}{2}~\frac{21}{1}~\frac{(20)^*}{1}
\neea
%
The reason behind this is we want to count the number of $p$ in the fraction, by grouping the terms in the numerator and denominator that contain $p$ we reduce the problem into analyzing those terms only. These terms are of the form $(p^n - yp)/(p^r - yp)$ if $(y,p^n) = p^{t-1}$ we can express $yp = xp^t$ with $(x,p) = 1$ and 
%
\nbea
\frac{p^n - xp^t}{p^r - xp^t} & = & \frac{p^{n-t} - x}{p^{r-t} - x}
\neea
%
now the numerator and denominator no longer have any factor of $p$ since $x$ is co-prime to $p$, thus the only terms in brackets that contain $p$ are
%
\nbea
\left(\frac{p^n}{p^r}\right) ~~~~~{\rm and} ~~~~~ \frac{(p^{n}-p^{r})^*}{1}
\neea
%
thus the binomial coeffcient is just $k\cdot p^n$ since
%
\nbea
\left(\frac{p^n}{p^r}\right) \cdot \frac{(p^{n}-p^{r})}{1} & = & \frac{p^n (p^{n-r}-1)}{1}
\neea
%
and $(p, (p^{n-r}-1)) = 1$. Now if $j = p^r$ then we don't have the $()^*$ term, the binomial stops one term earlier, thus the binomial is equal to $l\cdot p^{n-r}$.

{\bf Proposition E.0.1}. For odd prime $p$ and $m,n > 0$ we have
%
\nbea
(1 + xp^m)^{p^n} & \equiv & 1 + xp^{m+n} \pmod{p^{m+n+1}}
\neea
%

{\it Proof}. First we expand 
%
\nbea
(1 + xp^m)^{p^n} & = & 1 + \sum_{j=1}^{p^n} \binom{p^n}{j} (xp^m)^j
\neea
%
From Proposition E.0 we know that 
%
\nbea
\binom{p^n}{j} (xp^m)^j & = & \left\{
\begin{array}{l l}
k\cdot x^j p^{n+jm} & ~~~~~~~ {\rm if ~} \gcd(j,p^n) = 1 \\
l\cdot x^{yp^r} p^{n-r + m\cdot yp^r} & ~~~~~~~ {\rm if ~} \gcd(j,p^n) = p^r \to j = yp^r, ~\gcd(y,p) = 1, r > 0
\end{array}\right.
\neea
%
For the first case $n+jm > n+m$ if $j > 1$ and for the second case $myp^r > r$ for any $y$ and $r$ thus
%
\nbea
\binom{p^n}{j} (xp^m)^j & = & \left\{
\begin{array}{l l}
xp^{n+m} & ~~~~~~~ {\rm if ~} j = 1 \\
v\cdot p^{n + m + 1} & ~~~~~~~ {\rm if ~} j > 1
\end{array}\right.
\neea
%
therefore
%
\nbea
(1 + xp^m)^{p^n} & = & 1 + xp^{n+m} + v\cdot p^{n + m + 1} \\
& \equiv & 1 + xp^{n+m} \pmod{p^{n + m + 1}}
\neea
%

Going back to our proposed solution $(1 + x\cdot p^{\alpha-\beta})^{p^\beta}$, utilizing Proposition E.0.1 with $m = \alpha-\beta$ and $n=\beta$ we get
%
\nbea
(1 + x\cdot p^{\alpha-\beta})^{p^\beta} & = & 1 + x\cdot p^{\alpha-\beta +\beta} + v\cdot p^{\alpha-\beta+\beta + 1} \\
& = & 1 \pmod{p^\alpha}
\neea
%


The question now is how many of these numbers ($1 + x\cdot p^{\alpha-\beta}$ incongruent modulo $p^\alpha$) there are, this is equivalent to the number of unique values for $x$. The obvious answer is $0 \le x < p^\beta$, meaning we have $p^\beta$ solutions for $x^{p^\beta} - 1 \equiv 0$.

Are there more than that? What if we found a number $a^{p^\beta} - 1 \equiv 0$ besides the above? say there's the case then
%
\nbea
a^{p^\beta} & \equiv & 1 \pmod{p^\alpha} \\
\to a^{p^\beta} & \equiv & 1 \pmod{p}
\neea
%
but by Fermat's Littler Theorem this is forbidden because this means that either $p^\beta|(p-1)$ or $(p-1)|p^\beta$, except for $a \equiv 1 \pmod{p}$. So the only other possible roots is in the form $1 + xp$.

To prove that such roots do not exist we again use Proposition E.0.1, we start with
%
\nbea
(1 + x'\cdot p^{\alpha-\beta-1})^{p^\beta} & = & 1 + x'p^{\alpha - 1} + v p^{\alpha}
\neea
%
we want the RHS to be $\equiv 1 \pmod{p^\alpha}$ thus
%
\nbea
1 + x'p^{\alpha - 1} + v p^{\alpha} & = & 1 + w p^\alpha \\
x' & = & p(w - v) \\
p & | & x'
\neea
%
but this means that the necessary condition for $(1 + x'\cdot p^{\alpha-\beta-1})^{p^\beta} \equiv 1 \pmod{p^\alpha}$ is that $p | x'$. We now repeat the process with 
%
\nbea
(1 + x''\cdot p^{\alpha-\beta-2})^{p^\beta} & = & 1 + x''p^{\alpha - 2} + v' p^{\alpha-1}
\neea
%
and we still want the RHS to be $\equiv 1 \pmod{p^\alpha}$ therefore
%
\nbea
1 + x''p^{\alpha - 2} + v' p^{\alpha-1} & = & 1 + w' p^\alpha \\
x'' & = & p(pw' - v') \\
p & | & x''
\neea
%
so the necessary condition is still that $p|x''$ but this means that the solution is of the form $1 + x'\cdot p^{\alpha-\beta-1}$ but even in this case $p|x'$ so the solution is still $1 + x\cdot p^{\alpha-\beta}$, we can keep repeating with $p^{\alpha-\beta-3}$ and so on until we reach $p^{1}$ and working back up we will see that the solution must be of the form $1 + x\cdot p^{\alpha-\beta-1}$. So we have shown that $x^{p^\beta} - 1 \equiv 0$ has exactly $p^\beta$ solutions.

Next case is when $d|(p-1)$, this one is a bit trickier and we need induction. By Proposition 2.5.5 of ent.pdf we know that $x^d - 1\equiv 0 \pmod{p}$ has exactly $d$ solutions, the problem we have now is that $\mathbb{Z}/p^\alpha\mathbb{Z}$ is no longer a field. But we can build the proof case by case.

Base case is $x^d - 1\equiv 0 \pmod{p}$, based on this how can we find the solutions to $x^d - 1\equiv 0 \pmod{p^2}$? Well, we know that if there is a solution, say $a$, then $a$ has to satisfy
%
\nbea
a^d & \equiv & 1 \pmod{p^2} \\
\to a^d & \equiv & 1 \pmod{p}
\neea
%
so $a$ is also a root $\pmod{p}$ this means that it is of the form $a + np$, our task is to see whether we can find such $n$, ($a$ is guaranteed by Proposition 2.5.5). Let's see how this works
%
\nbea
(a + np)^d & = & a^d + da^{d-1}np + p^2 t, ~~~~~~~~ a^d = 1 + mp \\
& \equiv & 1 + mp + da^{d-1}np \pmod{p^2}
\neea
%
$a^d = 1 + mp$ since $a^d \equiv 1 \pmod{p}$. We want this whole thing to be $1 \pmod{p^2}$ so
%
\nbea
1 + mp + da^{d-1}np & \equiv & 1 \pmod{p^2} \\
\to m\bcancel{p} + da^{d-1}n\bcancel{p} & \equiv & 0 \pmod{p^{\bcancel{2}}} \\
da^{d-1}n\bcancel{p} & \equiv & -m \pmod{p} \\
n & \equiv & -m(da^{d-1})^{-1} \pmod{p}
\neea
%
the inverse is guaranteed since $da^{d-1}$ is co-prime to $p$ (this is a crux of the proof as we shall see soon) and $\mathbb{Z}/p\mathbb{Z}$ is a field since $p$ is prime, so $n$ is guaranteed to exist.

So we have the following solutions 
%
\nbea
a + (n + sp)p, ~~~~~ s \ge 0
\neea
%
however, for $s \ge 1$, $a + np + sp^2$ is bigger than $p^2$, so we only have one such $a < p^2$ (note that $n < p$). So for every root $a^d \equiv 1 \pmod{p}$ we have a unique root $a^d \equiv 1 \pmod{p^2}$. Using this info we can prove a unique root $\pmod{p^3}$, but this time we substitute $a = 1 +mp^2$ instead of $a = 1 + mp$, and in this way the induction is complete.

The uniqueness part is crucial because we want to show that there are exactly $d$ roots, so since we prove that there is a unique $a$ we are done.

The proofs above give us an idea on how to prove that $2^\alpha$ with $\alpha \ge 3$ doesn't have primitive roots. Say we elevate the roots of $x^2 - 1 \pmod{4}$ to modulo 8 just like before, we will get 4 roots, 1, 3, 5, 7, but $x^4 - 1 \pmod{8}$ only has 4 roots, but they are already covered by $x^2 - 1$, so that means that there are no primitive roots. Again, taking 1, 3, 5, 7, and elevating them to 9, 11, 13, 15, these eight are the roots of $x^4 - 1 \pmod{16}$ and they are also all the roots of $x^8 - 1\pmod{16}$ so 16 has no primitive roots and so on. Let's see this in detail. 

{\bf Proposition E.2}. Hypothesis, $1 + 2c$ with $0 \le c < 2^{\alpha-1}$ are all roots of $x^{\varphi(2^\alpha)/2} - 1 \equiv 0 \pmod{p^\alpha}$, so there are $2^{\alpha-1}$ roots and these roots are also roots of $x^{\varphi(2^\alpha)} - 1 \equiv 0 \pmod{2^\alpha}$, this is true for $\alpha > 2$.

{\it Proof}. We will use induction, base case is 8, with $a = 1, 3, 5, 7$ and from the assumption we know that $a^2 \equiv 1 \pmod{8} = 1 + 8m$ and so going from mod $8\to16$,
%
\nbea
a^2 & = & 1+8m \\
a^4 & = & (1 + 8m)^2 \\
& = & 1 + 16m + 64m^2 \\
& \equiv & 1 \pmod{16}
\neea
%
we know extend the solutions mod 8 from $1, 3, 5, 7$ to $9, 11, 13, 15$, so basically $a \to a + 8$, going to mod 16 we get
%
\nbea
(a + 8)^2 & = & a^2 + 16a + 64 ~~~~~~~ a^2 = 1+ 8m \\
\to & \equiv & 1 + 8m \pmod{16}
\neea
%
and so $(a + 8)^2 \equiv 1 \pmod{16}$ as well. And now we can repeat the inductive process to complete the proof which is a straightforward process by replacing 8 with $2^n$ and 16 with $2^{n+1}$ and therefore omitted here :)

So we are basically saying that all odd numbers $< 2^\alpha$ are roots of $x^{2^{\beta-1}} - 1 \pmod{2^\alpha}$ but again we also know that the roots of $x^{2^\beta} - 1 \pmod{2^\alpha}$ must also be a root of $x^{2^\beta} - 1 \pmod{2}$, which is all the odd numbers less than $2^\alpha$, so we have covered all possible roots of $x^{\varphi(2^\alpha)} - 1 \pmod{2^\alpha}$. And we see why it doesn't work for $\alpha = 2$, this is because $\varphi(2)/2 = 1$ and $x^1 - 1$ can only have one root and not two.

{\bf Proposition E.3}. As a consequence of Proposition E.2, $2^\alpha$ with $\alpha > 2$ has no primitive roots :)

The last case we have is a combination of the above two, $d|p^{\alpha-1}(p-1)$, the proof is therefore also a combination of the above two :) Let's denote $d = xy$ with $x|p^{\alpha-1} \to x = p^\beta$ with $\beta < \alpha$ and $y|(p-1)$ and by virtue of $(p,p-1) = 1$ we have $(x,y) = 1$ as well.

To tackle this first we find the solutions for $a^y - 1 \equiv 0 \pmod{p^{\alpha-\beta}}$ where $y|(p-1)$. But this is exactly the same as our previous case, the roots are the same as $a^y-1 \equiv 0 \pmod{p}$ elevated to $\pmod{p^{\alpha-\beta}}$ by the induction method above.

After finding all roots of $a^y - 1 \equiv 0 \pmod{p^{\alpha-\beta}}$, we then use these roots and extend them
%
\nbea
(a + w\cdot p^{\alpha-\beta})^{xy} & = & (a + w\cdot p^{\alpha-\beta})^{p^\beta y} \\
& = & (a^y)^{p^\beta} + \sum_{j=1}^{p^\beta} \binom{p^\beta}{j} (w\cdot p^{\alpha-\beta})^j \\
& = & (1 + s\cdot p^{\alpha-\beta})^{p^\beta} + \sum_{j=1}^{p^\beta} \binom{p^\beta}{j} (w\cdot p^{\alpha-\beta})^j \\
& = & 1 + \sum_{i=1}^{p^\beta} \binom{p^\beta}{i} (s\cdot p^{\alpha-\beta})^i + \sum_{j=1}^{p^\beta} \binom{p^\beta}{j} (w\cdot p^{\alpha-\beta})^j
\neea
%
just like before, using Proposition E.1, the sums are $0 \pmod{p^\alpha}$, the challenge now is to show that there are no other roots other than the ones shown above. Suppose there is another root, $b$, then
%
\nbea
b^{xy} & = & (b^{y})^{p^\beta} \equiv 1 \pmod{p^\alpha} \\
\to (b^{y})^{p^\beta} & \equiv & 1 \pmod{p^\delta} ~~~~~~~ 0 \le \delta < \alpha
\neea
%
but by Fermat's Little Theorem the only solution for the above is $b^y \equiv 1 \pmod{p}$ but this is what we have by extending these roots for every mod $p^{\alpha-\beta}$ so there are no other roots.

For $2p^\alpha$ we can repeat the whole process again or we can just utilize the following fact that if the order of $x$ modulo $a$ is $o_a$ and the order of $x$ mod $b$ is $o_b$ and $\gcd(a,b) = 1$ then the order of $x$ mod $ab$ is lcm$(o_a,o_b)$.




-=-=-=-=-=-=-=-=-=-=-=-=-=-=-=-=-=-=-=-


%
\nbea
(a + n\cdot2)^2 & = & a^2 + 4an + 4n^2 \\
& = & 1 + 2m + 4an + 4n^2 \\
\to 1 + 2m +4an & \equiv & 1 \pmod{4} \\
m + 4an & = & 2k
\neea
%


-=-=-=-=-=-=-=-=-=-=-=-=-=-=-=-=-=-=-=-



































\end{document}